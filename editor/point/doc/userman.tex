\documentstyle{article}

% $Header: /a/cvs/386BSD/ports/editor/point/doc/userman.tex,v 1.1 1994/02/15 22:12:47 jkh Exp $

\sloppy

\textwidth      6.5in
\oddsidemargin  0in
\evensidemargin 0in

\topmargin      0in
\headheight     0in
\headsep        0in
\textheight     8.5in
\topskip        0in

\parindent      0in
\parskip        5pt

\begin{document}

\thispagestyle{empty}

\vspace*{2in}

\begin{center}
\Huge
The Point Text Editor for X \\
\vspace{0.35in}
\Large
Version 1.63 \\
13 Jan 1994 \\
\vspace{1in}
Charles Crowley \\
Computer Science Department \\
University of New Mexico \\
Albuquerque, New Mexico 87131 \\
505-277-5446 (office) or 505-277-3112 (messages) \\
crowley@cs.unm.edu
\end{center}

\newpage

\tableofcontents

\newpage



\section{Getting Started}

I like to do most editing tasks with a mouse rather than with the
keyboard.
It is easy to find editors that allow you to do all your editing from
the keyboard but it is much harder to find an editor that will
allow you to do all your editing with the mouse.
Point is a text editor for X that uses the mouse much more extensively
than most text editors.
You can do almost all editing using only the mouse
(the one exception is entering new text).
You can also do almost everything for the keyboard if you prefer.
I find that a combination of mouse commands and keyboard commands
allows for the fastest editing,
The editor was designed for programmers and hence it is optimized
for browsing programs.
I find that I spend much more time looking at programs,
comparing programs and searching for text strings
than actually editing them in the sense of modifying their text.

Point supports any number of windows and makes it easy to move and copy
text within and between windows and provides a range of
search commands.
Point uses Tcl (tool command language) in its implementation.
Tcl is the macro language of the editor and provides an interprocess
communication mechanism as well.



\subsection{Building Point}

The README file in the Point distribution tells how to build Point.
It uses the Tk toolkit which should be available from the same ftp
site you obtained Point from (and from allspice.Berkeley.edu).
It requires X11 R4 (or R5) but does not use the Xt intrinsics
or any Xt-based widget set.



\subsection{Learning about Point}

I recommend you start by reading the manual page and try Point out first.
The manual page has everything you need to get started using Point
in a more concise form than here in the reference manual which is
very detailed (as reference manuals tend to be).

In  sections \ref{sect:browser} and \ref{sect:text}
I will describe the default configuration of Point.
This will show you almost all of the features of Point but after
you are familiar with it you will probably want to customize
Point to your own tastes.
In sections \ref{sect:customize1} and \ref{sect:customize2}
I describe how to customize Point.


\subsection{Starting Point}

You start Point with the command:
\begin{verbatim}
     point [-nobrowser] [-nb] [file ...]
\end{verbatim}

Point starts with a file browser (unless you specified the
{\tt -nobrowser} or {\tt -nb} command line option) and zero
or more text windows (depending on how many file names you
put on the command line).

When Point begins it looks for a configuration file first in {\tt .ptrc}
in your home directory.
Normally this file consists of the single line of the form:
\begin{verbatim}
     source pathName/ptsetup.tcl
\end{verbatim}
If Point does not find {\tt .ptrc} then it looks for {\tt ptsetup.tcl}
in the current directory.
If Point fails to find that it uses the default {\tt ptsetup.tcl}
file which is set up when Point is installed.

After reading its startup file, Point looks for a local (per directory)
setup file named {\tt .ptdirrc} in the current directory.
If this file is found, it is also read and interpreted.

In fact, whenever Point changes directories, it looks for a {\tt .ptdirrc}
file in the new directory and reads and interprets it if it is present.
For example, you might want to change the text colors so you could tell
from looking at the colors in a window which directory it was started from.
Or you could open a window displaying an information file for the directory
you are changing to.

The idea behind this is that that in general you will use
{\tt .ptrc} so you can start Point from any directory
and still get your customized version of {\tt ptsetup.tcl}.
But when you are just starting to use Point it is easiest just
to have {\tt ptsetup.tcl} in the current directory and Point will
find it there.

The file {\tt ptsetup.tcl} contains setup information that 
Point requires to run correctly.
Point will not work if this file is not found
and that is why there is a default version for Point
to use if if cannot find a custom one.
You can change parts of this file to reconfigure Point.
The descriptions that follow describe one configuration of Point.
Most of the behaviors described can be changed by editing {\tt ptsetup.tcl}
or while Point is running through the PREFS pull down menu.
This is explained in more detail
in sections \ref{sect:customize1} and \ref{sect:customize2}.



\subsection{Communicating with Point While It Is Running}

It is faster to start Point once and always leave it running
because it takes a fairly long time to start (5 to 15 seconds).
When Point starts it creates a file browser that can stay on
your screen.
When you want to edit a new file you can select it from the
file browser while point is running.

If the Point executable is named {\tt pt} then it does not start up
Point but instead tries to communicate with an already running
copy of Point.\footnote{
	Thus the Point executable is really two programs in one.
	I do it this way since separate copies of the executables
	shared so much library code that it was more efficient
	in disk space to combine them into one program.}
The command to use Point this way looks like this:
\begin{verbatim}
     pt [-wait] [-w] [-create] [-c] [-digit] [-interp name] [file ...]
\end{verbatim}
This command sends a message to Point telling it to open
a window on each of the specified files.
The command line options modify the details of how it does this.

Normally {\tt pt} sends a message to {\tt point} to open the windows
and then exits but
if {\tt -wait} (or {\tt -w}, its abbreviation) is specified
{\tt pt} waits for all of the windows it requested to be closed
and then {\tt pt} will exit.
If the {\tt -interp} option is specified the next argument is the
name of the interpreter to which {\tt pt} will send messages.
The default value is {\tt point}.
This is only necessary if you rename the Point executable from
{\tt point} to something else.

Point normally asks the user for verification before it creates
a new file.
This verification can be suppressed by the use of the {\tt -create}
(or {\tt -c}, its abbreviation) option which causes Point to create
the file without asking (for this file only).

The Point startup file {\tt ptsetup.tcl} defines three variables
{\tt location1}, {\tt location2} and {\tt location3} which
specify three window geometries.
By default, {\tt pt} puts new windows at {\tt location1},
but {\tt pt} also takes arguments of the form {\tt -1}, {\tt -2}, ...
{\tt -9} and will place the windows of subsequent file names
in the respective {\tt locationN}.
You can specify as many of these {\tt -N} flags and filenames
as arguments.
Each {\tt locationN} applies to all files that follow it until another
{\tt locationN} option is encountered.

There are two major uses of {\tt pt}.
The first is to open a group of files that is easy to specify
with the shell filename wild card specifications.
For example you might execute:
\begin{verbatim}
     pt *.c
\end{verbatim}
as a fast way of opening windows on all the {\tt .c} files.
The second use of {\tt pt} is to use Point as an editor
that is called as a subprocess of another program to
edit a file.
A primary example is a mail program such as {\tt mail},
{\tt elm} or {\tt xmail}.
For such programs you specify ``{\tt pt -w -c}'' as your editor of choice.
It will start {\tt pt} with a file name of a temporary file
(for the body of a mail message for example) and {\tt pt} will
open a Point window on the file and wait until the window is closed.




\section{File Browser Window} \label{sect:browser}

A file browser comes up when Point begins
(unless you specified the {\tt -nobrowser} command line option).
You can create new file browsers at any time.
You can create a file browser from the text window menu.
If you close all the file browsers and all the text windows
you can still start up new ones with the {\tt pt} command.

The directory the browser is displaying is the current directory in Point.
Some Point commands use the current directory.
For example, the tags command looks in the current directory for
the {\tt tags} file to use.
If you open a code file and then change the browser directory,
the tags command will not find the correct (or any) {\tt tags} file.

It is possible to have several browsers open at the same time.
If this is the case, the last browser the mouse entered determines
the current directory.
This is shown visually by the string {\tt CD:} in front of the directory
name in the browser title bar.



\subsection{Screen Layout}

Point puts up two types of main windows: file browsers and text windows.
A file browser contains a list of all open windows and a list of
files in the current directory (plus some titles and menus).
The contents of these windows are described in detail in this and the
next section.

The defaults for window locations are set up with the following
screen layout in mind.
(This can all be changed by the user with setup files.)
There are four main places for windows:
the northeast, northwest, southeast and southwest quadrants of the screen.
The southeast and southwest quadrants contain xterm windows,
a file browser is in the upper middle between the northwest quadrant and the
northeast quadrant
and the northwest and northeast quadrants contains stacks of text windows.
Normally you only look at one text window at a time so they are
stacked on top of one another.
When you need to look at two windows simultaneously you put one
in the northeast quadrant and the other in the northwest quadrant.
You can zoom the windows vertically while you are working on
them to see more text.



\subsection{Title bar}

Point assumes the window manager will provide a title bar for
each file browser.
The name of the window is set to the name of the directory
displayed in the file browser window.
The ``\verb+~+'' notation for your home directory is used in
the directory name in the title bar.\footnote{
	The format of the title line can be specified by the user.
	See the {\it titleFormat} and {\it browserIconFormat} options.}



\subsection{Open window list}

This pane shows the names of the files in the open windows.
This list is both for your reference and for topping and moving the
listed windows.
Clicking on the name of an open window with the left mouse button
will bring that window to the top of the stacking order
and move it to geometry 502x460+0+0.
The middle and right mouse buttons will move the window to different
predefined locations on the screen.
Clicking on a file name in the open window pane with the middle
(or right) mouse button causes the window to be moved to the second (or third)
predefined window locations with geometry 502x460-200+465
(or 502x460-0+0).\footnote{These geometries and the effect each each
mouse button can be redefined using the ptsetup.tcl startup file
or through the PREFS menu.}

An asterisk ({\tt *}) is put at the end of the file name
if the file has been edited but not yet saved.

The open window list is made three lines long when the browsers is created.
It has a scroll bar so you can scroll if there there more than three
lines but sometimes you want to have more or fewer lines.
You can increase the number of lines in the open window list by
clicking on the two arrow buttons above the list.
The up arrow reduced the size of the open window list and the down
arrow increases it.
The extra lines are taken from (or given to) the file list.



\subsection{File browser menu bar}

A menu of file selection commands appears above the list of files.
In this section we will go through each menu and item and
give (in parentheses) the Point name
for the command and a short description of the command.
A later section in this manual describes all the Point commands in detail.

This describes the file browser menu using the default Point
configuration file {\tt ptsetup.tcl}.
The menu bar contains only three items because the browser is
is a single column list of file names and is thin.

\begin{description}

\item[MENU] (pull down menu)
This menu provides access to most of the other menus.
They are all pull right menus and are described below.

\item[DIRS] (pull down menu)
This is the same as the DIRS pull right menu described below.

\item[New Browser] (Browser)
This command allows you to create new browsers.
Each browser will display the current directory.
If you click with the middle mouse button you get a browser right
next to the initial one.
If you click with the right mouse button you get a browser one browser
width away from the initial one.
If you click with the left mouse button you get a browser in the same
position as the initial one.


\item[PREFS] (pull right menu)
This pull right menu contains a number of cascaded submenus which are
noted by a {\tt =>} at the end of the item name.
When the mouse is moved to such an item the submenu appears beside
the item.
An item ending in ``{\tt ...}'' will bring up a dialogue box
and is used for options that require you to type in a string.
There are also check boxes for boolean options and radio buttons
for ``one-of-several'' options.
Most Point options can be changed (or examined) from this menu.

\item[SHELL/PTY] (pull right menu)
This pull right menu contains four commands that allow you to
run other Unix programs inside a Point window.
``{\tt Run csh in window}'' creates anew window and runs a csh in it.
``{\tt Run csh in file}'' runs a csh in the window containing the selection.
``{\tt Run and replace selection}'' and ``{\tt Run selection in file}''
both run the selected Unix command in the window containing the
selection.
While the program is running all input to the window (e.g., keystrokes)
is sent to the program running in the window and all output from
that program is inserted in the file in the window.

\item[New Window] (OpenFileOrCD)
This uses the current X selection as the name of a file or a directory.
If it is a file, Point will open a new window on that file.
If it is a directory then the file browser will change to that directory.
The X selection is the one made most recently and can be
from any program that can claim the X selection.
For example you can select a name in an {\tt xterm} window
and open the file (or change to the directory) with this command.
The list of files in the file browser claims the X selection for
a file name you select so if you click on a file name and
the click on this command that file will be opened.

\item[DIRS] (pull right menu) (CD directoryName)
This is a menu of directories you can jump to by selecting a menu item.
You will want to put the directories you go to frequently
on this list.
You can do that by editing the {\tt ptsetup.tcl} file.

\item[MISC] (pull right menu)
This is a menu of commands that are used  infrequently.
	\begin{description}

	\item[Change key bindings ...] (MakeKeyBindingsBox)
	Pops up a dialogue box that allows you to change the
	bindings of any key.
	First you specify the command to bind to the key.
	There is a scrolling list of common ones but you can type
	in any command into the text box.
	Arguments are shown for the commands but this is only for
	your reference; the arguments will not be copied into
	the text entry field when you click on the command.
	Only the command is given in the first text box so if you
	want to give the command some arguments.
	These must be fixed arguments because you will be invoking the
	command with a keystroke.
	Finally you give the key to bind to.
	The scrolling list has almost all of the keys you are likely to
	want to bind.
	Remember that you can find out the ``official'' X windows
	name of any key on your keyboard by running {\tt xev}, moving
	the mouse inside the xev window, pressing the key and noting
	the name in the information printed in the {\tt xterm} window you
	started {\tt xev} from.

	\item[Save Point Options ...] (MakeSaveOptionsBox)
	This allows you to write a file that contains the current
	setting of all Point options.
	This is a tcl file that can be read at startup to reestablish
	these options.

	\item[Load scratch file] (OpenWindow scratch \$location1 doNotAsk)
	This is a convenience function to pop up a window with a scratch
	file in it.

	\item[About Point ...] (MakeAboutBox)
	Pop up a dialogue box giving information about
	the version of Point you are using.

	\item[Cancel copy mode] (CancelModes)
	Cancel duplicate (also called copy) mode.

	\item[Print Statistics] (PrintStats)
	Print some statistics about the effectiveness of Point's caches.

	\item[Delete File] (exec rm [selection get])
	Delete the selected file.  A convenience function.

	\item[Insert ASCII ...] (MakeAsciiBox)
	Begin up a dialogue box that allows you to insert
	a character using its numerical ASCII code in decimal.

	\item[Save All Unsaved Files] (SaveAllFiles)
	Save all files that have been edited but not yet saved.

	\item[Set debug ...] (MakeDebugBox)
	Set an internal debugging variable.
	This usually has no effect in distributed versions.

	\item[Information] (print [Sel get])
	Display (in the {\tt xterm} window Point was started in)
	the character position
	of the first and last characters of the selection.

	\item[6x13] (BrowserFont 6x13)
	Change the font used in the file browser for displaying
	file names to {\tt 6x13}.

	\item[6x13bold] (BrowserFont 6x13bold)
	Change the font used in the file browser for displaying
	file names to {\tt 6x13bold}.

	\item[5x8] (BrowserFont 5x8)
	Change the font used in the file browser for displaying
	file names to {\tt 5x8}.

	\item[*clean-medium-r*6*50*] (BrowserFont *clean-medium-r*6*50*)
	Change the font used in the file browser for displaying
	file names to a font that is about the smallest font in
	X that might be considered legible by some people.
	This will allow you to see a LOT of file names.

	\item[8x13] (BrowserFont 8x13)
	Change the font used in the file browser for displaying
	file names to {\tt 8x13}.

	\end{description}

\item[ Refresh w/* ] (Option set filePattern *)
This causes the list of files to be regenerated.
It is used when you know the directory has changed.
Point does not monitor changes in the directory being
displayed so this allows you to be sure you are seeing
an accurate file name list.

\item[New Browser] (Browser)
Creates a new file browser.

\item[Del file] (exec rm [selection get])
Deletes the selected file.

\item[Close] (CloseBrowser)
Close this file browser but do not exit Point.

\item[QUIT] (pull right menu)
This command brings up a menu that allows you to choose one of
three ways of quitting Point.
You can automatically save all unsaved files,
discard the edits for all unsaved files
or have Point ask you individually whether to save
each unsaved file.

\end{description}



\subsection{File list}

The largest pane of the file browser window contains a list of files
and directories in the current directory.
The directories are listed first and are displayed with a ``{\tt /}''
appended to the directory name.
You can change directories by double clicking on the directory name
in the file list.
(You can also jump directly to some directories using the ``DIRS''
menu from the menu line.)
If you double click with the left mouse button the browser changes its
display to that directory.
If you double click with the middle mouse button a new browser is created
in the position next to the initial browser.
That new browser will display the directory you double clicked on.
If you double click with the right mouse button a new browser is created
in the position one browser width over from the position
of the initial browser.

If you select a file name from the list, it is highlighted,
and it the object of the next command from the menu bar.
It also becomes the X (primary) selection and can be pasted
into an {\tt xterm} window by clicking the middle mouse button.
If you double click on a file name a new window is created
that shows that file.
Double clicking on a file name is the most common way to open a new window.

If you double click on the file name with the left
mouse button the new window is placed in the first predefined
location (geometry of 502x460+0+0).
If you click with the middle mouse button you get the second
predefined location (geometry of 502x460-200+465) and the right mouse
button gets you the third predefined location (geometry of 502x460-0+0).
These are the same locations you get for existing windows by clicking
on their names in the open window list.\footnote{
Remember that these locations and the meaning of the various mouse
button clicks can be changed by editing the startup file ptsetup.tcl.}

The scroll bar in the file list (and the open window list) can
work in two distinct ways.
See \ref{sect:scrolling} for more information on this.



\subsection{Typical usage of the file browser window}

The main use of a file browser is to select file names of files to open.
Double clicking on a file name opens a window on that file.
The {\tt twm} window manager has a vertical zoom function
that is handy for quickly expanding and reducing
the file browser window.

Typically I open most windows in the NW position ---
my main position for windows.
I usually zoom vertically the NW window I am currently looking at and unzoom
it when I finish using the window or
when I need to use the SW quadrant.
Zooming and unzooming is easy with the {\tt twm} window manager
or with the Point {\tt Zoom} command.
I open a window in the NE quadrant when I want to use it
together with the window in the NW quadrant.
For example, if I am copying code from one window to another
or if I am looking up a definition of a procedure or global
data object.

A typical situation would be the case where I want to add a
timeout to some part of the code (say to do auto-repeating).
I get the window in which I want to put the timeout in the NW quadrant
and in the NE quadrant I put up a window containing a file where
I have already implemented a timeout facility.
Rather than look up how to use the facility in the manual,
I just copy the setup calls and the timeout callback procedures
and modify them for the current situation.
Then I might bring up the resource file in the NE quadrant
and modify it by adding resources for the new widgets I am adding.

If I need to look at three files at the same time or if I need
to quickly look at another file while already looking at two files,
I use the SE quadrant location.

The ``New Window'' command can be used if you have the name of the
desired file in a window somewhere, either in a Point text
window or in an {\tt xterm} window or any window that sets the
X primary selection.
{\it But remember:} relative path name must be relative to the directory
shown in the file browser window since that is Point's
working directory.

The DIRS menu allows you to move around the file tree
by jumps --- it contains the names of directories
you use a lot.\footnote{
Point caches directory listings so if you jump to a directory
that you have seen before (and it hasn't changed since then)
it should come up pretty fast.}
You can also move up and down the directory tree
by clicking on directory names (including {\tt ../}) in the file name listing.


\subsection{ Tk Window Hierarchy for a Browser Window}

It is sometimes useful to know the names of the parts of a browser
window.
In the following hierarchy the Tk window name is given first and
then the Tk widget type in parentheses.

.bwNNNNN (toplevel) --- The ``NNNNN'' is a window number.
	Window numbers are unique across both browser
	and text windows and start with ``00001''.
	\begin{itemize}
	\item .bwNNNNN.openList (frame) --- the list of open windows
		\begin{itemize}
		\item .bwNNNNN.openList.scroll (scrollbar)
		\item .bwNNNNN.openList.list (listbox)
		\end{itemize}
	\item .bwNNNNN.menu (frame) --- this contains some number of buttons:
		\begin{itemize}
		\item .bwNNNNN.menu.buttonName (button) ---
			``buttonName'' is given in the menu specification
		\item .bwNNNNN.menu.menuName (menubutton) ---
			``menuName'' is given in the menu specification
			\begin{itemize}
			\item .bwNNNNN.menu.menuName.m (menu)
			\end{itemize}
		\end{itemize}
	\item .bwNNNNN.fileList (frame) --- the list of files in
			the directory
		\begin{itemize}
		\item .bwNNNNN.fileList.scroll (scrollbar)
		\item .bwNNNNN.fileList.list (listbox)
		\end{itemize}
	\end{itemize}



\section{Text Windows} \label{sect:text}

Each open file is displayed in a text window.
A file can be displayed in several windows at the same time.
Point will keep the display in each window consistent
with the other windows showing the same file.
Text windows can be moved and resized with the window manager.

\subsection{The Active Window}

One of the text windows on the screen will be the {\em active window}.
This is the window that commands act on.
The last window the mouse sprite entered is the active window.
If you execute a command from the menu in a window then the commands
will always affect that window
(because the mouse cursor must be in that window in order to
access the menu).
The active window has a ``@'' as the first character of its
title bar.


\subsection{Text Window Menu Bar}

The menu bar contains both direct commands and pull-down menus.
Here I will list the commands on the menu bar and briefly
indicate their function.
In this section I will describes the menu bar of the default
Point configuration as specified in the file {\tt ptsetup.tcl}
distributed with Point.
All of these commands are described in more detail in a later section.
The Point command names are in parentheses.

\begin{description}

\item[FILE] A menu of file and window related commands.
	\begin{description}

	\item[Set text colors ...] (MakeColorBox)
	A dialogue box is opened which will allow you to
	change the foreground and background color of normal
	text and of selected text.

	\item[Set text font $=>$] (pull right menu)
	A submenu of fonts to change the text font of this window
	(only).

	\item[PREFS] (pull right menu)
	This pull right menu contains a number of cascaded submenus which are
	noted by a {\tt =>} at the end of the item name.
	When the mouse is moved to such an item the submenu appears beside
	the item.
	An item ending in ``{\tt ...}'' will bring up a dialogue box
	and is used for options that require you to type in a string.
	There are also check boxes for boolean options and radio buttons
	for ``one-of-several'' options.
	Most Point options can be changed (or examined) from this menu.

	\item[Zoom vertical] (Zoom)
	The window is resized to be the full height
	of the screen (and the same width as it was before).

	\item[Zoom full] (Zoom \{\} full)
	The window is resized to be the full size of the screen.

	\item[Move window $=>$] (pull right menu)
	A submenu is displayed allowing you to move
	the window to one of the four quadrants of
	the screen.
	\begin{description}

		\item[Move to NE] (MoveWindow 502x460+0+0)

		\item[Move to NW] (MoveWindow 502x460-200+465)

		\item[Move to SE] (MoveWindow 502x460-0+0)

		\item[Move to SW] (MoveWindow 502x460+0+485)

	\end{description}

	\item[Save file] (SaveFile)
	The file in the active window is saved to disk.

	\item[Save as ...] (SaveAs)
	A new name is requested (with a dialogue box)
	and the file is written to disk with that name.

	\item[Open selected file name] (OpenWindow [selection get])
	The X selection is used as the name of a file 
	and a window is opened to display that file.
	A new window is created even if there is an existing
	window open on that file.

	\item[Open New Browser ...] (Browser 135x445+510+0)
	A file browser is created with the specified geometry.

	\item[Close window $=>$] (pull right menu)
	The window is closed and the submenu selection determines
	whether the file in the window is automatically saved
	(if it has been changed), is not saved (even if it has
	been changed) or if the user is asked whether to save
	the file (if it has been changed).
	\begin{description}

		\item[and save] (CloseWindow save)

		\item[and ask] (CloseWindow ask)

		\item[and discard edits] (CloseWindow nosave)

	\end{description}

	\item[About Point ...] (MakeAboutBox)
	Pop up a dialogue box giving information about
	the version of Point you are using.

	\item[Quit Point $=>$] (pull right menu)
	Point is exited and the submenu selection determines
	whether the changed files are automatically saved
	(if they have been changed), not saved (even if they have
	been changed) or if the user is asked whether to save
	each file (if it has been changed).
	\begin{description}

		\item[And save all] (QuitPoint save)

		\item[And ask] (MakeQuitBox)

		\item[And discard edits] (QuitPoint discard)

	\end{description}

	\end{description}

\item[EDIT] A menu of editing related commands.
	\begin{description}

	\item[Insert X selection] (InsertSelectedString)
	The X (primary) selection is inserted at the
	insertion point.

	\item[Cancel copy mode] (CancelModes)
	Cancel copy (also called duplicate) mode.

	\item[Execute Selection as Tcl] (ExecSel)
		The selection is interpreted by Point's tcl
		interpreter.  This can be used to define
		(or redefine) a macro (if the selected text
		is a macro definition).

	\item[Define Selected Macro] (DefineMacro)
		The selection will be used as the name of the
		macro that the {\tt ExecMacro} will run.

	\item[Execute Macro] (ExecMacro)
		Execute the macro whose name was defined by the
		{\tt DefineMacro} command.

	\item[Search and Replace ...] (MakeReplaceBox)
	The command pops up a dialogue box that allows you to search
	for a string and replace it with another string.
	The search and replace operation can be interactive
	(asking you for verification of each replace) or batch.

	\item[Regex Search and Replace ...] (MakeRegexReplaceBox)
	The command pops up a dialogue box that allows you to search
	for a string that matches a regular expression
	and replace it with another string.
	The search and replace operation can be interactive
	(asking you for verification of each replace) or batch.

	\item[Indent selected lines] (IndentSelection)
		A macro that indents (by one tab) every line that
		is spanned by the selection.

	\item[Outdent selected lines] (IndentSelection 1)
		A macro that removes the first character of
		every line that is spanned by the selection.

	\item[Delete $=>$] (pull right menu)
	A submenu allows you to either:
	\begin{description}

		\item[Delete to end of line] (DeleteLine) deletes the
			characters in the line from the selection to
			the end of the line

		\item[Delete entire line] (DeleteLine 1) deletes the
			line the selection is on entirely 

		\item[Delete selection] (DeleteToScrap) delete the
			selection

	\end{description}

	\item[Scrap $=>$] (pull right menu)
	A submenu allows you to either:
	\begin{description}

		\item[Insert from] (InsertFromScrap) insert the contents
			of the scrap buffer at the insertion point

		\item[Delete sel to] (DeleteToScrap) delete the selection
			to the scrap buffer

		\item[Copy sel to] (CopySelToScrap) copy the selection to
			the scrap buffer

		\item[Exchange sel with] (ExchangeScrap) exchange the
			contents of the scrap buffer with the selection

	\end{description}

	\item[Copy $=>$] (pull right menu)
	A submenu allows you to either:
	\begin{description}

		\item[Note destination] (CopyToHereMode) remember the
			destination of the copy

		\item[Sel to destination] (CopyToHereMode) copy the
			selection to the remembered destination

		\item[Cancel copy mode] (CancelModes) cancel a pending copy

	\end{description}
	The mouse sprite changes to a pointing hand when
	you select a destination and changes back to an arrow
	when you complete the move.

	\item[Change Case $=>$] (pull right menu)
	A submenu allows you to either:
	\begin{description}

		\item[To upper] (ChangeCaseOfSel toupper) change the case
			of the selection to upper case

		\item[To lower] (ChangeCaseOfSel tolower) change the case
			of the selection to lower case

		\item[Toggle case] (ChangeCaseOfSel toggle) to switch the
			case of the selection

	\end{description}
	Only letters in the selection are affected.
	All other characters in the selection are unaffected.

	\item[Undo/Again/Redo $=>$] (pull right menu)
	A submenu of commands for undoing, redoing and repeating commands:
	\begin{description}
		\item[Repeat last edit] (Again mostrecent)
		The last edit is repeated but using the current selection.

		\item[Last edit (this file)] (Again thisfile)
		The last edit in this file is repeated
		but using the current selection.

		\item[Undo one edit] (Undo 1)
		A previous edit is undone.
		The first undo undoes the most recent edit and subsequent
		undoes undo earlier edits.

		\item[Redo one edit] (Redo 1)
		The most recently undone edit is redone.

		\item[Begin block undo] (Undo begin)
		Mark the beginning of a sequence of edits that you want to
		be all undone (and redone) together.

		\item[End block undo] (Undo end)
		Mark the end of a sequence of edits that you want to
		be all undone (and redone) together.

		\item[Show command history] (ShowUndoStack)
		Pop up a dialogue box which shows the command history for
		the file in this window.
	\end{description}

	\item[Redraw Window] (Redraw)
	Redraws the text in the window.

	\end{description}

\item[GOTO] A menu of commands that move the window around in the text.

	\begin{description}

	\item[Goto Selected Line \#] (GotoLine [selection get] lof)
		Go to the selected line number.  This command is useful
		for jumping to lines mentioned in compiler error
		messages.

	\item[Goto Line \# ...] (MakeGotoBox) A dialogue box pops up
		to allow you to type in a line number

	\item[Goto Selection] (ShowSelection) move to the last place
		you jumped from in this file

	\item[Find Selected CTag] (CTag [selection get])
		Find the selected C tag

	Point implements C tags (like vi and emacs).
	It uses the {\tt tags} file in the current directory
	(created with the Unix {\tt ctags} command).
	Point also will search certain files to determine which of them
	contain a keyword.
	The option {\tt keywordPattern} determines which files are searched.

	\item[Find CTag ...] (MakeCtagBox) A dialogue box pops up
		to allow you to type in a C tag

	\item[Find Selected Keyword] (GetSelectedKeyword)
		Find the selected keyword

	\item[Find Keyword ...] (MakeKeywordBox)
		A dialogue box pops up to allow you to
		type in a keyword

	\item[Find Matching bracket] (FindMatchingBracket)
		If the first character of the
		selection is a bracket then it will find the
		matching bracket (taking nesting into consideration).
		This works for: (, ), [, ], \{, and \}.
			
	\item[Search for string $=>$] (SearchForString)
	Forward is towards the end of the file and backward to towards
	the beginning of the file.
	A submenu allows you to either:

	\begin{description}

		\item[$>>$ for selected RE] (SearchForSel 1 forward)
			search forward for a string that matches the
			selected regular expression

		\item[$<<$ for selected RE] (SearchForSel 1 backward)
			search backward for a string that matches the
			selected regular expression

		\item[$>>$ for last RE] (RegexSearch \{\} forward)
			search forward for the string that
			matches the last regular expression used

		\item[$<<$ for last RE] (RegexSearch \{\} backward)
			search backward for the string that
			matches the last regular expression used

		\item[$>>$ for last string] (RepeatSearch forward)
			search forward for the last
			string that was searched for

		\item[$<<$ for last string] (RepeatSearch forward)
			search backward for the last
			string that was searched for

		\item[$>>$ for selection] (SearchForSel 0 forward)
			search forward for the next
			instance of the selected string

		\item[$<<$ for selection] (SearchForSel 0 backward)
			search backward for the next
			instance of the selected string

		\item[For string ...] (MakeSearchBox normal)
			a dialogue box pops
			up which allows you to enter a string to search for.
			The dialogue box also allows you to change two
			search options: whether case is considered in
			the search and whether the string can be part
			of another string or must be a whole word.
			The dialogue box also allows you to search
			forward or backwards.

	\end{description}

	\item[Move in file to $=>$] (pull right menu)
	A submenu allows you to either:

	\begin{description}

		\item[Beginning] move to the first line of the file
			(GotoLine 1 top)

		\item[End] move to the last line of the file (MoveToEndFile)

		\item[Last place] move to the last place you jumped
			from in this file (MoveToLastPlace)

	\end{description}

	In addition you can move the mouse sprite into the ``GOTO'' box and
	type a line number (ending it with any non-digit) and the
	window will go to that line number.
	This saves waiting for a dialogue box to pop up
	(as selecting ``Line \# ...'' would require) but you have no
	feedback as you type the digits.

	\end{description}

\item[$<<$] Searching backward.

	\begin{description}

	\item[Left button] (SearchForSel 0 backward)
		Search backward (towards the beginning of the file)
		for the next occurrence of the selected string.

	\item[Middle button] (MakeSearchBox normal) Bring up a dialogue box
		asking for a string to search forward for.

	\item[Right button] (RepeatSearch backward)
		Search backward (towards the beginning of the file)
		for the last string that was searched for.

	\end{description}

\item[$>>$]  Searching forward.

	\begin{description}

	\item[Left button] (SearchForSel 0 forward)
		Search forward (towards	the end of the file) for
		the next occurrence of the selected string.

	\item[Middle button] (MakeSearchBox regex) Bring up a dialogue box
		asking for a string to search forward for.

	\item[Right button]  (RepeatSearch forward)
		Search forward (towards the end of the file)
		for the last string that was searched for.

	\end{description}
	
In addition you can type an incremental search string into
either ``$<<$'' or ``$>>$''.
Move the mouse sprite inside either search button
(no button press is necessary),
type a non-ASCII character to clear the search string
(a function key is usually a handy choice),
then type in the letters of the string to search for.
The search is incremental in that after each character of the string
is typed, the string so far is searched for.
If you type a backspace the last character in the search string is
erased (although the search will {\em not} back up to where it was before
that character was typed).

\item[Close] Closing the window.

	\begin{description}

	\item[Left button] (CloseWindow save)
		Close the window and save the file (if ti has been changed).

	\item[Middle button] (CloseWindow ask) Close the window and ask about
		saving the file (if it has been changed).

	\item[Right button] (CloseWindow nosave)
		Close the window and do not save the file.

	\end{description}

\item[Sv] Saving the file in the window.

	\begin{description}

	\item[Left button] (SaveFile) Save the file in the window.

	\item[Middle button] (SaveAs) Save the file in the window
		with a new name.

	\item[Right button] (Redraw) Redraw the window.
		(This doesn't fit here but there was a free slot.)

	\end{description}

\item[Jump] moves the window around in the file it is showing.

	\begin{description}

	\item[Left button] (GotoLine 1 top) Jump to the beginning
		of the file.

	\item[Middle button] (MoveToLastPlace) Jump to the last place
		in the file that you jumped from (for any reason).

	\item[Right button] (MoveToEndFile) Jump to the end of the file.

	\end{description}

\item[Tag] finds tags and keywords in certain files.
	This command is described more fully in the sections that follow.

	\begin{description}

	\item[Left button] (CTag [selection get]) Find the selected tag.

	\item[Middle button] (MakeKeywordBox)
		Search files in the current directory for a keyword.
		The keyword is typed into the text box (clicking the
		middle mouse button copies in the X selection).
		The default is to search all files but they can be
		modified by changing the ``in files:'' text box.
		A dialogue box is put up with a scrolling list of all
		the files found to contain the keyword at least once.
		You can click on the file names in this scrolling
		list to open a window on that file or to top the window
		continuing that file if one is already open.
		when you click on the file name in the scrolling list
		an asterisk will be prepended to the name so you can
		keep track of which ones you have already looked at.

	\item[Right button] (GetSelectedKeyword)
		Search for a word in the files specified in the
		MakeKeywordBox dialogue box.

	\end{description}

\item[+Do-] Again and undo.

	\begin{description}

	\item[Left button] (Redo 1) Redoes the last edit that was
		undone with ``Undo'' and has not yet been redone.

	\item[Middle button] (Again mostrecent) Repeats the last edit using
		the current selection.

	\item[Right button] (Undo 1) Undoes the most recent edit.
		Repeated invocations will undo earlier and earlier
		edits.

	\end{description}

\item[Zz] Affects the case of all the letters in the selection.

	\begin{description}

	\item[Left button] (ChangeCaseOfSel toupper) Changes them to all
		upper case.

	\item[Middle button] (ChangeCaseOfSel toggle) Switches the case
		of each letter in the selection.

	\item[Right button] (ChangeCaseOfSel tolower) Changes them to all
		lower case.

	\end{description}

\item[MoveW] Moving windows to predefined locations.

	\begin{description}

	\item[Left button] (MoveWindow 502x460+0+0)
		Top the window and move it to the specified geometry.

	\item[Middle button] (MoveWindow 502x460-200+465)
		Top the window and move it to the specified geometry.

	\item[Right button] (MoveWindow 502x460-0+0)
		Top the window and move it to the specified geometry.

	\end{description}

\item[Line\#] jumps the window to a line number.

	\begin{description}

	\item[Left button] (GotoLine [selection get] lof)
		Jump to the line number indicated by the selection.
		Either the X selection or
		the Point selection, whichever was made more recently.

	\item[Middle button] (MakeGotoBox) Bring up a dialogue box
		to ask for a line number to go to.

	\item[Right button] (SetLineNumbers) Turns line numbers on and off
		in the display.  This is useful for finding line
		numbers in the file.  Do not try to edit the file with
		the line numbers on as it does not work well.

	\end{description}

	In addition you can move the mouse sprite into the ``Line\#'' box and
	type a line number
	(ending it with any non-digit---a function key is typically used
	although {\tt vi} users might want to use {\tt G})
	and the window will go to that line number.
	This saves waiting for a dialogue box to pop up but you have no
	feedback as you type the digits.

\item[HELP] (HelpMenu) A menu of help topics.
	When you select one of them, a text window is opened on a file
	of help text about that topic.

\end{description}



\subsection{Text Pane}

This shows a part of the file being edited.
Usually some text will be {\it selected}.
Selected text is shown in the selected text colors or is underlined.\footnote{
Underlining looks nicer and is less obtrusive
than reverse video on some monochrome displays.}
Many commands operate on the selected text.
The position just in front of the first selected
character is called the {\it insertion point}.

\subsubsection{Typing in text}

New text is entered by typing.
Text is inserted at the insertion point (in front of the selection).
The backspace or delete keys both erase the previous character.
Shift-backspace and shift-delete erase the previous word.
The backspace, delete, shift-backspace and shift-delete keys
work any time, not just during text insertion.\footnote{
	All of these key bindings can be changed.}

Typed text is entered in front of the insertion point even
if the mouse cursor is in a different Point window than
the selection.

Typed text replaces the selection if the selection is more than
one character and the option {\tt insertReplaces} is true.


\subsubsection{Selecting text}

The selection is the focus of almost all editing and many other commands
in Point.
The selection always contains at least one character.

There are two methods of selecting text.
You can select text by drawing through it:
\begin{enumerate}
\item move the mouse sprite to the beginning of the intended selection,
\item press the left mouse button,
\item move the mouse sprite to the end of the intended selection,
\item release the left mouse button.
\end{enumerate}

Or you can select text by clicking on the two end points of the selection,
that is, by extending the selection:
\begin{enumerate}
\item move the mouse sprite to the beginning of the intended selection,
\item click (press and release) the {\em left} mouse button,
\item move the mouse sprite to the end of the intended selection,
\item click the {\em right} mouse button.
\end{enumerate}

Normally text is selected by character (this is called
{\em character mode}) but
if you double click the left mouse button
you change to {\em word mode}, which
means that the selection will be extended (or contracted) by full words.
A word is either a single punctuation character or a contiguous
string of letters, digits and underscores ({\tt \_}).
If you triple click the mouse button you change to {\em line mode}
and the selection is extended (or contracted) in units of whole lines.
To use the draw through method of selection for word and line mode,
you hold down the mouse button on the last of the multiple
clicks (the second click for word mode and the third click for line mode)
and drag the mouse over the words or lines you wish to select.
If you select in word (line) mode then the selection will extend
by words (lines) also.

Triple clicking on a line not only selects the line but it also
causes the line number to be displayed in the message line.
This is an easy way to determine the line number of a line.\footnote{
Another way is to turn on line numbering.}

Multiple clicks must be made within a short time period to be counted
as multiple clicks (rather than several single clicks).



\subsubsection{Scrolling}				\label{sect:scrolling}

Point implements two distinct kinds of scrolling.
The first kind will be called ``line to top'' scrolling
which is the default.
It works as follows.

\paragraph{Line to top scrolling}

The scroll bar on the left indicates where you are in the file
and how much of the file is showing in the window.
Clicking the right mouse button on the scroll bar scrolls down
(towards the end of the file) by a certain number of lines.
This command scrolls the line the mouse sprite is at to the top
of the window.
Clicking the left mouse button scrolls up so that the top line of the
window moves to the line the mouse sprite is on.

Setting the {\tt button1ScrollsDown} option to {\tt False} will reverse
the meaning of the left and right buttons.

If you hold down either the left or right mouse buttons,
the scrolling will continue (in units determined by where
the mouse sprite is) as long as you hold down the button.

Clicking the middle mouse button on the scroll bar jumps the
window to that part of the file, that is, the scrollbar slider will
move to where the mouse sprite is.
If you press and hold down the middle mouse button you can
drag the text up and down by moving the mouse.
The display will follow your movements until
you release the middle mouse button.

The arrow buttons on the top and bottom of the scroll bar are not
used in ``line to top'' scrolling and the slider is for your information
only, it does not affect scrolling in any way.

\paragraph{Macintosh scrolling}

The second kind of scrolling will be called ``Macintosh'' scrolling
and it is the usual default for Tk toolkit scrollbars.
It uses only the left mouse button
(although the other buttons also work and do the same thing
as the left mouse button).

The arrow buttons on the top and bottom of the scrollbar scroll
the text by one line up or down when clicked on.
If you press and hold down the left mouse button on an arrow button
the scrolling will autorepeat at regular intervals after a short
initial delay.

If you click below the slider the text will scroll down
(towards the end of the file) by one screenfull.
Similarly a click above the scrollbar scrolls up one screenfull.
Both of these will autorepeat as well.

If you press the left mouse button on the slider and move the mouse,
the display and the slider will scroll with the mouse
until you release the mouse button.

The default is ``line to top'' scrolling.
If you want ``Macintosh'' scrolling you must set the Point option
{\tt tkScrolling} to ``true''.

Point does not wrap long lines.\footnote{
	I plan to add it as an option in the near future.}
Instead you can scroll horizontally using the scroll bar on the bottom
of the text window.


\subsubsection{Mouse menu commands}

The middle and right mouse buttons can be used to execute commands.
There are five commands associated with each of these two buttons:
no motion, north (up), east (right), south (down) and west (left).
To execute one of these five commands, you press the mouse button,
move the mouse in one of these four directions
(or do not move for the no-motion command).
A menu pops up after a short delay (0.6 seconds)
but you do not need to wait for the menu to pop up in order
to execute a command.
The menu comes up as a reminder if you hesitate.
If you set the {\tt mouseSpriteMenu} option to true,
then the mouse sprite is used to display the menu items.

If you want to cancel in the middle of a mouse menu command,
click the left mouse button.

The commands on the {\bf MIDDLE MOUSE BUTTON} are:
\begin{description}
\item[No motion---Dup] (CopyToHereMode) this command is executed in pairs.
	The first time you execute this command the insertion point is
	recorded by Point.
	Then you can select some text in this window or in another window.
	The second time this command is executed, the selected text
	is inserted at the insertion point recorded by the first time
	and the new insertion point is at the end of the newly copied text.
\item[North---Del] (DeleteToScrap) the selected text is deleted.
\item[South---Ins] (InsertFromScrap) the text in the scrap buffer (usually
	the last text deleted) is inserted at the insertion point.
\item[East---Copy] (CopySelToMouse) the selected text is inserted at the
	point where the mouse sprite was when you began the mouse menu command.
	The insertion is by characters, words or lines depending on how
	the text was selected.
\item[West---Move] (MoveSelToMouse) the selected text is inserted at the
	point where the mouse sprite was when you began the mouse menu
	command and deleted from where it was.
	The insertion is by characters, words or lines depending on how
	the text was selected.
\end{description}

The commands on the {\bf RIGHT MOUSE BUTTON} are:
\begin{description}
\item[No motion---Ext] (ExtendSelection)
	The selection is extended to this point.
\item[North---$<<$] (Search [selection get] backward)
	The selected text is searched for towards the beginning of the file.
\item[South---$>>$] (Search [selection get] forward)
	The selected text is searched for towards the end of the file.
\item[East---Undo] (Undo 1) The last edit is undone.
\item[West---Again] (Again mostrecent)
	The last edit is repeated using the current selection.
\end{description}


\subsection{ Tk Window Hierarchy for a Text Window}

It is sometimes useful to know the names of the parts of a browser
window.
In the following hierarchy the Tk window name is given first and
then the Tk widget type in parentheses.

.twNNNNN (toplevel) --- The ``NNNNN'' is a window number.
	Window numbers are unique across both browser
	and text windows and start with ``00001''.
	\begin{itemize}
	\item .twNNNNN.menu (frame) --- this contains any number of buttons:
		\begin{itemize}
		\item .twNNNNN.menu.buttonName (button) ---
			``buttonName'' is given in the menu specification
		\item .twNNNNN.menu.menuName (menubutton) ---
			``menuName'' is given in the menu specification
			\begin{itemize}
			\item .twNNNNN.menu.menuName.m (menu)
			\end{itemize}
		\end{itemize}
	\item .twNNNNN.msg (entry) --- the message line
	\item .twNNNNN.vScrollAndText (frame) 
		\begin{itemize}
		\item .twNNNNN.vScrollAndText.vScroll (scrollbar) ---
			the vertical scroll bar
		\item .twNNNNN.vScrollAndText.text (frame) --- the text
			is written in this frame
		\end{itemize}
	\item .twNNNNN.splitterAndHScroll (frame) 
		\begin{itemize}
		\item .twNNNNN.splitterAndHScroll.splitter (frame) --- just
			a spacer for now
		\item .twNNNNN.splitterAndHScroll.hScroll (scrollbar) ---
			the horizontal scroll bar
		\end{itemize}
	\end{itemize}







\section{Keystrokes} \label{sect:keys}

Keystrokes meanings are defined in the startup file {\tt ptsetup.tcl}.
Here is a table of the key bindings in the distributed {\tt ptsetup.tcl} file.

\begin{tabular}{llllll}
F1 & delete selection to scrap & F6 & redo last undone edit \\
F2 & insert scrap buffer & F7 & scroll up \\
F3 & repeat last search backwards & F8 & scroll down \\
F4 & repeat last search forwards & F9 & undo last edit \\
F5 & repeat last edit & F10 & open selected file name \\
Up & move cursor one line up & Down & move cursor one line down \\
Left & move cursor left & Right & move cursor right \\
Control-Left & move left one word & Control-Right & move right one word  \\
Meta-Left & move left one word & Meta-Right & move right one word  \\
Home & move to beginning of line & End & move to end of line  \\
Prior & scroll up & Next & scroll down  \\
Control-s & begin incremental search \\
\end{tabular}



\subsection{Hints for using function keys}

I have found that, in practice, I only use a few function keys
in normal editing.
Because the function keys are not labeled it is hard to remember
what commands they invoke.
For this reason it does not seem to be useful to put commands you
do not use quite often on function keys.
Labeling the function keys on your keyboard can help but I find
that I use very few function keys even though I define them all.

My editing style is to do certain commands exclusively from function keys.
The function keys I use regularly are: delete (F1), insert (F2),
repeat search forward (F4) and again (F5).
I use them together with mouse commands to get a higher command bandwidth.
For example, in making a number of identical changes I select the
text to change with the mouse and use F5 (again) to make the changes.
I sometimes use F4 (repeat last search) to select the text to change.
Then I can get into a rhythm and make the changes quickly.\footnote{
	This is similar to using `n' and `.' in vi to effect an
	interactive search and (selective) replace.}

I am sure you will develop a style of function key use that
fits your editing style.
The idea is to get the maximum command bandwidth and that is done by
executing some commands with keys and some with the mouse.
The right mix is a matter of personal preference.






\section{Useful Concepts} \label{sect:concepts}

In this section we will discuss a variety of things that are
useful in understanding how Point works.


\subsection{ Managing the Selection }

When you select something in Point it also becomes the X (primary)
selection.
This means that you can copy Point's selection into {\tt xterm} windows
by clicking the middle mouse button.
Point notes when it loses the X selection but it {\it does not}
deselect the Point selection.
It remains the Point selection and it remains highlighted.
But it does color the selection in a different color.
Normal text is displayed with the {\tt textForeground} and
{\tt textBackground} colors.
Selected text is displayed with the {\tt selectedTextForeground} and
{\tt selectedTextBackground} colors
if it is also the primary X selection.\footnote{
	X allows for several selections but almost all X software only
	uses the primary selection.  So when you see ``X selection''
	is almost always means the ``primary X selection''.}
Selected text is displayed with the {\tt deSelectedTextForeground} and
{\tt deSelectedTextBackground} colors if it is not the primary X selection.

There is a command to insert the X selection into the
text (at the insertion point in front of the selection, as always).
This could be the Point selection but is more likely to some text
you selected from another window.
If Point had deselected its selection you would not be able to see
where the X selection will be copied.

All the Point commands that use the selection as an argument
(search for selection, goto selected line number,
find selected ctag, find selected keyword, open selected file name,
delete selected file name and CD to selection)
use the X selection, not necessarily the Point selection
(although usually the Point selection will also be
the X selection).

The file browser file name list makes the file name you select
the X selection.



\subsection{ Typing in characters and replacing the selection }

Normally typed characters are inserted in front of the selection
(at the {\it insertion point}).
If the {\it overType} option is True then typed characters replace
characters of the text.

If the {\it insertReplaces} option is True and the selection contains
more than one character,
then the selection is deleted before the character is inserted.
This provides behavior that is similar to the way most Macintosh
text editors work.
Both replace styles are useful and the convenience of each
depends on what you are used to.
With {\it insertReplaces} False you can select the insertion point
quickly with double and triple clicks but you have to hit the
delete key (F1) before typing in replacement text.
With {\it insertReplaces} True you can replace text easily without
using the delete key but you have to be careful to set the
insertion point to be a single character.




\subsection{ Moving and Copying }

Moving and copying is very common in editing code.
Whenever I have to write new code I try to find some similar code,
copy it over and modify it rather than starting from scratch.
There are two kinds of copy and Point implements both kinds.
The first I will call ``copy-to-from'' and it is implemented
by the Point command {\tt CopyToHereMode}.
It is where you first specify
the destination of the copy and then the source.
This copy is in the ``Copy $=>$'' submenu of the Edit Menu.
It is also attached to the no-motion command of the middle
mouse button.

A typical use is to duplicate one or more lines.
First select the line(s).
Then execute the {\tt CopyToHereMode} command twice.
The easiest way to do this is to click the middle mouse button twice
(if you click quickly the mouse menu will not come up).
The first {\tt CopyToHereMode} sets the to location as the beginning of
the selected lines.
The second {\tt CopyToHereMode} copies the selection to that point.
Selecting the lines serves the dual purpose of setting the insertion
point (the copy destination) for the first {\tt CopyToHereMode}
and setting the selection (the text to be copied) for the
second {\tt CopyToHereMode}.

The other typical use of {\tt CopyToHereMode} is to copy in a word
or phrase while you are typing.
You are typing and you get to a point where you want to type a word that
is already visible on the screen.
Why type it again?
Stop typing, click the middle mouse button (the first {\tt CopyToHereMode}),
select the word (double click with the left mouse button), click
the middle mouse button again (the second {\tt CopyToHereMode})
and continue typing.
Note that you do note lose the insertion point when you use
{\tt CopyToHereMode}.

The {\tt CopyToHereMode} command is, as the name implies,
a moded command.
It is executed in pairs and after the first execution you are in
what is called ``copy mode'' or ``duplicate mode''.
This is indicated by a pointing hand cursor (``hand1'')
(there is an option to
set the copy mode cursor to any cursor font cursor you like,
for example you might want to use ``gumby'').
Being in copy mode does not restrict you in any way,
all Point commands work exactly the same and you can stay
in copy mode for hours if you want or quit Point while in
copy mode.
The only effect of copy mode is that a second {\tt CopyToHereMode}
will complete the copy and copy the selected text to the destination
remembered when you executed the first {\tt CopyToHereMode}.

There is a {\tt MoveToHereMode} which works exactly the same
except the text is moved rather than copied.
I rarely use this command and you will note it is not
on the menus.
You might find a use for it.

The other type of copy I will call ``copy-from-to'' and it is
implemented by the Point command {\tt CopySelToMouse}.
It is attached to the move-right motion of the middle button
mouse menu.
To copy text: select the text to copy,
move the mouse sprite over the destination,
press the middle mouse button, move right a bit and then release
to execute {\tt CopySelToMouse}.
I use this to copy lines from one place to another.
Note that if you select the text in line mode,
the {\tt CopySelToMouse} command will insert the text in front of
the line over which you execute the command.
This means you don't have to be that accurate and select the very
first character of the line.
These comments also apply to word mode, {\it mutatis mutandis}.

Note that both ``copy''s are on mouse menus.
This is because I copy things a lot and like to do it quickly
and without thinking about how to do it
or diverting my attention from the text.

There is also a {\tt MoveSelToMouse} which I use frequently
but less often than {\tt CopySelToMouse}.
It works just like {\tt CopySelToMouse} except the text is
moved rather than copied.
It is attached to the move-left motion of the middle mouse menu.
I use it to rearrange text and I almost always use it on whole lines.



\subsection{ Moving Windows Around and Screen Layout}

Your window manager allows you to move windows around.
I have move window attached to the left mouse button on the
title bar of my windows.\footnote{Actually it is a {\it twm}
function containing the commands \{f.move f.deltastop f.raiselower\}.}
But in practice I rarely move Point text windows around that way.
I have fixed on a preferred screen layout and windows go in certain
positions in that layout.
I use Point's move window commands to move windows between set
locations.
Mainly I use three locations: NE, NW and SE.
These correspond to left, right and middle button clicks on the
``MoveW'' command on the text window menu bar.
(They are also on the ``Move Window $=>$'' submenu of the FILE menu.)
I move windows around in a single jump using a mouse click.

You will want to develop your own screen layouts but when you do
I recommend using the move window commands to move windows
between fixed window locations.


\subsection{Backup Files}

Backup files are great when you need then but a nuisance when you don't
need them.
The solution I have found is to store backups in a subdirectory out
of the way.
The option {\tt backupNameFormat} determines how the path names of
the backup files will be determined.
This is discussed in section \ref{sect:options}.
This allows you to specify that backups go into a subdirectory.

The option {\tt backupDepth} determines the number of backups to keep.
I use six and have often been glad I did.
This does use up disk space but text files are typically not that
large so it is not a major problem.


\subsection{Undo, Redo and Again}

Each command is recorded in the command history.
There is a pointer to the most recently executed
(and not undone) command.
The undo command undoes that command and moves the
pointer back to the previous command.
Thus another undo will undo that previous command.
You can continue this way and undo every command you have
executed since the file was loaded.
You can undo edits even if you have saved the file since making
the edit.

There is a separate command history for each file
(not each window, each file).
This is most noticeable when undoing a move from one file to another.
The delete is in the command history of the ``from'' file and the
insert is in the command history of the ``to'' file.
They must be undone separately.

Undo undoes the effect of the last command:
deleted text is inserted again,
copied text is deleted,
moved text is moved back,
changed text is changed back.
If you select text, delete it and type in new text,
this whole action is undone as a unit.

There are two cases when a single undo will undo more than
one command.
The first is a delete followed by an insert at the same location.
This is taken to be a replace and is considered a single operation
and so it is undone (and redone) as a unit.
The second case is when a sequence of edits has been grouped
explicitly as a block command.
There are commands to mark the beginning and the end of a block
of commands.
For example, one considers a search and replace operation to be
a single command and so they are blocked.

If you undo a command and decide you really didn't want
to undo it you can use the redo command to redo (or un-undo) it.
In terms of the command history,
redo moves the pointer up to the next command and redoes that
command.
Thus undo moves backward in the command history list and
redo moves forward in the command history list.

You can undo and redo as much as you want but as soon as you
make an ordinary edit (not an undo or a redo),
the edits that were undone but not redone are discarded.
This is a bug (it is a lot of work to handle this case)
that will be corrected in a future version.

Again repeats the last command (that has not been undone)
in the current context.
That is, if the last edit was to delete the selection and
replace it with some text, then redo will replace
the current selection with that text.
This is the most common use of the again command.
Again is often used with search to get the effect of a
``replace with verification on each change'' command.

Again takes an optional argument that determines what
``last command'' actually means.
You can choose the last command overall, the last command
in the file in the selection window or the last command
in the file in a named window.
The last command overall is the default.

You can call up a dialogue box that lists the command
history and allows you to undo, redo and repeat commands.
This box can be popped up by with the menu command
{\tt EDIT/Undo/Again/Redo/Show command history}.




\subsection{Line numbers}

The default window titles indicates the range of line number shown
in the  window.
If you triple click on the line, a message is displayed
(in the message line of the browser usually, although
this can be controlled by the user with the {\it messageFlags} option)
showing the line number.
This is the fastest way to determine the line number of a line.

There is a command to turn line numbering on and off in a window.
A quick way to do this is by clicking the right mouse button on the
``Line\#'' button on the text window menu bar.

There are commands to jump to a specified line number or to
the selected line number.



\subsection{Using the scrap}

Deleted text goes into a buffer called the {\it scrap buffer}.
The insert command inserts the text in the scrap buffer.
There are commands to copy the selection into the scrap buffer
and to exchange the selection with the scrap buffer.
You can use these command to exchange two strings.
These commands are all on the ``Scrap $=>$'' submenu of the EDIT menu.



\subsection{Searching, C tags, and keywords}

There are many searching commands in Point because searching
is a common thing to do when looking at programs.

There are two general types of searches in Point.
In this section we discuss string search which searches for
the next instance of a specific string.
There are no special characters and the search string can contain
anything (including newlines) and it must be matched exactly.
In the next section we will discuss regular expression searching where 
you search for an instance of a string pattern that might match
a large number of strings.

You can search forward (or backward) for the next (or previous)
instance of the selected text.
This is the most common form of searching and so is on a mouse menu
(right mouse button move-down is search forward for selection
and right mouse button move-up is search backward for selection)
for quick access.
These same commands are attached to the left mouse button click
on the ``$<<$'' (for search backward) and ``$>>$'' (for search
forward) buttons on the menu bar.
The mouse menu is faster for a single search and the menu bar
button is faster and easier for repeated searches for the same string.

Sometimes you search for something and then after you find it you
change the selection.
For example you might make a small edit.
The {\tt RepeatSearch} command will search for the last string
you searched for without you having to select it again.
Indeed, the edit you made might mean that that exact string is
no longer there to select anyway.
The {\tt RepeatSearch} command is attached to the right mouse
button click on the ``$<<$'' (for search backward)
and ``$>>$'' (for search forward) buttons on the menu bar.
These commands are also attached to the F3 (search backward)
and F4 (search forward) keys.

I often use the F4 and F5 (repeat last edit) key together to do an
impromptu search and replace with verify on each replacement.

There is a search dialogue box you can bring up and type in any string
to search for.
Click the middle mouse button on the ``$<<$'' or ``$>>$'' menu
bar buttons to bring up the search dialogue box.
From the search dialogue box you can type in a search string
and you can change the {\tt ignoreCase} and {\tt findWholeWords}
searching options.

You can also type a search string directly into the ``$<<$'' or ``$>>$''
menu bar buttons by moving the cursor to ``$<<$'' or ``$>>$'',
pressing any function key (or any non-ASCII key) to
clear the string, and typing the characters to search for.
The search is ``incremental'' (like {\tt emacs} searches) in that it searches
for the string typed in so far after each character is typed.

You can also search for a matching bracket.
This command (and all search commands) is on the GOTO menu.

Point also implements Unix tags.
You specify the tag by selecting it or entering it into a dialogue box.
Point then finds the file the tag is in, opens a window on that file
(or tops an existing window on that file),
positions the window at the tag and selects the tag.

A different sort of search is the keyword search which is an internal
form of {\tt grep}.
You can activate a keyword search using the selection as the keyword
or you can bring up a dialogue box and type in the keyword
to search for.
The keyword search looks through all the {\tt *.c and *.h} files
in the current directory\footnote{The files to be searched are
changed via Point options.}
and brings up a dialogue box with a scrolling list of all the
files that contain the keyword at least once.
When you click on a file name on this scrolling list three things happen:
\begin{enumerate}
\item A window containing the file is displayed.
If the file is in an existing open window that window will be topped.
Otherwise a new window will be created.
\item That window will be scrolled so that the first instance
of the keyword is visible in the window and that instance will
be selected (in character mode).
\item The file name in the scrolling list will have an asterisk
prepended to it.
This is done so you can easily keep track of the files you have
already looked at.
\end{enumerate}


\subsection{Searching for regular expressions}

It is often useful to search for a string pattern rather than a
specific string.
For example, you might want to search for the next thing within
parentheses.
The regular expression for that would be:
\begin{verbatim}
     (.*)
\end{verbatim}
This pattern will not work correctly if there are two consecutive
sets of parentheses.
To be sure to only find the first set we would use the pattern:
\begin{verbatim}
     ([^)]*)
\end{verbatim}

A regular expression is represented in a special notation that
describes the pattern you wish to search for.
Often most of the characters in the pattern are literal characters that
you want to match exactly but some will be special codes that match
a class of characters or special features in the string.

First let us define what a regular expression is.
A {\it regular expression} is one of:

\begin{enumerate}

\item {\bf char} matches itself, unless it is a special
                character (metacharacter): . \ [ ] * + \verb+^+ \$

\item {\bf .} matches any character.

\item {\bf \verb+\+} matches the character following it, except
		when followed by a left or right round bracket,
		a digit 1 to 9 or a left or right angle bracket. 
		(see [7], [8] and [9])
		It is used as an escape character for all 
		other meta-characters, and itself. When used
		in a set ([4]), it is treated as an ordinary
		character.

\item {\bf [set]} matches one of the characters in the set.
                If the first character in the set is "\verb+^+",
                it matches a character NOT in the set, i.e. 
		complements the set. A shorthand A-E is 
		used to specify a set of characters A up to 
		E, inclusive. The special characters "]" and 
		"-" have no special meaning if they appear 
		as the first characters in the set.
                Some examples:
		\begin{itemize}
                \item {\bf [a-z]} any lower case alpha

                \item {\bf [\verb+^+]-]} any char except ] and -

                \item {\bf [\verb+^+A-Z]} any char except upper case alpha

                \item {\bf [a-zA-Z]} any alpha
		\end{itemize}

\item {\bf *} any regular expression form [1] to [4], followed by
                closure char (*) matches zero or more matches of
                that form.

\item {\bf +} same as [5], except it matches one or more.

\item a regular expression in the form [1] to [10], enclosed
                as \verb+\+(form\verb+\+) matches what form matches.
                The enclosure creates a set of tags, used for [8] and for
                pattern substitution. The tagged forms are numbered
		starting from 1.

\item {\bf \verb+\+D} a \verb+\+ followed by a digit 1 to 9 matches
		whatever a
                previously tagged regular expression ([7]) matched.

\item {\bf \verb+\+\verb+<+} a regular expression starting
		with a \verb+\+\verb+<+ construct and/or ending
		with a \verb+\+\verb+>+ construct, restricts the
		pattern matching to the beginning of a word, and/or
		the end of a word. A word is defined to be a character
		string beginning and/or ending with the characters
		A-Z a-z 0-9 and \_. It must also be preceded and/or
		followed by any character outside those mentioned.

\item a composite regular expression xy where x and y
                are in the form [1] to [10] matches the longest
                match of x followed by a match for y.

\item {\bf \verb+^+} a regular expression starting with a \verb+^+ character
		and/or ending with a \$ character, restricts the
                pattern matching to the beginning of the line,
                or the end of line.  Elsewhere in the
		pattern, \verb+^+ and \$ are treated as ordinary characters.

\end{enumerate}

Enter the regular expression as the search string and begin the search.
In the next section we will discuss how to use regular expression to do
search and replace operations.


\subsection{Regular expression search and replace}

This operation is done from a dialog box.
You specify a regular expression to search for and a replacement pattern.
The replacement pattern allows you to replace the matched string with
a string composes of literal text and pieces of the matched string.
For example to find two strings inside curly braces and interchange
them you would use the search pattern:
\begin{verbatim}
     {\([^}]*\)}\(.*\){\([^}]*\)}
\end{verbatim}
and use the replacement pattern:
\begin{verbatim}
     {\3}\2{\1}
\end{verbatim}

The search patterns are the same as used in regular expression search.
The replacement patterns describe how to construct the string that
will replace the string that was matched.
You can construct the replacement string using parts of the matched string.

The replacement string is constructed by reading the characters
of the replacement pattern one at a time and interpreting them
as follows:

\begin{enumerate}

\item {\bf char} put that character into the replacement string,
		unless it is a special character (metachar): \& or \verb+\+.

\item {\bf \&} put the entire matched string into the replacement string.

\item {\bf \verb+\+D} put the Dth matched substring into the replacement
	string.  A matched substring is a part of the search pattern
	enclosed in \verb+\+( and \verb+\+).
	D must be a digit from 1 to 9.

\end{enumerate}

After a string is matched it is replaced by the replacement string
constructed according the replacement pattern.
You can decide whether you want to verify each replacement by
setting a toggle in the dialog box.





\section{Hints For Using Point} \label{sect:hints}

In this section I will describe how a number of common
editing tasks are best performed in Point.
This is to give you an idea of some styles of editing
that the commands will support.
I will use the notations LMB, MMB and RMB for the left,
middle and right mouse buttons.

{\bf To make many copies of a line(s):}
\begin{enumerate}
\item Select the line(s) to copy (triple click and drag while holding down
the third click for multiple lines).
\item DeleteToScrap (F1) (to copy the lines into the scrap buffer).\footnote{
Copy to scrap would also work here but it is a less common command
and not as instinctive as delete for most people.}
\item InsertFromScrap (F2) to replace the deleted copy.
\item Select an insertion point and InsertFromScrap (F2).
\item Repeat 4 as many times as necessary.
\end{enumerate}

{\bf To make one copy of a some line(s) right next to the line(s):}
\begin{enumerate}
\item Select the lines.
\item Duplicate, duplicate (Click the MMB twice).
\end{enumerate}

{\bf To make one copy of a some line(s) somewhere else:}
\begin{enumerate}
\item Select at the location you want to copy the lines to.
\item Duplicate (Click the MMB).
\item Select the lines.
\item Duplicate (Click the MMB).
\end{enumerate}

{\bf To make one copy of a some line(s) somewhere else (another method):}
\begin{enumerate}
\item Select the lines.
\item Move the mouse to the location you want to copy the lines to.
\item CopySelToMouse (press MMB and move right)
\end{enumerate}

{\bf To see more a file:}
\begin{itemize}
\item Zoom the window, or
\item Change to a smaller font (File menu ``SetTextFont $=>$'' submenu).
\end{itemize}

{\bf Note:} I zoom any window that I am going to be using for more than
a minute or two.
And then I unzoom it when I am finished working with it.
I zoom and unzoom windows quite frequently.

{\bf To look at two windows at the same time:}
\begin{enumerate}
\item Move them to different quadrants.
\item Do this by clicking a mouse button on the ``MoveW'' menu bar item.
(LMB for the NW quadrant, MMB for the SE quadrant, RMB for the NE quadrant)
\end{enumerate}

{\bf To go the a line number that appears on the screen:}
\begin{enumerate}
\item Make the line number the X selection.
In an {\tt xterm} window this can usually be done by double clicking on it.
\item Click on the {\tt Line\#} button on the menu bar of the window
where the line is.
\end{enumerate}

{\bf To search for a string that appears on the screen:}
\begin{enumerate}
\item Make the string the X selection.
In an {\tt xterm} window this can usually be done by double clicking on it.
\item Click on the {\tt>>} button on the menu bar of the window
where you want to search.
\item Alternatively you can use the mouse menu command:
move into the window, press the right mouse button,
move down a bit (say an inch or more) and release the mouse button.
\end{enumerate}

{\bf To use point as your mail editor:}
\begin{enumerate}
\item Find out where to specify the editor command for your mail program
(or any other program that executes an editor as a subprocess).
\item Specify the command as {\tt pt -c -w}.
\end{enumerate}

{\bf To switch between two places in a file:}
\begin{enumerate}
\item Jump to the other place.
\item Use jump to last place to switch (click the MMB on ``Jump'').
\end{enumerate}
{\bf or}
\begin{enumerate}
\item Use forward and backward search to switch between two
instances of a name (such as the definition and a use of a procedure).
\end{enumerate}

{\bf To look at all instances of a name in a collection of files:}
\begin{enumerate}
\item Invoke search for selected keyword (RMB on the ``Tag'' menu bar item).
\item Fill in the form and search.
\end{enumerate}

{\bf To move a procedure header to the top of the window:}
\begin{enumerate}
\item Click with the LMB on the scroll bar beside the top
line of the procedure.
\end{enumerate}

{\bf To replace words with nearby words:}
\begin{enumerate}
\item Select the word to replace (double click with the LMB)
\item Delete (F1)
\item Duplicate (click the MMB) to mark the insertion point
\item Select the new word
\item Duplicate (click the MMB) to duplicate the selected word
to the remembered insertion point
\end{enumerate}

{\bf To insert a nearby word while typing:}
\begin{enumerate}
\item Type up to the word
\item Duplicate
\item Select the word
\item Duplicate
\item Continue typing
\end{enumerate}

{\bf To do a search and replace:}
\begin{enumerate}
\item Use the replace command.
\end{enumerate}

{\bf To do a search and replace (another method):}
\begin{enumerate}
\item Select the string to search for and search forward and then backward
(this Sets the remembered search string).
\item Search for the string until you find one you want to replace.
\item Replace the string.
\item Search for the next one with RepeatSearch (F4).
\item Optionally replace with Again (F5).
\item Repeat until done.
\end{enumerate}




\section{ Point Options} \label{sect:options}

There are a number of Point options that affect the way
things are done in Point.
In this section we will discuss all the Point options.
In the next section we will discuss the various ways
of setting these options.

In the following subsections the name of the option is
followed by a colon and its default value.


\subsection{Browser Appearance Related Options}

\begin{description}

\item[browserFont:fixed]
This sets the font used to display file names in a file browser.
This option is the default font used for new file browsers.
The font of an individual file browser can be set with the
{\tt browserFont} command (several of which are available
on the ``MISC'' menu on the browser menu bar).

\item[browserGeometry:135x445+510+0]
This is the default geometry for new file browsers.
You can change the geometry of an existing window
with the {\tt MoveWindow} command or with the window manager.

\item[browserIconFormat:"Dir:\%d"]
This option determines the contents of the window icon name that
Point gives the window.
The window manager typically uses this name as the default icon.
The {\tt browserIconFormat} is in the same form as the
{\tt browserTitleFormat} (see below).

\item[browserTitleFormat:\%a.CD:.\%d]
This option determines the contents of the window title that
Point gives the window.
The window manager typically displays this title in the title bar.
The {\tt browserTitleFormat} is in the form of a format string similar
in spirit to the one given to the C {\tt printf} function.
The title is generated by going through the {\tt browserTitleFormat} string
character by character.
All characters except '\%' and the character following it
are copied literally to the title.
A '\%' indicates that some specific string related to the window
is to be copied into the title.
The character after the '\%' determined which string.
The possible values are (and I am including the '\%' here for clarity):
\begin{itemize}
\item {\bf \%d} --- the full path name of the directory showing in the
	browser.  C shell like ``\verb+~+'' is done of this path name.
\item {\bf \%a/activeMsg/} --- ``activeMsg'' if the browser is the active
		browser and ``'' otherwise.
		The matching `/'s can be any character that does not
		appear in "activeMsg".
\end{itemize}

\item[filePattern:*]
This option sets the filter used in displaying file names
in a file browser.

\item[noBrowser:False]
If this option is true then no browser will be created when Point
starts up.

\item[showDirsFirst:True]
If this option is true, Point will list all directories first
(in alphabetical order) and then the ordinary files
(in alphabetical order).
If this option is false the directories and files will
be interspersed (in overall alphabetical order).

\item[showSizes:False]
If this option is true, the file browser will show file sizes
next to file names.

\end{description}



\subsection{Text Appearance Related Options}

\begin{description}

\item[eofChar:1]
This option determines which character is used to mark the end of
the file in a text window.
It is specified as the numerical equivalent of its ASCII code.
The default is a small diamond in most fonts.

\item[selectedTextBackground:black]
The option determines the background color for selected text
(if underlineSelection is 0).
This is used only when the Point selection is also the primary X selection.
The value must be a color name in the X color database.

\item[selectedTextForeground:white]
The option determines the foreground color for selected text
(if underlineSelection is 0).
This is used only when the Point selection is also the primary X selection.
The value must be a color name in the X color database.

\item[deSelectedTextBackground:black]
The option determines the background color for selected text
(if underlineSelection is 0).
This is used only when the Point selection is {\it not}
the primary X selection.
The value must be a color name in the X color database.

\item[deSelectedTextForeground:white]
The option determines the foreground color for selected text
(if underlineSelection is 0).
This is used only when the Point selection is {\it not}
the primary X selection.
The value must be a color name in the X color database.

\item[SetLineNumbers:Not an option]
Line numbering is determined by window not by a global option.
SetLineNumbers is really a command that changes the state
of line number display in a window.
See the SetLineNumbers for details.

\item[showPartialLines:False]
If only part of a line will fit at the bottom of the window
then the line is not drawn unless this option is true.

\item[textBackground:white]
This option determines the background color for text.
The value must be a color name in the X color database.

\item[textForeground:black]
The option determines the foreground color for text.
The value must be a color name in the X color database.

\item[textFont:fixed]
This option sets the default font used for newly created windows.
You can set the font of an existing window with the {\tt textFont}
command.
It can be any font name that can be found in the X font directories.

\end{description}




\subsection{Text Window Appearance Related Options}

\begin{description}

\item[autoZoom:False]
This option causes new windows to be opened in the zoomed state.

\item[busySpriteName:watch]
The shape of the mouse sprite in text windows to indicate
that Point is busy doing something.
It can be the name of any character in the X cursor font.

\item[copySpriteName:hand1]
The shape of the mouse sprite in text windows to indicate
that Point is in copy mode.
It can be the name of any character in the X cursor font.

\item[mouseMenuFont:fixed]
The font used in the popup mouse menus.
It can be any font name that can be found in the X font directories.

\item[messageFlags:2]
This option determines how error messages are displayed.
It can be any number from 0 to 15 and is the sum of:
1 if you want messages to appear in popup dialogue boxes,
2 if you want messages to appear in the message line of all text windows,
4 if you want messages to appear in the title bar of the active window,
8 if you want messages to appear in the {\tt xterm} window for which Point
was started.

\item[pathNames:False]
This option determines whether full path names are used in text window
title bars and the open file list (if pathNames is True)
or only the last component of the path name is used.

\item[spriteBackground:white]
The background color used for the mouse sprite in text windows.

\item[spriteForeground:black]
The foreground color used for the mouse sprite in text windows.

\item[spriteName:left\_ptr]
The shape of the mouse sprite in text windows.

\item[textGeometry:502x460+0+0]
The default geometry for new text windows
(if one is not specified in the new window command).
The geometry of existing text windows can be changed with the
window manager.

\item[textIconFormat:Edit:\%n]
This option determines the contents of the window icon name that
Point gives the window.
The window manager typically uses this name as the default icon.
The {\tt textIconFormat} is in the same form as the {\tt textTitleFormat}
(see below).

\item[textTitleFormat:\%a.@.\%n\%r. readOnly. [\%l-\%L]\%c. (modified).]
This option determines the contents of the window title that
Point gives the window.
The window manager typically displays this title in the title bar.
The {\t textTitleFormat} is in the form of a format string similar in spirit
to the one given to the C {\tt printf} function.
The title is generated by going through the {\tt textTitleFormat} string
character by character.
All characters except '\%' and the character following it
are copied literally to the title.
A '\%' indicates that some specific string related to the window
is to be copied into the title.
The character after the '\%' determined which string.
The possible values are (and I am including the '\%' here for clarity):
\begin{itemize}
\item {\bf \%a/activeMsg/} --- ``activeMsg'' if the window is the active
		window and ``'' otherwise.
		The matching `/'s can be any character that does not
		appear in "activeMsg".
\item {\bf \%c/modMsg/} --- ``modMsg'' if the file has changed and
		``'' otherwise.
		The matching `/'s can be any character that does not
		appear in ``modMsg''.
\item {\bf \%l} --- the line number of the top line in the window.
\item {\bf \%L} --- the line number of the bottom line in the window.
\item {\bf \%n} --- the file name (affected by pathNames option).
\item {\bf \%N} --- the full path name of the file.
\item {\bf \%o/overTypeMsg/} --- ``overTypeMsg'' if Point is in over type
		mode and ``'' otherwise.
		The matching `/'s can be any character that does not
		appear in ``overTypeMsg''.
\item {\bf \%p} --- the character number (position) of the first character
		in the window.
\item {\bf \%P} --- the character number (position) of the last character
		in the window.
\item {\bf \%r/readOnlyMsg/} --- ``readOnlyMsg'' if the file is read only
		and ``'' otherwise.
		The matching `/'s can be any character that does not
		appear in ``readOnlyMsg''.
\item {\bf \%s} --- the final component of the file name.
\item {\bf \%S} --- the size of the file (in characters).
\item {\bf \%v} --- the column number of the leftmost column in the window.
\item {\bf \%V} --- the column number of the rightmost column in the window.
\end{itemize}

\item[underlineSelection:0]
This option determines how the selection is shown in text windows.
A value of 0 causes Point to use the colors specified in the
options {\tt [selected]text\{foreground$\mid$background\}}.
A value of 1 causes selections to be underlined with a one pixel wide line.
A value of 2 causes selections to be underlined with a two pixel wide line.

\end{description}


\subsection{File and Backup Related Options}

\begin{description}

\item[backupByCopy:False]
If this option is true, then backups will be made by copying the
original file contents into a backup file and then copying the
new contents into the original file.
This way the original file contains the newest version and
hard links are not lost.
If this option is false, then the original file will be
renamed to the backup file and the new contents written to
a new file with the same name as the original file.
This saves one copy but loses hard links.

\item[backupDepth:1]
This option determines how many levels of backup point maintains.
If backupDepth is set to 0 then no backups are made.
Otherwise it must be a value from 1 to 9.

\item[backupNameFormat:\%n.\%v]
This option controls how the path name of backup files is generated.
It is like a {\tt printf} format string.
The backup file path name is generated by reading this format string
one character at a time.
All characters except ``\%'' are copied to the path name unchanged.
Each ``\%'' is followed by a format character that indicates
what string value should replace the \% and the format character.
The following format characters are recognized:
	\begin{description}
	\item[N] Insert the name of the file (not the path name but
		just the last component of the path name).
	\item[n] Insert the full path name of the file.
	\item[D] Insert the full path name of the directory the file
		is in.
	\item[b] Insert the base name of the file, that is, the name
		without the extension.
		The base name is the the name up but not including
		the last ``.'' in the name.
	\item[v] The version number.
		The most recent backup will be version 1, the next
		most recent version 2, etc.
		Only 9 versions are supported so this will always
		be a single digit (1 to 9).
	\item[other] Anything else will be passed through (without the \%).
	\end{description}
These formats allows relative or absolute path names.
For example you can have PC style backups with:
\begin{verbatim}
	set backupNameFormat %D/%b.bak
\end{verbatim}
You can have simple {\tt emacs} style backups with:
\begin{verbatim}
	set backupNameFormat %n~
\end{verbatim}
You can have complex {\tt emacs} style backups with:
\begin{verbatim}
	set backupNameFormat %n~%v~
\end{verbatim}
You can have a general backups directory "backups" in your login directory:
\begin{verbatim}
	Option set backupNameFormat "~/backups/%N.%v"
\end{verbatim}
You can have backups go into a backups directory in the
same directory as the filed being backed up:
\begin{verbatim}
	Option set backupNameFormat "%D/backups/%N.%v"
\end{verbatim}
You can put backups in the same directory
as the file being backed up:
\begin{verbatim}
	Option set backupNameFormat "%n.%v"
\end{verbatim}
I use:
\begin{verbatim}
	set backupNameFormat %D/bak/%N.%v
\end{verbatim}
It will always be true that
\begin{verbatim}
%D/%N=%n.
\end{verbatim}

\item[maxFiles:200]
The maximum number of files that can be edited at one time.
This limit will be removed soon.

\item[nBuffers:100]
This option sets the number of internal buffers Point uses for text.
Each buffer is 1024 bytes long.
This value must be 25 or greater.
A small number of buffers means that Point does more I/O.
If you have a large buffer cache in the operating system
this will not slow you down very much and you will save
redundant buffering.

\item[readOnly:False]
This option determines whether new windows will be marked as read-only.
If so, no editing will be allowed.
If the UNIX file permissions do not allow writing the file,
the window will be marked read-only in any case.

\end{description}



\subsection{Interaction Style Options}

\begin{description}

\item[autoIndent:True]
This option determines whether autoindenting occurs when you
start a new line (with the Return key).
If set to True, it causes each new line to be spaced over by exactly the
same spaces and tabs as the previous line.
Point will not use tabs if the previous line used spaces.

\item[button1ScrollsDown:False]
If this option is True then the scrollbars work such that
button 1 (the left mouse button) scrolls
down (towards the end of the file) and button 3 (the right mouse
button) scrolls up (towards the beginning of the file).
Setting this option to False reverses the meaning of button 1
and button 3.

\item[insertReplaces:False]
If this option is True and the selection contains more than one
character, then a typed character will delete the selection
before being inserted.
If this option is false the selection is only used to determine
the insertion point and is not replaced by the insertion.

\item[keepSelectionVisible:False]
If this option is false then it is possible for the window on
the file containing the selection to be scrolled so that the
selection is not visible.
If this option is true then scrolling the selection window
will cause the selection to be moved so that it always remains
visible.

\item[menuDelay:600]
This option determines the number of milliseconds of delay
after a mouse menu button is pressed before the mouse menu
comes up.

\item[menuTolerance:10]
This option sets the number of pixels you have to move (after pressing
the mouse button and bringing up a mouse menu) before you leave the
``no motion'' menu item.

\item[mouseSpriteMenu:False]
This option determines whether mouse menus are displayed using
the mouse cursor (if mouseSpriteMenu is True) or with a circular
menu drawn over the text (if mouseSpriteMenu is False).
There is a delay (determined by the {\tt menuDelay} option)
before the menu comes up
so that fast mouse menu commands will not bring up the menu.

\item[overType:False]
This option determines whether typed characters are inserted
at the insertion point or whether they replace the first character
of the selection.
Setting this to True allows you to type over text instead of inserting
in front of it.

\item[returnString:""]
This option is used to coordinate modal dialogue boxes.
When Point puts up a dialogue box where it needs to wait for 
an answer, it sets {\tt returnString} to the empty string and
then processes events while waiting for it to become non-empty.
The {\tt WaitForReturnString} command is used to wait until the
{\tt returnString} option is set to a non-empty value.

\item[rightMargin:999]
This option determines when typed lines will be automatically broken.
The intention of the default of 999 is to not break lines but allow
them to be long.
This option also affects the margins used by the {\tt JustifySel} command.
In fact, that command will only be useful if this option is set to the
line length you want to justify to.

\item[tabWidth:8]
This option sets the location of the tab stops in the text.
Tab stops are every {\tt tabWidth} characters.
Tabs must be equally spaced.

\item[tkScrolling:False]
Determines the scrolling style in text window.
If this option is true the ``Macintosh'' scrolling style is used.
If this option is false the ``line to top'' scrolling style is used.

\item[undoMotion:False]
Point records motion in the file (scrolling and jumping) in its
undo history but normally it is ignored.
If this option is True then undo will undo motion as well
as edits.

\end{description}



\subsection{Search Options}

\begin{description}

\item[findWholeWords:False]
This option determines whether searched for strings can be surrounded
by alphanumeric character.
Setting {\tt findWholeWords} to True is useful when looking for variable
names like {\it i} which occur frequently within other names.

\item[ignoreCase:True]
This option determines whether case is considered in searches.

\item[keywordPattern:*]
This option specifies the files that are scanned for keywords in the
keyword search commands.

\item[linesOverFind:999]
When a string is searched for and the window must by jumped in
order to show the string just found,
Point needs to know how to position the window
in relation to the line the found string is on.
This option determines how many lines down from the top of the
window a found string will be placed.
A value of 0 means that it will be placed on the top line.
If the value is greater than the number of lines in the window,
the found string will be placed in the middle of the window
(this is the default case).
This option also affects text positioning in the window after a
{\tt GotoLine} command if the {\tt lof} argument is specified.

\item[wrapAroundSearches:False]
If this option is set to True then forward searches will
wrap around the end of the file, that is,
a search started in the middle of a file will search
to the end of the file then continue searching at the
beginning of the file until it either finds the word
or has searched the entire file.

\end{description}



\subsection{Mouse Menu Options}

\begin{description}

\item[lmm1: Ext]
The text in the no-motion entry of mouse menu 1.

\item[cmm1:ExtendSelection]
The command executed by the no-motion entry of mouse menu 1.

\item[lmm1n: $<<$]
The text in the move-north entry of mouse menu 1.

\item[cmm1n:Search backward [selection get]]
The command executed by the move-north entry of mouse menu 1.

\item[lmm1e:Undo]
The text in the move-east entry of mouse menu 1.

\item[cmm1e:Undo]
The command executed by the move-east entry of mouse menu 1.

\item[lmm1s: $>>$]
The text in the move-south entry of mouse menu 1.

\item[cmm1s:Search forward [selection get]]
The command executed by the move-south entry of mouse menu 1.

\item[lmm1w:Again]
The text in the move-west entry of mouse menu 1.

\item[cmm1w:Again]
The command executed by the move-west entry of mouse menu 1.

\item[lmm2: Dup]
The text in the no-motion entry of mouse menu 2.

\item[cmm2:CopyToHereMode]
The command executed by the no-motion entry of mouse menu 2.

\item[lmm2n:Del ]
The text in the move-north of mouse menu 2.

\item[cmm2n:Delete]
The command executed by the move-north of mouse menu 2.

\item[lmm2e:Copy]
The text in the move-east entry of mouse menu 2.

\item[cmm2e:CopySelToMouse]
The command executed by the move-east entry of mouse menu 2.

\item[lmm2s:Ins ]
The text in the move-south entry of mouse menu 2.

\item[cmm2s:Insert]
The command executed by the move-south entry of mouse menu 2.

\item[lmm2w:Move]
The text in the move-west entry of mouse menu 2.

\item[cmm2w:MoveSelToMouse]
The command executed by the move-west entry of mouse menu 2.

\end{description}






\section{Customizing Point with Point Options} \label{sect:customize1}

The simplest way to customize Point is to change the values
of the Point options.
There are several times when Point options are set.
We will look at these times in order.
Note that the latest setting is the one the will apply so options
that are set and later reset will lose all setting except the
last one.

\begin{enumerate}

\item {\it Point source code} ---
defaults for all the options are compiled into the Point code.
The values of these defaults are described in the previous section.
You can change these defaults only by changing the Point source
code and recompiling.
There are easier ways to set the options.

\item {\it X resource database} ---
as Point is initializing it looks for each Point option in the X
resource database and sets the option to the resource database
value if it finds it.
It looks for exact matches between the X resource database resource
name and the Point option name with three exceptions.
Point looks for a resource named {\tt font} and if found,
uses it to set the options {\tt browserFont} and {\tt textFont}.
Point looks for a resource named {\tt foreground} and if found,
uses it to set the options {\tt textForeground}
and {\tt selectedTextBackground}.
Point looks for a resource named {\tt background} and if found,
uses it to set the options {\tt textBackground}
and {\tt selectedTextForeground}.
These three exceptions are done before exact matches in the
resource database are searched for so an exact match will
override one of these exceptions.

\item {\it tclLib/options.tcl} ---
{\tt ptsetup.tcl} reads in the file {\tt tclLib/options.tcl}
which sets almost all the options using tcl commands which set
Point options.  Here is an example:
\begin{verbatim}
     Option set textForeground blue
\end{verbatim}
The {\tt tclLib/options.tcl} file can be created by Point while
it is running with the command ``MENU/MISC/Save Point Options ...''.
This is used to save the current option setting and preserve them
for later Point sessions.
{\bf Important point:}
Note that this file, as written by Point, sets almost every option
and so will override any options set in the code or in the X
resource database.
You should delete option setting lines from {\tt tclLib/options.tcl}
if you would rather set these options from the X resource database.

\item {\it ptsetup.tcl} ---
Specific option settings can go in this file (after the call
to read in {\tt tclLib/options.tcl}).

\item {\it PREFS Menu} ---
Most Point options can be set while Point is running through the
PREFS menu (available from both the text window and the browser
menu bars).
These setting only affect the current Point session but you can
save the options in {\tt tclLib/options.tcl} and have them
read in each time Point is started.

\end{enumerate}



\subsection{Options you may want to change}

There are many Point options and in this section I want to list the ones
that you are most likely to want to change.

\begin{itemize}

\item {\bf tkScrolling}
I prefer line-to-top style scrolling myself since I use it to exactly
position code where I can look at as much as possible.

\item {\bf button1ScrollsDown}
I learned this style of scrolling using line-to-top scrolling at Xerox.
It depends on whether your mental model has the mouse rotated to 3
o'clock or to 9 o'clock.

\item {\bf backupNameFormat} and {\bf backupDepth}
I like to put the backups in a subdirectory where they don't clutter
up my directory but they are there when I need them.
Since they are out of the way I keep six backups.
This seems to be enough for almost all screw-ups.
My format is ``bak/\%n.\%v''.

\item Text colors:
These are the options {\bf textForeground}, {\bf textBackground},
{\bf selectedTextForeground} and {\bf selectedTextBackground}.
What can I say?  Color is fun.

\item Title and icon formats:
These are the options {\bf browserIconFormat}, {\bf browserTitleFormat},
{\bf textIconFormat} and {\bf textTitleFormat}.

\item {\bf textFont}
I use {\tt 6x13} because I like to see as much as possible.
If you use a larger font you will want to have some smaller fonts
on the menus so that you can temporarily change to a smaller font
to see more stuff.

\item {\bf findWholeWords}
This is handy when you are looking for instance of the identifier
{\it i} and in other similar cases.

\item {\bf ignoreCase}
Normally I have this true since I hate to type shifted letters,
but once in a while you want case to matter to separate the wheat
from the chaff.

\item {\bf linesOverFind}
Ordinarily you do not change this interactively but people have
different ideas about where found string should be.
I use the middle of the window and so set this to 999.

\item {\bf lmm1n etc.}
The mouse menu commands are most useful when they are customized
to the commands you use frequently.

\end{itemize}



\subsection{Setting local options by directory with .ptdirrc}

Each time Point changes to a new directory it looks for a
file named {\tt .ptdirrc} in the new directory.
If there is one, it reads it in as a setup file.
This allows you to set options by directory.

For example you might want to change the option {\tt textForeground}.
Then you could tell by the color of the text which directory
it had come from.

Or you might want to have different {\tt backupDepth}s and
different {\tt backupNameFormat}s in different directories.
You might even want to have a message window pop up when
you change to a directory, an active reminder of something.






\section{Customizing Point Using Tcl} \label{sect:customize2}

Point is implemented using the Tcl command language and the Tk X toolkit.
This means that all user actions cause Tcl code to be interpreted.
All the Point commands are added to the Tcl interpreter hence all the
functionality of Point is available from Tcl
as well as  all of the Tk toolkit functionality.
Together these make a highly configurable text editor.

All the functionality described in this manual is achieved though
the Tcl code in {\tt ptsetup.tcl}.
You might want to look through that file (using Point of course!)
to get an idea of what is there.

All customization of Point is done by editing the {\tt ptsetup.tcl} file.
In this section we will describe what kinds of changes to make
to perform various levels of customization in Point.


\subsection{Setting Point Options}

You will notice that near the beginning of {\tt ptsetup.tcl}
there are a number of lines that set Point options that look something like:
\begin{verbatim}
     Option set selectedTextBackground lawngreen
\end{verbatim}
Any Point option can be set with a similar line.
The keywords {\tt Option set} determines the command and is
always the same.
The third string is the name of the Point option.
These are listed in a section \ref{sect:options}.
The fourth string is the new value of the option.
That's all there is to it.

Another easy thing to do is to change the default geometries
used in various commands.
Above the {\tt Option set} lines you see three lines:
\begin{verbatim}
     set location1 "502x410+0+0"
     set location2 "502x390+530+415"
     set location3 "502x390+530+0"
\end{verbatim}
By changing these geometries you can change where the command
places windows.

Unfortunately not all geometries are specified here
(I'm working on changing that).
So you might search through {\tt ptsetup.tcl} for other
geometry specifications that you can change.



\subsection{ Rebinding Keys }

There is a command to rebind any key to any command.
This is done from a dialogue box that you can bring up
from the menu choice on the browser menu bar
``MENU/MISC/Change key bindings ...''.
You must specify three things for each binding:

\begin{enumerate}
\item {\it The command to execute:} This is just the command name.
It must be typed into the first text entry box.
There is a scrolling list of command names.
When you click on an item in this list the command name is
inserted into the text entry box.
Some of the scrolling list entries have command names and command
arguments but only the command name will be copied into the
text entry box.
The argument specifications are for your reference only.
\item {\it Fixed arguments:}  Most commands require arguments and
so the arguments to the command (if any) are specified in the
second text entry box.
These arguments must be fixed strings.
If you want different arguments, you must rebind different keys.
\item {\it The key to bind it to:}  This specifies the key that
will execute the specified command.
There is a scrolling list of most of the keys you might want
to rebind.
Clicking on a key on this list enters it into the third
text entry box.
\end{enumerate}

Once you specify these three things you click on the ``Remap Key''
button and the key binding will take effect for all existing
windows.
Windows created after you rebind the key will not have this new
key binding.



\subsection{ Changing the Menus }

The next level of customization involves changing the menus.
I have set things up so the menus are specified using Tcl
lists which are structured a bit like Lisp lists except brackets
are used instead of parentheses.
Thus you can change the menus without having to learn too much about Tcl
and the Tk toolkit commands.

{\bf A word of warning:} Tcl lists are fragile in the sense that a missing
bracket makes the list invalid.
The way Point is currently set up, if the Tcl menu list is invalid
then the menu doesn't get made and lots of bad things happen.
Thus you should always change a copy of {\tt ptsetup.tcl}.
I use {\tt egrep -n '\{|\}' ptsetup.tcl | more} to find sticky
problems with mismatched brackets.

I will try to make Point more robust in this respect in the future
but a possible workaround is to read your old version of
{\tt ptsetup.tcl} first and then the new one.
This way, if the new one fails on a tcl syntax error, everything
will already be defined anyway and so Point will start and work
correctly (although without the changes you made).

Let us look at the structure of menu specifications.
The browser menu bar is created using the Tcl variable {\tt BrowserMenuSpec}
and it is set as follows.
For this explanation I am using a reduced version of the 
distributed menu bar specification.
The full version can be found in {\tt ptsetup.tcl}.
It has more items on it but they are similar to these items.

\begin{verbatim}
set BrowserMenuSpec {
     {button New "  New  " {
          {OpenFileOrCD 502x410+0+0}
          {OpenFileOrCD 502x390+530+415}
          {OpenFileOrCD 502x390+530+0}
     }}
     {menu   DIRS "DIRS"   DIRS}
     {button DelFile "Del File" {
          {exec rm [selection get];Option set filePattern "*"}
          Bell
          Bell
     }}
     {button Close "  Close  " {
          CloseBrowser
          Bell
          Bell
     }}
     {menu   QUIT "QUIT"   QuitMenu}
}
\end{verbatim}

A {\it menu bar specification} is a list of items each of which is
also a list.
The specification above has five sublists
(button, menu, button, button, menu).
Each sublist will create one button on the menu bar.
The only two types allowed are ``button'' and ''menu''.

A ``button'' sublist starts with the keyword ``button''.
The second item is the name of the button.
This is used to bind other events to that button.
We will ignore it for now.
The third string is the text that will appear on the button.
You can put spaces here to space out the buttons.
The fourth item on the list is a sublist of three Tcl commands.
These are the commands that will be executed when the user clicks
the left, middle and right mouse buttons on this menu bar button.
The one for ``Close'' is the simplest.
Note that newlines can be used within bracketed lists for readability.
The left mouse button action is the Point command ``CloseBrowser''.
In general any Point command can be put here.
The Point commands are all listed in section \ref{sect:commands}.
Note that we use default (ring the bell) actions for the middle
and right mouse buttons if we do not want to use them.
The command must be there and the list must have exactly three commands.
You can use ``Bell'' and ``DoNothing'' as dummy commands.

The ``DelFile'' button is a little more complicated.
Its left mouse button command consists of two Tcl commands.
Note that if the command is more than a single string it
must be a sublist and enclosed in brackets.
If several Tcl/Point commands are to be executed then they
must be separated by semicolons or newlines.
The phrase {\tt [selection get]} is a Tcl expression that inserts
the current X selection as that argument.
The {\tt exec} command is a Tcl command to execute a Unix command,
in this case the {\tt rm} command.
Thus the first command will delete the file that is named in the
X selection.
The second command is a Point command to set the {\tt filePattern} option.
This will have the effect of forcing Point to reread the directory and
rewrite the list of files.
The ``New'' button has a different command for each mouse button.
Each command has an argument so it must be enclosed in brackets
so that the command is a single item on the list.
Each command is the Tcl procedure {\tt OpenFileOrCD} which is given below:
\begin{verbatim}
proc OpenFileOrCD {geometry} {
     set name [selection get]
     if [file isdirectory $name] \
          "CD $name" \
          "OpenWindow $name $geometry"
}
\end{verbatim}
It gets the X selection and tests if it is a file or a directory.
If it is a directory the browser changes to that directory and if it is a file
it opens a window on that file.

A ``menu'' sublist starts with the keyword ``menu''.
The second string is the name of the button.
This is used to bind other events to that button.
The third string is the text that will appear on the button.
The fourth string is the {\it name} of the string where
the submenu is specified.
Note that the submenu itself {\it cannot} appear here.
The submenu must be specified as another Tcl variable and
the name of that Tcl variable is put here.

Let's look at the specification of {\tt QuitMenu}:
\begin{verbatim}
set QuitMenu {
     {command "And save all" {QuitPoint save}}
     {command "And ask" MakeQuitBox}
     {command "And discard edits" {QuitPoint discard}}
}
\end{verbatim}

Menu specifications are similar to, but a bit different from,
menu bar specifications.
They also consist of a series of sublists,
each of which specifies one item on the menu.
There are five types of sublists,
but only the ``command'' type is used in {\tt QuitMenu}.
The ``command'' specification is similar to the ``button'' specification.
It starts with the keyword ``command'' and then the text of the
menu item.
(No name is required as it is with a ``button''.)
The third item is a single command to execute when the button is selected.
(You cannot give a command for each mouse buttons --- you should
use a submenu instead.)
{\tt QuitPoint} is a Point command and {\tt MakeQuitBox} is a Tcl procedure.
You might look at {\tt MakeQuitBox} in {\tt ptsetup.tcl}
but to understand it requires familiarity with the Tk toolkit.

You can also look up the {\tt DIRS} menu specification string in
{\tt ptsetup.tcl}.
It is not difficult to understand.

The Tcl variable {\tt TextMenuSpec} specifies
the text window menu bar.
It works just like the browser menu bar specification.

As a larger example of submenu specifications
let us look at the {\tt EditMenu} specification
used in {\tt TextMenuSpec}.
Below is an abbreviated version of it:
\begin{verbatim}
set EditMenu {
     {command "Redraw window" {Redraw}}
     {cascade "Copy =>" {
          {command "Note destination" {CopyToHereMode}}
          {command "Sel to destination" {CopyToHereMode}}
          {command "Cancel copy mode" {CancelModes}}
     }}
     {separator}
     {command "Repeat last edit" {Again mostrecent}}
}
\end{verbatim}
This shows two more menu item types: ``separator'' and ``cascade''.
The ``separator'' one is obvious:
it generates a horizontal separator line in the menu.
The ``cascade'' type creates a pull-right (or walking) submenu.
The second string is the text to put in the menu item.
The third item in the sublist is yet another menu specification.
But this time the menu string itself must appear {\it not}
the name of a variable that it is set to as in menu bar specifications.
This is an admitted inconsistency based on the idea that these
submenus would be fairly small but menu bar menus would be larger.
You could have several levels of submenus but they get tedious for
the user if they are nested too deeply.

So that is the format of the menu strings.
To change the browser menu, change the value the Tcl variable
{\tt BrowserMenuSpec} is set to and similarly for {\tt TextMenuSpec}.
You might try small changes in the existing menus as build up as you
gain confidence.
It is quite easy to remove or rearrange menu items
and it is fairly easy to add your own items.
Cascade submenus are also quite easy except for keeping
the brackets matched.

\subsubsection{ Interactively Changing Menu Bars }

If you change the menu specifications they do not automatically
go into effect.
If you save the changed menu specifications, quit Point and
restart it, then the new menus will be in effect.
You can also do this while Point is running but it is more work.
You should do the following things in order:

\begin{enumerate}

\item {\it Change the menus:}  make the changes you want.

\item {\it Tell Point:}  Select the new menu specification and
execute the command from the text window menu bar
``EDIT/Execute Selection as Tcl''.
This will read the new menu specification into Point
(actually into the tcl interpreter).
Look in the {\tt xterm} window where you started Point for error
messages that reading the tcl code might generate.

\item {\it Remake the menus:}  Execute the command {\tt RemakeTextMenus}
or {\tt RemakeBrowserMenus} (depending on which menu bar you changed).
This command is not on any menu so there are two ways to execute it.
The first is to rebind some key to that command.
The second is to enter the string {\tt RemakeTextMenus} or
{\tt RemakeBrowserMenus} in a scratch window, select it and
execute the ``EDIT/Execute Selection as Tcl'' command.

\end{enumerate}

Changes made this way will affect windows created after the
change is made.


\subsection{ Interactively Playing with Tcl }

The command on the text menu bar ``EDIT/Execute Selection as Tcl''
allows you to access the tcl interpreter and do a lot
of different things.
You start by opening up a scratch window.
Then you enter tcl commands, select them and then execute
then as tcl using the above command.

Here are some tcl commands you might find useful to try.
Note many of them are of the form ``{\tt pr [command]}''
where the {\tt command} generates some information and the
{\tt pr} tcl proc prints its argument which in this case is
the output of the command.
The output is printed in the {\tt xterm} window Point was
started from.
A common mistake is to generate the information but forget
to print it out.

\begin{itemize}
\item {\tt pr [GetTextWindowList]} ---
This command prints out a list of all text windows.
You often need to know the name of a window to get more
information about it.
\item {\tt pr [GetBrowserList]} --- prints a list of the browser windows.
\item {\tt pr [winfo children .]} --- this shows all the children of
the main Point window, that is, all the text and browser windows.
\item {\tt pr [winfo children .tw00001]} --- once you find out the
name of a text window you can print out its children to see the
way it is constructed.
\item {\tt pr [bind .tw00001.vScrollAndText.text <Shift-Left>]} ---
This is one way to find out the binding of a key in a text window.
The window name is gotten from {\tt GetTextWindowList}.
\item {\tt bind .tw00001.vScrollAndText.text <Shift-Left>
	{MoveSel word left}} --- this is a way to bind
	a command to a key.
\item {\tt pr [GetWindowInfo .tw00001]} --- print information
about a window.
\item {\tt pr [GetFileInfo .tw00006]} --- print information
about the file in a window.
\item {\tt pr [selection get]} --- print the X selection
\item {\tt pr [Sel get]} --- print the Point selection information
\item {\tt pr [expr 55*78]} --- you can execute arbitrary tcl code.
\end{itemize}



\subsection{ Changing the scroll bars to the right }

You can easily change the scroll bars in text window from the
left to the right.
Look in the file {\tt tclLib/windows.tcl} and find the tcl proc
{\tt TextWindow}.
Inside that proc (it has only one line) change ``left'' to ``right''.
This only changes the scroll bars in text windows, not for browser
windows.
Changing the browser windows requires changes to tk.




\subsection{ Extending Point Using Tcl as a Macro Language }

Since Tcl is an interpreted language you can use it to write Point macros.
Let's look at a couple of examples.

This is a macro that indents each line that contains some
part of the selection:
\begin{verbatim}
proc IndentSelection {} {
     set sel [Sel get]
     set here [lindex $sel 0]
     set stop [lindex $sel 1]
     for {} 1 {} {
          MoveSel line left0
          set here [lindex [Sel get] 0]
          if {$here>$stop} \
               break;
          InsertString \t
          set stop [expr $stop+1]
          MoveSel char down
     }
}
\end{verbatim}

The first three lines get the bounds of the selection.
The loop beginning in the fourth line continues until a {\tt break}
is executed.
We move the selection to the very first character of the line
and get its location.
If we are beyond the original selection then the loop is done.
Otherwise we insert a tab to indent the line.
Since we have added a character we have to adjust {\tt stop}.
Finally we move down to the next line and repeat the loop.

This example sends the selection through a Unix command and
replaces it with the output of the Unix command.
this is analogous to the ``!'' command in {\tt vi}.
\begin{verbatim}
proc Filter {{cmd fmt}} {
     set s [Sel return]
     set ret [catch {exec $cmd < $s} ns]
     if {$ret==0} DeleteToScrap
     InsertString $ns
}
\end{verbatim}
Note that {\tt fmt} is the default {\tt cmd}.
This fills the text to 72 character lines.
First we get the selection and send it to the Unix command.
We use {\tt catch} in case the Unix command fails.
If the command succeeds we delete the selection
and replace it with the standard output of the command.
Otherwise we insert the error message in front of
the original selection.

The following three Tcl procedures make it easy to
experiment with macros:
\begin{verbatim}
proc DefineMacro {{id 0}} {
     set name Macro$id
     global $name
     set $name [selection get]
}

proc ExecMacro {{id 0}} {
     set name Macro$id
     global $name
     # Note the two levels of indirection on name
     eval [set $name]
}

proc ExecSel {{id 0}} {
     eval [selection get]
}
\end{verbatim}
``DefineMacro'' defines the selection as the name of the macro
and ``ExecMacro'' executes the macro with that name.
``ExecSel'' is used to read in the macro definition.
As I was writing the above indenting macro I worked as follows.
First I defined ``IndentSelection'' as the macro with ``DefineMacro''.
Then I wrote the ``proc'' definition of ``IndentSelection'', selected
it and entered the definition with ``ExecSel''.
Then I selected the sample text and executed ``ExecMacro''.
When there was an error, I corrected the definition,
entered it with ``ExecSel'' and retried it with ``ExecMacro''.





\section{Point Commands} \label{sect:commands}

In this section I will describe each of the commands that Point understands.
Each command entry begins with the command name and then gives a description
of the effect of the command.
Some commands have arguments which are given after the command name.
All arguments are strings.

Commands that insert text insert it at the insertion point.
Commands that operate on the selection operate on the selection window.
Other commands operate in the active window (the last window the
mouse entered) unless the optional ``tkPathName'' argument is present.
If it is present then it is the Tk path name for the window
which will be affected by the command.

Some of these commands are native Point commands and some of them
are implemented with tcl procedures.




\subsection{Insertion Commands}

\subsubsection{InsertAscii numericChar}
This command allows you to insert any eight bit character into the file.
The argument ``numericChar'' can specify the character as a decimal integer.

\subsubsection{InsertFromScrap}
The text in the scrap buffer is inserted at the insertion point.
The contents of the scrap buffer are unchanged and can be inserted again
in the same or a different place.
The other commands that change the scrap buffer are:
DeleteToScrap, ExchangeWithScrap and CopySelToScrap.

{\bf Usage hint:} {\it  to duplicate a line (or other section of text)
a number of times: select and delete the lines and then
insert them as many times as required.}

\subsubsection{InsertString string}
The specified string is inserted at the insertion point.

\subsubsection{InsertSelectedString}
The X selection is inserted at the insertion point.

\subsubsection{InsertFile}
The selection is used as a file name.
That file replaces the selection in the window.




\subsection{ Selection Editing Commands}

\subsubsection{ChangeCaseOfSel \{toupper$\mid$tolower$\mid$toggle\}}
This command changes the case
of each letter in the selection.
Non-letter characters are unaffected.
The argument determines how the case is changed:
to upper case, to lower case or change case
(lower goes to upper and upper goes to lower).
If the argument is missing or empty it defaults to ``toggle''.

\subsubsection{DeleteToScrap}
The selected text is deleted and placed in the scrap buffer.
The previous contents of the scrap buffer are lost.

\subsubsection{ExchangeWithScrap}
The contents of the scrap buffer and the selection are exchanged.

{\bf Usage hint:} {\it to exchange two pieces of text:
select one piece of text
and invoke CopySelToScrap, then select the other piece of text and
invoke ExchangeWithScrap, finally select the first piece of text
and invoke ExchangeWithScrap again.}

\subsubsection{DeleteLine [all]}
Deletes from the beginning of the selection to the end
of the line.
If {\tt all} is 1 (true) the the entire line is deleted.

\subsubsection{IndentSelection [outdent]}
A tab is inserted at the beginning of each line that contains
at least one selected character.
If {\tt outdent} is 1 (true) then the first character
of each line is deleted.
Presumably this is a tab character.

\subsubsection{Filter [cmd]}
The selection is provided to {\tt cmd} as its standard
input and the standard output of {\tt cmd} replaces
the selection.
The default {\tt cmd} is {\tt fmt}.

\subsubsection{JustifySel}
This command acts on whole lines.
It acts on all lines in which one or more characters
in the line are selected.
All of the lines are justified between column 1 and the column
specified by the {\bf rightMargin} option variable.

{\bf Usage hint:} {\it To indent and justify, first justify to column 1 and
then shift all the lines to the right.}




\subsection{Move and Copy Commands}

\subsubsection{CopySelToScrap}
Copies the selection to the scrap buffer.
The selection is unaffected.
The previous contents of the scrap buffer are lost.

{\bf Usage hint:} {\it to copy the selected text somewhere:
invoke CopySelToScrap, move to the insertion location and
select it and then invoke InsertFromScrap.}

\subsubsection{CopySelToMouse}
The selection is copied to the location of the mouse pointer
when the command was executed.
The copy is by characters, words or lines depending on the mode
of the selection.
If the selection mode is by words or lines the insert is at the
beginning of the word or line that contains the location
of the mouse pointer.

{\bf Usage hint:} {\it to copy a line (or lines) of text:
select the lines in line mode (with a triple click)
and invoke the CopySelToMouse command anywhere on the line {\bf after}
you want the lines copied to.}

{\bf Note:} This command is only suitable to attach to a mouse
button or a mouse menu.

\subsubsection{MoveSelToMouse}
The selection is moved to the current location of the mouse pointer.
The move is by characters, words or lines depending on the mode
of the selection.
If the selection mode is by words or lines the insert is at the
beginning of the word or line that contains the location
of the mouse pointer.

{\bf Usage hint:} {\it to move a line (or lines) of text:
select the lines in line mode (with a triple click)
and invoke the MoveSelToMouse command anywhere on the line {\bf after}
you want the lines moved to.}

{\bf Note:} This command is only suitable to attach to a mouse
button or a mouse menu.

\subsubsection{CopyToHereMode}
This command must be invoked twice to complete a copy.
The first time you invoke the {\tt CopyFromHereMode} command you set the
location to which the text will be copied.
You are then in {\em duplicate} mode and the mouse sprite becomes
a hand with a right pointing hand (this can be changed
to any cursor font cursor with the {\tt copySpriteName} option).
You can then continue using Point and execute any commands you like.
When you invoke the {\tt CopyFromHereMode} command while in duplicate mode
the current selection will be copied to the location remembered
from the first invocation of {\tt CopyFromHereMode}.

{\bf Usage hint:} {\it the advantage of this command is that you can copy
text into your typing stream without losing your insertion point.
Suppose you are typing a line in C and want to type a variable
name that appears on a nearby line.
First invoke the {\tt CopyFromHereMode} command, then select the nearby word
(with a double click of the left mouse button) and then invoke
{\tt CopyFromHereMode} again.
The word is copied into the line and the insertion point is at the
end of the copied word so you can continue typing the line without
readjusting the insertion point.
You can create a line by grabbing pieces of nearby lines one after
the other.
A new line of code in an existing routine can usually be created without
the keyboard at all.}

My rule about copying nearby words is that the break-even point for
effort is around five or six characters, that is, if the variable name
you are about to type is longer than that it is probably easier to
copy it than to retype it.
Another advantage of copying is that there is no chance you will mistype
the word.

If the word you want to copy is not visible on the screen it is usually
not worth the effort to copy it because that would require window
rearranging.
It may be worth it if it is a global variable with a long name
(and global variables should have long descriptive names)
that would be hard to type correctly.
And since you will have to type in an external reference to the
variable anyway you might as well find a copy of the external reference,
copy it in as a complete line and then copy the global variable name
from the copied in line.

\subsubsection{MoveFromHereMode}
This is identical to the {\tt CopyFromHereMode} command except that the text
is moved (that is, deleted from its original location) when the
second {\tt MoveFromHereMode} is invoked.





\subsection{Again and Undo Commands}

\subsubsection{Again [tkPathName$\mid$mostrecent$\mid$thisfile]}
The last command (not counting deletes) is repeated in the current context.
If the last command copied text into the insertion point then that same
text is copied into the current insertion point.
If the last two commands were to delete the current selection and replace
it with new text (either typed in or copied in from somewhere else) then
this command will replace the current selection with that same text.

The optional argument determines which command is repeated.
The default is {\tt mostrecent} which means that the most
recent edit is repeated.
If the argument is {\tt thisfile} then the most recent edit
in the file containing the selection is repeated.
Finally if a window name is given the most recent edit in the
file in that window is repeated.

{\bf Usage hint:} {\it to perform a selective replace, that is, to replace
some but not all instances of one text string with another text string:
search for the string using a SearchForString selection box until you
find the first instance you want to replace.
Then search for the following instances of the string
(either with the dialogue box or with the RepeatSearch command).
For all the ones you want to replace just invoke Again.}

\subsubsection{Undo [nToUndo] [tkPathName]}
The last undone command is undone:
deleted text is reinserted, copied text is deleted,
moved text is put back where it was moved from.
If this command is repeated the undo itself is undone
so only one previous command can be undone with UndoOne.
If this command is repeated the command before the last command
is undone and so on.
The argument {\tt nToUndo} (which defaults to 1) determines how
many edits will be undone.

\subsubsection{Redo [nToRedo] [tkPathName]}
The last undone command that has not been redone is redone.
This command can be used only to redo commands that have been undone.
The argument {\tt nToRedo} (which defaults to 1) determines how
many edits will be redone.




\subsection{Search and Replace Commands}

\subsubsection{Search stringToSearchFor
		[\{forward$\mid$backward\}] [tkPathName] [update]}
The search is performed in the active window
(not the selection window), that is, the
last window the mouse sprite was in unless the third
argument is given, in which case it names the window to search in.
If the selection is in that window then
the search begins at the selection, otherwise the search begins
at the first character of the file.
If the fourth argument is missing, empty or is the string ``update''
then the window will be updated after the string is found.
Otherwise it will not be updated.
The search direction is either forward or backward.
There are several options that affect how the search is done:
the {\tt ignoreCase} option determines whether case is significant
in the search, the {\tt findWholeWords} option determines whether
the string can be a substring of a larger string or not.
The {\tt linesOverFind} option affects the positioning of the found string
in the window if the window has to be jumped in order to show the
found string.
If the second argument is missing or empty then ``forward''
is assumed.
If the third argument is given the second argument must appear also
(although it can be the empty string).

\subsubsection{RepeatSearch \{forward$\mid$backward\} [tkPathName]}
The last string searched for is used as the search string
for a new search.
The search direction is either forward or backward.
If the first argument is missing or empty then ``forward''
is assumed.
If the second argument is given the first argument must appear also
(although it can be the empty string).

\subsubsection{RegexSearch regexToSearchFor
		[\{forward$\mid$backward\}] [tkPathName] [update]}
This command searches for a regular expression in the text.
It is identical to the ``Search'' command given above except that
it is a regular expression search instead of a fixed text search and
if {\tt regexToSearchFor} is empty then the last regular expression
that was searched for will be used again.

{\bf Usage hint:} {\it Normally you want a fixed text search.
Then you do not have to worry about escaping special characters
and it is a bit faster as well.
Just use the regular expression search when you really want to
search for a pattern rather than a fixed string.}

\subsubsection{RepeatSearch \{forward$\mid$backward\} [tkPathName]}
The last string searched for is used as the search string
for a new search.
The search direction is either forward or backward.
If the first argument is missing or empty then ``forward''
is assumed.
If the second argument is given the first argument must appear also
(although it can be the empty string).

\subsubsection{CTag ctag}
The {\tt ctag} is looked up in the file {\tt tags}
(created by the Unix command {\tt ctags}) in the current directory.
the file containing the flag is loaded into a new window
or, if the file is already in a window, that window is raised
to the top.
The window is jumped to show the {\tt ctag} string and the {\tt ctag} string
is selected.

\subsubsection{SearchCharacter char [tkPathName]}
If the `char' is an ASCII character then it is appended
to the search string maintained by this command and the
string is searched for (forwards in the file).
Thus the command implements incremental search.
If `char' is a non-ASCII character (e.g., a function key)
then the search string is cleared.
This search string is different from the search string for the
search command.

{\bf Usage hint:} {\it I bind this to the menu bar item
for string search.
Then I can type characters into that menu item and get incremental
search.
It never throws away characters unless you clear the search string
so get in the habit of pressing a function key before starting
your search string.}

\subsubsection{Replace searchString replaceString [inselection] [tkPathName]}
This command replaces each occurrence of the ``searchString'' with the
``replaceString'' either in the entire file (if the third argument is
missing or empty) or within the selection (if the third argument is
``inselection'').
The search is affected by the normal search options
({\tt ignoreCase} and {\tt findWholeWords}).
The replace is done in the active window unless the fourth
argument names a valid text window.

\subsubsection{RegexReplaceOne regexString replaceString [tkPathName]}
This command finds the next match of the regular expression
``regexString'' and replaces it with ``replaceString''.

\subsubsection{RegexReplaceAll regexString replaceString [inselection]}
This command finds all matches of the regular expression
``regexString'' and replaces them with ``replaceString''.

\subsubsection{FindMatchingBracket}
The following brackets are matched in pairs:
( and ), \{ and \}, [ and ].
If one of these characters is the first character of the selection,
Point will search (forward for (, \{ and [; backward for ), \} and ])
for the matching bracket.
By matching, we mean that it keeps a count of left and right brackets
and finds the matching bracket when count is zero.





\subsection{Scrolling and Jumping Commands}

\subsubsection{MoveToEndFile [tkPathName]}
The window is repositioned so that it shows the end of the file.
The file will be positioned so that the end of file marker is
on the last line of the window (unless the entire file will
fit in fewer lines).

\subsubsection{ShowSelection}
The window containing the selection is brought to the front and
repositioned so that the beginning of the selection is visible
in the window.
The {\tt linesOverFind} option determines the placement of the first
line of the selection in the window.

\subsubsection{MoveToLastPlace [tkPathName]}
This repositions the window to the last place you have jumped from.
Point remembers a separate last place for each window.

{\bf Usage hint:} {\it this command is useful for switching between
two different places in the file.}

\subsubsection{GotoLine lineNumber \{lof$\mid$top\} [tkPathName]}
The window is repositioned so that the line number specified
is show in the window.
If `lof' is specified then the {\tt linesOverFind} option
determines where the line is positioned in the window.
If `top' is specified then the line number is positioned 
at the top of the window.
See the description of that option for details.
If the second argument is missing or empty then ``lof''
is assumed.

\subsubsection{GotoDigit digit [tkPathName]}
The `digit' is added to a line number that is collected.
If `digit' is not a digit, then the line number collected
so far is jumped to.
The {\tt linesOverFind} option determines the placement
of the line gone to in the window.

\subsubsection{ScrollWindow \{up$\mid$down\} \{numberOfLines$\mid$page\}
		[tkPathName]}
The window is scrolled by `numberOfLines'.
If the second argument is ``page'' or missing or empty,
then the window is scrolled down the number of lines in
the window minus two (for context).
The first argument determines the scroll direction.
If it is missing or empty then ``down'' is assumed.




\subsection{Cursor Positioning Command}

\subsubsection{MoveSel \{char$\mid$word$\mid$line\}
		\{up$\mid$down$\mid$right$\mid$left$\mid$left0\}}
This command moves the selection.
The movement can be by character, by word or to the limits of the line.
The second argument determines the direction of movement.
You can only use ``up'' and ``down'' with ``char''.
For ``line'', ``right'' moves the selection to the last character
of the line (which will always be the newline character
that ends the line), ``left'' moves to the first non-white space
character of the line and ``left0'' moves to the first
character of the line.
These commands are usually attached to keys
or used in macros (like indent selected lines).





\subsection{Window Creation Commands}

\subsubsection{OpenWindow fileName geometry doNotAsk}
A new window is created and 'fileName' is loaded into the new window.
The `geometry' argument specifies the geometry of the new window.
If `geometry' is missing or empty then the default
{\tt textGeometry} is used.
If {\tt fileName} does not exist then the user will be asked
for verification before it is created unless the {\tt doNotAsk}
argument is present and is literally the string ``doNotAsk''.
This command returns a string that is the Tk path name
of this window.
It will always be of the form ``.twDDDDD'' where each D
is a decimal digit.
Point assigns window names with a counter so the first window
will be ``00001'', the second window ``00002'' and so on.
Text windows, browser windows, and some popup dialogue boxes
all share the same counter so the numbers of the text windows
will not necessarily be sequential.

\subsubsection{CloseWindow \{save$\mid$nosave$\mid$ask\} [tkPathName]}
The window is closed and removed from the screen.
If the file in the window has been edited and this is the last window
open on the file then the argument determines whether the
file will be saved or not.
If `save' is specified it will be saved, if 'nosave' is specified
it will not be saved and if 'ask' is specified a dialogue box
will pop up asking the user whether to save the file or not.
If the first argument is missing or empty then ``ask'' is assumed.

\subsubsection{Browser geometry [tkPathName]}
This command creates another file browser.
The `geometry' argument determines the geometry of the new browser.
If it is missing or empty then the default {\tt browserGeometry} will be used.
If the second argument is missing or empty then ``big'' is assumed.

{\bf Usage hint:} Use this command to get a file from another directory
while keeping a file browser showing the current directory.

\subsubsection{CloseBrowser [tkPathName]}
The specified browser (the active browser is the default) is closed.




\subsection{File Writing Commands}


\subsubsection{SaveFile [tkPathName]}
The file in the active window (or the window named by {\tt tkPathName})
is saved on disk.

\subsubsection{SaveAs [tkPathName]}
The file in the active window (or the window named by {\tt tkPathName})
is written to disk.
The name to use for the file is provided by the user through a dialogue box.

\subsubsection{SaveAllFiles}
All files that have been changed but not yet saved are saved.





\subsection{Window Size and Location Commands}

\subsubsection{LowerWindow [tkPathName]}
Move the active window (or the window named by {\tt tkPathName})
behind all other windows in the display
(both Point windows and the windows of other X applications).

\subsubsection{MoveWindow geometry [tkPathName]}
The window is moved and resized according to `geometry'.

\subsubsection{RaiseListWindow numInList [geometry]}
This command affects the window which is the `numInList'th
item on the open window list.
That window is raised and resized according to `geometry'
(which defaults to {\tt textGeometry}).

\subsubsection{RaiseWindow [tkPathName]}
Bring the active window (or the window named by {\tt tkPathName})
to the front of all other windows in the display
(both Point windows and the windows of other X applications).

\subsubsection{Zoom [\{vertical$\mid$full\}] [tkPathName]}
The window is enlarged to be the full height of the screen
(if the first argument is {\tt vertical}
or the full size of the screen (if the first argument is {\tt full}.
If the first argument is missing {\tt vertical} is assumed.





\subsection{Font Commands}

\subsubsection{BrowserFont fontName [tkPathname]}
This command changes the font
(in the specified browser where the active browser is the default)
to `fontName'.
If the font change fails the font in the active browser
is changed to the default font {\tt browserFont}.

\subsubsection{TextFont fontName [tkPathname]}
This command changes the font
(in the specified window where the active window is the default)
to `fontName'.
If the font change fails the font in the active window
is changed to the default font {\tt textFont}.





\subsection{ Selection and Mode Commands }

\subsubsection{CancelModes}
This command cancels duplicate or extract mode
if either in in effect.
Otherwise the command has no effect.
This command is used when you inadvertently enter
duplicate and extract mode with a erroneous button
click or keypress.
The mouse sprite will change back to normal
(from the right pointing hand that marks duplicate mode).

\subsubsection{ExtendSelToLines}
The selection is extended so that it consists of only complete lines.

\subsubsection{Sel \{set$\mid$get$\mid$return\} selBegin selEnd [tkPathName]
			[\{char$\mid$word$\mid$line\}]}
If the first argument is ``get'' then there are no other arguments
and the command returns a Tcl list of four strings.
The first two strings are decimal numbers and are the character positions
of the first and last character in the selection.
The third string is the Tk path name of the window containing
the selection.
The fourth string is the selection mode and is one of the
string {\tt char}, {\tt word} or {\tt line}.
If the first argument is ``set'' then the selection is changed
so that it begins at character position ``selBegin'' and ends
at character position ``selEnd''.
If the third argument is present the selection window is changed to
the window specified by the Tk path name.
If the fourth argument is present is changes the
selection mode.
If the first argument is ``return'' then the contents of the
selection are returned.

\subsubsection{SetLineNumbers how}
This command determines whether line numbers are displayed
with the text in the window.
Each window has a line number flag and this command changes it.
If {\tt how} is "1" then line numbers are displayed.
If {\tt how} is "0" then line numbers are not displayed.
If {\tt how} is anything else then the line numbering flag is toggled.

\subsubsection{SetTextColor colorName \{normal$\mid$selected\}
     \{foreground$\mid$background\} }
This command sets the text color in the active window.
You can use it to set either the foreground or the background
of either normal text or selected text.
If the second argument is missing or empty then ``normal''
is assumed.
If the third argument is missing or empty then ``foreground''
is assumed.

\subsubsection{ToggleReadOnly [tkPathname]}
A file can be specified ''read-only'', in which case,
Point will not allow you to modify it.
This command toggles the read-only status of the file
in the active window (or the window named by {\tt tkPathname}).








\subsection{ Subprocess Commands }

\subsubsection{ConnectToPty arg1 arg2 arg3 arg4 arg5 arg6}
Create a Unix pseudo-tty.
Execute the command {\tt arg1} with arguments {\tt arg2} through
{\tt arg6} as a subprocess.
The output of the subprocess is inserted into the window
containing the selection when this command is executed.
Input into that window is sent to the subprocess over
the pseudo-tty.

\subsubsection{WaitForProcess pid}
Delay until subprocess {\tt pid} finishes.

\subsubsection{RunProgramInFile delsel}
The selection is used as the name of a program to execute
in a subprocess and the window containing the selection is connected
to the subprocess via a Unix pseudo-tty.
If {\tt delsel} is 1 (true) then the selection is deleted
after it is used.

\subsubsection{RunProgramInWindow prog}
{\tt prog} is used as the name of a program to execute
in a subprocess.
A window is created and connected to the subprocess via a Unix pseudo-tty.





\subsection{ Information and Status Commands }

\subsubsection{MessageLine message}
The text string {\tt message} is displayed on the message line of
each text window.
This command can be used from tcl/tk macros to display messages
to the user.

\subsubsection{PrintStats}
Information about the effectiveness of Point's caches is
printed out to the {\tt xterm} window in which Point was started.

\subsubsection{ShowUndoStack}
This command pops up a dialogue box that shows the command
history.
For each command it gives information about the command
including whether it has been undone (and not redone) or not.

\subsubsection{GetTextWindowList}
This command returns a list of the Tk name of all existing text windows.

\subsubsection{GetBrowserList}
This command returns a list of the Tk name of all existing browsers.

\subsubsection{GetRowCol offset [tkPathName]}
This command returns a Tcl list of two integers.
These are the row and column in the window of the
character in character position ``offset''.

\subsubsection{WindowName set \{active$\mid$selected\} tkPathName}
Make the specified window (in {\tt tkPathName}) either the
selection window or the active window.

\subsubsection{WindowName get \{active$\mid$selected\}}
Return the Tk path name of the selection window or the active window.

\subsubsection{GetWindowInfo [tkPathName]}
This command returns a Tcl list of information about the
window whose Tk path name is given by ``tkPathName''
(the default is the active window).
If the argument is omitted then the active window is assumed.
This list consists of (in order):
the character position of the first character displayed in the window,
the character position after the last character displayed in the window,
the line number of the first line displayed in the window,
the line number of the last line displayed in the window plus one,
the number of columns the window is scrolled horizontally,
the number of lines displayed in the window,
the number of columns displayed in the window,
the X window id of the window,
the Tk window id of the toplevel window,
the Tk window id of the text window.

\subsubsection{GetFileChars firstByte lastByte [tkPathName]}
Return the characters from the file in the specified window.
If no window is given the active window is used.

\subsubsection{GetFileInfo [tkPathName]}
This command returns a Tcl list of information about the
file displayed in the window whose Tk path name is given by ``tkPathName''.
If the argument is omitted then the active window is assumed.
This list consists of (in order):
the name of the file displayed in the window,
the current size of the file in bytes,
the original size of the file when it was loaded
(before any editing was done),
a flags field which is the {\it or}ing together
of the flags:
0x1 if a backup file has been made,
0x2 if the file is read only,
0x4 if the file has been changed (and not yet saved),
and 0x8 if the file cannot be changed to read/write.

\subsubsection{winfo children .}
This is really a straight Tk command.
It returns the path names of all the children of the root window.
For Point this will be all the toplevel windows.
All the text windows will have names of the form ``.twDDDDD''.
All browser windows will have names of the form ``.bwDDDDD''.
Here each ``D'' is a decimal digit.
Popup dialogues will have various other names
but none of these names will start with ``.bw'' or ``.tw''.
This command will provide you with a list of all of Point's
text windows.
You can use other commands to get information about each
text window (and the file it displays).




\subsection{ Option Commands }

\subsubsection{Option get optionName}
A string is returned that is the value of the specified
Point option.
Integers are returned as decimal strings.
Booleans are returned as 0 (False) or 1 (True).

\subsubsection{Option set optionName optionValue}
The specified Point option is set to ``optionValue''.
Integer arguments can be in octal, decimal or hex.
Boolean options can be 0, 1, true or false.
Case is not considered in the strings ``true'' and ``false''.





\subsection{Miscellaneous Commands}

\subsubsection{QuitPoint \{save$\mid$discard$\mid$ask\}}
Quit Point after checking each file to see if it has been changed
but not yet saved.
The argument determines whether these files will be saved.
If the argument is missing then ``ask'' is assumed.

\subsubsection{Redraw}
This command redraws the entire window so that it corresponds to
the true state of the file.

\subsubsection{CD directoryName [tkPathname]}
The Unix current directory of the specified browser
(the default is the active browser) is changed to `directoryName'.
If the first argument is missing or empty then the user's home directory
is assumed.

\subsubsection{SendOnClose tkPathName interpName command}
This command is intended to be used by a process that
is sending commands to Point via the Tcl ``send'' command.
The ``tkPathName'' is the Tk window name that Point
assigns to the window (and that is returned by the
OpenWindow command).
The ``interpName'' is the name of the Tcl interpreter to send
``command tkPathName'' to when that window is closed.
That is, when window ``tkPathName'' is closed,
Point will execute the Tcl command:
\begin{verbatim}
     send interpName command tkPathName
\end{verbatim}

\subsubsection{DoNothing}
This command does nothing.
It is a placeholder command if you have to specify
a command but do not want it to actually do anything.

\subsubsection{WaitForReturnString}
This command runs a local Tk event loop and does not return
until the Point option {\tt returnString} is non-empty.
This is used to create a modal dialogue box.
You set up the dialogue box so that it sets returnString to
a non-empty string when it has its input.
Then you pop up the dialogue box and execute this command.
When it returns the input will be available.
As an example look at {\tt proc MakeVerifyBox} and
{\tt proc DoReplace} in {\tt ptsetup.tcl}

\subsubsection{RemakeTextWindows}
The menu bar of each text window is destroyed and recreated.
This is used after the menu bar specification has been changed
and you want the menus to reflect the new specification.

\subsubsection{RemakeBrowserWindows}
The menu bar of each browser window is destroyed and recreated.
This is used after the menu bar specification has been changed
and you want the menus to reflect the new specification.

Change the cursor of all the text and browser windows.
The new cursor can be given by name (it must be in the X font cursor
font) or it can be one of the Point cursors (busy, current, main or dup).





\subsection{ Event Callback Commands }

The commands in this section are used by event handlers to get
Point to do certain things.

\subsubsection{VScroll \{press$\mid$release$\mid$motion\} newY button}
This command is used by the scrollbar to scroll a window.

\subsubsection{HScroll \{press$\mid$release$\mid$motion\} newY button}
This command is used by the scrollbar to scroll a window.

\subsubsection{Key keysym shiftState}
Tells Point that a certain key was pressed.

\subsubsection{Mouse tkPathName command x y clicks}
Tells Point about a mouse event.
Command must be one of: BeginSelection, BeginExtend, ExtendSelection,
EndExtending, BeginMouseMenu1, BeginMouseMenu2, ContinueMouseMenu,
CancelMouseMenu or EndMouseMenu.

\subsubsection{Configure tkPathName}
Tells point that a configure event occurred.

\subsubsection{EnterText tkPathName}
Tells point that the mouse cursor entered a text window.

\subsubsection{EnterBrowser tkPathName}
Tells point that the mouse cursor entered a browser.

\subsubsection{Expose tkPathName x y width height count}
Tells point that an expose event occurred.





\subsection{ Dialogue Box Commands }

\subsubsection{MakeAsciiBox}
This command allows you to insert any eight bit character into the file.
A dialogue box comes up and you can enter the character as a decimal
(starting with [1-9]) integer, an octal (starting with 0) integer or
a hexadecimal (starting with 0x) integer.

\subsubsection{MakeAboutBox}
This command brings up a box that gives information
about Point.
Clicking on the dialogue box closes it.

\subsubsection{MakeMsgBox msg}
Pops up a message box displaying the message in ``msg''.

\subsubsection{MakemmBox}
Pops up a dialogue box that allows you to set the parameters
of the mouse menus, that is, the text and command for each
direction of the two mouse menus.

\subsubsection{MakeColorBox}
Pops up a dialogue box that allows you to select the color
or the foreground and background of normal and selected text.
You select the colors from a scrolling list of of color names from
the X color database.

\subsubsection{MakeCtagBox}
Pops up a dialogue box that allows you to search for a C tag.

\subsubsection{MakeKeywordBox}
Pops up a dialogue box that allows you to search for a keyword.

\subsubsection{FillKeywordBox}
Starts a search for a keyword based on the keyword in
the keyword box and fills the list with the file names
containing that keyword.

\subsubsection{MakeGotoBox}
Pops up a dialogue box that allows you to jump to a line number.

\subsubsection{MakeDebugBox}
Pops up a dialogue box that allows you to change the value
of the debug variable.

\subsubsection{MakeQuitBox}
Pops up a dialogue box that allows you to choose quit options.

\subsubsection{MakeReplaceBox}
The {\tt MakeReplaceBox} command brings up a dialogue box
that allows you to specify
the string to search for and the string to replace it with.
There are three checkbox options you can set.
The first is to verify each replacement.
If this is turned on then each time the search string is found
the window is made to show the string found, the string is
selected and you
are presented with a dialogue box asking whether you want
to replace the string or not.
If you do not choose the verify option you might want to use the
internal (C) version of search and replace rather than the external
(Tcl) version which is considerably slower.
The final checkbox option is to restrict the replace to the selection
only.
This is useful, for example, for changing one identifier to another
over a few lines of code.
The search uses the standard search options (ignore case, find
whole words, etc.)

\subsubsection{MakeSearchBox}
Pops up a dialogue box that allows you to enter a search
string, change the {\tt findWholeWords}
and {\tt ignoreCase} search options
and set the search direction as backward or forward.

\subsubsection{OpenFileOrCD name geometry}
If ``name'' is a directory the active browser
is changed to that directory.
If ``name'' is an ordinary file then a window is opened
up with geometry ``geometry'' that displays file ``name''.

\subsubsection{MakeSearchOptionsBox}
Pops up a dialogue box that allows the user to change
all the search related options.

\subsubsection{MakeOtherOptionsBox}
Pops up a dialogue box that allows the user to change
several string valued options.







\subsection{ Point Commands --- Alphabetical List }

This is a list of Point commands and tcl procs that are defined
in the tcl code distributed with Point.

\begin{description}

\item {Again [tkPathName$\mid$mostrecent$\mid$thisfile]}
\item {Browser geometry \{big$\mid$small\} [tkPathName]}
\item {BrowserFont fontName [tkPathname]}
\item {CD directoryName [tkPathname]}
\item {CTag ctag}
\item {CancelModes}
\item {ChangeCursor
	\{busy$\mid$current$\mid$main$\mid$dup$\mid$FontCursorName\}}
\item {CloseBrowser [tkPathName]}
\item {CloseWindow \{save$\mid$nosave$\mid$ask\} [tkPathName]}
\item {Configure tkPathName}
\item {ConnectToPty arg1 arg2 arg3 arg4 arg5 arg6}
\item {CopySelToMouse}
\item {CopySelToScrap}
\item {CopyToHereMode}
\item {DeleteLine [all]}
\item {ChangeCaseOfSel \{toupper$\mid$tolower$\mid$toggle\}}
\item {DeleteToScrap}
\item {DoNothing}
\item {EnterBrowser tkPathName}
\item {EnterText tkPathName}
\item {ExchangeWithScrap}
\item {Expose tkPathName x y width height count}
\item {ExtendSelToLines}
\item {FillKeywordBox}
\item {Filter [cmd]}
\item {FindMatchingBracket}
\item {GetBrowserList}
\item {GetFileChars firstByte lastByte [tkPathName]}
\item {GetFileInfo [tkPathName]}
\item {GetRowCol offset [tkPathName]}
\item {GetTextWindowList}
\item {GetWindowInfo [tkPathName]}
\item {GotoDigit digit [tkPathName]}
\item {GotoLine lineNumber \{lof$\mid$top\} [tkPathName]}
\item {HScroll \{press$\mid$release$\mid$motion\} newY button}
\item {IndentSelection [outdent]}
\item {InsertFile}
\item {InsertFromScrap}
\item {InsertSelectedString}
\item {InsertString string}
\item {JustifySel}
\item {Key keysym shiftState}
\item {LowerWindow [tkPathName]}
\item {MakeAboutBox}
\item {MakeAsciiBox}
\item {MakeColorBox}
\item {MakeCtagBox}
\item {MakeDebugBox}
\item {MakeGotoBox}
\item {MakeKeywordBox}
\item {MakeMsgBox msg}
\item {MakeOtherOptionsBox}
\item {MakeQuitBox}
\item {MakeReplaceBox}
\item {MakeSearchBox}
\item {MakeSearchOptionsBox}
\item {MakemmBox}
\item {Mouse tkPathName command x y clicks}
\item {MoveFromHereMode}
\item {MoveSel \{char$\mid$word$\mid$line\} \{up$\mid$down$\mid$right$\mid$left$\mid$left0\}}
\item {MoveSelToMouse}
\item {MoveToEndFile [tkPathName]}
\item {MoveToLastPlace [tkPathName]}
\item {MoveWindow geometry [tkPathName]}
\item {OpenFileOrCD name geometry}
\item {OpenWindow fileName geometry doNotAsk}
\item {Option get optionName}
\item {Option set optionName optionValue}
\item {PrintStats}
\item {QuitPoint \{save$\mid$discard$\mid$ask\}}
\item {RaiseListWindow numInList [geometry]}
\item {RaiseWindow [tkPathName]}
\item {Redo [nToRedo] [tkPathName]}
\item {Redraw}
\item {RegexReplaceAll regexString replaceString [inselection]}
\item {RegexReplaceOne regexString replaceString [tkPathName]}
\item {RegexSearch regexToSearchFor [\{forward$\mid$backward\}] [tkPathName] [update]}
\item {RemakeTextWindows}
\item {RemakeBrowserWindows}
\item {RepeatSearch \{forward$\mid$backward\} [tkPathName]}
\item {RepeatSearch \{forward$\mid$backward\} [tkPathName]}
\item {Replace searchString replaceString [inselection] [tkPathName]}
\item {RunProgramInFile delsel}
\item {RunProgramInWindow prog}
\item {SaveAllFiles}
\item {SaveAs [tkPathName]}
\item {SaveFile [tkPathName]}
\item {ScrollWindow \{up$\mid$down\} \{numberOfLines$\mid$page\} [tkPathName]}
\item {Search stringToSearchFor [\{forward$\mid$backward\}] [tkPathName] [update]}
\item {SearchCharacter char [tkPathName]}
\item {Sel \{set$\mid$get$\mid$return\} selBegin selEnd [tkPathName] [\{char$\mid$word$\mid$line\}]}
\item {SendOnClose tkPathName interpName command}
\item {SetLineNumbers how}
\item {SetTextColor colorName \{normal$\mid$selected\} \{foreground$\mid$background\} }
\item {ShowSelection}
\item {ShowUndoStack}
\item {TextFont fontName [tkPathname]}
\item {ToggleReadOnly [tkPathname]}
\item {Undo [nToUndo] [tkPathName]}
\item {VScroll \{press$\mid$release$\mid$motion\} newY button}
\item {WaitForProcess pid}
\item {WaitForReturnString}
\item {WindowName set \{active$\mid$selected\} tkPathName}
\item {winfo children .}
\item {Zoom [\{vertical$\mid$full\}] [tkPathName]}

\end{description}




\end{document}

