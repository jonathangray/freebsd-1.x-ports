\section{Point Commands} \label{sect:commands}

In this section I will describe each of the commands that Point understands.
Each command entry begins with the command name and then gives a description
of the effect of the command.
Some commands have arguments which are given after the command name.
All arguments are strings.

Commands that insert text insert it at the insertion point.
Commands that operate on the selection operate on the selection window.
Other commands operate in the active window (the last window the
mouse entered) unless the optional ``tkPathName'' argument is present.
If it is present then it is the Tk path name for the window
which will be affected by the command.

Some of these commands are native Point commands and some of them
are implemented with tcl procedures.




\subsection{Insertion Commands}

\subsubsection{InsertAscii numericChar}
This command allows you to insert any eight bit character into the file.
The argument ``numericChar'' can specify the character as a decimal integer.

\subsubsection{InsertFromScrap}
The text in the scrap buffer is inserted at the insertion point.
The contents of the scrap buffer are unchanged and can be inserted again
in the same or a different place.
The other commands that change the scrap buffer are:
DeleteToScrap, ExchangeWithScrap and CopySelToScrap.

{\bf Usage hint:} {\it  to duplicate a line (or other section of text)
a number of times: select and delete the lines and then
insert them as many times as required.}

\subsubsection{InsertString string}
The specified string is inserted at the insertion point.

\subsubsection{InsertSelectedString}
The X selection is inserted at the insertion point.

\subsubsection{InsertFile}
The selection is used as a file name.
That file replaces the selection in the window.




\subsection{ Selection Editing Commands}

\subsubsection{ChangeCaseOfSel \{toupper$\mid$tolower$\mid$toggle\}}
This command changes the case
of each letter in the selection.
Non-letter characters are unaffected.
The argument determines how the case is changed:
to upper case, to lower case or change case
(lower goes to upper and upper goes to lower).
If the argument is missing or empty it defaults to ``toggle''.

\subsubsection{DeleteToScrap}
The selected text is deleted and placed in the scrap buffer.
The previous contents of the scrap buffer are lost.

\subsubsection{ExchangeWithScrap}
The contents of the scrap buffer and the selection are exchanged.

{\bf Usage hint:} {\it to exchange two pieces of text:
select one piece of text
and invoke CopySelToScrap, then select the other piece of text and
invoke ExchangeWithScrap, finally select the first piece of text
and invoke ExchangeWithScrap again.}

\subsubsection{DeleteLine [all]}
Deletes from the beginning of the selection to the end
of the line.
If {\tt all} is 1 (true) the the entire line is deleted.

\subsubsection{IndentSelection [outdent]}
A tab is inserted at the beginning of each line that contains
at least one selected character.
If {\tt outdent} is 1 (true) then the first character
of each line is deleted.
Presumably this is a tab character.

\subsubsection{Filter [cmd]}
The selection is provided to {\tt cmd} as its standard
input and the standard output of {\tt cmd} replaces
the selection.
The default {\tt cmd} is {\tt fmt}.

\subsubsection{JustifySel}
This command acts on whole lines.
It acts on all lines in which one or more characters
in the line are selected.
All of the lines are justified between column 1 and the column
specified by the {\bf rightMargin} option variable.

{\bf Usage hint:} {\it To indent and justify, first justify to column 1 and
then shift all the lines to the right.}




\subsection{Move and Copy Commands}

\subsubsection{CopySelToScrap}
Copies the selection to the scrap buffer.
The selection is unaffected.
The previous contents of the scrap buffer are lost.

{\bf Usage hint:} {\it to copy the selected text somewhere:
invoke CopySelToScrap, move to the insertion location and
select it and then invoke InsertFromScrap.}

\subsubsection{CopySelToMouse}
The selection is copied to the location of the mouse pointer
when the command was executed.
The copy is by characters, words or lines depending on the mode
of the selection.
If the selection mode is by words or lines the insert is at the
beginning of the word or line that contains the location
of the mouse pointer.

{\bf Usage hint:} {\it to copy a line (or lines) of text:
select the lines in line mode (with a triple click)
and invoke the CopySelToMouse command anywhere on the line {\bf after}
you want the lines copied to.}

{\bf Note:} This command is only suitable to attach to a mouse
button or a mouse menu.

\subsubsection{MoveSelToMouse}
The selection is moved to the current location of the mouse pointer.
The move is by characters, words or lines depending on the mode
of the selection.
If the selection mode is by words or lines the insert is at the
beginning of the word or line that contains the location
of the mouse pointer.

{\bf Usage hint:} {\it to move a line (or lines) of text:
select the lines in line mode (with a triple click)
and invoke the MoveSelToMouse command anywhere on the line {\bf after}
you want the lines moved to.}

{\bf Note:} This command is only suitable to attach to a mouse
button or a mouse menu.

\subsubsection{CopyToHereMode}
This command must be invoked twice to complete a copy.
The first time you invoke the {\tt CopyFromHereMode} command you set the
location to which the text will be copied.
You are then in {\em duplicate} mode and the mouse sprite becomes
a hand with a right pointing hand (this can be changed
to any cursor font cursor with the {\tt copySpriteName} option).
You can then continue using Point and execute any commands you like.
When you invoke the {\tt CopyFromHereMode} command while in duplicate mode
the current selection will be copied to the location remembered
from the first invocation of {\tt CopyFromHereMode}.

{\bf Usage hint:} {\it the advantage of this command is that you can copy
text into your typing stream without losing your insertion point.
Suppose you are typing a line in C and want to type a variable
name that appears on a nearby line.
First invoke the {\tt CopyFromHereMode} command, then select the nearby word
(with a double click of the left mouse button) and then invoke
{\tt CopyFromHereMode} again.
The word is copied into the line and the insertion point is at the
end of the copied word so you can continue typing the line without
readjusting the insertion point.
You can create a line by grabbing pieces of nearby lines one after
the other.
A new line of code in an existing routine can usually be created without
the keyboard at all.}

My rule about copying nearby words is that the break-even point for
effort is around five or six characters, that is, if the variable name
you are about to type is longer than that it is probably easier to
copy it than to retype it.
Another advantage of copying is that there is no chance you will mistype
the word.

If the word you want to copy is not visible on the screen it is usually
not worth the effort to copy it because that would require window
rearranging.
It may be worth it if it is a global variable with a long name
(and global variables should have long descriptive names)
that would be hard to type correctly.
And since you will have to type in an external reference to the
variable anyway you might as well find a copy of the external reference,
copy it in as a complete line and then copy the global variable name
from the copied in line.

\subsubsection{MoveFromHereMode}
This is identical to the {\tt CopyFromHereMode} command except that the text
is moved (that is, deleted from its original location) when the
second {\tt MoveFromHereMode} is invoked.





\subsection{Again and Undo Commands}

\subsubsection{Again [tkPathName$\mid$mostrecent$\mid$thisfile]}
The last command (not counting deletes) is repeated in the current context.
If the last command copied text into the insertion point then that same
text is copied into the current insertion point.
If the last two commands were to delete the current selection and replace
it with new text (either typed in or copied in from somewhere else) then
this command will replace the current selection with that same text.

The optional argument determines which command is repeated.
The default is {\tt mostrecent} which means that the most
recent edit is repeated.
If the argument is {\tt thisfile} then the most recent edit
in the file containing the selection is repeated.
Finally if a window name is given the most recent edit in the
file in that window is repeated.

{\bf Usage hint:} {\it to perform a selective replace, that is, to replace
some but not all instances of one text string with another text string:
search for the string using a SearchForString selection box until you
find the first instance you want to replace.
Then search for the following instances of the string
(either with the dialogue box or with the RepeatSearch command).
For all the ones you want to replace just invoke Again.}

\subsubsection{Undo [nToUndo] [tkPathName]}
The last undone command is undone:
deleted text is reinserted, copied text is deleted,
moved text is put back where it was moved from.
If this command is repeated the undo itself is undone
so only one previous command can be undone with UndoOne.
If this command is repeated the command before the last command
is undone and so on.
The argument {\tt nToUndo} (which defaults to 1) determines how
many edits will be undone.

\subsubsection{Redo [nToRedo] [tkPathName]}
The last undone command that has not been redone is redone.
This command can be used only to redo commands that have been undone.
The argument {\tt nToRedo} (which defaults to 1) determines how
many edits will be redone.




\subsection{Search and Replace Commands}

\subsubsection{Search stringToSearchFor
		[\{forward$\mid$backward\}] [tkPathName] [update]}
The search is performed in the active window
(not the selection window), that is, the
last window the mouse sprite was in unless the third
argument is given, in which case it names the window to search in.
If the selection is in that window then
the search begins at the selection, otherwise the search begins
at the first character of the file.
If the fourth argument is missing, empty or is the string ``update''
then the window will be updated after the string is found.
Otherwise it will not be updated.
The search direction is either forward or backward.
There are several options that affect how the search is done:
the {\tt ignoreCase} option determines whether case is significant
in the search, the {\tt findWholeWords} option determines whether
the string can be a substring of a larger string or not.
The {\tt linesOverFind} option affects the positioning of the found string
in the window if the window has to be jumped in order to show the
found string.
If the second argument is missing or empty then ``forward''
is assumed.
If the third argument is given the second argument must appear also
(although it can be the empty string).

\subsubsection{RepeatSearch \{forward$\mid$backward\} [tkPathName]}
The last string searched for is used as the search string
for a new search.
The search direction is either forward or backward.
If the first argument is missing or empty then ``forward''
is assumed.
If the second argument is given the first argument must appear also
(although it can be the empty string).

\subsubsection{RegexSearch regexToSearchFor
		[\{forward$\mid$backward\}] [tkPathName] [update]}
This command searches for a regular expression in the text.
It is identical to the ``Search'' command given above except that
it is a regular expression search instead of a fixed text search and
if {\tt regexToSearchFor} is empty then the last regular expression
that was searched for will be used again.

{\bf Usage hint:} {\it Normally you want a fixed text search.
Then you do not have to worry about escaping special characters
and it is a bit faster as well.
Just use the regular expression search when you really want to
search for a pattern rather than a fixed string.}

\subsubsection{RepeatSearch \{forward$\mid$backward\} [tkPathName]}
The last string searched for is used as the search string
for a new search.
The search direction is either forward or backward.
If the first argument is missing or empty then ``forward''
is assumed.
If the second argument is given the first argument must appear also
(although it can be the empty string).

\subsubsection{CTag ctag}
The {\tt ctag} is looked up in the file {\tt tags}
(created by the Unix command {\tt ctags}) in the current directory.
the file containing the flag is loaded into a new window
or, if the file is already in a window, that window is raised
to the top.
The window is jumped to show the {\tt ctag} string and the {\tt ctag} string
is selected.

\subsubsection{SearchCharacter char [tkPathName]}
If the `char' is an ASCII character then it is appended
to the search string maintained by this command and the
string is searched for (forwards in the file).
Thus the command implements incremental search.
If `char' is a non-ASCII character (e.g., a function key)
then the search string is cleared.
This search string is different from the search string for the
search command.

{\bf Usage hint:} {\it I bind this to the menu bar item
for string search.
Then I can type characters into that menu item and get incremental
search.
It never throws away characters unless you clear the search string
so get in the habit of pressing a function key before starting
your search string.}

\subsubsection{Replace searchString replaceString [inselection] [tkPathName]}
This command replaces each occurrence of the ``searchString'' with the
``replaceString'' either in the entire file (if the third argument is
missing or empty) or within the selection (if the third argument is
``inselection'').
The search is affected by the normal search options
({\tt ignoreCase} and {\tt findWholeWords}).
The replace is done in the active window unless the fourth
argument names a valid text window.

\subsubsection{RegexReplaceOne regexString replaceString [tkPathName]}
This command finds the next match of the regular expression
``regexString'' and replaces it with ``replaceString''.

\subsubsection{RegexReplaceAll regexString replaceString [inselection]}
This command finds all matches of the regular expression
``regexString'' and replaces them with ``replaceString''.

\subsubsection{FindMatchingBracket}
The following brackets are matched in pairs:
( and ), \{ and \}, [ and ].
If one of these characters is the first character of the selection,
Point will search (forward for (, \{ and [; backward for ), \} and ])
for the matching bracket.
By matching, we mean that it keeps a count of left and right brackets
and finds the matching bracket when count is zero.





\subsection{Scrolling and Jumping Commands}

\subsubsection{MoveToEndFile [tkPathName]}
The window is repositioned so that it shows the end of the file.
The file will be positioned so that the end of file marker is
on the last line of the window (unless the entire file will
fit in fewer lines).

\subsubsection{ShowSelection}
The window containing the selection is brought to the front and
repositioned so that the beginning of the selection is visible
in the window.
The {\tt linesOverFind} option determines the placement of the first
line of the selection in the window.

\subsubsection{MoveToLastPlace [tkPathName]}
This repositions the window to the last place you have jumped from.
Point remembers a separate last place for each window.

{\bf Usage hint:} {\it this command is useful for switching between
two different places in the file.}

\subsubsection{GotoLine lineNumber \{lof$\mid$top\} [tkPathName]}
The window is repositioned so that the line number specified
is show in the window.
If `lof' is specified then the {\tt linesOverFind} option
determines where the line is positioned in the window.
If `top' is specified then the line number is positioned 
at the top of the window.
See the description of that option for details.
If the second argument is missing or empty then ``lof''
is assumed.

\subsubsection{GotoDigit digit [tkPathName]}
The `digit' is added to a line number that is collected.
If `digit' is not a digit, then the line number collected
so far is jumped to.
The {\tt linesOverFind} option determines the placement
of the line gone to in the window.

\subsubsection{ScrollWindow \{up$\mid$down\} \{numberOfLines$\mid$page\}
		[tkPathName]}
The window is scrolled by `numberOfLines'.
If the second argument is ``page'' or missing or empty,
then the window is scrolled down the number of lines in
the window minus two (for context).
The first argument determines the scroll direction.
If it is missing or empty then ``down'' is assumed.




\subsection{Cursor Positioning Command}

\subsubsection{MoveSel \{char$\mid$word$\mid$line\}
		\{up$\mid$down$\mid$right$\mid$left$\mid$left0\}}
This command moves the selection.
The movement can be by character, by word or to the limits of the line.
The second argument determines the direction of movement.
You can only use ``up'' and ``down'' with ``char''.
For ``line'', ``right'' moves the selection to the last character
of the line (which will always be the newline character
that ends the line), ``left'' moves to the first non-white space
character of the line and ``left0'' moves to the first
character of the line.
These commands are usually attached to keys
or used in macros (like indent selected lines).





\subsection{Window Creation Commands}

\subsubsection{OpenWindow fileName geometry doNotAsk}
A new window is created and 'fileName' is loaded into the new window.
The `geometry' argument specifies the geometry of the new window.
If `geometry' is missing or empty then the default
{\tt textGeometry} is used.
If {\tt fileName} does not exist then the user will be asked
for verification before it is created unless the {\tt doNotAsk}
argument is present and is literally the string ``doNotAsk''.
This command returns a string that is the Tk path name
of this window.
It will always be of the form ``.twDDDDD'' where each D
is a decimal digit.
Point assigns window names with a counter so the first window
will be ``00001'', the second window ``00002'' and so on.
Text windows, browser windows, and some popup dialogue boxes
all share the same counter so the numbers of the text windows
will not necessarily be sequential.

\subsubsection{CloseWindow \{save$\mid$nosave$\mid$ask\} [tkPathName]}
The window is closed and removed from the screen.
If the file in the window has been edited and this is the last window
open on the file then the argument determines whether the
file will be saved or not.
If `save' is specified it will be saved, if 'nosave' is specified
it will not be saved and if 'ask' is specified a dialogue box
will pop up asking the user whether to save the file or not.
If the first argument is missing or empty then ``ask'' is assumed.

\subsubsection{Browser geometry [tkPathName]}
This command creates another file browser.
The `geometry' argument determines the geometry of the new browser.
If it is missing or empty then the default {\tt browserGeometry} will be used.
If the second argument is missing or empty then ``big'' is assumed.

{\bf Usage hint:} Use this command to get a file from another directory
while keeping a file browser showing the current directory.

\subsubsection{CloseBrowser [tkPathName]}
The specified browser (the active browser is the default) is closed.




\subsection{File Writing Commands}


\subsubsection{SaveFile [tkPathName]}
The file in the active window (or the window named by {\tt tkPathName})
is saved on disk.

\subsubsection{SaveAs [tkPathName]}
The file in the active window (or the window named by {\tt tkPathName})
is written to disk.
The name to use for the file is provided by the user through a dialogue box.

\subsubsection{SaveAllFiles}
All files that have been changed but not yet saved are saved.





\subsection{Window Size and Location Commands}

\subsubsection{LowerWindow [tkPathName]}
Move the active window (or the window named by {\tt tkPathName})
behind all other windows in the display
(both Point windows and the windows of other X applications).

\subsubsection{MoveWindow geometry [tkPathName]}
The window is moved and resized according to `geometry'.

\subsubsection{RaiseListWindow numInList [geometry]}
This command affects the window which is the `numInList'th
item on the open window list.
That window is raised and resized according to `geometry'
(which defaults to {\tt textGeometry}).

\subsubsection{RaiseWindow [tkPathName]}
Bring the active window (or the window named by {\tt tkPathName})
to the front of all other windows in the display
(both Point windows and the windows of other X applications).

\subsubsection{Zoom [\{vertical$\mid$full\}] [tkPathName]}
The window is enlarged to be the full height of the screen
(if the first argument is {\tt vertical}
or the full size of the screen (if the first argument is {\tt full}.
If the first argument is missing {\tt vertical} is assumed.





\subsection{Font Commands}

\subsubsection{BrowserFont fontName [tkPathname]}
This command changes the font
(in the specified browser where the active browser is the default)
to `fontName'.
If the font change fails the font in the active browser
is changed to the default font {\tt browserFont}.

\subsubsection{TextFont fontName [tkPathname]}
This command changes the font
(in the specified window where the active window is the default)
to `fontName'.
If the font change fails the font in the active window
is changed to the default font {\tt textFont}.





\subsection{ Selection and Mode Commands }

\subsubsection{CancelModes}
This command cancels duplicate or extract mode
if either in in effect.
Otherwise the command has no effect.
This command is used when you inadvertently enter
duplicate and extract mode with a erroneous button
click or keypress.
The mouse sprite will change back to normal
(from the right pointing hand that marks duplicate mode).

\subsubsection{ExtendSelToLines}
The selection is extended so that it consists of only complete lines.

\subsubsection{Sel \{set$\mid$get$\mid$return\} selBegin selEnd [tkPathName]
			[\{char$\mid$word$\mid$line\}]}
If the first argument is ``get'' then there are no other arguments
and the command returns a Tcl list of four strings.
The first two strings are decimal numbers and are the character positions
of the first and last character in the selection.
The third string is the Tk path name of the window containing
the selection.
The fourth string is the selection mode and is one of the
string {\tt char}, {\tt word} or {\tt line}.
If the first argument is ``set'' then the selection is changed
so that it begins at character position ``selBegin'' and ends
at character position ``selEnd''.
If the third argument is present the selection window is changed to
the window specified by the Tk path name.
If the fourth argument is present is changes the
selection mode.
If the first argument is ``return'' then the contents of the
selection are returned.

\subsubsection{SetLineNumbers how}
This command determines whether line numbers are displayed
with the text in the window.
Each window has a line number flag and this command changes it.
If {\tt how} is "1" then line numbers are displayed.
If {\tt how} is "0" then line numbers are not displayed.
If {\tt how} is anything else then the line numbering flag is toggled.

\subsubsection{SetTextColor colorName \{normal$\mid$selected\}
     \{foreground$\mid$background\} }
This command sets the text color in the active window.
You can use it to set either the foreground or the background
of either normal text or selected text.
If the second argument is missing or empty then ``normal''
is assumed.
If the third argument is missing or empty then ``foreground''
is assumed.

\subsubsection{ToggleReadOnly [tkPathname]}
A file can be specified ''read-only'', in which case,
Point will not allow you to modify it.
This command toggles the read-only status of the file
in the active window (or the window named by {\tt tkPathname}).








\subsection{ Subprocess Commands }

\subsubsection{ConnectToPty arg1 arg2 arg3 arg4 arg5 arg6}
Create a Unix pseudo-tty.
Execute the command {\tt arg1} with arguments {\tt arg2} through
{\tt arg6} as a subprocess.
The output of the subprocess is inserted into the window
containing the selection when this command is executed.
Input into that window is sent to the subprocess over
the pseudo-tty.

\subsubsection{WaitForProcess pid}
Delay until subprocess {\tt pid} finishes.

\subsubsection{RunProgramInFile delsel}
The selection is used as the name of a program to execute
in a subprocess and the window containing the selection is connected
to the subprocess via a Unix pseudo-tty.
If {\tt delsel} is 1 (true) then the selection is deleted
after it is used.

\subsubsection{RunProgramInWindow prog}
{\tt prog} is used as the name of a program to execute
in a subprocess.
A window is created and connected to the subprocess via a Unix pseudo-tty.





\subsection{ Information and Status Commands }

\subsubsection{MessageLine message}
The text string {\tt message} is displayed on the message line of
each text window.
This command can be used from tcl/tk macros to display messages
to the user.

\subsubsection{PrintStats}
Information about the effectiveness of Point's caches is
printed out to the {\tt xterm} window in which Point was started.

\subsubsection{ShowUndoStack}
This command pops up a dialogue box that shows the command
history.
For each command it gives information about the command
including whether it has been undone (and not redone) or not.

\subsubsection{GetTextWindowList}
This command returns a list of the Tk name of all existing text windows.

\subsubsection{GetBrowserList}
This command returns a list of the Tk name of all existing browsers.

\subsubsection{GetRowCol offset [tkPathName]}
This command returns a Tcl list of two integers.
These are the row and column in the window of the
character in character position ``offset''.

\subsubsection{WindowName set \{active$\mid$selected\} tkPathName}
Make the specified window (in {\tt tkPathName}) either the
selection window or the active window.

\subsubsection{WindowName get \{active$\mid$selected\}}
Return the Tk path name of the selection window or the active window.

\subsubsection{GetWindowInfo [tkPathName]}
This command returns a Tcl list of information about the
window whose Tk path name is given by ``tkPathName''
(the default is the active window).
If the argument is omitted then the active window is assumed.
This list consists of (in order):
the character position of the first character displayed in the window,
the character position after the last character displayed in the window,
the line number of the first line displayed in the window,
the line number of the last line displayed in the window plus one,
the number of columns the window is scrolled horizontally,
the number of lines displayed in the window,
the number of columns displayed in the window,
the X window id of the window,
the Tk window id of the toplevel window,
the Tk window id of the text window.

\subsubsection{GetFileChars firstByte lastByte [tkPathName]}
Return the characters from the file in the specified window.
If no window is given the active window is used.

\subsubsection{GetFileInfo [tkPathName]}
This command returns a Tcl list of information about the
file displayed in the window whose Tk path name is given by ``tkPathName''.
If the argument is omitted then the active window is assumed.
This list consists of (in order):
the name of the file displayed in the window,
the current size of the file in bytes,
the original size of the file when it was loaded
(before any editing was done),
a flags field which is the {\it or}ing together
of the flags:
0x1 if a backup file has been made,
0x2 if the file is read only,
0x4 if the file has been changed (and not yet saved),
and 0x8 if the file cannot be changed to read/write.

\subsubsection{winfo children .}
This is really a straight Tk command.
It returns the path names of all the children of the root window.
For Point this will be all the toplevel windows.
All the text windows will have names of the form ``.twDDDDD''.
All browser windows will have names of the form ``.bwDDDDD''.
Here each ``D'' is a decimal digit.
Popup dialogues will have various other names
but none of these names will start with ``.bw'' or ``.tw''.
This command will provide you with a list of all of Point's
text windows.
You can use other commands to get information about each
text window (and the file it displays).




\subsection{ Option Commands }

\subsubsection{Option get optionName}
A string is returned that is the value of the specified
Point option.
Integers are returned as decimal strings.
Booleans are returned as 0 (False) or 1 (True).

\subsubsection{Option set optionName optionValue}
The specified Point option is set to ``optionValue''.
Integer arguments can be in octal, decimal or hex.
Boolean options can be 0, 1, true or false.
Case is not considered in the strings ``true'' and ``false''.





\subsection{Miscellaneous Commands}

\subsubsection{QuitPoint \{save$\mid$discard$\mid$ask\}}
Quit Point after checking each file to see if it has been changed
but not yet saved.
The argument determines whether these files will be saved.
If the argument is missing then ``ask'' is assumed.

\subsubsection{Redraw}
This command redraws the entire window so that it corresponds to
the true state of the file.

\subsubsection{CD directoryName [tkPathname]}
The Unix current directory of the specified browser
(the default is the active browser) is changed to `directoryName'.
If the first argument is missing or empty then the user's home directory
is assumed.

\subsubsection{SendOnClose tkPathName interpName command}
This command is intended to be used by a process that
is sending commands to Point via the Tcl ``send'' command.
The ``tkPathName'' is the Tk window name that Point
assigns to the window (and that is returned by the
OpenWindow command).
The ``interpName'' is the name of the Tcl interpreter to send
``command tkPathName'' to when that window is closed.
That is, when window ``tkPathName'' is closed,
Point will execute the Tcl command:
\begin{verbatim}
     send interpName command tkPathName
\end{verbatim}

\subsubsection{DoNothing}
This command does nothing.
It is a placeholder command if you have to specify
a command but do not want it to actually do anything.

\subsubsection{WaitForReturnString}
This command runs a local Tk event loop and does not return
until the Point option {\tt returnString} is non-empty.
This is used to create a modal dialogue box.
You set up the dialogue box so that it sets returnString to
a non-empty string when it has its input.
Then you pop up the dialogue box and execute this command.
When it returns the input will be available.
As an example look at {\tt proc MakeVerifyBox} and
{\tt proc DoReplace} in {\tt ptsetup.tcl}

\subsubsection{RemakeTextWindows}
The menu bar of each text window is destroyed and recreated.
This is used after the menu bar specification has been changed
and you want the menus to reflect the new specification.

\subsubsection{RemakeBrowserWindows}
The menu bar of each browser window is destroyed and recreated.
This is used after the menu bar specification has been changed
and you want the menus to reflect the new specification.

Change the cursor of all the text and browser windows.
The new cursor can be given by name (it must be in the X font cursor
font) or it can be one of the Point cursors (busy, current, main or dup).





\subsection{ Event Callback Commands }

The commands in this section are used by event handlers to get
Point to do certain things.

\subsubsection{VScroll \{press$\mid$release$\mid$motion\} newY button}
This command is used by the scrollbar to scroll a window.

\subsubsection{HScroll \{press$\mid$release$\mid$motion\} newY button}
This command is used by the scrollbar to scroll a window.

\subsubsection{Key keysym shiftState}
Tells Point that a certain key was pressed.

\subsubsection{Mouse tkPathName command x y clicks}
Tells Point about a mouse event.
Command must be one of: BeginSelection, BeginExtend, ExtendSelection,
EndExtending, BeginMouseMenu1, BeginMouseMenu2, ContinueMouseMenu,
CancelMouseMenu or EndMouseMenu.

\subsubsection{Configure tkPathName}
Tells point that a configure event occurred.

\subsubsection{EnterText tkPathName}
Tells point that the mouse cursor entered a text window.

\subsubsection{EnterBrowser tkPathName}
Tells point that the mouse cursor entered a browser.

\subsubsection{Expose tkPathName x y width height count}
Tells point that an expose event occurred.





\subsection{ Dialogue Box Commands }

\subsubsection{MakeAsciiBox}
This command allows you to insert any eight bit character into the file.
A dialogue box comes up and you can enter the character as a decimal
(starting with [1-9]) integer, an octal (starting with 0) integer or
a hexadecimal (starting with 0x) integer.

\subsubsection{MakeAboutBox}
This command brings up a box that gives information
about Point.
Clicking on the dialogue box closes it.

\subsubsection{MakeMsgBox msg}
Pops up a message box displaying the message in ``msg''.

\subsubsection{MakemmBox}
Pops up a dialogue box that allows you to set the parameters
of the mouse menus, that is, the text and command for each
direction of the two mouse menus.

\subsubsection{MakeColorBox}
Pops up a dialogue box that allows you to select the color
or the foreground and background of normal and selected text.
You select the colors from a scrolling list of of color names from
the X color database.

\subsubsection{MakeCtagBox}
Pops up a dialogue box that allows you to search for a C tag.

\subsubsection{MakeKeywordBox}
Pops up a dialogue box that allows you to search for a keyword.

\subsubsection{FillKeywordBox}
Starts a search for a keyword based on the keyword in
the keyword box and fills the list with the file names
containing that keyword.

\subsubsection{MakeGotoBox}
Pops up a dialogue box that allows you to jump to a line number.

\subsubsection{MakeDebugBox}
Pops up a dialogue box that allows you to change the value
of the debug variable.

\subsubsection{MakeQuitBox}
Pops up a dialogue box that allows you to choose quit options.

\subsubsection{MakeReplaceBox}
The {\tt MakeReplaceBox} command brings up a dialogue box
that allows you to specify
the string to search for and the string to replace it with.
There are three checkbox options you can set.
The first is to verify each replacement.
If this is turned on then each time the search string is found
the window is made to show the string found, the string is
selected and you
are presented with a dialogue box asking whether you want
to replace the string or not.
If you do not choose the verify option you might want to use the
internal (C) version of search and replace rather than the external
(Tcl) version which is considerably slower.
The final checkbox option is to restrict the replace to the selection
only.
This is useful, for example, for changing one identifier to another
over a few lines of code.
The search uses the standard search options (ignore case, find
whole words, etc.)

\subsubsection{MakeSearchBox}
Pops up a dialogue box that allows you to enter a search
string, change the {\tt findWholeWords}
and {\tt ignoreCase} search options
and set the search direction as backward or forward.

\subsubsection{OpenFileOrCD name geometry}
If ``name'' is a directory the active browser
is changed to that directory.
If ``name'' is an ordinary file then a window is opened
up with geometry ``geometry'' that displays file ``name''.

\subsubsection{MakeSearchOptionsBox}
Pops up a dialogue box that allows the user to change
all the search related options.

\subsubsection{MakeOtherOptionsBox}
Pops up a dialogue box that allows the user to change
several string valued options.







\subsection{ Point Commands --- Alphabetical List }

This is a list of Point commands and tcl procs that are defined
in the tcl code distributed with Point.

\begin{description}

\item {Again [tkPathName$\mid$mostrecent$\mid$thisfile]}
\item {Browser geometry \{big$\mid$small\} [tkPathName]}
\item {BrowserFont fontName [tkPathname]}
\item {CD directoryName [tkPathname]}
\item {CTag ctag}
\item {CancelModes}
\item {ChangeCursor
	\{busy$\mid$current$\mid$main$\mid$dup$\mid$FontCursorName\}}
\item {CloseBrowser [tkPathName]}
\item {CloseWindow \{save$\mid$nosave$\mid$ask\} [tkPathName]}
\item {Configure tkPathName}
\item {ConnectToPty arg1 arg2 arg3 arg4 arg5 arg6}
\item {CopySelToMouse}
\item {CopySelToScrap}
\item {CopyToHereMode}
\item {DeleteLine [all]}
\item {ChangeCaseOfSel \{toupper$\mid$tolower$\mid$toggle\}}
\item {DeleteToScrap}
\item {DoNothing}
\item {EnterBrowser tkPathName}
\item {EnterText tkPathName}
\item {ExchangeWithScrap}
\item {Expose tkPathName x y width height count}
\item {ExtendSelToLines}
\item {FillKeywordBox}
\item {Filter [cmd]}
\item {FindMatchingBracket}
\item {GetBrowserList}
\item {GetFileChars firstByte lastByte [tkPathName]}
\item {GetFileInfo [tkPathName]}
\item {GetRowCol offset [tkPathName]}
\item {GetTextWindowList}
\item {GetWindowInfo [tkPathName]}
\item {GotoDigit digit [tkPathName]}
\item {GotoLine lineNumber \{lof$\mid$top\} [tkPathName]}
\item {HScroll \{press$\mid$release$\mid$motion\} newY button}
\item {IndentSelection [outdent]}
\item {InsertFile}
\item {InsertFromScrap}
\item {InsertSelectedString}
\item {InsertString string}
\item {JustifySel}
\item {Key keysym shiftState}
\item {LowerWindow [tkPathName]}
\item {MakeAboutBox}
\item {MakeAsciiBox}
\item {MakeColorBox}
\item {MakeCtagBox}
\item {MakeDebugBox}
\item {MakeGotoBox}
\item {MakeKeywordBox}
\item {MakeMsgBox msg}
\item {MakeOtherOptionsBox}
\item {MakeQuitBox}
\item {MakeReplaceBox}
\item {MakeSearchBox}
\item {MakeSearchOptionsBox}
\item {MakemmBox}
\item {Mouse tkPathName command x y clicks}
\item {MoveFromHereMode}
\item {MoveSel \{char$\mid$word$\mid$line\} \{up$\mid$down$\mid$right$\mid$left$\mid$left0\}}
\item {MoveSelToMouse}
\item {MoveToEndFile [tkPathName]}
\item {MoveToLastPlace [tkPathName]}
\item {MoveWindow geometry [tkPathName]}
\item {OpenFileOrCD name geometry}
\item {OpenWindow fileName geometry doNotAsk}
\item {Option get optionName}
\item {Option set optionName optionValue}
\item {PrintStats}
\item {QuitPoint \{save$\mid$discard$\mid$ask\}}
\item {RaiseListWindow numInList [geometry]}
\item {RaiseWindow [tkPathName]}
\item {Redo [nToRedo] [tkPathName]}
\item {Redraw}
\item {RegexReplaceAll regexString replaceString [inselection]}
\item {RegexReplaceOne regexString replaceString [tkPathName]}
\item {RegexSearch regexToSearchFor [\{forward$\mid$backward\}] [tkPathName] [update]}
\item {RemakeTextWindows}
\item {RemakeBrowserWindows}
\item {RepeatSearch \{forward$\mid$backward\} [tkPathName]}
\item {RepeatSearch \{forward$\mid$backward\} [tkPathName]}
\item {Replace searchString replaceString [inselection] [tkPathName]}
\item {RunProgramInFile delsel}
\item {RunProgramInWindow prog}
\item {SaveAllFiles}
\item {SaveAs [tkPathName]}
\item {SaveFile [tkPathName]}
\item {ScrollWindow \{up$\mid$down\} \{numberOfLines$\mid$page\} [tkPathName]}
\item {Search stringToSearchFor [\{forward$\mid$backward\}] [tkPathName] [update]}
\item {SearchCharacter char [tkPathName]}
\item {Sel \{set$\mid$get$\mid$return\} selBegin selEnd [tkPathName] [\{char$\mid$word$\mid$line\}]}
\item {SendOnClose tkPathName interpName command}
\item {SetLineNumbers how}
\item {SetTextColor colorName \{normal$\mid$selected\} \{foreground$\mid$background\} }
\item {ShowSelection}
\item {ShowUndoStack}
\item {TextFont fontName [tkPathname]}
\item {ToggleReadOnly [tkPathname]}
\item {Undo [nToUndo] [tkPathName]}
\item {VScroll \{press$\mid$release$\mid$motion\} newY button}
\item {WaitForProcess pid}
\item {WaitForReturnString}
\item {WindowName set \{active$\mid$selected\} tkPathName}
\item {winfo children .}
\item {Zoom [\{vertical$\mid$full\}] [tkPathName]}

\end{description}


