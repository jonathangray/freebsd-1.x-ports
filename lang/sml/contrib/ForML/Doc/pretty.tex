\newcommand{\bsl}{/}
\documentstyle{article}
% ForML Version 0.6 - 25 January 1993 - er@cs.cmu.edu
\oddsidemargin 0pt \evensidemargin 20pt \headheight 0pt
\headsep 0pt    %25pt
\topmargin 0pt  %27pt
\footheight 12pt \footskip 25pt \textheight 660pt \textwidth 469pt

\title{ForML - a Pretty-Printing Facility for SML}
\author{Ekkehard Rohwedder}
\date{25 January 1993}
\begin{document} 
\maketitle 
\begin{abstract}
This is the documentation accompanying Release 0.6
  of ForML, a prettyprinting
  module written in SML that supports the formatting of concrete syntax in
  languages or language-development tools.\\
\end{abstract}

\section{Introduction}

A number of languages and language-development tools (e.g.\ CAML, Ppml)
offer prettyprinting facilities to support the unparsing and formatting
of abstract syntax. 
ForML, a formatter written in SML of New Jersey, adresses some
of these issues for the SML language. 

After examining basic goals involved in the design of a pretty printer,
we will introduce the ForML formatting primitives and give examples of 
their use.
Next we  describe the entire ForML interface, and finally we give a more
involved example of ForML's use.

The ForML ideas are taken from the CAML pretty-printing
primitives~\cite{Weis88},
which in
turn are based on the Ppml pretty-printing facility of the Centaur
system~\cite{Ppml99}.
In contrast to the pretty printing algorithm by Oppen~\cite{Oppen80}, 
the ForML formatting is not based on a stream interface: an
intermediate formatting structure is generated in memory before the
final output is printed.


\subsection*{Unparsing and Formatting}
Lexer and parser transform input text into trees
containing the abstract syntax of the input language.
In language-development environments, however, one often wants to
take the reverse route producing nicely formatted output from some
abstract syntax trees.
It is advantageous to structure this process into two steps that are
interfaced by a formatting language:
\begin{itemize}
  \item Unparsing --- precedences and associativity
    in the abstract syntax tree are disabiguated (e.g.\ by
    inserting ``('' and ``)'' or other bracketing constructs),
    and the appropriate output functions for the subtrees are selected
    (allowing, for example, the elision of deeply nested structures).
  \item A device-independent formatting language interfaces
    the unparser and the formatter.
  \item Formatting --- the breakpoints in the output are determined, and
    the mapping to the output device(s) is performed.
\end{itemize}

ForML does {\em not\/} take you all of the way: the unparsing of abstract
syntax trees will have to be implemented by hand. However, ForML offers
a formatter-language interface and output routines for a
monospaced device.\footnote{Future versions will support other output
devices, like LaTeX or PostScript. Please contact the author at
{\tt er@cs.cmu.edu} if you need to produce output for other than a
monospaced device.}

Processing with ForML is efficient, allowing its use in an interactive
environment, and its formatting language can express common formating demands
directly and easily, while relying on perspicuous and usable concepts.
It has been successfully employed as the formatting tool for the
Elf-language~\cite{Pfenning91D}.

In the next section we describe the primitives provided by ForML to
build formats and how they affect the final output.
SML program text will be set in {\tt teletype font}, and we also assume that
the {\tt Formatter} structure has been opened, making its primitives directly
available.

\section{ForML Formats}

  \subsection{Boxes}

  A {\tt format} is either a string ({\tt String "}{\it a string}{\tt "}),
  a space {\tt Space}, a breakpoint {\tt Break},
  or a box containing a list of formats.
  Breakpoints indicate mandatory or potential locations where
  a new line (and possibly increased indentation) is to be
  inserted into the output.

  Boxes come in various formats, and ---depending upon the flavor of the
  box in which they are situated and upon the available printwidth---
  the {\tt Break}s inside of them are interpreted differently.
  To illustrate we will fill various boxes with the following list of formats
  \begin{center}
  {\tt [String "XXXX", Break, String "XXXX", Break, String "XXX",
        Break, String "XXXXX"]}
  \end{center}
  and observe the respective outputs\footnote{The suggestive frame drawn
    around the boxes is {\em not} part of the actual output.}.
  \begin{itemize}
   \item Horizontal boxes {\tt Hbox} {\em never} break at their breakpoints.
         {\tt Break}s are simply converted to whitespace.
    \begin{center}
      \begin{tabular}{|l|}
       \hline
        XXXX XXXX XXX XXXXX\\ 
       \hline
      \end{tabular}
    \end{center}
   \item Vertical boxes {\tt Vbox} {\em always} break at all their
     breakpoints. The {\tt Break}s are transformed into
     newlines, and a certain amount of indentation from the left border of
     the box is inserted.
    \begin{center}
      \begin{tabular}{|l|}
       \hline
        XXXX \\
        ~~XXXX\\
        ~~XXX\\
        ~~XXXXX\\ 
       \hline
      \end{tabular}
    \end{center}
    \item Horizontal-or-vertical boxes {\tt HOVbox}
      can behave as a horizontal
      box if the available pagewith accomodates the width of the
      box, otherwise they behave exactly like vertical boxes.
    \begin{center}
      \begin{tabular}{|l|}
       \hline
        XXXX XXXX XXX XXXXX\\ 
       \hline
      \end{tabular}
      ~~~or~~~
      \begin{tabular}{|l|}
       \hline
        XXXX \\
        ~~XXXX\\
        ~~XXX\\
        ~~XXXXX\\ 
       \hline
      \end{tabular}
    \end{center}
   \item Horizontal-vertical boxes {\tt HVbox} allow each
     breakpoint {\em individually} to behave as if it was in a horizontal
     box or in a vertical box, depending upon whether the material
     after the break (and up to the next break) would still fit into
     the available pagewidth.
     Thus ---depending on the page width--- the above example could be
     displayed quite differently:
    \begin{center}
      \begin{tabular}{|l|}
       \hline
        XXXX XXXX XXX XXXXX\\ 
       \hline
      \end{tabular}
      ~~~or~~~
      \begin{tabular}{|l|}
       \hline
        XXXX XXXX XXX\\
        ~~XXXXX\\
       \hline
      \end{tabular}
    \end{center}
    \begin{center}
      or~~~
      \begin{tabular}{|l|}
       \hline
        XXXX XXXX\\
        ~~XXX XXXXX\\
       \hline
      \end{tabular}
      ~~~or~~~
      \begin{tabular}{|l|}
       \hline
        XXXX \\
        ~~XXXX\\
        ~~XXX\\
        ~~XXXXX\\ 
       \hline
      \end{tabular}
    \end{center}
  \end{itemize}
  
\subsection{Example: Pretty-Printer for $\lambda$-Calculus}
Let us delve into a concrete example and put these primitives to work by
writing a formatter for a $\lambda$-calculus language
presented with the following concrete grammar:
\begin{center}
\begin{tabular}{lcl}
$exp$ & $::=$ & $\lambda x. exp ~|~ let ~x = exp ~in~ exp ~|~ exp_A$ \\
$exp_A$ & $::=$ & $exp_A~exp_B ~|~ exp_B  $ \\
$exp_B$ & $::=$ & $var ~|~ {\mbox{\rm ``(''}} exp {\mbox{\rm ``)''}}$ \\
\end{tabular}
\end{center}
An SML datatype {\tt exp} which captures the abstract syntax 
of such $\lambda$-expressions might look as follows:
\begin{verbatim}
datatype exp =
         Var of string
      |  Lam of string * exp
      |  App of exp * exp
      |  Let of string * exp * exp
\end{verbatim}

\subsubsection{A first cut}
Let us first design the ``boxes'' for the different language elements:
\begin{itemize}
 \item variable names will be output directly 
 \item applications $e_1 ~ e_2$ may be split if they do not fit into a
       line:
\begin{center}
       \begin{tabular}{|l|} \hline
          $expression_a$ \\
          $~~~~~~expression_b$ \\ \hline
       \end{tabular}
\end{center}
 \item no breaks are allowed immediately after ``('' or immediately before ``)``
 \item $\lambda$-abstractions allow a break after the ``.'':
\begin{center}
       \begin{tabular}{|l|} \hline
          $\lambda x .$\\
          $~~~expr$ \\ \hline
       \end{tabular}
\end{center}
  \item {\tt let}-expressions also allow a break:
\begin{center}
       \begin{tabular}{|l|} \hline
          $let~$\begin{tabular}{|c|}\hline $x=exp_1$\\ \hline\end{tabular}\\
          $~in~ exp_2$\\ \hline
       \end{tabular}
\end{center}
Additionally we want ---if necessary--- ``$x=exp_1$'' to break after ``$=$''.
\end{itemize}
Armed with this design specification,
we now can write a simple
pretty-printer for {\tt lambda}-expressions along the
concrete syntax as follows:
\begin{verbatim}
fun format (Let(ID,exp1,exp2)) =
           HOVbox[String "let", Space,
                  HOVbox[String ID, String "=", Break, format exp1],
                  Break, String "in", Space, format exp2]
  | format (Lam(ID,exp)) =
           HOVbox[ String "\\", String ID, String ".", Break, format exp]
  | format exp = format_A exp
and format_A (App(expa,expb)) = HOVbox[ format_A expa, Break, format_B expb ]
  | format_A exp = format_B exp
and format_B (Var ID) = String ID
  | format_B exp = HOVbox[ String "(", format exp, String ")" ]
\end{verbatim}
Here is sample output from this formatter with the {\tt Pagewidth} 20:
\begin{verbatim}
               \x.
                  \y.
                     \z.
                        let x= y
                           in y (x y)
                                 (z y)
\end{verbatim}


\subsubsection{Some improvements}
There are a number of points about this particular formatter that we are not
quite satisfied with. 
\begin{itemize}
\item We do not like that {\tt let x=\ exp in ...} inserts a white space after
      {\tt =} in horizontal output mode. However, we are in luck: ForML allows
      us to fine-tune breaks by specifying:
      \begin{itemize}
        \item how many spaces to output for the break in horizontal mode
              (default: 1)
        \item how many spaces to indent from the left border of the box, when
              the break actually is taken (default: 3)
      \end{itemize}
      The fine-tuning break primitive is called {\tt Break0} and takes these
      two parameters as arguments.
      Thus in the {\tt format (Let ...)}-clause of the above 
      {\tt format}-function we need to change the first {\tt Break} to
      {\tt Break0 0 3} to get the desired effect.
\item Similarly, we might want to indent the {\tt in} by one rather
      than by three spaces to align it under the {\tt let}. We can achieve this
      by changing the second {\tt Break} in the {\tt Let}-clause to read
      {\tt Break0 1 1}.
\item A long series of $\lambda$-abstractions may lead to breaks in nested 
      boxes, although it may be nicer if the could be treated as a list, 
      i.e.\ instead
of\\
\begin{tabular}{|l|} \hline
  $\lambda x .$ \\
  ~\ ~\ \begin{tabular}{|l|} \hline
        $\lambda y .$ \\
        ~\ ~\ \begin{tabular}{|l|} \hline
              $\lambda z .$ \\
              ~\ ~\ \begin{tabular}{|l|} \hline inner expression \\ \hline
                    \end{tabular}
              \\ \hline
          \end{tabular}
       \\ \hline
    \end{tabular}
   \\ \hline
\end{tabular}
\ ~ \ ~ \ ~we would like to have\ ~\ ~\ ~\ 
\begin{tabular}{|l|} \hline
  $\lambda x.$ $\lambda y.$ $\lambda z.$ \\
              ~\ ~\ \begin{tabular}{|l|} \hline inner expression \\ \hline
                    \end{tabular}
       \\ \hline
\end{tabular}\\
To achieve this we change the line ``{\tt | format (Lam ...)}'' to read
``{\tt | format (e as Lam(ID,exp)) = format\_lam e nil}'', and we add two
clauses to define the auxiliary function {\tt format\_lam}:
\begin{verbatim}
and format_lam (Lam(ID,exp)) absl =
                format_lam exp (ll @ [String("\\" ^ ID ^ "."), Break])
  | format_lam exp ll = HVbox(ll @ [format exp])
\end{verbatim}
\end{itemize}
With these improvements in place, the formatter now produces:
\begin{verbatim}
               \x. \y. \z.
                  let x=y
                   in y (x y) (z y)
\end{verbatim}

\subsection{The signature FORMATTER}
We now explain the interface to the formatter by describing the components of
its {\tt FORMATTER} signature in {\tt formatter.sig}:
\subsubsection{Defaults and Switches}
Several references contain default values which may be changed by the
user:
\begin{verbatim}
      val Indent : int ref
      val Blanks : int ref
      val Skip   : int ref
\end{verbatim}
{\tt Indent}, {\tt Blanks}, and {\tt Skip} correspond to
the values for added indentation, horizontal spacing, and vertical skip
that are used by the default boxes. Their respective values are initially
{\tt 3}, {\tt 1}, and {\tt 1}, but they may be changed at any point.
\begin{verbatim}
      val Pagewidth     : int ref
\end{verbatim}
{\tt Pagewidth} contains the display width for the page. It is initially set to
{\tt 80}.
\begin{verbatim}
      val Bailout       : bool ref
      val BailoutSpot   : int ref
      val BailoutIndent : int ref
\end{verbatim}
When you turn the {\tt Bailout} flag on, ForML attempts to fail more gracefully
whenever the output overruns the display width: such text will be output with
an initial indentation of {\tt BailoutIndent} from the left margin.
Since ForML views the text boxes it formats as rectangles rather than
octagons, it tends to ``panic'' when you turn the {\tt Bailout} flag on, and
outputs a number of text lines indented flushly by {\tt BailoutIndent},
rather than just continuing normal indentation until the situation on the page
really becomes tight again.
To ameliorate this situation, set {\tt BailoutSpot}
to a value between {\tt BailoutIndent} and {\tt Pagewidth}, and then a
Bailout will only be triggered when the leftmost edge of text would be output
after the {\tt BailoutSpot}. 
Sounds like a bailout, doesn't it?
Initially, {\tt Bailout} is set to {\tt true}, {\tt BailoutSpot} is {\tt 40},
and {\tt BailoutIndent} is {\tt 0}.

\subsubsection{Formats}
The abstract datatype {\tt format} holds the actual formats. The minimum
and maximum display widths of a format may be inquired with {\tt Width}.
\begin{verbatim}
      type format
      val  Width: format -> (int * int)
\end{verbatim}
A whole slew of functions are provided to assemble formats, and you have already
seen a cross-section of these earlier.
\begin{verbatim}
      val  Break: format
      val  Break0: int -> int -> format     (* blanks, indent *)
      val  String: string -> format
      val  String0: int -> string -> format (* output width *)
      val  Space: format
      val  Spaces: int -> format
      val  Newline: unit -> format
      val  Newlines: int -> format
      val  Newpage: unit -> format
      val  Vbox: format list -> format
      val  Vbox0: int -> int -> format list -> format  (* indent, skip *)
      val  Hbox: format list -> format
      val  Hbox0: int -> format list -> format         (* blanks *)
      val  HVbox: format list -> format
      val  HVbox0: int -> int -> int -> format list -> format  (* blanks, indent, skip *)
      val  HOVbox: format list -> format
      val  HOVbox0: int -> int -> int -> format list -> format (* blanks, indent, skip *)
\end{verbatim}
The functions ending in {\tt 0} take additionally arguments for indentation,
horizontal spacing, or vertical skip, whereas their cousins (without the 
{\tt 0})
simply use the default values stored in {\tt Indent}, {\tt Blanks}, and 
{\tt Skip}. The {\tt String}, {\tt Space}, {\tt Spaces},
{\tt Newline}, {\tt Newlines}, and {\tt Newpage} functions do what you would
expect them to. {\tt String0} requires you to additionally enter the width to
be used in pagewidth calculations.
Be forewarned, though, that ForML is pretty stupid about {\tt Newline}({\tt s})
issued inside of boxes: it thinks they are just like any other string, except
their length is 0, so you could get some strange formatting results --- rather
use {\tt Break}s in {\tt Vbox}es if you need to force new lines.
% 
% \begin{verbatim}
%       val  Separator : format list -> format (* format for separators *)
% \end{verbatim}
% The {\tt Separator} format needs a little more explanation:
% It may only be used inside of boxes, and you cannot inquire about its
% {\tt Width}.
% The format list given as argument to the {\tt Separator}-format will be inserted
% whenever the {\tt Separator} is encountered, {\em except\/} if {\tt Separator}
% is the last item inside of a box, in which case nothing will be inserted.
% Typically, a {\tt Separator} would hold the format for list-element separators
%when formatting a list structure.


\subsubsection{Shipping it out}
The two functions
\begin{verbatim}
      val  makestring_fmt:        format -> string
      val  print_fmt:             format -> unit
\end{verbatim}
turn formats into strings and print them out, respectively.
If you want to put formatted output into an {\tt outstream}, consider
at the following:
\begin{verbatim}
      type fmtstream
      val  open_fmt:              outstream -> fmtstream
      val  close_fmt:             fmtstream -> outstream
      val  output_fmt:            (fmtstream * format) -> unit
      val  file_open_fmt:         string -> ( (unit -> unit) * fmtstream )
      val  with_open_fmt:         string -> (fmtstream -> 'a) -> 'a
\end{verbatim}
The function {\tt open\_fmt} creates a ``format-stream'' {\tt fmtstream}
associated with a particular outstream,
and {\tt close\_fmt} closes it, but leaves the {\tt outstream} open.
You use {\tt output\_fmt} to print a format on the {\tt fmtstream}.
Note that once you have closed an {\tt outstream} associated with a
{\tt fmtstream}, any further output to that {\tt fmtstream} will result in
an I/O error.
% \footnote{And
% there is also {\tt val  debug\_output\_fmt: (fmtstream * format) -> unit},
% which decorates its output with additional information for debugging ForML.}

{\tt file\_open\_format} expects a file name as an argument. It opens the
file and returns a pair consisting of a function to later close this file,
as well as a {\tt fmtstream} onto which to output formats.

Finally, the function {\tt with\_open\_format} expects the name of a file,
and a function whose argument is a {\tt fmtstream}. It will open the file, call
the function with the resulting {\tt fmtstream}, and afterwards close the
file and return the result of the function call.


\section{An extended example: Mini-ML}
We now extend the previous example to a subset of an ML-like language with the
following abstract syntax:
$$\begin{array}{lcl}
        exp & ::=  & x ~|~ n ~|~ {\tt true} ~|~ {\tt false}
                   ~|~ exp_1~exp_2 ~|~ \lambda x.exp\\
            & &  ~|~ {\tt let}~x=exp_1~{\tt in}~ exp_2
                 ~|~ {\tt if}~ exp_1 ~{\tt then}~ exp_2 ~{\tt else}~ exp_3\\
            & &  ~|~ [exp_1,\ldots,exp_n] ~|~  exp_1 ~op~ exp_2\\
    \end{array}
$$
where $x\in Var$, $n\in Nat$,
       $\{{\tt true}, {\tt false}\} \in Bool$,
        $op = \{ =, <, +, -, *, /, \ldots\}$, and $[\ldots]$ denotes a list.

A corresponding SML datatype for abstract syntax trees might be as follows
(purely for convenience we also decorate {\tt Op}-nodes with the operator
precedence):
 \begin{verbatim}
datatype mexp = Var of string | Int of int | Bool of bool
                  | App of mexp * mexp
                  | Lam of string * mexp
                  | Let of string * mexp * mexp
                  | If of mexp * mexp * mexp
                  | List of mexp list
                  | Op of string * int * mexp * mexp
 \end{verbatim}
The formatter for {\tt mexp}s now also has to provide unparsing, i.e.\ it needs
to take operator precedences into account.
Below is an example of a formatting function for this language.
Rather than mimicking some suitable concrete grammar,
it deals with nested {\tt App}s
similarly as the second version of the {\tt format}-function did with nested 
{\tt Lam}s.  The reader may want to reconstruct the basic box designs.

Before we start, however, we need to define an auxiliary function:
\begin{verbatim}
infixr 5 @@

fun (l1 @@ nil) = nil
  | (l1 @@ l2) = l1 @ l2
\end{verbatim}
Now on to the main formatting routine:
\begin{verbatim}
fun mformat (Var s) = String s
  | mformat (Int n) = String (makestring n)
  | mformat (Bool b) = String (if b then "true" else "false")
  | mformat (e as Lam _) = mformat_lam e []
  | mformat (e as App (e1,e2)) = mformat_app e1 [mformat_op 50 e2]
  | mformat (Let(s,e1,e2)) =
            HOVbox[String "let", Space, HOVbox[String s, String "=", Break0 3 1,
                                               mformat e1], Break0 1 1,
                   String "in", Space, mformat e2]
  | mformat (If(e1,e2,e3)) = HOVbox[ String "if", Space, mformat e1, Break,
                                     String "then", Space, mformat e2, Break,
                                     String "else", Space, mformat e3]
  | mformat (List l) =
            Hbox[ String "[",
                  HVbox (fold (fn (el, fmtl) =>
                                  [ mformat el ] @
                                        [String ",", Break0 1 0] @@ fmtl)
                              l
                              nil),
                  String "]" ]
  | mformat (e as Op(_,p,_,_)) = mformat_op p e
\end{verbatim}
In formatting {\tt App}s above (and also below), we force surrounding
parentheses by calling the formatting routine for operators with the
highest priority for the ``outside'' operator.
Finally, expressions consisting of nested {\tt Lam}s, {\tt App}s, 
and {\tt Op}s are
formatted by the auxiliaray functions
{\tt mformat\_lam}, {\tt mformat\_app}, and {\tt mformat\_op}
respectively:
\begin{verbatim}
and mformat_lam (Lam(s,e)) l =
                mformat_lam e (l @ [String("\\" ^ s ^ "."), Break])
  | mformat_lam e l = HVbox(l @ [mformat e])
and mformat_app (App(e1,e2)) l = mformat_app e1 ((mformat_op 50 e2)::Break::l)
  | mformat_app e l = HVbox((mformat_op 50 e)::Break::l)
and mformat_op p' (Op(s,p,e1,e2)) =
    HOVbox( (if p'>p then [String "("] else [])
            @ [mformat_op p e1] @ [ Break0 0 1, String s]
            @ [mformat_op p e2] @
            (if p'>p then [String ")"] else []) )
  | mformat_op p e = mformat e
\end{verbatim}
A sample output with a {\tt Pagewidth} of 20 might look as follows:
\begin{verbatim}
            \x. \y. \z.
               let silly=
                    \x. \y. \z.
                       if y+z>0
                          then x
                          else y
                in silly
                      [1, ~1, 0,
                       true,
                       false,
                       \x. x+1]
                      (1+(2+x)*3
                          /(4-5)
                       -6*7+8)
                      15
\end{verbatim}

\bibliographystyle{plain}
\bibliography{pretty}

\end{document}
