% RCS LOG
%
% $Log: user.tex,v $
% Revision 1.1  1994/02/08 00:23:12  jkh
% Initial revision
%
%Revision 1.3  1991/10/22  20:03:47  db
%Minor formatting changes.
%
%Revision 1.2  91/04/11  11:16:19  11:16:19  db (Dave Berry)
%Fixed minor bugs in description of Make system.
%Added functor specifications to appropriate entries.
%
%Revision 1.1  91/03/27  16:41:38  16:41:38  db (Dave Berry)
%Initial revision
%

\chapter{Using the Library.}	\label{user}

\section{Getting Started.}

To use the library, you need a version of SML with the library loaded.
The person who installed the library on your system should have created
such a version for you.  If not, you'll have to read Chapter~\ref{install}
and do it yourself.

Suppose that you want a function to convert integers to strings,
and a function to prompt the user for input.  On reading the rest of this
chapter, you will find that you want the {\tt Int} and {\tt User} library
entries, specifically the functions {\tt Int.string} and
{\tt User.prompt}.

There are two ways that the library could be set up.
Type {\tt loadLibrary "User"}.  If this returns the
string {\tt "All library entries have been loaded"},
then the entries are ready for use.  If this is not the case, the
your call to {\tt loadLibrary} will compile the {\tt User} entry for you.
The {\tt Make} system will ensure that all entries required by the
{\tt User} entry itself will be loaded as well.

Once you've loaded the {\tt User} and {\tt Int} entries, you can use
the {\tt Int.string} and {\tt User.prompt} functions just like any
other components of structures.  Most library entries are implemented
as structures instead of functors. This saves space and makes the
library easier to use.
\footnote{The library can also be loaded without loading the {\tt Make}
system or any entries at all.  This is intended for minimising space
when building stand-alone applications rather than for general use.}


\section{Entries.}

Most library entries are structures.  Each structure is documented by a
signature and a sample (possibly inefficient) implementation.
Structure names are written in mixed case,
with the corresponding signature names written in upper case.  For
example, the {\tt Int} structure implements the {\tt INT} signature.

The signature gives the usual type information, with a short
description of each object (in English), and some introductory material.
The introductory material includes details of the author, maintenance
arrangements and so forth.  Signatures may include Extended ML axioms
(in comments), if the implementer provides them.

The implementation is a reference implementation of the objects
declared in the signature.  The actual implementation may be more
complex than this, but it must produce the same results and side-effects.
The only exception to this rule is that implementations may produce
debugging or tracing output in addition to their normal effects.

Some entries are functors.  Since SML doesn't have a notion of ``functor
signature'', these entries are described by comments at the start of
each functor.

The detailed descriptions of the current entries are given later in this
report.  This section gives an overview of the whole library, to give
you some idea of what the library contains.

In addition to the entries described in this report, the library also
includes space for contributions that don't fit the library format.
To see these, find out where the library is installed on your system
and browse the {\tt contrib} sub-directory. 



\section{Structures.}

The library currently contains the following structures:

\begin{description}
  \item[\tt General.]
	Types, exceptions and functions that are widely used or that
	don't fit anywhere else.  In particular, the types {\tt Result},
	{\tt Option} and {\tt Nat} are made global and are used in many
	other library entries.

  \item[\tt Bool, Instream, Int, List, Outstream, Real, Ref, String.]
	Basic operations on the pervasive types.
 
  \item[\tt BoolParse, IntParse, ListParse, StringParse.]
	Functions to parse the basic types from strings or instreams.
	The {\tt ListParse} structure matches the {\tt SEQ\_PARSE}
	signature; the others match the {\tt PARSE} signature.

  \item[\tt Ascii, StringType, StringListOps.]
	{\tt Ascii} defines constants for the non-printing Ascii
	characters.  {\tt StringType} defines functions for testing
	whether the first character in a string is a letter, digit,
	control character, etc.  {\tt StringListOps} defines functions
	on strings that mimic those in the {\tt List} entry.

  \item[\tt ListSort.]
	Functions to sort and permute lists.

  \item[\tt AsciiOrdString, LexOrdString, LexOrdList.]
	Orderings on sequences.  {\tt AsciiOrdString} compares strings
	when case difference is significant; by contrast {\tt LexOrdString}
	ignores case.

  \item[\tt Array, ArrayParse, Byte, ByteParse, Vector, VectorParse.]
	Types and functions for arrays, bytes, and constant vectors.

  \item[\tt Pair, PairParse, ListPair, StreamPair.]
	Operations on pairs of objects.  {\tt StreamPair} also
	includes suggested functions for interacting with the host file
	system using pairs of one instream and one outstream.
 
  \item[\tt EqSet, Set.]
	Polymorphic sets.  {\tt EqSet} defines sets over equality
	types.  {\tt Set} defines sets over arbitrary types, but
	defines several functions that require and equality function
	to be passed as an argument.

  \item[\tt Hash, EqFinMap.]
	Hash tables and finite maps.  The keys to finite maps must
	be equality types.   
  
  \item[\tt User.]
	Functions to prompt the user for input.  A window system should
	provide an appropriate replacement for this structure.

  \item[\tt Make.]
	The {\tt Make} system mentioned in the {\tt Getting Started}
	section.  This will be described in more detail in
	Section~\ref{make}.

  \item[\tt Const.]
	Creates unique copies of objects.

  \item[\tt System.]
	Some suggested functions for interacting with the host file
	system and the SML compiler.  Many of these functions are not
        implemented in the portable version of the library.

  \item[\tt Combinator.]
	Simple combinator functions.

  \item[\tt Memo.]
	Memoising functions.
\end{description}



\section{Signatures, Functors, and Monomorphic Types.}	\label{generic}

Most of the structures mentioned in the previous section have
corresponding signatures with the same name (but in upper case).  The
exceptions are the Parse structures, which are match the {\tt generic}
signatures {\tt PARSE} and {\tt SEQ\_PARSE}.  (Exception: the
{\tt ArrayParse} structure matches the {\tt ARRAY\_PARSE} signature).

There are several other generic signatures.  One example is the
{\tt EQUALITY} signature.  This specifies a type {\tt T} and an equality
function {\tt eq: T $\rightarrow$ T $\rightarrow$ bool}.  By convention,
when a library entry defines a new type, it gives it the name {\tt T} in
addition to its normal name.  For example, the types {\tt Int.int} and
{\tt Int.T} are identical (and are also identical to the built-in type
{\tt int}).  This means that many of the structures in the library will
match the {\tt EQUALITY} signature.

Now suppose that you define a functor that 
takes an argument structure specified by the {\tt EQUALITY}
signature.  For example, the functor {\tt MonoSet} takes
such an argument and returns a set of elements of type {\tt T}.
The library naming conventions mean that this functor can
be applied to many of the types defined in the library.

The library provides several such generic signatures.  {\tt ORDERING}
defines a type {\tt T} and an ordering function.  {\tt PRINT} defines
a type {\tt T}, a function to convert values of that type to strings,
and a function to print the string representation of a value on an
outstream.  {\tt ORD\_PRINT}, {\tt EQ\_ORD}, and {\tt EQ\_PRINT} are
obvious combinations of these, while {\tt OBJECT} combines all of
them in a single signature.  {\tt EQTYPE\_ORD} and {\tt EQTYPE\_PRINT}
are versions of {\tt ORDERING} and {\tt PRINT} in which {\tt T} is an
equality type.   

The functors {\tt MonoSet}, {\tt MonoList}, {\tt MonoVector} and
{\tt MonoArray} use generic signatures.  {\tt MonoSet} uses {\tt EQUALITY},
and the others use {\tt EQ\_PRINT}.  They all return structures that
are similar to their polymorphic equivalents, except that the objects
they define contain elements of a fixed type.  This is sometimes
desirable.

One reason for creating monomorphic values in this way is that they
can sometimes be implemented more efficiently then
polymorphic ones.
For example, a vector of booleans can be implemented as a bitset.

Another reason is that many functions over monomorphic collections
have a simpler type, as they
don't need a function parameter to apply to the elements of the vector.
For example, the equality functions on polymorphic vectors and monomorphic
vectors have the types
{\tt ('a $\rightarrow$ 'a $\rightarrow$ bool) 
$\rightarrow$ 'a Vector $\rightarrow$ 'a Vector $\rightarrow$ bool}
and {\tt MonoVector $\rightarrow$ MonoVector $\rightarrow$ bool}
respectively.

The library also contains the {\tt ByteVector}, {\tt BoolVector},
{\tt ByteArray} and {\tt BoolArray} structures.  These can be created
by applying
the respective functors, or implemented directly.
The library also contains the generic signatures {\tt SEQUENCE} and
{\tt SEQ\_ORD}, which are matched by these structures.

MonoVectors and MonoArrays can be extended to include parsing functions
using the {\tt MonoVectorParse} and {\tt MonoArrayParse} functors.
These take a MonoVector or MonoArray and a structure that matches the
{\tt PARSE} signature, and produce structures that match the generic
{\tt MONO\_SEQ\_PARSE} signature.  The type {\tt T} of the structure
that matches the {\tt PARSE} signature must be the type of the elements
of the MonoVector or MonoArray.  This is enforced by a sharing specification.

\section{Library Conventions.}

The previous section introduced the convention that types in
library	entries can be referred to by the name {\tt T}, and showed how
this is used to implement generic signatures.  This section describes
some more conventions that make library entries
consistent and therefore easier to use.

\begin{description}
  \item[\em Identifiers.]
	We have already seen that structure and functor names are mixed
	case, while signatures names are all upper case.
	All words in alphanumeric identifiers are capitalised (as we have
	already seen for structure names), except for two cases:
   \begin{enumerate}
    \item 
	Alphanumeric labels and value identifiers (not including
	constructors and exceptions) begin with a lower case letter.
	This is so that values and constructors can be distinguished
	easily.
    \item As we have seen, signature identifiers are written in upper
	case, with multiple words separated by underscores.  This is
	a widely used convention, and is useful when writing about
	structures and signatures in documentation or tutorial material.
   \end{enumerate} 	
	Apart from signature identifiers, alphanumeric identifiers never
	include underscores.  Symbolic identifiers are only used for
	functions or constructors, and then rarely.

  \item[\em Infix functions.]
	Functions that are intended to be used infix are declared as
	such, with a suggested precedence, even though this status
	doesn't propagate when a structure is opened.

  \item[\em Currying.]
	Nonfix functions are usually curried.  Entries don't usually
	provide both curried and uncurried or infix versions of a function.

  \item[\em Imperative functions.]
	Imperative functions usually return the unit value.

  \item[\em Exceptions.]
	In the current entries, exceptions are only used to signal
	exceptional circumstances, such as
	a function called with unsuitable arguments.  If a function
	sometimes returns a value and sometimes doesn't, then usually
	its result type will be an instance of the {\tt Result} type
	defined in the {\tt General} structure (and available globally).
	Using these
	types forces you to use a case expression instead of a handler to
	check the result, which means that the compiler will report missing
	cases.

	However, this convention is not a hard and fast rule.  For example,
	if someone wants to implement a version of the parsing
	functions that raises an exception on incorrect input, they
	may add this version to the library.

	Exceptions usually return some information about the values that
	caused the exception, and the name of the function in which it
	occurred
	(if more than one function can raise that exception).

  \item[\em Equality.]
	If an entry is defined over equality types, the name of that entry
	usually begins with {\tt Eq}.  Sometimes there will be two
	versions of an entry, one restricted to equality types, the other
	taking an explicit equality function.  The {\tt Set} and
	{\tt EqSet} entries are an example of this.

	If an entry uses an explicit equality function, then this function
	is passed to the functions that manipulate values of the type.
	It is not passed to functions that {\tt create} values of the
	type.  The latter approach can result in erroneous programs,
	where two values have the same type but different associated
	equality functions.

  \item[\em Monomorphic types.]
	Often the library includes both monomorphic and polymorphic
	versions of an entry.  In such cases the monomorphic version
	is implemented as a functor.  The result signature of the functor
	is the same as that of the polymorphic structure, with the
	monomorphic type substituted for occurrences of the polymorphic
	type, and the elements type(s) substituted for the corresponding
	type variables.

  \item[\em Sharing.]
	Most objects defined in a library entry are unique to that entry.
	A few shared types are defined in the {\tt General} entry.
\end{description}


\section{Standard Names.}	\label{names}

Many library entries define a type and operations on that type.
These entries use several names in common, so that frequently-used
operations are easy to remember.  (this was inspired by the SmallTalk
and Eiffel libraries.)  It also means that the entries can be used with
the generic signatures, as we saw in Section~\ref{generic}.

\begin{description}	
  \item[\tt eq, ne, lt, le, gt, ge.]
		The comparison functions.
  
  \item[\tt fixedWidth.] 
		True if the string representation of the type takes a
		known amount of space (e.g. for Bytes).

  \item[\tt string.] 	Converts an object to its string representation.

  \item[\tt print.]	Prints the string representation of a value on an
		outstream.

  \item[\tt parse.] 	Converts the string representation of an object to that
		object.

  \item[\tt read.]		Like parse, but takes an InStream.

  \item[\tt create.]	Create an object from components.

  \item[\tt tabulate.]	Generate a sequence by applying a function to the
		indices of the elements.

  \item[\tt size.]		Return the number of elements in a sequence.

  \item[\tt empty.]	Return true if a sequence is empty.

  \item[\tt sub, nth.]	Return the nth element of a sequence.

  \item[\tt extract.]	Return a subsequence of a sequence.

  \item[\tt map, apply, iterate, iterateApply.]
		Apply a function to each element of a sequence
		(four variations of the basic idea).

  \item[\tt foldL, foldR, foldL', foldR'.]
		Combine the elements of a sequence to produce a single
		value (four variations of the basic idea).

  \item[\tt pairwise.]	Test that all adjacent pairs of elements in a
		sequence
		satisfy a binary predicate.

  \item[\tt rev.]	Reverse a sequence.

  \item[\tt \^{}.]	Catenate two sequences.
\end{description}

In addition, there is a convention for naming conversion functions --
functions that convert values of one type to values of another type.
If the function converts values of type {\tt X} to values of type
{\tt Y}, and is defined in the entry {\tt X} that defines the type {\tt X},
then the function is called {\tt X.y}.  Conversely, if the function is
defined in the entry {\tt Y} that defines type {\tt Y}, then it is called
{\tt Y.fromX}.  For example, the function to convert a vector to a list
is called {\tt Vector.list}, and the function to do the opposite conversion
is called {\tt Vector.fromList}.

The rationale behind this convention is that functions are usually
called by their qualified name, and the name of the structure is the same
as one of the types in the conversion.  Therefore it's redundant to
have both type names in the name of the function itself.

  
\section{Using the Make System.}	\label{make}

The {\tt Make} system is used by the library to ensure that, when you load
a new entry, all entries that the new entry requires are loaded first
In addition, you can use {\tt Make} to minimise the amount of recompilation
you do when you change your program.
{\tt Make} works by keeping track of which pieces of your code depend on
which others.  You must give {\tt Make} this information by annotating
your code with special comments.  Then, when you tell {\tt Make} to compile
a piece of code, it automatically loads any code that the new code requires
that isn't already loaded.

The dependency information is given by preceding each piece of code
by a tag declaration in a special comment.
Keeping the dependency information with each
piece of code, instead of in a separate file like the {\small UNIX}
{\tt make} command, makes it easier to keep the two in sync.
Tag declaration
comments are a subset of ordinary comments, so files containing them can
be compiled without using make.

Tag declaration comments begin with the string {\tt ($\mit\ast$\$} instead of
the usual {\tt ($\mit\ast$}, and must start at the beginning of a line.
Formatting
characters are forbidden before and after the initial {\tt ($\mit\ast$} string,
but
are permitted between tags and dependencies.
For example, the following tag declaration associates the tag {\tt Foo}
with
the code that follows it:

{\tt ($\mit\ast$\$Foo $\mit\ast$)}

\noindent
The following tag declaration associates the tag {\tt Bar} with the code that
follows it, and states that this code depends on the code associated
with the tags {\tt Foo1} and {\tt Foo2}:

{\tt ($\mit\ast$\$Bar: Foo1 Foo2 $\mit\ast$)}

\noindent
The code associated with a tag is terminated by the next tag declaration
or the end of a file.

The easiest way to consult the dependency information is to list
the files containing the code in a file (called a tag file), and to call
{\tt loadFrom} with the name of that tag file.  This loads all the
dependency
information into the make system.  If you change the dependency
information, use {\tt consultDecl} or {\tt consultFile} to update the stored
dependencies.

Once the dependency information has been loaded, call {\tt make tag} to compile
the code associated with the tag and all the code that it depends on.
The system will write all the relevant code into a temporary file (called
{\tt \%Make.tmp\%} by default) and call {\tt use} on that file.

Once a piece of code has been read from a file, it is stored in a cache
When the code associated with a tag is changed, call {\tt touch tag} to
tell the {\tt Make} system that the current entry in the cache is obsolete.
{\tt consultDecl} and {\tt consultFile} do this automatically for the tags
that they read.  If a piece of code fails to compile, then that entry in
the cache is automatically touched as well.
The system does not check the modification dates of files.

Once all the changed pieces of code have been touched, call {\tt again ()} to
re-make your program.


\section{Infix Operators.}

This is the list of infix operators provided by the pervasives and
by the library:

\begin{tabbing}
\hspace{0.3in}\={\tt infix  0 General.before}\\
\>{\tt infix  1 Bool.or}\\
\>{\tt  infix  2 Bool.\&}\\
\>{\tt infix  3 o  Bool.implies}\\
\>\hspace{0.5in}{\tt Combinator.oo  Combinator.co
	Combinator.fby  Combinator.over}\\
\>{\tt infix  4 =  $<>$  $<$  $>$  $<$=  $>$=}\\
\>{\tt infix  5 Int.--}\\
\>{\tt infixr 5 ::  @}\\
\>{\tt infix  6 +  -  \verb+^+}\\
\>{\tt infix  7 /  $\mit\ast$ div  mod  Int.divMod  Int.quot Int.rem  Int.quotRem}\\
\>{\tt infix  8 Int.$\mit\ast\ast$ Real.$\mit\ast\ast$}\\
\>{\tt infix  9 X.sub}, for most {\tt X}
\end{tabbing}

Note that @ associates to the right in the library.
Remember that infix status is local to each structure.

Equivalent curried functions for some of the above are as follows:

\begin{tabular}{ll}
  {\tt =}   &   {\tt X.eq}, for most {\tt X}\\
  {\tt $<>$}  &   {\tt X.eq}, for most {\tt X}\\
  {\tt $<$}   &   {\tt Int.lt, Real.lt}\\
  {\tt $>$}   &   {\tt Int.gt, Real.gt}\\
  {\tt $<$=}  &   {\tt Int.le, Real.le}\\
  {\tt $>$=}  &   {\tt Int.ge, Real.ge}\\
  {\tt @}   &   {\tt List.appendLast}\\
  {\tt (,)} &   {\tt Pair.create}
\end{tabular}


\section{Future Entries.}

The main aim of the library is to provide a framework for future
development.  The only way that the library can grow is by users
contributing code. 
If you would like to add an entry to the library, please read
Chapter~\ref{implement}.
The following list suggests some entries that would be useful:

\begin{itemize}
  \item Trees, of various kinds, and implementations of maps etc. based
on trees.

  \item Regular expressions and substitutions over strings.

  \item Lazy lists.

  \item Complex numbers, rational numbers.

  \item Stacks, priority queues, etc.

  \item Tries.

  \item Scanner generators, parser generators.

  \item A portable, easy-to-use interface to host window systems.
\end{itemize}
