% RCS LOG
%
% $Log: intro.tex,v $
% Revision 1.1  1994/02/08 00:23:12  jkh
% Initial revision
%
%Revision 1.1  91/03/27  16:41:20  16:41:20  db (Dave Berry)
%Initial revision
%

\chapter{Introduction.}

The Edinburgh SML Library currently provides over 200 functions
on several basic types, including sets, hash tables and the built-in types.
By using the library you can avoid having to
write your own versions of these functions.  What's more, you will be
sure that they will run on any implementation of SML, sparing you or your
users the need to port your program between different compilers.  Using
the library will also make your programs easier for other library users
to understand.

The library provides a consistent framework that can be extended by
the addition of new modules.  Users of the library are invited to
add their own code to the library, provided that they follow the
guidelines described in this report.  The framework ensures that the
library entries present a consistent user interface.  It also makes
practical the development of more sophisticated tools, such as a
hypertext browser.

Part of this framework is software.
The {\tt Make} system recompiles
the part of your program that have changed, and no more. The coding styles
used in the library are supported by a small number of types and functions,
which the library makes available at top-level. Generic signatures let
you write functors that can be applied to most library entries.
Generic signatures also serve to structure the library.

The rest of the framework consists of conventions that library entries
should follow. These are just as important as the software. They include
conventions that ensure a consistent user interface, such as consistent
use of upper-case and lower-case letters in identifiers, the preference of
curried functions over tuple parameters, and several standard names.
The conventions also include a standard format for comments. This makes
each signature broadly equivalent to a UNIX on-line manual page.

The library also provides a standard interface for
common extensions to the language, such as vectors and arrays.
A portable version of the library defines the objects that
it provides, and can be used with any implementation of the language.
Implementers are encouraged to provide more efficient versions
of library entries.

This notion of portability extends to features that can't be defined
in pure SML, such as the {\tt use} and {\tt cd} functions provided by
most compilers.  Although the portable version of the library
can't define these functions, it can specify their types;
compiler-specific versions of the relevant library entries can
implement them.

The Edinburgh SML Library has been developed over a number of years with
input from several people.  The current version will be distributed
with most SML compilers.  The library is free, and people are welcome
to develop it as they see fit.
