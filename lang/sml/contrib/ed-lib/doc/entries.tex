% RCS LOG
%
% $Log: entries.tex,v $
% Revision 1.1  1994/02/08 00:23:12  jkh
% Initial revision
%
%Revision 1.2  1991/10/22  19:59:56  db
%line 46: CONSTANTS --> CREATORS.
%
%Revision 1.1  91/03/27  16:40:59  16:40:59  db (Dave Berry)
%Initial revision
%

\chapter{The Entries.}		\label{entries}

This chapter consists of the on-line documentation for the current entries in
the Edinburgh SML library.  As explained in Chapter~\ref{user},
most entries are documented by an annotated signature.  The others
are described by generic signatures.

Each entry begins with a header section.  This is a comment that
includes a title, author, creation date, maintenance details, a
description of the entry, and optional notes or references to related
entries.  In the actual files the header section also contains an
RCS log.

Like {\small UNIX} manual pages, the header sections have a standard
syntax.  This is specified in Chapter~\ref{implement}.


The other objects defined in the entry are grouped under convenient
sub-headings.  These depend on the entry.  The following list gives
some sub-headings that I've used extensively; other authors may decide
not to use these.

\begin{description}
  \item[{\small\tt PERVASIVES.}]
	Most of the {\small SML} pervasives are provided in library
	entries, so that they can be accessed even if their identifiers are
	rebound at top level.  Pervasive constructors, overloaded
	identifiers, and the equality function are not rebound in this
	way, because the language won't allow it.
  \item[{\small\tt SYSTEM.}]
	Some functions can't be defined in terms of the {\small SML}
	pervasives.  These are usually grouped under the {\small\tt SYSTEM}
	sub-heading.
  \item[{\small\tt TYPES.}]
	Any other type or types defined by an entry, including synonyms
	of the main type.
  \item[{\small\tt CONSTANTS.}]
	Constant values.
  \item[{\small\tt CREATORS.}]
	Functions that create composite values from elements.
  \item[{\small\tt CONVERTORS.}]
	Functions that convert values from one type to another.
  \item[{\small\tt OBSERVERS.}]
	Functions that return information about a value, such as membership
	of a class, number of elements, etc.
  \item[{\small\tt SELECTORS.}]
	Functions that return sub-components of a value.
  \item[{\small\tt ITERATORS.}]
	Functions that apply a parameter function to every element
	of a sequence.
  \item[{\small\tt MANIPULATORS.}]
	General functions.
\end{description}

Each object in an entry is documented by a comment that follows the
definition.  This comment may begin with an example call; if present,
this should be separated from the main text by a semi-colon.  For
simple values, the comment may include the definition.  I've only done
this for one-liners.


