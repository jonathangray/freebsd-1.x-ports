\input paperhead    %  Read private macro package
\DoContents         %  Maintain a table of contents
\centerline{\titlefont A Guide to \TeX}
\vskip 4pt
\centerline{\titlefont for the Troff User}
\vskip 8pt
\centerline{\it Mike Urban}
\centerline{\sl TRW Software Productivity Project}
\vskip .5in
\Section{Introduction}
Donald Knuth's \TeX\ system has several advantages over the
\def\Troff/{{\it troff\/}}% Includes the italic correction
standard \Troff/ program distributed with Berkeley or Bell \Unix
\note{UNIX is a Registered Trademark of AT\&T Bell Laboratories.  Some
say that's the problem.}.
\TeX\ is portable, as it
is written in |Pascal|, and is available
on a variety of systems such as \Unix, VMS, TOPS-20%
\note{VMS and TOPS-20 are Trademarks of Digital
Equipment Corporation}, and IBM's
VM/CMS%
\note {VM/CMS is a Trademark of IBM} system. It is inexpensive (\$50 for
the \Unix\ port from the University of Washington),
and it generates device-independent output.  It
also can produce beautiful copy, especially when setting mathematics.
Finally, it has arbitrary-length command
and variable names, which can
provide much greater readability than the 1- and 2-character
symbols of \Troff/
(this property is shared with the
|Scribe|$^{\rm TM}$ system from Unilogic, Inc.).
For example, the first few lines of the file which produced
this paper are:
\begintt
\input paperhead    % Read private macro package
\DoContents         %  Maintain a table of contents
\centerline{\titlefont A Guide to \TeX}
\vskip 4pt
\centerline{\titlefont for the Troff User}
\vskip 8pt
\centerline{\it Mike Urban}
\centerline{\sl TRW Software Productivity Project}
\vskip .5in
\Section{Introduction}
Donald Knuth's \TeX\ system has several advantages over the
\endtt
The only cryptic item in this excerpt is the |\vskip|
primitive, which stands for {\bf Vertical Skip}; the remaining
commands are fairly transparent.

\TeX\ is, however, substantially different in
its general architecture from \Troff/.  Its input is
``stream-oriented'' rather than consisting of discrete lines,
with embedded formatting commands preceded by an {\sl escape\/},
or distinguished
character (the backslash, |\|).  Its treatment of different typefaces is
different from the approach used
by \Troff/, which was designed to drive a particular
(and now somewhat obsolete) variety of typesetter.
Finally, \TeX\ sets up a page of
type using an algorithm based on ``gluing'' together ``boxes'' of
characters, in a manner quite analogous to the way that
traditional printers locked up ``composing sticks'' containing
rows of letters by adjusting external wedges called ``quoins.''
To the casual user setting up a simple document, the difference
may appear to be minor.  However, the consequences can be felt
when one is attempting an operation that is even slightly extraordinary.

For these reasons, using \TeX\ for the first time
may prove to be a strange,
perhaps even frustrating experience for an experienced \Troff/
user, even if a wealth of ``user-friendly'' macros (analogous
to the {\bf --mm} or {\bf --me} macro packages) has been
provided.  This document will attempt to assist the prospective
\TeX\ user in the transition from \Troff/.  It does not pretend
to be a complete description of \TeX, nor is it a ``cookbook''
of conversion techniques for \Troff/ documents.  It is simply
an introduction to the most basic elements of \TeX\ that
assumes some \Troff/ experience.  When in doubt,
the reader is urged to try
out the various features of \TeX\ to see how they behave in
practice, rather than relying on ``thought experiments''
based on the material in this paper or ``{\sl The \TeX book}''.
\input control
\input fonts
\input dimen
\input pages
\input tables
\input mathmode
\input defs
\input unix
\input example
\input specials
\input fontsample
\input biblio
\PrintTOC
\input title
\bye
