\Section{Fonts and Sizes}
\SubSection{Fonts}
{\it Troff\/} is oriented towards a
Wang Laboratories, Inc.,
C/A/T~Phototypesetter\note{A device-independent
\Troff/ is available as a separate program product from Western
Electric or maybe some other AT\&T subsidiary.},
and this causes several peculiarities and limitations, such as the
four-font restriction and the limitation of text width to 
$7{1\over2}$ inches.  Furthermore, the phototypesetter produces different
sizes of type by magnifying the letters with lenses.  This causes
the smaller point sizes to be cramped and hard to read, while
the much larger sizes may be more widely spaced than aesthetics
might dictate.

\TeX\ produces device-independent output files with a theoretical
resolution smaller than the wavelength of visible light (65536
basic units to a printer's point).  Characters on the page
are symbols (often called {\sl glyphs\/}) from particular fonts.
The size of the symbol is defined by the font in which it occurs,
just as its shape is.  Thus, ten-point roman (which you are reading
now) is a font completely separate from eight-point roman ({\eightpoint
which you are reading now}).
Plain \TeX\ assumes that the document
will be set in ten-point type, and sets the distance between
lines to account for that.  Plain \TeX\ also defines five control
sequences for changing fonts:
$$\vbox{\halign{\indent#\hfil\qquad&#\hfil\cr
|\rm| switches to the normal ``roman'' typeface:&Roman\cr
|\sl| switches to a slanted roman typeface:&\sl Slanted\cr
|\it| switches to italic style:&\it Italic\cr
|\tt| switches to a typewriter-like face:&\tt Typewriter\cr
|\bf| switches to an extended boldface style:&\bf Bold\cr}}$$
At the beginning of a run you get roman type.  As we saw
earlier, the best way to switch fonts is within a {\sl group\/};
this obviates the need for explicitly switching back to the
previous font.  Notice that these control sequences all change
to the ten-point version of these type styles, regardless of
the point size currently being used.

Users of \Troff/, who are accustomed to using {\it italics\/} for
emphasis and special material, should note carefully the difference
between {\it italics\/} and {\sl slanted letters.  Slanted letters
are just skewed roman letters, \it  but italic letters are drawn
in a completely different style\/}.  Consider, for example,
{\tenu unslanted italic letters}.  Usually, slanted letters can
be used for emphasis or for cited book titles; italics can be
used for special material such as section headings, or names
of \Unix\ programs such as \Troff/.  However, there are no established
conventions for the use of italics versus slanted letters, and the
writer is free to use them as he or she sees fit.

In addition to the
five plain \TeX\ fonts, one can easily define a control sequence
which will switch to 
any single font in the font library stored in the \Unix\ file system
(which normally includes the Computer Modern font family you are
now reading).  In effect, one is creating an association between
a font name, used as a control sequence within \TeX, and the name
of the file on disk where the information about the font can be found.
For example, the word `\Unix' which appears
throughout this document is set in
a ten-point small-caps
font.  The \Unix\ file name for this font is |amcsc10|\note
{At present, the fonts have names like {\tt am\it xxx\/}, standing
for ``Almost Computer Modern''.  It is expected that these
names will progress through {\tt bm\it xxx\/} (Becoming Computer
Modern) and finally reach {\tt cm\it xxx\/}.
}.
The \Unix\ file names have no intrinsic significance to \TeX,
so to use
|amcsc10| in this paper, a |\csc| control sequence was defined
and associated with the |amcsc10| disk file
by saying
\begintt
\font\csc=cmcsc10
\endtt
After the |\font| control sequence, a switch to title font
is accomplished by simply issuing the |\csc| control sequence.
Samples of some of the available fonts have been appended to
this paper.

Plain \TeX\ predefines some control sequences for certain
smaller character
sizes.  For example, |\sevenbf| selects {\sevenbf seven-point boldface}
type.  When switching to a different size of type, one should be
aware that the interline spacing is {\sl not\/} changed, nor are
the definitions for |\sl|, |\bf|, etc. changed; they still refer
to ten-point slanted, boldface, etc.\note{The macro package used
to set this paper includes an {\tt\\eightpoint} macro that changes
all the appropriate definitions
to eight-point type, as in this footnote.
See your \TeX\ guru to see if such a macro is publicly available.}
To use the other available character sizes, such as nine-point
roman,
one must use |\font| to
associate a new control word (e.g. |\ninerm|) with the appropriate
font information file (in this case, |amr9|).

When switching from a slanty font like {\it italics\/} to a vertical
one like this roman font, a little extra space should be inserted
to prevent the top of the slanty text from getting too close to
the unslanted material.  Otherwise,
it ends up `{\it printed} like this' instead of
`{\it printed\/} like this.'  This so-called {\sl italic
correction\/} should be set with the
control symbol |\/| in the slanty font whenever such a 
switch is made.  For example, the previous sentence began:
\begintt
This so-called {\sl italic correction\/} should be set
\endtt
This isn't too hard to get used to, and can prevent some
uglinesses that \Troff/ doesn't understand or attempt to
deal with.
\tracingpages=1
\SubSection{Special Characters Revisited}
Not every font has every character.  For example, the
roman font doesn't have the inequality signs `$<$' and
`$>$' (they must be typed in Mathematics Mode, to be
described later).  However, if you know the symbols that
{\sl are\/} on the fonts you're using\note{The \TeX book
displays font layout tables in Appendix~E.}, you can produce
these characters using the special |\char| control sequence.
For example, the visible characters in the |\tt| font correspond
to ASCII terminal characters.  Thus, to produce a typewriter-style
backslash (`{\tt\\}', ASCII code 92 decimal), one can type
\begintt
{\tt\char92}
\endtt
By using an ``accent gr\`ave'' or left-quote character, one
can specify the character directly.  For example, the backslash
could also be produced by
\begintt
{\tt\char`\\}
\endtt
Remember, this example only applies to Typewriter font.  Position
92 of the roman font doesn't contain a backslash; thus
|\char`\\| would only produce the double-quote (\char92).

As described in the  later section on ``Macros And Definitions,''
these |\char| sequences can be assigned to {\sl macros\/} or abbreviations,
if they are to be used many times in a single document.
\SubSection{Magnification}
Many of the large type sizes can be obtained satisfactorily
by magnifying ten-point type to larger sizes just as
\Troff/ does.  Indeed, such magnified fonts may be the only
available fonts in the desired size.  One may specify a
magnified font by using the word |scaled| in the font
definition as shown:
\begintt
\font\twelverm=cmr10 scaled 1200
\endtt
The scaling factor (``1200'' in this example) is based on
1000 as a 1:1 ratio; in this example, it specifies that
the normal ten-point letters will be magnified by a factor
of 1.2000 to a size of twelve points.

Plain \TeX\ provides a series of standard magnification steps
that give a series of magnifications of 1.2 times.  They
are specified by a macro called |\magstep| as follows:
$$\vbox{\halign{\tt#\hfil&\hfil\quad#\quad\hfil&\quad#\hfil\cr
\hfil\sl specification&\hfil\sl value&\hfil\sl Changes 10pt to\cr
\noalign{\smallskip\hrule\smallskip}
\\magstep 0&1000&10pt (no change)\cr
\\magstephalf&1095&11pt\cr
\\magstep 1&1200&12pt\cr
\\magstep 2&1440&14pt (a little more)\cr
\\magstep 3&1728&18pt (a little less)\cr
\\magstep 4&2074&21pt (a little less)\cr
\\magstep 5&2488&24pt (a little more)\cr
}}$$
Thus, a specification like
\begintt
\font\elevenrm=cmr10 scaled \magstephalf
\endtt
produces a reasonable eleven-point roman font.

Finally, one can use magnifications to
enlarge the entire contents of
a paper (if all the fonts are available in the 
appropriate magnifications on your installation) by
saying |\magnification=|$n$ where $n$
is an integer such as 1000 (for no change) or can be
a |\magstep| macro.  Thus, if you say
|\magnification=\magstephalf| at the beginning of your paper,
its entire contents will be scaled up to eleven-point (well~$\ldots$
10.95-point) type.
Consult your local \TeX\ guru for a
list of the fonts and sizes available on your facility.
\SubSection{Ligatures}
As with \Troff/, many \TeX\ fonts provide ligatures that
automatically occur when certain consecutive characters are read.
For example, the letters~|ff| in the input will produce the
single glyph~`ff' on the page.
This is intended to improve the appearance of sentences such
as ``If you flex your fingers in a coffin, you can baffle a
giraffe.''
In \TeX, as with \Troff/, this happens automatically, and
the writer rarely needs to think about it.  However,
in \TeX, there are additional
special ligatures that deserve particular attention.
Two consecutive hyphens (|--|) produce an en-dash
(`--'; used for ranges such as `pages 13--34'); Three consecutive
hyphens (|---|) produce a regular punctuation dash (---).
Two consecutive closing single quote characters
(\'{}\'{}) will produce a
closing double-quote ('') while two consecutive opening single-quotes
(\`{}\`{} ---often
referred to on terminal keyboards
as `grave accent' characters) will produce
an opening double-quote (``).  The double-quote character |"| should
not be used.
Finally, two ``Spanish'' ligatures are available: the inverted
question mark `?`' can be typed as a question mark
followed by an open quotation mark(|?|\`{}), and the inverted
exclamation point `!`' can be typed as |!|\`{}.
These are {\sl not\/} control sequences, but are
a property built into the definition of each font.  Thus, they may
not occur in every available font (e.g. typewriter font), although
all the italic, bold, slanted and roman fonts can be expected
to behave this way.
