\Section{A Short Example}
\def\boxit#1{\vbox{\hrule\hbox{\vrule\kern3pt\vbox{\kern3pt#1
\kern3pt}\kern3pt\vrule}\hrule}}
\vfil
\centerline{\parfillskip=0pt\boxit{\boxit
{\vbox{\hsize 4in\tolerance 800\noindent
\advance\count0 by 1
Pages \number\count0
\advance\count0 by 1
\ and \number\count0
\ contain a short sample of \TeX\ input,
demonstrating some of the features discussed in this
document.  The result is on page
\advance\count0 by 1
\number\count0.  Note
that the page number in the sample output follows the
page numbering in this document, rather than being
`1' as it would be if the sample were run separately.
The reader should inspect closely the left and right
single quotes in the input, which can be rather hard to
distinguish in the `tt' font.}}}}
\vfil\eject
\begintt
\font\titletype=cmbx10 scaled \magstep 1 % 12-point boldface
\footline={\TeX\ Sample \hfil Page \folio}
\headline={\hfil{\bf S A M P L E}\hfil}
\vskip .25in
\line{\hfil Sample Output}  % Right-justified
\line{\hfil Feb 30, 1986}   % text due to \hfil
\vskip 12pt
\centerline{\titletype \TeX\ Sample}
\vskip 1cm
This material, which appears in 10-point type (possibly
magnified) , demonstrates
several features: (1)~Ties.  (2)~Floating Keeps for figures.
(3)~General syntax. (4)~Equations and tables.  (5)~Verbatim
mode using {\tt obeylines}.
\beginsection Footnotes and Accents.
|bigskip
This is a new paragraph.  It contains a footnote\footnote{*}
{Which doesn't tell you much}, and contains some fascinating
accent marks like the accent gr\`ave, and the accents in
phrases like ``\c c'est la vie'' and ``\^gis revido.''  This
paragraph also demonstrated the double-quote ligatures---which
are somewhat hard to read in the `tt' font---and the long
dashes.
|bigskip
Here's yet another new paragraph, this one
indented, since it is not at the beginning
of a section.  Notice that the interparagraph
spacing is zero, unlike the main body of the paper.
In Figure~1,
we see an equation from ``{\sl A System for Typesetting Mathematics\/}''
by Kernighan and Cherry.
\midinsert % A displayed Equation from Kernighan and Cherry
\def\emx{e^{mx}} % This is done with EQN defines in Kernighan and Cherry
\def\mab{m\sqrt{ab}}
\def\sa{\sqrt a}
\def\sb{\sqrt b}
$${\int\limits{dx\over{a\emx-be^{-mx}}}}=\cases{
{1\over{2\mab}}\log
{{\sa\emx-\sb}\over{\sa\emx+\sb}}&\cr
&\cr
{1\over{\mab}}\tanh^{-i}({\sa\over\sb}\emx)&\cr
&\cr
{-1\over{\mab}}\coth^{-i}({\sa\over\sb}\emx)&\cr
}$$
\centerline{Figure 1.  An Equation}
\endinsert
Furthermore, here is a table from M. Lesk's
``{\sl Tbl---A Program to Format Tables\/}'':
\endtt\vfil\eject\begintt
\vskip 10pt
\settabs\+Language\quad&Carnegie-Mellon\quad&\cr
\+\hfil Language\hfil & \hfil Authors\hfil & \quad Runs on\cr %centered
\+ \cr % Skip a line
\+ Fortran & Many & Almost anything\cr
\+ PL/1 & IBM & 360/370\cr
\+ C & BTL & 11/45,H6000,370\cr
\+ BLISS & Carnegie-Mellon & PDP-10,11\cr
\+ IDS & Honeywell & H6000\cr
\+ Pascal & Stanford & 370\cr
\endtt\begintt
\bigskip
Finally, here's an example of {\tt obeylines}:
{\obeylines\parindent=1in\parskip=0pt % Single-space, 1-inch indent
\endtt\begintt
This material
has no interparagraph gap
between the lines, but each
line is, in effect, a new paragraph} % End of Obeylines
\endtt\begintt
Now to finish up.
\bye
\endtt
\vfil\eject % Throw a new page.
{
\parskip=0pt plus 1pt
%\font\titletype=cmbx10 at 12 truept% 12-point boldface if magnified
\font\titletype=cmbx10 scaled \magstep 1 % 12-point boldface
\footline={\TeX\ Sample \hfil Page \folio}
\headline={\hfil{\bf S A M P L E}\hfil}
\vskip .25in
\line{\hfil Sample Output}  % Right-justified
\line{\hfil Feb 30, 1986}   % text due to \hfil
\vskip 12pt
\centerline{\titletype \TeX\ Sample}
\vskip 1cm
This material, which appears in 10-point type (possibly
magnified), demonstrates
several features: (1)~Ties.  (2)~Floating Keeps for figures.
(3)~General syntax. (4)~Equations and tables.  (5)~Verbatim
mode using {\tt obeylines}.
\beginsection Footnotes and Accents.

This is a new paragraph.  It contains a footnote\footnote{*}
{Which doesn't tell you much}, and contains some fascinating
accent marks like the accent gr\`ave, and the accents in
phrases like ``\c c'est la vie'' and ``\^gis revido.''  This
paragraph also demonstrated the double-quote ligatures---which
are somewhat hard to read in the `tt' font---and the long
dashes.

Here's yet another new paragraph, this one
indented, since it is not at the beginning
of a section.  Notice that the interparagraph
spacing is zero, unlike the main body of the paper.
In Figure~1,
we see an equation from ``{\sl A System for Typesetting Mathematics\/}''
by Kernighan and Cherry.
\midinsert % A displayed Equation from Kernighan and Cherry
\def\emx{e^{mx}} % This is done with EQN defines in Kernighan and Cherry
\def\mab{m\sqrt{ab}}
\def\sa{\sqrt a}
\def\sb{\sqrt b}
$${\int{dx\over{a\emx-be^{-mx}}}}=\cases{
{1\over{2\mab}}\log
{{\sa\emx-\sb}\over{\sa\emx+\sb}}&\cr
&\cr
{1\over{\mab}}\tanh^{-i}({\sa\over\sb}\emx)&\cr
&\cr
{-1\over{\mab}}\coth^{-i}({\sa\over\sb}\emx)&\cr
}$$
%
%$${\int\limits{dx\over{ae^{mx}-be^{-mx}}}}=\cases{
%{1\over{2m\sqrt{ab}}}\log
%{{\sqrt{a}e^{mx}-\sqrt{b}}\over{\sqrt{a}e^{mx}+\sqrt{b}}}&\cr
%&\cr
%{1\over{m\sqrt{ab}}}\tanh^{-i}({\sqrt{a}\over\sqrt{b}}e^{mx})&\cr
%&\cr
%{-1\over{m\sqrt{ab}}}\coth^{-i}({\sqrt{a}\over\sqrt{b}}e^{mx})&\cr
%}$$
\centerline{Figure 1.  An Equation}
\endinsert

Furthermore, here is a table from M. Lesk's
``{\sl Tbl---A Program to Format Tables\/}'':
\vskip 10pt
\settabs\+Language\quad&Carnegie-Mellon\quad&\cr
\+\hfil Language\hfil & \hfil Authors\hfil & \quad Runs on\cr %centered
\+ \cr % Skip a line
\+ Fortran & Many & Almost anything\cr
\+ PL/1 & IBM & 360/370\cr
\+ C & BTL & 11/45,H6000,370\cr
\+ BLISS & Carnegie-Mellon & PDP-10,11\cr
\+ IDS & Honeywell & H6000\cr
\+ Pascal & Stanford & 370\cr

\bigskip
Finally, here's an example of {\tt obeylines}:
{\obeylines\parindent=1in\parskip=0pt % Single-space, 1-inch indent

This material
has no interparagraph gap
between the lines, but each
line is, in effect, a new paragraph} % End of Obeylines

Now to finish up.
\vfil\eject
\headline={\line{\hss}}
\footline={\hss\folio\hss}
}
