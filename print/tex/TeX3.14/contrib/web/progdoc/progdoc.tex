% This is MAKEPROG.WEB                    as of 29 Nov 88
%---------------------------------------------------------
% (c) 1988 by J.Schrod.
%        Major parts of this program have been taken from TANGLE.WEB
%           by D.E.Knuth, which is not copyrighted.

%
% Version 1.0 was released in November, 1988.
%


\def\title{MAKEPROG (V\version)}
\def\version{1.0}
\def\years{1988}

\font\ninerm=cmr9
\let\mc=\ninerm % medium caps for names like EBCDIC

\def\hang{\hangindent 3em\indent\ignorespaces}
\def\item#1{                             % fixed item of 30pt width
   {\parskip=0pt\endgraf\noindent}%      % new line
   \hangindent 30pt %
   \hbox to 30pt{\hfil#1\kern 1em}%
   \ignorespaces                         % ignore following blanks
   }

\def\PASCAL{Pascal}
\def\MAKEPROG{\leavevmode\hbox{\mc MAKEPROG\/}}
\def\ASCII{\leavevmode\hbox{\mc ASCII\/}}
\mathchardef\BA="3224      % double arrow as relation

\def\topofcontents{\null\vfill
   \centerline{\titlefont The MAKEPROG processor}
   \vskip 15pt
   \centerline{(Version \version)}
   \vskip 2cm}
\def\botofcontents{\vfill
   \hangindent 3em
   \noindent $\copyright$ \years{} by J.~Schrod.\hfil\break
   All rights reserved.
   \par \vskip 1ex
   \noindent The source of this program may be used for
   noncommercial purposes, but credits must be given to the origin.
   \par}





@* Introduction.                                            % sect.    1

\noindent This program converts a documentation file to a
program file, i.e.\ all files between \.{\\beginprog} and
\.{\\endprog} are copied verbatim.  It was written by
J.~Schrod in September, 1987.  This program is written in
\.{WEB}.

The program uses a few features of the local \PASCAL\
compiler that may need to be changed in other installations:

\item{1)} Case statements have a default.

\item{2)} Input-output routines may need to be adapted for
use with a particular character set and/or for printing
messages on the user's terminal.

\noindent System-dependent portions of \MAKEPROG{} can be
identified by looking at the entries for `system
dependencies' in the index below.
@!@^system dependencies@>

\noindent The ``banner line'' defined here should be changed
whenever \MAKEPROG{} is modified.  The copyright notice must
not be changed.

@d banner=='This is MAKEPROG, Version 1.0.'
@d copy_right=='    (c) 1988 by J.Schrod.'
@d rights_res=='        All rights reserved.'


% from progdoc.doc   05 Dec 88
%------------------------------
% These lines were inserted by hand (not with MAKEPROG).
\chardef\escape=0
\chardef\letter=11
\chardef\other=12
%\chardef\active=13              % is defined in Plain already

\chardef\atcode=\catcode`\@
\chardef\uscode=\catcode`\_

\catcode`\@=\letter
\catcode`\_=\letter
\font\tentex=cmtex10            % typewriter extended ASCII 10pt
\let\ttex=\tentex               % only for PLAIN with base size 10pt

\def\setup_verbatim{%
   \def\do##1{\catcode`##1\other}\dospecials
   \parskip\z@skip \parindent\z@
   \obeylines \obeyspaces \obeytabs \frenchspacing
   \ttex
   }

\let\tab=\space
\begingroup
   \catcode`\^^I=\active%       % Attention: no tabs!
   \gdef\obeytabs{\catcode`\^^I=\active\def^^I{\tab}}
   \global\let^^I=\tab%         % if an active tab appears in a \write
\endgroup
\let\origvert=|
\chardef\vbar=`\|

\catcode`\|=\active

\def|{%
   \leavevmode
   \hbox\bgroup
      \let\par\space \setup_verbatim
      \let|\egroup
   }
\newif\if@print

\def\beginprog{%
   \endgraf
   \bigbreak
   \begingroup
      \setup_verbatim \catcode`\|\other
      \@printtrue
      \ignore_rest_line
   }
\let\end_verbatim=\endgroup             % internal command !
\begingroup
   \obeylines%          % ^^M is active! ==> every line must end with %
   \gdef\ignore_rest_line#1^^M{\set_next_line}%
   \gdef\set_next_line#1^^M{\do_set{#1}}%
\endgroup

\def\do_set#1{%
   \endgraf
   \check_print{#1}%
   \if@print  \indent \print_char#1\end_line\end_line
   \else  \let\set_next_line\end_verbatim
   \fi
   \set_next_line
   }
\let\end_line=\relax
\begingroup
\obeyspaces\obeytabs
\gdef\check_print#1{\cut_at_tab#1^^I\end_line}
\gdef\cut_at_tab#1^^I#2\end_line{\check_first_part#1 \end_line}
\gdef\check_first_part#1 #2\end_line{\do_check{#1}}
\endgroup
\def\do_check#1{%
   \def\@line{#1}%
   \ifx \@line\@endverbatim  \@printfalse
   \fi
   }

{\catcode`\|=\escape  \catcode`\\=\other % | is temporary escape char
   |gdef|@endverbatim{\endprog}
}                       % he \endgroup can't be used
\newcount\char_count  \char_count\@ne

\def\print_char#1#2\end_line{%
   \print_first_char{#1}%
   \print_rest_of_line{#2}%
   }
{\obeytabs\gdef\@tab{^^I}}

\def\print_first_char#1{%
   \def\@char{#1}%
   \advance \char_count\@ne
   \ifx \@char\@tab  \print_tab
   \else  \@char
   \fi
   }
\newcount\count_mod_viii
\def\mod_viii#1{%
   \count@ #1\relax  \count_mod_viii\count@
   \divide \count@ 8\relax
   \multiply \count@ 8\relax
   \advance \count_mod_viii -\count@
   }
\def\print_tab{%
   \loop  \space \mod_viii\char_count
      \ifnum \count_mod_viii>\z@
         \advance \char_count\@ne
   \repeat
   }
\def\print_rest_of_line#1{%
   \def\@line{#1}%
   \ifx \@line\empty  \char_count\@ne
        \def\next##1\end_line{\relax}%
   \else  \let\next\print_char
   \fi
   \next#1\end_line
   }
