% This is PROGDOCU.TEX                    as of 05 Dec 88
%---------------------------------------------------------
% (c) 1988 by J.Schrod. Put into the public domain.

%
% english documentation for the MAKEPROG system
% just one sheet---it is very primitive
% Plain TeX and typographically unpleasant.
%

%
% first version (for ftp/Bitnet)                                   (88-12-05)
%






%%%%%%%%%%%%%%%%%%%%%%%%%%%%%%%%%%%%%%%%%%%%%%%%%%%%%%%%%%%%%%%%%%%%%%%%%%
%
%     local macros
%

\chardef\letter=11
\chardef\other=12
\chardef\atcode=\catcode`\@         % save old catcode of @
\catcode`\@=\letter                 % for the usage of private macros



% fonts

\font\mc=cmr9                      % medium caps for acronyms
\font\sc=cmcsc10                   % caps and small caps 10pt
\font\tentex=cmtex10               % typewriter extended ASCII 10pt
\let\ttex=\tentex                  % only for PLAIN with base size 10pt
\font\ftnrm=cmr12 scaled \magstep1 % roman 14pt


% short hands

\def\LaTeX{{\rm L\kern-.36em\raise.3ex\hbox{\sc a}\kern-.15em\TeX}}
\def\MAKEPROG{{\mc MAKEPROG}}
\def\CWEB{{\mc CWEB\/}}
\def\WEB{{\tt WEB\/}}
\def\TANGLE{{\tt TANGLE\/}}
\def\WEAVE{{\tt WEAVE\/}}
\def\TIE{{\tt TIE\/}}
\def\DVI{{\tt DVI\/}}


% local formatting

\def\item#1{%                 % item with a fixed indention of 3em
   {\parskip\z@skip \par \noindent}% % parskip is restored again
   \hbox to 3em{\hfil #1\quad}%
   \hangindent 3em
   \ignorespaces
   }

% |...| verbatim from progdoc.doc. look there for the explanations.

\catcode`\|=\active
\def|{%
   \leavevmode
   \hbox\bgroup
      \let\par\space \setupverb@tim
      \let|\egroup
   }
\def\setupverb@tim{%
   \def\do##1{\catcode`##1\other}\dospecials
   \parskip\z@skip \parindent\z@
   \obeylines \obeyspaces \frenchspacing
   \ttex
   }



\catcode`\@=\atcode

%
%%%%%%%%%%%%%%%%%%%%%%%%%%%%%%%%%%%%%%%%%%%%%%%%%%%%%%%%%%%%%%%%%%%%%%%%%%


\centerline{\ftnrm The MAKEPROG System}
\vskip 2mm
\centerline{\sc Joachim Schrod}



\beginsection Introduction

D.~Knuth has introduced the concept of ``Literate Programming'' where
programmers should explain to potential human readers of their
programs what they want the computer to do.  To support this for
Pascal programs he has created the {\sl \WEB{} system of structured
documentation}, an extension of Pascal.  In \WEB{} a program is
splitted into sections, every section contains a documentation and a
program part (both parts can be empty).%
\footnote*{In fact there is a third part, the macro part, which is not
important in this context.}
Such a \WEB{} program is transformed by the \TANGLE{} processor into a
program source and the \WEAVE{} processor produces an output which can
be fed into \TeX{} to get a fine looking document.

At the Technical University of Darmstadt we have used this
concept---with \CWEB{}---in many of our projects, e.g.\ for our
portable \DVI{} driver family.  But for many programming languages a
\WEB{} system is not available and is often difficult to make; e.g.,
to prettyprint \TeX{} macros is a non-trivial task due to \TeX's
dynamic lexical analysis.  To stop this gap I have developed the
\MAKEPROG{} system.  It makes it possible to document programs with
\TeX{} in a \WEB{} like fashion, the program parts of the
documentation file can be extracted to yield the program file.  During
the extraction process the documentation file can be altered with
change files.

The \MAKEPROG{} system consists of two parts:  (1)~the \MAKEPROG{}
processor which does the extraction and (2)~the macro file |progdoc|
which makes formatting facilities available.  The \MAKEPROG{}
processor is derived from \TANGLE{}, therefore there should be no
difficulties for every site running \WEB{} to install \MAKEPROG{}.
The macros in |progdoc| are implemented with \MAKEPROG{}, the file
|progdoc.doc| is the definitive source.  |progdoc.tex| is only
delivered to allow the printing of |progdoc.doc|.

The following deficiencies are known to me (some are not inherent but
there wasn't the time to do it until now):

\item{---} \MAKEPROG{} does not rearrange the code as \TANGLE{} does.
This is the most important feature which is lacking.  The support of
stepwise refinements is one of \WEB{}'s main advantages.  But for a
few programming languages---e.g.\ \TeX{}---this is not so problematical
because their identifiers are dynamically bound at run time.

\item{---} \MAKEPROG{} does not prettyprint the program part because
it knows nothing about the host language and therefore nothing about
the lexical and syntactical structure of the program.  Instead it just
prints the program part verbatim.  But compared with wide spread
verbatim setting macros they have the advantage that you can print
your program part if you have embedded tabs in it.  The macros will
replace every tab with one to eight spaces according to the current
column.

\item{---} \MAKEPROG{} produces no index.  This is impossible because
it does not know what an identifier looks like.  (This can even change
inmidst a program, cf.\ \TeX!)

\item{---} Because the documentation file is a \TeX{} file and \TeX{}
does not recognize change files the complete documentation (with a
change file) cannot be printed.  \TIE{} must be used to create a new
master file which can be printed afterwards.

\item{---} The macros in |progdoc| do not produce a title page and a
table of contents.  This could be done easily.

\item{---} There is no \LaTeX{} version of |progdoc| available.  But
this is simple, too---look at the comments in |progdoc.doc|.

\item{---} The page breaking is not very lucky.  I still have to play
with the penalties.

\item{---} \MAKEPROG{} should insert the actual date as a comment line
in front of the produced program file. The syntax of the comment line
(start and end of a comment) must be specifiable.




\beginsection How To Use \MAKEPROG{}

The documentation is produced with the macro set |progdoc| built on
Plain \TeX{}.  Therefore the documentation file must start with
|\input progdoc|.  Afterwards you can structure your document into
sections and the sections into groups.  The first section of a group
starts with |\chap|, it corresponds to the starred section (`|@*|') of
a \WEB{} program.  |\chap| has one parameter, the title of the section
group.  The parameter is ended by a dot.  The dot is printed by the
macro.  Every other section starts with |\sect|.  These macros produce
a number in front of each section; this number is incremented with
each new section.

Within a section one or more program part(s) can be specified with the
macros |\beginprog| and |\endprog|.  Both macros must start at the
beginning of a line.  If |\beginprog| does not start at the beginning
of a line verbatim typesetting will be switched on but no extraction
to the program file will result afterwards.  After |\beginprog| the
rest of the line is ignored.  If |\endprog| does not start at the
beginning of a line or if it is not followed by white space (blanks,
tabs, or end of line) neither verbatim typesetting nor extracting will
stop.

Outside of the program part---in the so called documentation
part---you can use the vertical bar to print small texts verbatim,
e.g.\ identifiers, macro names, etc.  A vertical bar starts the
verbatim mode, the next vertical bar stops it.  You can use
|\origvert| to get an original vertical bar.  |\vbar| is the character
with the {\mc ASCII} representation of a vertical bar in the actual
font.

After you have finished writing your document you can print it with
\TeX{} and you can run the \MAKEPROG{} processor to extract all
program parts into a program file.  During the extraction \MAKEPROG{}
will recognize change files like \TANGLE{} does.




\beginsection Installation

The first step is to install the \MAKEPROG{} processor.  Because it is
derived from \TANGLE{} this should be rather easy.  Just take your
local \TANGLE{} change file and you should have very few alterations
(perhaps much to delete).  You will need to put the \MAKEPROG{}
processor somewhere where your local command processor will find
it---perhaps you will even need a command script around it.  But this
will be the same as it was with \TANGLE{}.  By the way, you can
contact me if you need change files for MS-Pascal~V3.11 or higher, for
the HP-Pascal compiler on a HP-UX machine, or for the Pascal compiler
on a PCS~Cadmus System, contact me.  Also you can get an object file
for an Atari~ST from me.

You can test your new program by running |progdoc.doc| through it.
The output must be identical to |progdoc.tex|.  Well, the main work is
now done.  You still have to put |progdoc.tex| in a directory where
\TeX{} will find it and then have much fun.  (Don't worry---be
happy$\,\ldots$)




\beginsection Errors and Remarks

If you have found an error or if you have some remarks or suggestions
please contact me.  My Bitnet address is |XITIJSCH@DDATHD21| but please
note that you perhaps have to mail me again because our connection is
very instable and incoming mails are often lost.
{\it I will acknowledge every mail within a week}.

\medskip

\noindent You can also reach me with the old fashioned mail:

\smallskip

\begingroup
\obeylines \parskip=0pt \parindent=3em
Detig$\,\cdot\,$Schrod \TeX{}sys OHG
Joachim Schrod
Kranichweg 1
\vskip .5\baselineskip
D-6074 R\"odermark-Urberach
FR Germany
Phone: $(+60\,74)~16\,17$
\endgroup





\beginsection Distribution

You can give the \MAKEPROG{} system to everyone you want but it must
be the complete system, i.e.\ at least the three files |makeprog.web|,
|progdoc.doc|, and |progdocu.tex|.  {\it It is explicitly forbidden to
pass on the system without the documentation---if you distribute\/
|progdoc.tex| without |progdoc.doc| your bad conscience will torture
you until eternity.} Furthermore this restriction for distribution
must be told to everyone who gets the \MAKEPROG{} system from you.

Of course I do not make any warranties.  (The usual blablah should
follow here.)



\bye

%% That's it. Excuses to everybody out there who {\it can} write english...

