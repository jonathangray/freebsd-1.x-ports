
  \documentstyle[12pt]{report}
  \nofiles                          
  \def\LATEX{\LaTeX}
  \let\TEX = \TeX               
  \setcounter{totalnumber}{4}   
  \setcounter{topnumber}{2}     
  \setcounter{bottomnumber}{2}
  \renewcommand{\topfraction}{.5}
  \renewcommand{\bottomfraction}{.5}
  \setlength{\oddsidemargin}{3.9cm}     %real measurement 1.5in
  \setlength{\textwidth}{5.7in}         %right margin is now 1in
  \setlength{\topmargin}{1cm}
  \setlength{\headheight}{.6cm}
  \setlength{\textheight}{8.5in}
  \setlength{\parindent}{1cm}
  \renewcommand{\baselinestretch}{1.5}
  \raggedbottom
  \input{init.tex}
  \input{bonds.tex}
  \input{six.tex}
  \input{cright.tex}
  \input{cdown.tex}
  \input{tbranch.tex}
  \input{cto.tex}
  \input{cbranch.tex}
  \input{hetthreea.tex}
  \input{threering.tex}
  \begin {document}    
  \setcounter{page}{11}
  \textfont1=\tenrm
  \initial
  \setcounter{chapter}{3}
 
 \centerline{CHAPTER III}
 \vspace{0.4cm}
 \centerline{TEX/LATEX CODE FOR COMPONENTS OF ORGANIC}
 \centerline{CHEMICAL STRUCTURE DIAGRAMS}
 \vspace{0.4cm}
 \centerline{1. CONVENTIONS FOR DRAWING THE DIAGRAMS}
 \vspace{0.4cm}

 The chemical structure of a molecule is defined by the spatial
 arrangement of the atoms and the bonding between them.
 Chemists use several standard methods for representing the
 structures two-dimensionally by diagrams called structural
 formulas; and this thesis will develop mechanisms for printing
 such diagrams using the TeX/LaTeX system.

 A very common structure representation, sometimes called a
 dash structural formula, uses the element symbols for the 
 atoms and a dash for each covalent bond in the compound.
 Thus the dash represents the pair of shared electrons that
 constitutes the bond. Two dashes ($=$) represent a double
 bond and three dashes ($\equiv $) a triple bond. ---
 It is usually neither necessary nor practical to represent
 each bond in a molecule explicitly by a dash. 
 Some molecules and some bonds are so common that a complete
 dash formula would not be used except at a very introductory
 level of presenting chemical information. 
 A condensed structural formula is one alternative. It does
 not contain dashes but uses the convention that atoms
 bonded to a carbon are written immediately after that
 carbon and otherwise atoms are written from left to right
 in the order in which they occur in the real structure.
 The following two structural formulas are a dash formula
 and a condensed formula, respectively, for the same
 compound, ethanol.  
 \vspace{-0.5cm}
 \[ \parbox{4.5cm} {
    \begin{picture}(400,900)(0,-110)
     \put(0,0)   {\cbranch{H}{S}{H}{S}{C}{S}{}{S}{H} }
     \put(240,0) {\cbranch{H}{S}{}{Q}{C}{S}{O---H}{S}{H} }
    \end{picture}  }
    \hspace{1.5cm}
    {\rm CH_{3}CH_{2}OH}  \]
 \newpage
 Multiple bonds are usually not implied unless a very common
 group, such as the cyano group, is shown. It can be found
 as -C$\equiv $N or simply as -CN.

 Another alternative to a complete dash formula is a diagram
 where the symbols for carbon and for hydrogen on carbon are
 not shown. Each corner and each open-ended bond in these
 diagrams implies a carbon atom with as many hydrogen
 atoms bonded to it as there are free valences. This
 representation is the customary one for ring structures
 (structures with a closed chain of atoms). Thus, the
 following two diagrams both represent the compound
 cyclopropane.
 \[ \hetthree{Q}{H}{H}{H}{H}{S}{S}{C}
    \hspace{3cm}  \yi=330
    \threering{Q}{Q}{Q}{Q}{Q}{Q}{Q}{Q}{Q} \]

 \reinit
 The three different kinds of structure representation can
 be combined in one diagram, such that in part of the
 diagram all bonds are represented by dashes and all
 atoms by an element symbol, in another part a condensed
 structural formula fragment is used, and in still
 another part a cyclic fragment with implied carbon
 and hydrogen atoms occurs.

 The rest of this chapter describes how LaTeX can be used
 to position and typeset the bond lines and condensed
 formula strings that are the components of structure
 diagrams.

 It should be mentioned that there are no binding rules
 for many aspects of the two-dimensional representations
 of a chemical structure. Structures and fragments of
 structures can be oriented in different ways depending
 on the availability of space, the emphasis given to
 a certain part of a structure, or the spatial
 relationship of the parts to each other.
 Thus, a cyclopropane ring can be represented in various
 orientations, $\bigtriangleup $, $\bigtriangledown $,
 and others. Also, the angles between
 the bond lines can be different in different representations
 of one and the same compound. Since most molecules do not
 have all their atoms lying in one plane it would not 
 even be possible to reproduce all bond angles in a   
 two-dimensional representation. The structures shown in    
 this thesis adopt the orientations and bond angles
 found to prevail in Solomons' textbook (Solomons 84),
 the organic chemistry text used for several years at
 the University of Tennessee.
 
 There are some methods to indicate the real,
 three-dimensional structure (the stereochemistry)
 of a molecule in the two-dimensional representation:
 A dashed line and a wedge instead of a full bond
 line mean that the real bond extends below or
 above the plane, respectively. 
 
 \vspace{4mm}
 \centerline{2. BOND LINE DRAWING AND POSITIONING}
 \vspace{0.4cm}
 \begin{flushleft}
  \underline{A. Review of TeX/LaTeX Facilities for Line-Drawing}
 \end{flushleft}
 
 The easiest way to produce horizontal and vertical lines representing
 chemical bonds is by the use of keyboard characters and simple control
 sequences provided by TeX.  By typing one, two, or three hyphens,
 a normal hyphen, a medium dash designed for number ranges, and a
 punctuation dash are produced, - -- ---, respectively. When a hyphen
 is typed in TeX's math mode, it is interpreted as a minus sign and
 the spacing around it will be different from text mode. ---
 The equal sign can represent a double bond for chemistry typesetting.
 It can be typed in text mode and in math mode, again resulting in   
 different spacing around the symbol. --- The control sequence
 \verb+\+equiv can be used as a triple bond ($\equiv $). It has to
 be typed in math mode.
 
 Vertical lines are available through the keyboard character or the
 control sequences \verb+\+vert and \verb+\+mid, all three to be
 entered in math mode.
 A double vertical bar is produced by \verb+\+$|$ or \verb+\+Vert,
 again both in math mode.

 The spacing around all these symbols can be controlled by adding 
 extra (positive or negative) space with the horizontal spacing
 commands. The symbols, just as any other part of a line, can also
 be raised or lowered respective to the normal baseline. The length
 and height of the symbols however depend on the font currently 
 in use.

 Where control of length and height of the bond lines is needed,
 TeX's or LaTeX's command sequences for printing horizontal and 
 vertical ``rules'' can be used.  The systems recognize several
 length units, including the inch, centimeter, millimeter, and
 printer point (Knuth 84, p. 57). One printer point (pt), an often
 used unit in typesetting, measures about 0.35 mm. --- LaTeX's
 rule-printing command has the format \\                    
 \centerline{$\backslash $rule[raise-length] \{width\}
 \{height\} . }
 Thus it can be used to produce horizontal and vertical rules.
 Using the \verb+\+rule command one can also print multiple
 bond lines of user-controlled length, e.~g.  $\dbond{16}{19} $,
 $\tbond{16}{20} $, with the short control sequences \verb+\+dbond
 and \verb+\+tbond defined in this thesis. The vertical spacing
 between the bonds depends on the current line spacing in the
 document and may have to be adjusted. The control sequences
 are set up for math mode.
 
 When bond lines other than horizontal and vertical ones are to
 be printed, and when a coordinate system is needed to control
 placement of structure components relative to one another, 
 LaTeX's picture environment (Lamport 86, pp. 101-111) is a
 necessity. 

 A picture environment uses length units which are dimensionless
 and have to be defined by the user before entering the 
 environment. This is done by the \verb+\+setlength command.
 In this study, \verb+\+setlength \{\verb+\+unitlength\}  
 \{0.1pt\} is the definition used for most diagrams. Such a small
 unitlength was chosen to have fine control over the appearance
 of the diagram.

 The picture environment starts with the statement\\
 \centerline{$\backslash $begin\{ picture\} (width, height) }
 where picture width and height reserve space on the page 
 and are specified in terms of unitlengths. Optionally, one
 can include the coordinates of the lower left corner of the
 picture: \\
 \centerline{$\backslash $begin\{ picture\} (width, height)
 (${\rm x_i\mbox{,}y_i}$).}
 The default value for these coordinates is (0,0).---
 Objects are placed into the picture with the \verb+\+put
 command with their reference point at the coordinates (x,y):    
 \verb+\+put(x,y) \{ picture object\} .

 The picture objects of most interest to this study are
 straight lines. They are drawn by the \verb+\+line
 command:\\
 \centerline{$\backslash $line(${\rm x_s\mbox{,}y_s}$) \{length\} }
 where the coordinate pair specifies the slope of the line, 
 and the nonnegative value of length specifies the length
 of the projection of the line on the x-axis for all 
 nonvertical lines, and the length of the line for vertical
 lines. The reference point of a line is one of its ends. 
 Thus the statement \\
 \centerline{$\backslash $put(x,y) 
  \{$\backslash $line(${\rm x_s\mbox{,}y_s}$) \{len\} \}  }
 draws a line that begins at (x,y), has a slope of ${\rm y_s\mbox{/}x_s}$,
 and extends for length len as explained above.

 Only a limited number of slopes is available through the line
 fonts in LaTeX. The possible values for ${\rm x_s}$ and ${\rm y_s}$ are
 integers between -6 and +6, inclusive. These values translate
 into 25 different absolute angle values, which are listed  
 in appendix C.
 
 \vspace{0.4cm}
 \begin{flushleft}
  \underline{B. Bonds in Structural Formulas Written on One Line}
 \end{flushleft}
 
 The application of some of the bond-drawing mechanisms for this
 simplest type of structural diagrams is illustrated in figure 3.1.

 \begin{figure}\centering
  \begin{picture}(900,900)
   \put(0,700)  {3.1a \ $CH\equiv C-CH=CH_{2}$}
   \put(0,450)  {3.1b \ $CH$\raise.1ex\hbox{$\equiv$}$C-CH=CH_{2}$}
   \put(0,200)  {3.1c \ $CH\tbond{14}{20} C\sbond{14} 
                         CH\dbond{14}{19} CH_{2}$}
  \end{picture}
  \caption{One-line structural formulas}
 \end{figure}

 For figure 3.1a only keyboard characters and the TeX command
 \verb+\+equiv were used to produce the bonds. Figure 3.1b
 shows a slight improvement through raising the triple bond.
 Figure 3.1c was printed using the \verb+\+sbond, 
 \verb+\+dbond, and \verb+\+tbond command sequences from
 this thesis, choosing a length of 14 pt for the bonds.
 It can be seen that each of the formulas in figure 3.1 is a
 creditable representation of the structure. Depending on the
 design of the page, the reason for displaying the structure
 at a particular place, and the emphasis put on features of the
 structure in the text, one would choose shorter or longer
 bonds and take more or less trouble to produce the structure.

 The picture environment is not needed for one-line structural
 formulas, unless one of these formulas has to be attached to
 another structural fragment, as in figure 3.2. Then the 
 coordinate system of the picture environment makes it
 possible to fit the two fragments together (see chapter~V for
 details).
 
 % figure 3.2
 \begin{figure}[h]
  \hspace{5cm}
  \parbox{70 pt}  {
   \begin{picture}(400,200)
    \put(-155,0) {$CH_{3}-CH-CH_{2}-CH_{2}-CH_{2}-CH_{3}$}
   \end{picture}  }

  \hspace{5cm}   \yi=200  \pht=600
  \sixring{Q}{Q}{Q}{Q}{Q}{}{D}{D}{D} \\
  \caption{One-line structure in picture environment}
 \end{figure}
 \reinit

 \vspace{0.4cm}
 \begin{flushleft}
  \underline{C. Bonds in Acyclic Structures with Vertical Branches}
 \end{flushleft}
 
 Structure diagrams with vertical, single- or double-bonded, branches,
 going up or down, are frequently seen. Several experiments with TeX
 and LaTeX were made to see how this type of structure can be
 handled. One method is to align the vertical bonds by using the
 mechanisms for tabbing or for printing tables and matrices.
 Here a structure such as the one shown in figure 3.3 is treated
 as a set of columns as indicated by the vertical dividing lines
 drawn into the second version of this structure in figure 3.3.

 % figure 3.3
 \begin{figure}
  \hspace{1cm}
  \begin{minipage}{180pt}
  \begin{tabbing}
   $CH_{3}CH_{2}$\= $CH$\= $CHCH_{2}$\= $CHCH_{2}CH_{3}$\+ \kill
                    $Br$\>           \> $CH_{3}$       \\ [-10pt]
    \hspace{2pt}$\vert $\>           \> \hspace{2pt}$\vert $ \- \\ [-9pt]
   $CH_{3}CH_{2}$\> $CH$\> $CHCH_{2}$\> $CHCH_{2}CH_{3}$\+ \+ \\ [-9pt]
     \hspace{2pt}   $\vert $                                  \\ [-9pt]
                    $CH_{2}CH_{3}$
  \end{tabbing}
  \end{minipage}
  \hspace{2.5cm}
 \begin{minipage}{180pt}         
  \begin{tabbing}
   $CH_{3}CH_{2}$\= $\vert CH$\= $\vert CHCH_{2}$\= $\vert CHCH_{2}CH_{3}$
                                                     \+ \kill
                    $\vert Br$\> $\vert $        \> $\vert CH_{3}$
                                                    \\ [-10pt]
   $\vert $\hspace{2pt}$\vert $\> $\vert $       \> $\vert $\hspace{2pt}
                                                    $\vert $ \- \\ [-9pt]
   $CH_{3}CH_{2}$\> $\vert CH$ \> $\vert CHCH_{2}$\> $\vert CHCH_{2}CH_{3}$
                                                    \+ \\ [-9pt]
         $\vert $\> $\vert $\hspace{2pt}$\vert $  \> $\vert $
                                                    \\ [-9pt]
         $\vert $\> $\vert CH_{2}CH_{3}$
   \end{tabbing}
   \end{minipage}
   \caption{Vertical branches}
  \end{figure}

 The structure diagram in figure 3.3 uses \verb+\+vert for the vertical
 bonds and LaTeX's tabbing environment for the alignment. One can also
 use ``rules'' as the vertical bonds in order to give the horizontal
 and vertical bonds the same lengths. Furthermore, vertical bonds can
 also be double bonds. The following examples illustrate these features.

 \[ \tbranch{O}{D}{H_{2}N-}{C-NH_{2}}{}{}{13} \hspace{2cm}
    \tbranch{}{}{CH_{3}-CH_{2}-}{C-CH_{3}}{D}{NH}{13} \hspace{2cm}
    \tbranch{}{}{H-}{C=\ }{S}{Br}{13}\tbranch{}{}{}{C-H}{S}{Br}{13} \]

 Similar structures were also generated with TeX's \verb+\+halign
 mechanism which forms templates for the columns rather than setting
 tab stops. For the purpose of printing the structure diagrams, no
 clearcut advantage was seen in one or the other method of
 alignment. In each case the vertical spacing depends on the line
 spacing in the document.
 
 The alternative method of producing these structures is the use
 of the picture environment. It provides better control over
 horizontal and vertical spacing and over bond lengths. Also,
 as illustrated in section 2B. of this chapter,
 using a picture environment makes
 it possible to attach one structural fragment to another at 
 a specific place. Thus, although the picture environment is not
 necessary for drawing structures with vertical branches, it 
 has several advantages, and writing LaTeX code for this
 implementation is not more difficult than writing the code
 for the tabbing method of alignment. 
 \newpage

 \begin{flushleft}
  \underline{D. Bonds in Structures Containing Slanted Bond Lines}
 \end{flushleft}
  
 Structure diagrams with slanted bond lines are frequently used for
 acyclic compounds and have to be used to depict almost all cyclic
 structures. Two examples are shown here:

 \[ \cdown{$CH_{3}$}{S}{$N^{+}$}{D}{$O$}{S}{$O^{-}$}
    \hspace{3cm} \sixring{$COOH$}{$OCOCH_{3}$}{Q}{Q}{Q}{Q}{S}{S}{C} \]
 
 In developing diagrams for such structures in this thesis the 
 conventions described in section 1. of this chapter are followed. 
 Thus the symbol for carbon is not
 printed for the carbons that are ring members, but it is usually 
 printed in acyclic structures, unless the acyclic structure fragment
 is a long chain, or space for the diagram is limited.

 The picture environment is always needed for slanted lines. It was
 explained in section 2A. of this chapter that LaTeX can draw lines
 only with a finite number of slopes. This is not a severe limitation
 for creating the structure diagrams, since the conventions for
 structure representation allow variations in the angles.
 The representation does not have to reflect the true
 atomic coordinates. In fact many chemistry publications contain 
 structure diagrams with angles significantly deviating from the real
 bond angles, even where those could have been used easily. Thus,
 Solomons' text (Solomons 84) 
 shows the carboxylic acid group often in this form \\
 \pht=600
 \[ \cright{}{S}{C}{D}{O}{S}{OH} \]
 \pht=900
 with an angle of about $90^0$ between the OH and doublebonded O,
 whereas the true angle is close to $120^0$. --- The angles used
 in this thesis for the regular hexagon of the sixring deviate by
 % \parbox{4mm}{+\vspace{-18pt}\\ $-$}~$1^0$ from $120^0$ 
  $\pm 1^{0}$ from $120^{0}$
  because of LaTeX's limited
 number of slopes. This difference is not big enough to be
 detected as a flaw.
 
 To write the LaTeX statement for a slanted bond line, one chooses the
 origin and the slope and then uses trigonometric functions to calculate
 the LaTeX ``length'' of the line for the desired real length. Once the
 LaTeX length is determined, the coordinates of the end point of the 
 line can be calculated in case the end point is needed as the origin
 of a connecting line. --- The origin and length of slanted double
 bonds were also calculated with standard methods from trigonometry.
 As an example, figure 3.4 shows how coordinates of the origin were
 calculated for the inside part of a ring double bond that is at a
 distance d from the outside bond.

 \setlength{\unitlength}{1pt}   % figure 3.4
 
 \begin{figure}[b]
  \begin{picture}(300,250)(0,-100)
   \thicklines
   \put(0,0)       {\line(5,3)  {120}}
   \put(120,72)    {\line(5,-3) {120}}
   \put(240,0)     {\line(0,-1) {100}}
   \put(215,-6)    {\line(-5,3) {88}}
   \thinlines
   \put(120,72)    {\circle*{4}}
   \put(125,72)    {($x$,$y$)}
   \put(127,47)    {\circle*{4}}
   \put(132,47)    {($x_d$,$y_d$)}
   \put(120,72)    {\line(0,-1) {16}}
   \put(120,72)    {\line(-3,-5){9}}
   \put(111,56)    {\line(1,0)  {16}}
   \put(127,56)    {\line(0,-1) {9}}
   \put(111,56)    {\line(5,-3) {16}}
   \put(111,62)    {\scriptsize d}
   \put(116,46)    {\scriptsize d}
   \put(112,17)    {{\small $\theta =30^{0}$}}
   \put(114,28)    {\vector(0,1){27}}
   \put(270,35) {$x_{d}=x-d\sin ${\small $\theta $}$+d\cos ${\small $\theta $}}
   \put(270,5)  {$y_{d}=y-d\sin ${\small $\theta $}$-d\cos ${\small $\theta $}}
  \end{picture}
  \caption{Calculating position and length of double bond.}
 \end{figure}

 \reinit  

 The LaTeX command \verb+\+multiput is similar to \verb+\+put and provides
 a shortcut for the coding of structures where several bond lines of the
 same slope and length occur at regular intervals. Multiput has the
 format\\ 
 \centerline{$\backslash $multiput(x,y)(${\rm \Delta x\mbox{,}\Delta y}$) 
   \{n\}\{object\} ,  }
 where n is the number of objects, e. g. lines. A structure diagram
 for which several \verb+\+multiput statements are appropriate is
 the structure of vitamin A shown in figure 3.5. 

 \begin{figure}[b]
  \hspace{2cm}
  \parbox{5cm}   {
   \begin{picture}(900,900)(-300,-300)
    \put(342,200)   {\line(0,-1)  {200}}
    \put(342,0)     {\line(-5,-3) {171}}
    \put(171,-103)  {\line(-5,3)  {171}}
    \put(0,0)       {\line(0,1)   {200}}
    \put(0,200)     {\line(5,3)   {171}}
    \put(171,303)   {\line(5,-3)  {171}}
    \put(322,180)   {\line(0,-1)  {160}}
    \put(342,0)     {\line(5,-3)  {128}}
    \put(171,303)   {\line(5,3)   {128}}
    \put(171,303)   {\line(-5,3)  {128}}
    \multiput(342,200)(342,0){5}{\line(5,3){171}}
    \multiput(513,303)(342,0){4}{\line(5,-3){171}}
    \multiput(527,270)(342,0){4}{\line(5,-3){135}}
    \multiput(855,303)(684,0){2}{\line(0,1){160}}
    \put(1881,275){=O}
   \end{picture}   }
   \caption{Diagram using $\backslash $multiput}
 \end{figure}

 The size of objects in a picture environment can be scaled in a simple
 way by changing the unitlength. Figure 3.6 illustrates scaling and 
 two problems associated with it. Changing the unitlength changes the
 length of the lines only, not the width of the lines or the size of   
 text characters. Thus ``it does not provide true magnification and
 reduction'' (Lamport 86, p. 102). However, the size of the text
 characters can be varied separately, as will be discussed in the
 next section of this chapter.      
 \pht=750
 % figure 3.6
 \begin{figure}[b]\centering
   \setlength{\unitlength}{.07pt}
   \sixring{$OH$}{Q}{Q}{Q}{Q}{$Br$}{S}{D}{S}
   \hspace{1.5cm}
   \setlength{\unitlength}{0.08pt}
   \sixring{$OH$}{Q}{Q}{Q}{Q}{$Br$}{S}{D}{S}
   \hspace{1.5cm}  \yi=150
   \setlength{\unitlength}{0.15pt}
   \sixring{$OH$}{Q}{Q}{Q}{Q}{$Br$}{S}{D}{S}
   \caption{Scaling (unitlength=0.07pt, 0.08pt, 0.15pt)}
 \end{figure}

 \reinit
 The smallest diagram in figure 3.6 illustrates a limitation that
 is unfortunate for the printing of structure diagrams. The shortest
 slanted line that can be printed by LaTeX's line fonts is
 one with an x-axis projection of about 3.6 mm.
 If a shorter slanted line is requested, LaTeX just prints
 nothing. A chemist would occasionally want to draw shorter lines,
 especially for the purpose of generating dashed lines indicating
 stereochemical features.

 \vspace{0.4cm}   
 \centerline{3. ATOMIC SYMBOLS AND CONDENSED STRUCTURAL FRAGMENTS}
 \vspace{0.4cm}
 Special considerations for the printing of condensed structural
 fragments are required since many of them contain subscripts.
 TeX considers the printing of subscripts a part of mathematics 
 typesetting which has to be done in the special math mode.
 It was pointed out in chapter I that typesetting of mathematics
 documents is one of the strong points of TeX; the fonts of type 
 for the math mode are designed to agree with all conventions
 of high quality mathematics publishing. Each typestyle in math
 mode consists of a family of three fonts (Knuth 84, p. 153),
 a textfont for normal symbols, a scriptfont for first-level
 sub- and superscripts, and a scriptscriptfont for higher-level
 sub- and superscripts. When structural fragments such as
 ${\rm C_{2}H_{5}}$ are typeset, the textfont is used
 for the C and the H.

 As TeX enters math mode it selects textfont1 as the textfont
 unless otherwise instructed. Textfont1 is defined by the TeX
 macros as math italic, a typestyle that prints letters (not
 numbers) similar to the italic style, but with certain 
 features adapted for mathematics typesetting. The italic
 style letters, lower and upper case, are the ones commonly
 seen in typeset mathematical formulas. Chemical formulas
 on the other hand are not usually printed with slanted
 letters. In this thesis, basically two methods were employed
 to produce chemistry-style letters in TeX's math mode which
 has to be used because of the presence of subscripts.
 
 For a document that contains many chemical formulas it is
 convenient to redefine textfont1 at the beginning of the
 TeX input file. The statement \newline \verb+\+textfont1=\verb+\+tenrm
 was used at the beginning of this document and causes TeX
 to select the roman font as the textfont in math mode.
 The roman typestyle is the one normally used by TeX
 outside of math mode and it is the style in which this
 thesis is printed. The tenrm style, which is slightly
 smaller than the twelverm size of the text in this document,
 was chosen because it appears to look better for the
 chemical formulas which consist largely of capital letters.
 When different typesizes are used in this way, all the 
 atomic symbols and formulas in any one structure, even
 those without subscripts, have to be printed in math mode
 so that they all have the same size. --- It could be a       
 problem with this method of selecting the roman font for
 math mode that the lowercase Greek letters (and some other
 symbols used in mathematics) are not available in this
 font. To print these one can temporarily redefine
 textfont1 to math italic with the statement
 \verb+\+textfont1=\verb+\+tenmi. One can also switch to
 a math font different from the default textfont1.
 Using one of LaTeX's font definitions, \verb+\+small,
 a statement \{\verb+\+small\$\verb+\+theta \$\} will
 print the Greek letter.
 
 Another method for avoiding the math italic style for letters
 in chemical formulas is to select the roman style in each
 individual instance where a formula has to be printed in
 math mode. A statement such as \$\{\verb+\+rm C\_2H\_5\}\$
 produces ${\rm C_{2}H_{5}}$ at the size of type currently used
 in the document. When the typestyle is thus selected within
 math mode, enclosed by dollar signs, TeX changes the style
 of the letters of the alphabet only; the lowercase Greek
 letters and math symbols remain available.

 The size of the letters in chemical formulas can be changed
 with the ten size declarations provided by LaTeX (Lamport 86,
 p. 200) or with TeX's declarations. (Some of TeX's declarations
 are not defined in LaTeX (Lamport 86, p.205)). The size
 declaration has to be written outside of math mode.
 One place in chemistry typesetting where a 
 smaller typesize is desirable is the writing on reaction
 arrows. The size in the following example is scriptsize:
 \newpage

 \advance \yi by 100
 \[ HC\equiv CH + H_{2}O   
  \parbox{92pt} {\cto{Hg^{++}}{18\%\ H_{2}SO_{4},\ 90^{0}}{14}}     
                                         CH_{3}-CHO \]

 Finally, condensed structural formulas sometimes have to be
 right-justified to be attached to the main structural diagram.
 Figure 3.7 illustrates this for the positioning of the
 substituent in the 4-position of the pyrazole ring. LaTeX 
 makes this positioning convenient with the \verb+\+makebox
 command, especially in the picture environment where the command
 has the format \\
 \centerline{$\backslash $makebox(width,height)[alignment]\{content\} }
 (Lamport 86, p. 104). The one-line piece of text that constitutes
 the content of the (imaginary) box can be aligned with the
 top, bottom, left side, or right side of the box.
 \reinit

 \begin{figure}[h]\centering
  \parbox{\xbox pt}             {
   \begin{picture}(\pw,\pht)(-\xi,-\yi)
    \put(200,-84)        {\line(5,3)    {110}}        % bond 1,2
    \put(342,200)        {\line(0,-1)   {140}}        % bond 3,2
    \put(342,200)        {\line(-1,0)   {342}}        % bond 3,4
    \put(0,200)          {\line(0,-1)   {200}}        % bond 4,5
    \put(0,0)            {\line(5,-3)   {140}}        % bond 5,1
    \put(135,-130)       {$N$}                          % N-1 in ring
    \put(310,-30)        {$N$}                          % N-2 in ring
     \put(171,-137) {\line(0,-1)   {83}}         % subst. on
          \put(150,-283) {$C_{6}H_{5}$}                      %  on N-1
     \put(370,-17)  {\line(5,-3)   {100}}        % subst. on
          \put(475,-100) {$C_{6}H_{5}$}                      %  N-2
     \put(335,211)  {\line(5,3)  {128}}   % outside
                 \put(349,189)  {\line(5,3)  {128}}   %  double O
                 \put(475,250)  {$O$}                   %  on C-3
     \put(0,200)    {\line(-5,3)   {128}}        % single subst.
          \put(-430,234) {\makebox(300,87)[r]{$CH_{3}COCH_{2}CH_{2}$}}
     \put(-7,11)    {\line(-5,-3){128}}   % outside
                 \put(7,-11)    {\line(-5,-3){128}}   %  double O
                 \put(-200,-130){$O$}                   %  on C-5
  \end{picture}              }     % end pyrazole macro
  \caption{Right-justification of substituent formula}
 \end{figure}


 \end{document}

