\input eplain

\headline = {headline\hfil}
\topinsert
topinsert
\endinsert

\parskip = 0pt

\centerline{Xiaogen Yang and Max L. Deinzer}
\smallskip
\centerline{Department of Agricultural Chemistry}
\centerline{Oregon State University, Corvallis, Oregon 97331}

\doublecolumns
Humulene monoepoxides exist in hop essential oil and were suggested
as one of the important contributors of hop flavor to fermented
malt beverages [1-3]. However, it is more possible that the ``noble
hop'' aroma compounds are produced from certain hop components
during the brewing process, because the ``noble hop'' aroma is
distinct from ``dry hop'' aroma. The aging of hops, during which the
amount of oxidation products of humulene increases [4], is also
necessary to develop the ``noble hop'' aroma. One approach to find
out the flavor compounds and their origin is to examine the brewing
products of the oxygenated sesquiterpenes. The reactions of the
brewing process can be simplified to hydrolyzation and fermentation
as the first step. We have been investigated the hydrolysis of
humulene monoepoxides.
\singlecolumn
To simulate the hydrolyzation effect in the ``late hop'' brewing
process, humulene monoepoxides were boiled in water at
pH\thinspace 4 for 10 minutes. After boiling, a large amount of
the epoxides remain unreacted. When the solution was kept at
ambient temperature for several weeks, humulene epoxides were
then almost completely hydrolyzed. To accelerate the hydrolysis
process, humulene epoxides were suspended in a aqueous solution
buffered at pH\thinspace 4 and boiled for three hours under
reflux. All three humulene monoepoxides produced a complex
mixture after the reaction (Figure 1). The hydrolysis products
can be easily separated into two groups by extraction with
pentane and dichloro\-me\-thane subsequently. The pentane
extracts are less polar, more volatile and smell stron\-ger than
the dichloro\-methane extracts. Sensory evaluation of the
hydrolysis products of humulene epoxide II and III by sniffing
gas chromatography effluent indicates that some of the compounds
have the flavor notes which are close to the ``noble hop'' aroma
character described as ``spicy, citrus, floral'' (Table 1).

\bye
