%%@texfile{%
%% filename="pref.tex",
%% version="1.1",
%% date="21-JUN-1991",
%% filetype="AMS-LaTeX: documentation",
%% copyright="Copyright (C) American Mathematical Society, all rights
%%   reserved.  Copying of this file is authorized only if either:
%%   (1) you make absolutely no changes to your copy, including name;
%%   OR (2) if you do make changes, you first rename it to some other
%%   name.",
%% author="American Mathematical Society",
%% address="American Mathematical Society,
%%   Technical Support Department,
%%   P. O. Box 6248,
%%   Providence, RI 02940,
%%   USA",
%% telephone="401-455-4080 or (in the USA) 800-321-4AMS",
%% email="Internet: Tech-Support@Math.AMS.org",
%% checksumtype="line count",
%% checksum="78",
%% codetable="ISO/ASCII",
%% keywords="latex, amslatex, ams-latex",
%% abstract="This file is part of the AMS-\LaTeX{} package, version 1.1."
%%   It is part of the monograph sample, testbook.tex (q.v.)."
%%}
%%% end of file header
%
\chapter*{Preface}

This book is a study of polynomial operator pencils,
i.e., operator polynomials of the form
\begin{equation*}
A(\lambda)=A_0+\lambda A_1+\cdots+\lambda^nA_n,
\end{equation*}
where $\lambda$ is a spectral parameter and $A_0,\dots,A_n$ are linear
operators acting in a Hilbert space $\cal H$. In the simplest cases 
$A(\lambda)=
A-\lambda I$ and $A(\lambda)=I-\lambda A$ we come to the usual (linear)
spectral problems. 

Spectral problems for polynomial pencils arise naturally in diverse
areas of mathematical physics (differential equations and boundary value
problems, controllable systems, the theory of oscillations and waves,
elasticity theory, and hydromechanics). This explains the steady interest
in these problems over the last 35 years.

A consideration of the simplest model---a matrix pencil---enables us
to see the  essential differences between nonlinear and linear
spectral problems. If the coefficients of the pencil $A_k$
$(k=0,\dots,n)$ are matrices of order $m$, $\det A_n\neq 0$, and all
the roots $\{\lambda_k\}_1^{nm}$ of the characteristic equation $\det
A(\lambda) =0$ are distinct, then the pencil $A(\lambda)$ has $nm$
eigenvectors $\{\varphi _k\}_1^{nm}$ (i.e.,
$A(\lambda_k)\varphi_k=0$). One possible approach is to single out in
the system $\{\varphi_k\}_1^{nm}$ various subsystems
$\{\varphi_{k_j}\}_{j=1}^m$ forming bases in $\bold C^m$. Another
approach leads to the consideration of the system of vectors
$\{\varphi_k,\lambda_k \varphi_k,\dots,\lambda_k^{n-1}\varphi_k)$
$(k=1,\dots,nm)$, which forms a basis in the space $\bold C^{nm}$.
Both these approaches are fruitful and admit far-reaching
generalizations.

%% \CharacterTable
%%  {Upper-case    \A\B\C\D\E\F\G\H\I\J\K\L\M\N\O\P\Q\R\S\T\U\V\W\X\Y\Z
%%   Lower-case    \a\b\c\d\e\f\g\h\i\j\k\l\m\n\o\p\q\r\s\t\u\v\w\x\y\z
%%   Digits        \0\1\2\3\4\5\6\7\8\9
%%   Exclamation   \!     Double quote  \"     Hash (number) \#
%%   Dollar        \$     Percent       \%     Ampersand     \&
%%   Acute accent  \'     Left paren    \(     Right paren   \)
%%   Asterisk      \*     Plus          \+     Comma         \,
%%   Minus         \-     Point         \.     Solidus       \/
%%   Colon         \:     Semicolon     \;     Less than     \<
%%   Equals        \=     Greater than  \>     Question mark \?
%%   Commercial at \@     Left bracket  \[     Backslash     \\
%%   Right bracket \]     Circumflex    \^     Underscore    \_
%%   Grave accent  \`     Left brace    \{     Vertical bar  \|
%%   Right brace   \}     Tilde         \~}
\endinput
