%%@texfile{%
%% filename="amslatex.tex",
\def\filename{amslatex.tex}
%% version="1.1c",
\def\fileversion{1.1c}
%% date="24-FEB-1993",
\def\filedate{24-FEB-1993}
%% filetype="AMS-LaTeX: documentation",
%% copyright="Copyright (C) American Mathematical Society, all rights
%%   reserved.  Copying of this file is authorized only if either:
%%   (1) you make absolutely no changes to your copy, including name;
%%   OR (2) if you do make changes, you first rename it to some other
%%   name.",
%% author="American Mathematical Society",
%% address="American Mathematical Society,
%%   Technical Support Department,
%%   P. O. Box 6248,
%%   Providence, RI 02940,
%%   USA",
%% telephone="401-455-4080 or (in the USA) 800-321-4AMS",
%% email="Internet: Tech-Support@Math.AMS.org",
%% checksumtype="line count",
%% checksum="3919",
%% codetable="ISO/ASCII",
%% keywords="latex, amslatex, ams-latex",
%% abstract="This file is part of the AMS-\LaTeX{} package, version 1.1.
%%   It contains the User's Guide, in the form of a \LaTeX{} document.
%%   It can be printed using standard \LaTeX{}, BEFORE the installation
%%   of the AMS-\LaTeX{} add-on.  Installation instructions are in
%%   Appendix A."
%%}
%%% end of file header
\immediate\write16{%
AMS-LaTeX file `\filename' (\fileversion, \filedate)}
%
%% This file can be typeset on most computer systems using the command
%%
%%   latex amslatex.tex
%%
%% even before the rest of the installation has been carried out.
%% Installation instructions are in Appendix A.
%%
\ifx\undefined\selectfont
%  If we are using an old LaTeX format file that does not incorporate
%  the Mittelbach--Sch"opf font selection scheme, we will
%  use the following documentstyle declaration:
\documentstyle{article}
%  \ntt is defined differently below.
\newcommand{\ntt}{\tt}
%  Otherwise we will use this one:
\else
\documentstyle[oldlfont]{article}
%  With the Mittelbach--Sch"opf font selection scheme in effect,
%  we define \ntt so that it always produces \mediumseries
%  \normalshape\tt even if \ntt is used inside a section heading
%  or similar place where the current font is not \mediumseries
%  \normalshape. This is to avoid font substitution warnings
%  for odd combinations like \cmtt/bx/n/10 which are usually
%  not available.
\newcommand{\ntt}{\series m\shape n\tt}
\fi

%\makeindex % write index information to the file amslatex.idx

\newtheorem{prop}{Proposition}

%%%%%%%%%%%%%%%%%%%%%%%%%%%%%%%%%%%%%%%%%%%%%%%%%%%%%%%%%%%%%%%%%%%%%%
%% Special definitions for use in producing the
%% AMS-LaTeX User's Guide.

\makeatletter

% Indent a little on the left in the verbatim environment.
\renewcommand{\verbatim}{\interlinepenalty\@M \@verbatim
  \leftskip\@totalleftmargin\advance\leftskip2pc
  \frenchspacing\@vobeyspaces \@xverbatim}

%      The @ character requires special handling because it's
%      a special character for MakeIndex. To avoid trouble we
%      make a direct call to the \char command.
\newcommand{\atsign}{\char64\relax}

\chardef\bslash=`\\ % p. 424, TeXbook

% We define some macros to do automatic indexing of control sequences,
% environments, options, and file names. For the index part we
% let \- to \empty; without this, entries with and without discretionary
% hyphens would be treated as separate entities by the makeindex
% program. 4-AUG-1990 mjd

\newcommand{\autoindex}{\index}

% control sequence
\newcommand{\cs}{\protect\pcs}
\newcommand{\pcs}[1]{{\ntt\bslash#1}{\let\-\empty 
  \autoindex{#1@{\string\ntt\bslash#1}}}}

% LaTeX documentstyle name
\newcommand{\sty}{\protect\psty}
\newcommand{\psty}[1]{{\ntt#1}{\let\-\empty
  \autoindex{#1@{\string\ntt{}#1} documentstyle}}}

% LaTeX option name
\newcommand{\opt}{\protect\popt}
\newcommand{\popt}[1]{{\ntt#1}{\let\-\empty
  \autoindex{#1@{\string\ntt{}#1} option}}}

% environment name
\newcommand{\env}{\protect\penv}
\newcommand{\penv}[1]{{\ntt#1}{\let\-\empty
  \autoindex{#1@{\string\ntt{}#1} environment}}}

% file name
\newcommand{\fn}{\protect\pfn}
\newcommand{\pfn}[1]{{\ntt#1}{\let\-\empty
  \autoindex{#1@{\string\ntt{}#1}}}}

% to index a control sequence without printing it
\newcommand{\indexcs}[1]{{\let\-\empty
  \autoindex{#1@{\string\ntt\bslash#1}}}}

% We allow some slop at the right margin because we have some
% long control sequence names and verbatim text to deal with.
\hfuzz2pc

% Macros for the various macro package names.
\newcommand{\AmS}{{\protect\the\textfont2
        A\kern-.1667em\lower.5ex\hbox{M}\kern-.125emS}}

% ``LaTeX'' is always set in roman (per latex.tex). We substitute
% \scriptfont0 for \sc for the `A', to avoid font unavailability
% problems at odd sizes.
\renewcommand{\LaTeX}{{\rm L\kern-.36em\raise.3ex\hbox{%
  \protect\the\scriptfont0 A}\kern-.15emT\kern-.1667em%
  \lower.7ex\hbox{E}\kern-.125emX}}

\newcommand{\Textures}{{\it Textures}}

% `Meta' macro.
\def\<#1>{{$\langle$\it#1\/$\rangle$}}

% To introduce permissible breakpoints for line breaks in verbatim text:
\newcommand{\5}{\penalty500 }

% A modified form of \sloppypar, to be used at the end of a paragraph:
\renewcommand{\sloppypar}{{\tolerance9999\par}}

% Used in the first appendix, for listings of files:
\newenvironment{filelist}{\par
  \addvspace{\medskipamount}\hrule\nobreak\addvspace{\medskipamount}\noindent
  \begin{tabular*}{\columnwidth}[t]{p{7pc}@{\extracolsep{\fill}}p{20pc}}}%
{\end{tabular*}\par}

\newcommand{\BibTeX}{{\sc Bib\kern-.1em\TeX}}

\makeatother


\begin{document}
\begin{titlepage}
\pagestyle{empty}
\title{\protect\AmS-\protect\LaTeX{} Version 1.1\\User's Guide}
\author{American Mathematical Society}
\date{August 1991}

\maketitle
\end{titlepage}

% By not using arabic numbers for the table of contents pages, we
% avoid having to run this file three times to get correct page
% numbers. Setting \count1 helps to distinguish in the TeX log
% that these are special pages, not part of the normal
% numbering sequence.
\pagenumbering{roman}\count1=-1

\message{Table of contents}
%      We turn off \autoindex temporarily while doing the table of
%      contents, to avoid indexing things that use \opt, \cs, \sty,
%      or \env in section headings.
\renewcommand{\autoindex}[1]{}
\tableofcontents
\renewcommand{\autoindex}{\index}

\newpage \message{Part I}
\pagenumbering{arabic}\count1=0
\part{General}
\section{Introduction}

The necessary documentation for using the \AmS-\LaTeX{} package has two
parts: this {\em User's Guide\/}, and some sample files illustrating
the features available in the
\AmS-\LaTeX{} package.  The file used to produce this {\it User's Guide}
is \fn{amslatex.tex}; the sample files are named
\fn{testart.tex} and \fn{testbook.tex}.
Installation instructions for the
\AmS-\LaTeX{} package are found in Appendix~\ref{a:install}.
As explained there, installation requires
making a new \LaTeX{} format file.
This {\em User's Guide}, however,
can be typeset without the new format file,
so that users can read it before proceeding further if they wish.
As a consequence, though, it was
impractical in many cases to show sample output for commands from the
\opt{amstex} option; this is done instead in the sample file
\fn{testart.tex}.  In the {\em User's Guide\/}
approximate output has been shown for the purposes
of illustration when it was practical to do so in ordinary \LaTeX{}.

For best understanding, you should be reasonably familiar with the
\LaTeX{} manual: {\it\LaTeX{}: A document preparation system}, by Leslie
Lamport \cite{lm}. Reading the {\it Joy of \TeX\/} \cite{jt} (the manual for \AmS-\TeX{})
will help you get the most out of the \AmS-\LaTeX{} software, but is not
mandatory. For users whose background is in \AmS-\TeX{} rather than \LaTeX{},
there is an appendix describing the ways in which the \LaTeX{} \opt{amstex}
option differs from \AmS-\TeX{} 2.1.

\subsection{Notes}

The notation \<dimension>, \<number>,
\index{dimension@{\it dimension}}\ index{number@{\it number}}
and the like will be used to
indicate that an arbitrary dimension or number or whatever
is to be substituted by the user.  By {\it dimension\/} we mean a number
followed by one of \TeX{}'s standard units {\tt pt}, {\tt pc},
{\tt in}, {\tt mm}, {\tt cm}, and so forth.

It is important in this {\it User's Guide} that we distinguish between the
original, non-\LaTeX{} implementation of \AmS-\TeX{} and the modified form
of it that constitutes the \LaTeX{} option \opt{amstex}.
Typewriter type will be used for the \LaTeX{} option
\opt{amstex}, and the standard logo \AmS-\TeX{} will be used for the
original non-\LaTeX{} version.

\section{The \protect\AmS-\protect\LaTeX{} project}

\AmS-\TeX{} was originally released for general use in 1982.  Its main
strength is that it makes it easy for the user to typeset mathematics,
while taking care of the many details necessary to make the output
satisfy the high standards of mathematical publishing.  It provides a
predefined set of natural commands such as \cs{matrix} and \cs{text} that
make complicated mathematics reasonably convenient to type.  These
commands incorporate the typesetting experience and standards of the
American Mathematical Society, to handle
problematic possibilities without burdening the user: matrices within matrices,
or a word of text within a subscript, and so on.

\AmS-\TeX{}, unlike \LaTeX{} does not have certain features that are
very convenient for authors---automatic numbering that adjusts to
addition or deletion of material being the primary one.  There are
also labor-saving ways provided in \LaTeX{} for preparing such items as indexes,
bibliographies, tables, and simple diagrams.  These
features are such a convenience for authors that the use of \LaTeX{}
spread rapidly in the mid-80s (a
reasonably mature version of \LaTeX{} was available by the end of
1983), and the American Mathematical Society began to be asked by its
authors to accept electronic submissions in \LaTeX{}.

The obvious question to ask was whether the strengths of \AmS-\TeX{} could
be combined with the strengths of \LaTeX{}, and in 1987 the American
Mathematical Society began to investigate the possibility of doing just
that.  Work on the \AmS-\LaTeX{} project was carried out over
the next three years by Romesh Kumar, a \TeX{}
consultant in the Chicago area, and by West German \LaTeX{} experts Frank
Mittelbach and Rainer Sch\"opf, with assistance from Michael Downes
of the American Mathematical Society technical support staff.

The overall philosophy was to provide \AmS-\TeX{} commands to the \LaTeX{} user
while adhering to standard \LaTeX{} syntax as much as possible.
Thus, to make their syntax more like normal \LaTeX{} syntax, \AmS-\TeX{}
commands having the form \cs{something}\5$\ldots$\5\cs{endsomething} were
converted to \LaTeX{} environments, so that they now have the form
\cs{begin}\verb'{something}'\5$\ldots$\5\cs{end}\verb'{something}'.  For
example, a matrix is typed as
\cs{begin}\verb'{matrix}'\5\dots\5\cs{end}\verb'{matrix}' instead of
\cs{matrix}\5$\ldots$\5\cs{endmatrix}.  Also, some commands that have top
and bottom options were changed so that the option is specified using
\verb'[t]' or \verb'[b]' instead of by a prefix \verb'top' or \verb'bot'
in the command name.  See Appendix~\ref{a:diff} for more details.

A good part of the original \AmS-\TeX{} was whittled off in the creation of the
\opt{amstex} option.  Many commands were redundant and were simply
dropped; others seemed only marginally useful and were omitted in order
to conserve control sequence memory.  Some internal control sequences
were eliminated by restructuring the code.

\AmS-\LaTeX{} is different enough from the original \AmS-\TeX{} that using the
{\it Joy of \TeX\/} as documentation would be unsatisfactory.  Instead, this
{\it User's Guide\/} aims to be more or less self-sufficient.  The {\it Joy of \TeX\/}
is still recommended reading because it provides background information
that helps explain why some things are handled the way they are.

\section{Major components of the \protect\AmS-\protect\LaTeX{} package}

The first major part of the \AmS-\LaTeX{} package is an extensive modification
of \AmS-\TeX{} 2.0+ that allows it to be used in \LaTeX{} as a documentstyle
option.  In other words, if you are writing an article,
your documentstyle declaration should look like this:
\begin{verbatim}
\documentstyle[amstex]{article}
\end{verbatim}

The second major part of the \AmS-\LaTeX{} package is a pair of documentstyles
called \sty{amsart} and \sty{amsbook}, parallel to \LaTeX{}'s \sty{article}
and \sty{book}, which are designed to be used in preparing manuscripts for
submission to the AMS.
There is nothing to prohibit their use for other purposes;
some users have said that they like using these documentstyles,
even when they don't intend to submit their manuscripts to the AMS, just
because they find the general design pleasing.

When the \sty{amsart} and \sty{amsbook} style files
are used the \opt{amstex} option will be automatically included, so
that the documentstyle declarations would simply be
\cs{documentstyle}\verb'{amsart}' or \cs{documentstyle}\verb'{amsbook}'.

The analog in \AmS-\TeX{} of the \sty{amsart} documentstyle is the
documentstyle \sty{amsppt}  (``AMS preprint'').  In \sty{amsart} and
\sty{amsbook} the document structure commands of the {\tt amsppt} style
described in Appendix~A of the {\it Joy of \TeX\/} have been superseded
by their \LaTeX{} equivalents, where equivalents existed, and otherwise
have been reimplemented in \LaTeX{} form.  The bibliography commands
described in Appendix~C of the {\it Joy of \TeX\/} have been dropped in
favor of \BibTeX{}, partly because this saves a significant amount of
memory.

\message{Part II}
\newpage
\part{Font considerations}
\label{s:fonts}

\section{The font selection scheme of Mittelbach and Sch\"opf}

In order to provide not only access to the AMSFonts currently
available but a general, reliable mechanism for making new math fonts
accessible to the user, the Society enlisted Frank Mittelbach and
Rainer Sch\"opf to adapt their recently developed font selection scheme
to accommodate the needs of the \AmS-\LaTeX{} project.  This new scheme has
a couple of distinctive features: (1) fonts (even math fonts) need not
be preloaded but can be loaded on demand; (2) font switches work a bit
differently---attributes are independent, and only one is changed at a
time.  In \LaTeX{} terms this means that, for example, \verb'\bf\Large'
has the same effect as \verb'\Large\bf'.

At the present time the files for the new font selection scheme are
being distributed along with the \AmS-\LaTeX{} package, with the permission
of Mittelbach and Sch\"opf; in the future the new scheme is slated to
become an official part of \LaTeX{}, in place of the current scheme.  A
detailed description of the workings of the font selection scheme can be
found in an article by Mittelbach and Sch\"opf that appeared in
{\it TUGboat\/}, June 1990 (vol.~11, no.~2): {\it The new font family
selection---user interface to standard \LaTeX{}\/} \cite{msf}.  If you
don't have access to that article, see the file \fn{fontsel.tex}
in the {\tt fontsel} distribution.

\section{Basic concepts}

In normal use, the ordinary \LaTeX{} commands \cs{rm},
\cs{it}, \cs{tt}, \cs{bf} are defined in terms of more primitive
commands \cs{family} etc., and  still function in much the same way as
before.  Knowledge of the more primitive commands will not be
essential except in documentstyle design or similar tasks.

The Mittelbach--Sch\"opf font selection scheme classifies fonts
based on the attributes
{\em shape}, {\em series}, {\em size}, and {\em family}.
Each attribute can be changed independently using the commands
\cs{shape}, \cs{series}, \cs{size}, and \cs{family}.
For example, to change the family attribute to {\tt cmr} (Computer
Modern roman),
the command would be \cs{family}\verb'{cmr}'.
Note that these commands do
not actually select the new font, because it's not uncommon
for you to want to change several attributes at a time
before actually switching to the new font.  The command
for putting the new attributes into effect is \cs{selectfont}.
For example, if the current font is family {\tt cmr},
size 10/12
(10-point type with 12-point baselineskip), series {\tt m} (medium
weight and width), and shape {\tt it} (italic), then the command
\begin{verbatim}
\family{cmtt}\shape{n}\selectfont
\end{verbatim}
would switch to a Computer Modern typewriter font
in the ``normal,'' i.e., upright, shape.  The size and series
values used in the selection of the new font would remain the
same as before.

\subsection{Shape}
The {\em shape} attribute is either: normal ({\tt n}),
italic ({\tt it}), small caps ({\tt sc}),
slanted (or ``sloped'') ({\tt sl}), or upright italic ({u}).  The
first three of these are the shapes that were typically found
together in the same font case, in the days of manual typesetting.
The latter two are somewhat unusual
variant shapes that are present in the Computer
Modern fonts.

The command to switch to a particular shape, say {\tt sc},
without changing other font attributes would be
\begin{verbatim}
\shape{sc}\selectfont
\end{verbatim}
but there are abbreviations for the most common shape changes:
\cs{sc}, \cs{it}, \cs{sl}, and \cs{normalshape}.  These are
the same as in the previous font selection scheme, except for
\cs{normalshape}, which may be understood as a replacement for
\cs{rm}.  In the new font selection scheme \cs{rm} is a family-changing
command, not a shape-changing command.  If you are dismayed at
the prospect of typing many instances of \cs{normalshape}, which
is obviously much longer than \cs{rm}, you needn't be.  As you will
see, many former uses of \cs{rm}, especially in mathematics,
are better handled by other means.  With astute use of grouping,
most documents can be done without using \cs{normalshape} at all.

\subsection{Series}

The series attribute is actually a combination of two related
attributes, weight and width.  The font charts
of type manufacturers typically show weights of light,
medium, and bold, and widths of condensed, medium, and expanded,
with intermediate and extreme variations such as semibold,
extra bold, and ultra bold.  The full list of the weights and
widths allowed for in the Mittelbach--Sch\"opf scheme are as shown
in Table~\ref{weight-width} (adapted from
Table~1 in \cite{msf}), along with their corresponding
 abbreviations for use with the \cs{series} command.
Examples:
\begin{description}
\item[\cs{series}{\tt\char`\{ux\char`\}}\cs{selectfont}]
Switches to an ultra expanded version of the current font.

\item[\cs{series}{\tt\char`\{sbc\char`\}}\cs{selectfont}]
Switches to a semibold condensed version of the current font.

\item[\cs{series}{\tt\char`\{m\char`\}}\cs{selectfont}]
Switches to a medium weight, medium width version of the current font.
\end{description}
Only two series changes are common enough to require abbreviations:
\cs{bf} and \cs{mediumseries} are abbreviations for,
respectively,
\begin{verbatim}
\series{bx}\selectfont   \series{m}\selectfont
\end{verbatim}
or, in other words, ``bold'' and ``not bold''.\footnote{%
See the remarks in Subsection~\ref{s:otherfam}
about {\tt\bslash bfdefault}.}%

\begin{table}[htp]
%      Since we are not making a list of tables, we use the []
%      option of \caption to avoid putting the caption text
%      into the .lot file (which could cause
%      `TeX capacity exceeded---buffer size' if the caption
%      is long, as here.)
\caption[]{Font weights and widths, and their abbreviations.
For use in the {\tt\bslash series\-} command, combine the
weight and width abbreviations, dropping any
{\tt m}'s (for ``medium''), except in the case where
both weight and width are medium: then use a single
{\tt m}.  Examples: Ultra Bold Condensed: {\tt ubc};
Medium Condensed:~{\tt c}.}
\label{weight-width}
\begin{center}
\begin{tabular}{|ll@{\hspace{2\tabcolsep}}ll|}
\hline
Weight&& Width&\\ \hline
\vrule width0pt height10pt
Ultra Light& {\tt ul}&       Ultra Condensed& {\tt uc}\\
Extra Light& {\tt el}&       Extra Condensed& {\tt ec}\\
Light& {\tt l}&              Condensed& {\tt c}\\
Semilight& {\tt sl}&         Semicondensed& {\tt sc}\\
Medium (normal)& {\tt m}&    Medium& {\tt m}\\
Semibold& {\tt sb}&          Semiexpanded& {\tt sx}\\
Bold& {\tt b}&               Expanded& {\tt x}\\
Extra Bold& {\tt eb}&        Extra Expanded& {\tt ex}\\
Ultra Bold& {\tt ub}&        Ultra Expanded& {\tt ux}\\[3pt]
\hline
\end{tabular}
\end{center}
\end{table}

\subsection{Size}
Because a change in font size is usually accompanied by a  change in
baselineskip, the \cs{size} command is designed to take two arguments,
the new size and the new baselineskip.  To switch to 14-point type
with a baselineskip of 18 points, the command would be
\begin{verbatim}
\size{14}{18pt}\selectfont
\end{verbatim}
All the usual \LaTeX{} size-changing commands from \cs{tiny}
to \cs{Huge} have suitable definitions based on the \cs{size} command.

{\em Note}.\ In the specification for the baselineskip, it is necessary
to give the units, because in some situations a unit other than {\tt
pt} may be desirable, or a skip register (see \cite[p.~118]{kn}) might be
used instead of an explicit value.

\subsection{Family}\label{s:families}
We define a font {\em family} as a group
of fonts of various shapes, widths, and weights, that
share distinctive design features, such as
x-height, the relative thickness of horizontal and vertical strokes,
distinctive shapes of particular letters, and so forth.
In other words,
fonts in the same family share a resemblance that fonts from
different families don't share (though in some cases the
resemblance is obvious only to an experienced eye).
Table~\ref{fams} gives a classification of some of
the Computer Modern fonts according to family.

\begin{table}[htp]
\caption[]{Computer Modern font families}
\label{fams}
\begin{center}
\begin{tabular}{|p{12pc}|l|}
\hline
Font file name& Family (and abbreviation)\\
\hline
\tt \raggedright
cmr10, cmti10, cmsl10, cmcsc10, cmu10,
cmbx10, cmbxti, cmbxsl, cmb10&
Computer modern roman ({\tt cmr})\\
\hline
\tt \raggedright
cmss10, cmssi10, cmssbx10, cmssdc10&
Computer modern sans serif ({\tt cmss})\\
\hline
\tt \raggedright
cmtt10, cmitt10, cmsltt, cmtcsc10&
   Computer modern typewriter ({\tt cmtt})\\
\hline
\end{tabular}
\end{center}
\end{table}

The abbreviations \cs{rm}, \cs{tt}, and \cs{sf} are provided for
switching to the Computer Modern roman, typewriter, and sans serif
families.  (The definition of \cs{sf}, for example, is
\verb'\family{cmss}\selectfont'.)

\subsection{Using other font families}\label{s:otherfam}
If the base family of a document is Computer Modern roman, with other
families used only sporadically, the other families would be
selected using the \cs{family} command as described in
\S\ref{s:families}.  If you want to change the {\it base family\/}
of the document, however, say to Times Roman or Baskerville,
then the best way is to change the default family settings.
In a canonical setup with all Computer Modern fonts, the following
definitions are in effect:
\begin{verbatim}
\newcommand\rmdefault{cmr}
\newcommand\sfdefault{cmss}
\newcommand\ttdefault{cmtt}
\end{verbatim}
Some or all of these default settings can be changed using
\cs{renewcommand}.  For example, if you have families
{\tt pstr}, {\tt pshel}, and {\tt pstt} for respectively
PostScript Times Roman, PostScript Helvetica, and Postscript
Typewriter fonts, then you could make them the default
via the commands
\begin{verbatim}
\renewcommand{\rmdefault}{pstr}
\renewcommand{\sfdefault}{pshel}
\renewcommand{\ttdefault}{pstt}
\end{verbatim}
either in the preamble of an individual document, or in an
option file (which then could be used by more than one document).
After these changes, the commands \cs{rm}, \cs{sf}, and \cs{tt}
will select the PostScript families rather than Computer Modern families.
Computer Modern families would still be accessible through explicit
use of the \cs{family} command, e.g.,
\begin{verbatim}
\family{cmtt}\selectfont
\end{verbatim}

Note that in order to
use such alternate families you must
have on your computer system
a fontdef file that defines which fonts belong to the
families {\tt pstr}, {\tt pshel}, and {\tt pstt},
as well as what sizes, shapes, and weights
are available on your particular system;
see the file \fn{fontdef.max} for more details (and also
section~\ref{s:custom-fontdef}).

In addition to the family defaults, there are defaults for some other
font attributes: \cs{bfdefault}, \cs{itdefault}, \cs{scdefault}, and
\cs{sldefault}. These give further control over fonts. I.e., if you
wanted to have all the slanted fonts in a document come out in italic,
it could be done like this:
\begin{verbatim}
\renewcommand{\sldefault}{it}
\end{verbatim}
The normal values for these defaults are
\begin{verbatim}
\bfdefault      bx
\itdefault      it
\scdefault      sc
\sldefault      sl
\end{verbatim}

Notice that by default bold fonts come from the Bold Expanded series
rather than the Bold series.  A comparison of the bold Computer Modern
fonts provided in standard distributions of \TeX{} shows why:
\[\begin{tabular}{|l|lll|}
\hline
Bold& \multicolumn{3}{c|}{Bold Expanded}\\[2pt]
\hline
{\tt cmb10}&    {\tt cmbxsl8}&  {\tt cmbx5}& {\tt cmbx9}\\
        &       {\tt cmbxsl10}& {\tt cmbx6}& {\tt cmbx10}\\
        &       {\tt cmbxti7}&  {\tt cmbx7}& {\tt cmbx12}\\
        &       {\tt cmbxti10}& {\tt cmbx8}&\\
\hline
\end{tabular}\]


\subsection{The \opt{oldlfont} option}
\label{s:oldlfont}

When the Mittelbach--Sch\"opf font selection
scheme is in use, emulation of the old font selection scheme can
be obtained by adding the option \opt{oldlfont} to the documentstyle
options list.  When the \opt{oldlfont} option is used, size-changing
commands return to normal shape and medium series in addition
to changing the font size; \cs{rm} gives normal shape and medium
series; \cs{tt} gives the normal shape and medium series
of the typewriter font; and \cs{sf} gives the normal shape and
medium series of sans serif.

\subsection{Warnings}

Many combinations of font attributes are not available at the
present time because the corresponding fonts do not exist.
The combination
\begin{verbatim}
\family{cmr}\series{bx}\shape{sl}
\end{verbatim}
happens to be available, because the corresponding font file,
{\tt cmbxsl10}, is part of the standard \TeX{} distribution.
However, for the combination
\begin{verbatim}
\family{cmss}\series{sbux}\shape{sc}
\end{verbatim}
``Computer Modern sans serif semibold ultra expanded small caps,''
no font file currently exists.

When a combination of font attributes is selected that is
not available, the nearest available font will be substituted,
and a warning message---not an error message, just a warning
message---will appear on-screen during the processing of
the document file.  The warning message will indicate which
font was substituted.

Once in a while,
you may find surprising results from a few commands in standard
\LaTeX{}  because they do not reset all the font attributes in the new
font selection scheme.  For example, if the \cs{footnote} command
appears within italic text (e.g., in a theorem), then the text of the
footnote will also be italic, because the standard definition of
\cs{footnote} resets only the {\em size\/} attribute, not the {\em
shape\/} or {\em family\/} or {\em series\/} attributes.  Problems
of this nature have been corrected in the
\sty{amsart} and \sty{amsbook} documentstyles,
and will be rectified in future versions of \LaTeX{} for the
standard \LaTeX{} documentstyles.  In the meantime, you can add
explicit font commands where needed: to get a normal
footnote in italic text, type\indexcs{normalshape}
\begin{verbatim}
\footnote{\normalshape ...}
\end{verbatim}
instead of just \cs{footnote}.

When using the Mittelbach--Sch\"opf scheme, the font names listed in
an ``overfull hbox'' message won't look the same as before.  Each font
name will have the family, series, shape, and size, separated by
slashes. For example, 10-point Computer Modern bold extended will
appear as {\tt \bslash cmr/bx/n/10}.  Formerly it would have appeared
as \cs{tenbf}.

\subsection{Customization of fontdef files}
\label{s:custom-fontdef}

There are three {\it fontdef\/} files in the standard
collection of \AmS-\LaTeX{} files.  Two of them, \fn{fontdef.ori}
and \fn{fontdef.max},  originated with
Mittelbach and Sch\"opf, while the third, \fn{fontdef.ams},
was produced at the AMS by adapting \fn{fontdef.max}.
Although these standard fontdef files will work reasonably well for most
users, it will often be to users' advantage to make
a customized fontdef file for their own use.  This is
because two users chosen at random
seldom have exactly the same collection of
fonts, unless they bought the same product at the same
time from the same company.
Printer drivers that use bitmapped fonts (usually in the form
of .PK files) generally require a separate bitmap file for
each size of each font, so that a fairly frequent
problem is for a particular user to be missing a few of the
particular bitmaps needed for the fonts and sizes called
for in a document.  The action taken by printer
drivers when a font is not found varies from useful (offering
to substitute another font) to not at all useful (refusing
entirely to print the document).  The
Mittelbach--Sch\"opf font scheme offers a substitution
mechanism in the fontdef files for getting around the latter problem,
as described below.

\subsubsection{A typical font definition}
Here is a typical font definition from \fn{fontdef.max}, for
Computer Modern text italic:
\begin{verbatim}
\new@fontshape{cmr}{m}{it}{%
      <5>cmti7 at5pt%
      <6>cmti7 at6pt%
      <7>cmti7%
      <8>cmti8%
      <9>cmti9%
      <10>cmti10%
      <11>cmti10 at10.95pt%
      <12>cmti12%
      <14>cmti12 at14.4pt%
      <17>cmti12 at17.28pt%
      <20>cmti12 at20.74pt%
      <25>cmti12 at24.88pt%
      }{}
\end{verbatim}
We now discuss the definition piece by piece.
\begin{description}
\item[\tt\bslash new\atsign fontshape] This is the command used to
define each family/series/shape combination.  The \verb'@' character
in the command name indicates that this is an internal command.
It has five {\it arguments}, indicated by the pairs of curly
braces \verb'{'~\verb'}'.

\item[\tt\char`\{cmr\char`\}] The font family: argument 1.

\item[\tt\char`\{m\char`\}] The font series: argument 2.

\item[\tt\char`\{it\char`\}] The font shape: argument 3.

\item[\tt\char`\{<5>cmti7 ...\char`\}] A list of point sizes
and the external font descriptions of the fonts that will be used
for each point size.  For example, the line \verb'<7>cmti7'
means that if a seven-point font in this particular family/series/shape
combination is required, the external font file ``cmti7'' will be
used.  There shouldn't be any spaces in this list, except before
``at'' clauses, and since end-of-line normally produces a space,
percent signs are used in the point size list to
comment out those spaces.

\item[\tt at10.95pt, at14.4pt, ...] When scaling up fonts to sizes
larger than their original size, the best strategy normally is to
follow a {\it magstep\/} progression rather than using exact point
sizes:
\begin{center}
\begin{tabular}{|ll|}
\hline
at 10.95pt& instead of at 11pt\\
at 12pt& no change\\
at 14.4pt& instead of at 14pt\\
at 17.28pt& instead of at 17pt\\
at 20.74pt& instead of at 20pt\\
at 24.88pt& instead of at 25pt\\
\hline
\end{tabular}
\end{center}
The reason this is good strategy is that for printer drivers using
bitmapped fonts, you are more likely to have the right sizes if you
use the magstep values than if you specify exact point sizes.  (If you
have a PostScript printer driver or other vector-based driver, you
won't have to worry about this.)

\item[\tt\char`\{\char`\}] The fifth argument of the \cs{new\atsign
fontshape} command is normally empty. It is used for specifying, if
necessary, additional action to be taken when a particular font is
loaded.

\end{description}

There are two kinds of substitutions that can be done: substituting
for an individual size of a particular font, or substituting for
all sizes of a particular family/series/shape combination.

\subsubsection{Substitutions for individual sizes}
The smallest size in the above example is five-point,
\verb'<5>cmti7 at5pt', obtained by reducing the seven-point version
of the font to 5pt.  If you don't have a .PK (bitmap) file corresponding
to this particular size, and want to substitute \fn{cmti7} at its
natural size, then you would change this line in the font definition
to read \verb'<5>1cmti7'.  The number 1 at the beginning will
cause \LaTeX{} to print an informational message about the fact that
a substitution was made, if the font
is actually used at that size.

Alternatively, instead of substituting a larger size of exactly
the same font, you might prefer to get the proper size and substitute
a different font shape.  For example, if you wanted to
substitute five-point roman instead of seven-point italic,
you would change the line to read \verb'<5>2cmr5'.  (The number
2 at the beginning causes a slightly different informational
message to be given if and when that size is actually used.)

\subsubsection{Substitution of a different series, shape, or family}

Substitution of a different family/series/shape combination for
one that is unavailable to you is done with a command called
\cs{subst\atsign fontshape}.  It takes six arguments, the first
three the family, series, and shape that aren't available,
and the second three the family, series, and shape that you
wish to substitute.  For example, let's suppose you want to
substitute Computer Modern medium slanted for Computer Modern
medium italic.  The command would be
\begin{verbatim}
\subst@fontshape{cmr}{m}{it}{cmr}{m}{sl}
\end{verbatim}

Further examples of the use of \cs{subst\atsign fontshape}
can be found in the files \fn{fontdef.ori} and \fn{fontdef.max}.

\section{Names of math font commands}
\label{s:mathfonts}

The single biggest issue in the integration of \AmS-\TeX{} and \LaTeX{} font
usage was that in \AmS-\TeX{} math font commands work
differently than text font commands and have different names. Instead
of being a simple switch, whose scope is bounded by curly braces, a math
font command in \AmS-\TeX{} is a command with one argument.  This means that
in \AmS-\LaTeX{}, to obtain a single bold letter in math you type
\cs{bold}\verb'{A}' rather than \verb'{\bf A}', and two bold letters would
be typed \verb'\bold{A}'\5\verb'\bold{B}' instead of \verb'{\bf AB}'.
(A similar distinction between text accents and math accents
 already existed in \LaTeX{}.)
Having the font command apply only to a single letter in this way
is more natural in math formulas, because
letters are usually single variables rather
than components of a word, and different fonts are mixed in all
combinations; four consecutive letters might be from four different
fonts.

The full list of math font commands in the \opt{amstex} option is
\cs{mathrm}, \cs{bold}, \cs{cal}, with the addition of \cs{frak}
(Fraktur) and \cs{Bbb} (blackboard bold) if AMSFonts are available.
Math italic, the default font for letters in math, also has a name,
\cs{mit}, but this is never needed in ordinary use.  Tables
\ref{fonttable} and~\ref{mathfonts} give a comprehensive listing of
font change commands for convenient reference.

\begin{table}[htp]
\chardef\{=`\{ \chardef\}=`\}
\caption[]{Font commands used in text}
\label{fonttable}
\begin{center}
\begin{tabular}{|lll|}
\hline
\multicolumn{1}{|l}{Font command}&   Equivalent&   Font selected\\
\hline
\cs{normalshape}&  \tt\cs{shape}\{n\}&      normal, upright, ``roman''\\
\cs{it}&   \tt\cs{shape}\{it\}&             italic\\
\cs{em}&   \tt\cs{shape}\{it\}$^*$ &          emphasis\\
\cs{sl}&   \tt\cs{shape}\{sl\}&     slanted\\
\cs{sc}&   \tt\cs{shape}\{sc\}&     small caps\\
\cs{mediumseries}&  \tt\cs{series}\{m\}&     medium weight\\
\cs{bf}&            \tt\cs{series}\{bx\}&    bold extended weight\\
\cs{tt}&    \tt\cs{family}\{cmtt\}&          typewriter style\\
\cs{sf}&    \tt\cs{family}\{cmss\}&          sans serif\\
\cs{rm}&    \tt\cs{family}\{cmr\}&           roman\\
\hline
\multicolumn{3}{|l|}{\parbox{20pc}{$^*$\strut The command \cs{em} selects
shape {\tt it} if the current font is upright, otherwise it selects
shape {\tt n} (normal).}}\\[6pt]
\hline
\end{tabular}
\end{center}
\bigskip
\caption[]{Font commands used in math}
\label{mathfonts}
\begin{center}
\begin{tabular}{|lp{20pc}|}
\hline
\cs{bold}&  Used to obtain bold letters from the English alphabet.\\
\cs{boldsymbol}& Used to obtain bold numbers and other nonalphabetic
        symbols, as well as bold Greek letters.\\
\cs{pmb}&   ``Poor man's bold,'' used for math symbols when
        bold versions don't exist in the currently available fonts.\\
\cs{cal}&   Calligraphic letters. Only uppercase is available.\\
\cs{mit}&   Math italic.  This font is automatically selected
        in math mode, so the command \cs{mit} is not needed in
        normal use.\\
\cs{mathrm}& Roman, normal shape.  Note: most of the time, \cs{text}
        or \cs{operatorname} should be used instead of
        \cs{mathrm} to produce this font in math.\\
\cs{frak}&  Euler Fraktur alphabet.\\
\cs{Bbb}&   Blackboard bold alphabet.  Only uppercase is
        available.\\
\hline
\end{tabular}
\end{center}
\end{table}

To gain access to a new math alphabet, you use the
\cs{new\-math\-alpha\-bet} command in the preamble of your document.
If you have the AMSFonts package, for example, and you want to use
Russian letters in math, taking them from the University of Washington
Cyrillic fonts, then you need
to find out the family name assigned to the fonts and
the shapes and weights available.  See Table~\ref{fonts}
to see what family names are included in the standard font definition
file \fn{fontdef.max}.  If you made a custom fontdef file
to match your available fonts, look in that file to find
the information.  If you are running \LaTeX{} at a larger
institution where some technical person has been assigned
to handle arcane font matters, you may need to consult that
person.

\begin{table}[htp]
\caption[]{Font name assignments made in {\tt fontdef.max}}
\label{fonts}
\medskip
\begin{tabular}{|lllp{17pc}|}
\hline
\multicolumn{1}{|c}{Family}&     \multicolumn{1}{c}{Series}&
        \multicolumn{1}{c}{Shape}&  \\
\hline
cmr&   m&     n& Computer Modern Roman\\
cmr&   m&     sl& CM slanted\\
cmr&   m&     it& CM italic\\
cmr&   m&     sc& CM small caps\\
cmr&   m&     u& CM upright italic\\
cmr&   b&     n& CM bold\\
cmr&   bx&    n& CM bold extended\\
cmr&   bx&    sl& CM bold extended slanted\\
cmr&   bx&    it& CM bold extended italic\\
cmss&  m&     n& CM sans serif\\
cmss&  m&     sl& CM sans serif slanted\\
cmss&  sbc&   n& CM sans serif semibold condensed\\
cmss&  bx&    n& CM sans serif bold extended\\
cmtt&  m&     n& CM typewriter\\
cmtt&  m&     it& CM typewriter italic\\
cmtt&  m&     sl& CM typewriter slanted\\
cmtt&  m&     sc& CM typewriter small caps\\
cmm&   m&     it& CM math italic\\
cmm&   b&     it& CM bold math italic\\
cmsy&  m&     n& CM math symbols\\
cmsy&  b&     n& CM bold math symbols\\
lasy&  m&     n& \LaTeX{} extra symbols\\
lasy&  b&     n& \LaTeX{} bold extra symbols\\
msa&   m&     n& AMS extra symbols~A\\
msb&   m&     n& AMS extra symbols~B\\
euf&   m&     n& Euler fraktur\\
euf&   b&     n& Euler fraktur bold\\
eur&   m&     n& Euler roman\\
eur&   b&     n& Euler bold roman\\
eus&   m&     n& Euler script\\
eus&   b&     n& Euler bold script\\
euex&  m&     n& Euler math extension symbols\\
UWCyr& m&     n& University of Washington Cyrillic\\
UWCyr& m&     it& UW Cyrillic italic\\
UWCyr& m&     sc& UW Cyrillic small caps\\
UWCyr& b&     n& UW Cyrillic bold\\
UWCyss& m&    n& UW Cyrillic sans serif\\
ccr&   m&     n& Concrete Roman\\
ccr&   m&     it& Concrete italic\\
ccr&   m&     sc& Concrete small caps\\
ccr&   c&     sl& Concrete condensed slanted\\
ccm&   m&     it& Concrete math italic\\
\hline
\end{tabular}
\end{table}

Suppose, then, that the family name for the University of
Washington fonts is {\tt UWCyr}.  Decide on the name of
the command you'd like to use for Cyrillic,
let's say \cs{cy}.  In the preamble area
of your document, add the line
\begin{verbatim}
\newmathalphabet*{\cy}{UWCyr}{m}{n}
\end{verbatim}
Thenceforth \verb'\cy{A}', \verb'\cy{d}', and so on will give you
a Russian A, d, or whatever in math.  Since there is not a one-to-one
correspondence between the Russian alphabet and the English
alphabet, you may need to refer to your documentation to
find out how to obtain certain letters.  The {\it AMSFonts User's Guide\/}
\cite{amsfonts} gives a complete table.

If you also want to use bold Russian letters, you could define
another math alphabet and name it, say, \verb'\boldcy'.
Alternatively, you could set things up so that bold Russian
letters are accessible through the commands \cs{boldsymbol}
and \cs{boldmath}.
If you add the line
\begin{verbatim}
\addtoversion{bold}{\cy}{UWCyr}{b}{n}
\end{verbatim}
in your document's preamble,
then \verb'\cy{A}' would produce a normal-weight Russian A
and
\begin{verbatim}
{\boldmath  $ ... \cy{A} ... $ }
\end{verbatim}
would produce a bold Russian A (with the rest of the formula being
made bold as well).  Furthermore, you could then obtain
a bold Russian A in the midst of normal math using \cs{boldsymbol}:
{\samepage
\begin{verbatim}
$ ... \boldsymbol{\cy{A}} ... $
\end{verbatim}
}% End of \samepage

In the \opt{amstex} option \cs{boldsymbol} is to be used for
individual bold math symbols and bold Greek letters---everything in
math except for letters (where you would use \cs{bold}).  For example,
to obtain a bold $\infty$, $+$, $\pi$, or $0$, you would use the
commands \verb'\boldsymbol{\infty}', \verb'\boldsymbol{+}',
\verb'\boldsymbol{\pi}', or \verb'\boldsymbol{0}'.   Because they are
not included in the  standard distribution of \TeX{} fonts,  sizes
other than 10-point of bold fonts for math symbols, Greek, and math
italic ({\sc CMBSY} and {\sc CMMIB}) are provided in the AMSFonts
distribution (version 2.0+).

Since \cs{boldsymbol} takes rather a lot of typing, you would usually
put some definitions in the preamble of the form
\begin{verbatim}
\newcommand{\bpi}{\boldsymbol{\pi}}
\newcommand{\binfty}{\boldsymbol{\infty}}
\end{verbatim}
for any bold symbols you're going to use frequently.

For some math symbols \cs{boldsymbol} will not have any effect because
bold versions of those symbols do not exist in the currently available
fonts.  These include
extension symbols and large operator symbols from the font CMEX, as well
as the AMS extra math symbols from the fonts MSAM and MSBM.
``Poor man's bold'' (\cs{pmb}) can be used for some of the
things that aren't handled properly by \cs{boldsymbol}.  It works by
typesetting several copies of the symbol with slight offsets.
With large operators and extension symbols, however, \cs{pmb} does not
currently work very well because the proper spacing and treatment of
limits is not preserved.

To make an entire math formula bold (or as much of it as possible,
depending on the available fonts), use \cs{boldmath} preceding
the formula, as described in the \LaTeX{} manual.

The sequence \verb'{\bf\hat{a}}' (in ordinary \LaTeX{}) or
\verb'\bold{\hat{a}}' (in the \opt{amstex} option) produces a bold
accent character over the {\bf a}, as you would expect. However,
combinations like \verb'{\cal\hat{a}}' will not work in ordinary
\LaTeX{} because the \cs{cal} font does not have its own accents.  In
the \opt{amstex} option the font change commands are defined in such a
way that accent characters will be taken from the \cs{rm} font if they
are not available in the current font (in addition to the \cs{cal} font,
the \cs{Bbb} and \cs{frak} fonts don't contain accents).

In ordinary \LaTeX{} uppercase Greek can be made bold by, e.g.,
\verb'{\bf\Gamma}'.  In the \opt{amstex} option uppercase
Greek can be made bold only by using \cs{boldsymbol} (in other words,
uppercase Greek is handled the same as lowercase Greek).
\sloppypar

\section{The command \cs{newsymbol}}
\label{s:newsym}

The command \cs{newsymbol} is presently used only for symbols from the
AMS extra symbol fonts, MSAM and MSBM.  \cs{newsymbol} allows you to create a
control sequence that will properly produce a symbol from the extra symbol
fonts.  The use of \cs{newsymbol} is
explained in the {\em AMSFonts User's Guide\/}.
In a \LaTeX{} document
there is one main difference in usage, which is only applicable
if you want to use AMSFonts without using the \opt{amstex}
option: instead of using the additional
setting-up commands \cs{loadmsam} and \cs{loadmsbm},
you should put
``\opt{amsfonts}'' in the documentstyle options list.
Otherwise \cs{newsymbol} commands can be used exactly as
shown in the {\em AMSFonts User's Guide\/}.  Like \cs{newcommand}'s,
they should be placed in the preamble.

The \opt{amsfonts} option is geared to version 2.0 or later of the
AMSFonts  package. Font layout and font names for some of the fonts are
different than in earlier versions. If you have an earlier version, you
would need to contact the AMS for an upgrade  in order to use the
\opt{amsfonts} option successfully. See Appendix~\ref{a:help} for
information on how to obtain the AMSFonts.

\section{The \opt{amssymb} option}

If you are running a version of \LaTeX{} with extra memory available for
control sequence names, and you use quite a few of the extra math
symbols from the AMSFonts, it may be more convenient for you to use
the \opt{amssymb} documentstyle option, which will define all the symbol
names (about 200), so you won't have to include an individual
\cs{newsymbol} command in your document for each one.  You may prefer to
include it in the construction of a format file (see the installation
instructions, Appendix~\ref{a:install}) to save processing time; it is a
stand-alone option, so it can be included in the format file without
including the \opt{amstex} option.

\message{Part III}
\newpage
\part{Features of the \opt{amstex} option}

\section{Math spacing commands}
Both\indexcs{,}\indexcs{:}\indexcs{;}\indexcs{thinspace}%
\indexcs{negthinspace}\indexcs{medspace}\indexcs{negmedspace}%
\indexcs{thickspace}\indexcs{negthickspace}
the spelled-out and abbreviated forms of these commands are robust, and
in addition they can also be used outside of math.
The primary math spacing commands are:
\begin{center}
\begin{tabular}{|l|l||l|l|}
Abbrev.&Spelled out&Abbrev.&Spelled out\\
\verb'\,'&\verb'\thinspace'&\verb'\!'&\verb'\negthinspace'\\
\verb'\:'&\verb'\medspace'&&\verb'\negmedspace'\\
\verb'\;'&\verb'\thickspace'&&\verb'\negthickspace'\\
\verb'@,'&&\verb'@!'&\\
&\verb'\quad'&&\\
&\verb'\qquad'&&
\end{tabular}
\end{center}
\verb'@,'\index{"@,@{\tt{}\atsign,}} and
\verb'@!'\index{"@"!@{\tt{}\atsign"!}} give
one-tenth the space of \cs{,} and {\tt\bslash !}\index{"!@{\tt\bslash
"!}} respectively, for extra fine tuning where necessary.

\section{Multiple integral signs}
\cs{iint}, \cs{iiint}, and \cs{iiiint} give multiple
integral signs with the spacing between them nicely adjusted,  in both
text and display style.  \cs{idotsint} is an extension of the same
idea that gives two integral signs with dots between them.

\section{Over and under arrows}

There are some additional
over and under arrow operations provided in the \opt{amstex} option:
{\samepage
\begin{tabbing}
\qquad\={\tt\bslash overleftrightarrow\qquad}\=\kill
\> \cs{underleftarrow} \> \cs{underrightarrow} \+\\
\cs{overleftrightarrow}         \> \cs{underleftrightarrow}
\end{tabbing}
}
All over and under operations, including the previously available ones
(\cs{over\-right\-arrow}, \cs{over\-left\-arrow}), have been modified to scale
properly in subscript  sizes. (After you have installed \AmS-\LaTeX{}, you
can process and print the sample file \fn{testart.tex} to see  examples
of the arrows.)

\section{Dots} In the \opt{amstex} option, ellipsis dots should almost
always be typed as
\cs{dots}. Placement (on the baseline or centered) is selected
according to whatever follows the \cs{dots}.  If the next thing is
a plus sign, the dots will be centered; if it's a comma, they will be on
the baseline.  These default dot placements provided by the
\opt{amstex} option can be changed by the documentstyle if different
conventions are wanted.

If the dots fall at the end of a math formula, the next thing is
something like \cs{end} or \verb'\)' or \verb'$', which does not give any
information about how to place the dots.  Then you must help by using
\cs{dotsc} for ``dots with commas,'' or \cs{dotsb} for ``dots with
binary operators/relations,'' or \cs{dotsm} for ``multiplication dots,''
or \cs{dotsi} for ``dots with integrals.'' For example, the input
\begin{verbatim}
Then we have the series $A_1,A_2,\dotsc$,
the regional sum $A_1+A_2+\dotsb$,
the orthogonal product $A_1A_2\dotsm$,
and the infinite integral
\[\int_{A_1}\int_{A_2}\dotsi\].
\end{verbatim}
will produce low dots in the first instance and centered dots
in the others, with the spacing on either side of the dots
nicely adjusted.
\begin{quotation}
Then we have the series $A_1,A_2,\ldots\,$,
the regional sum $A_1+A_2+\cdots\,$,
the orthogonal product $A_1A_2\cdots\,$,
and the infinite integral
\[\int_{A_1}\int_{A_2}\cdots\,.\]
\end{quotation}

Specifying dots this way, in terms of their meaning rather than in terms
of their visual placement, is in keeping with the general philosophy of
\LaTeX{} and makes documents more portable between places where different
conventions prevail.  The control sequences \cs{ldots} and \cs{cdots}
are still available, however, for compatibility.

\section{Accents in math}

The following accent commands automatically
give good positioning of double accents:
\begin{verbatim}
 \Hat    \Check  \Tilde  \Acute  \Grave  \Dot    \Ddot
 \Breve  \Bar    \Vec
\end{verbatim}
In ordinary \LaTeX{} the second
accent will usually be askew:
$\hat{\hat A}$ (\cs{hat}\verb'{\hat A}').
In the \opt{amstex} option, if you type \cs{Hat}\5\verb'{\Hat A}'
(using the capitalized form for both accents) the second accent
will be properly positioned (see \fn{testart.tex} for examples).

As explained in the {\it Joy of \TeX\/}, this double accent operation is complicated
and tends to slow down the processing of a \TeX{} file.
If your document contains many double accents, you can
use \cs{accentedsymbol} in the preamble of your document to
help speed things up.  It stores the result of the double accent
command in a box register, for quick retrieval.  \cs{accented\-symbol}
is used like \cs{newcommand}:
\begin{verbatim}
\accentedsymbol{\Ahathat}{\Hat{\Hat A}}
\end{verbatim}

Some accents have a wide form: typing \verb'$'\cs{widehat}
\verb'{xy},'\cs{widetilde}\verb'{xy}$'
produces $\widehat{xy},\widetilde{xy}$.  Because these wide accents
have a certain maximum size, extremely long expressions are better
handled by a different notation:
$(AmBD)^{\widehat{\hphantom{x}}}$ instead of $\widehat{AmBD}$.  But getting
an accent into a superscript is a little tricky (try it),
so \opt{amstex} has the following control sequences
to make it easier:
\begin{verbatim}
 \sphat     \spcheck   \sptilde   \spdot
 \spddot    \spdddot   \spbreve
\end{verbatim}
The example above would be typed \verb'(AmBD)\sphat'.

Finally, \cs{dddot} and \cs{ddddot} are available to
produce triple and quadruple dot accents
in addition to the \cs{dot} and \cs{ddot} accents already available
in \LaTeX{}.

\section{Roots}

In ordinary \LaTeX{} the placement of root indices is sometimes not so
good: $\sqrt[\beta]{k}$ (\verb'\sqrt'\5\verb'[\beta]{k}').  In the
\opt{amstex} option \cs{leftroot} and \cs{uproot} allow you to adjust
the position of the root:
\verb'\sqrt'\5\verb'[\leftroot{-2}'\5\verb'\uproot{2}'\5\verb'\beta]{k}'
will move the beta up and to the right.
(See the sample file \fn{testart.tex}.) The negative argument
used with \cs{leftroot} moves the $\beta$ to the right. The units are
a small dimension that is a useful size for such
adjustments.

\section{Boxed formulas} The command \cs{boxed} puts a box around its
argument, like \cs{fbox} except that the contents are in math mode.

\section{Extensible arrows}

\verb'@>>>' and \verb'@<<<' produce arrows that extend automatically to
accommodate unusually wide subscripts or superscripts.  The text of a
superscript is typed in between the first and second \verb'>' or
\verb'<' symbols, and for a subscript, it's typed between the second and
third symbols. For example, \verb'@>\xi F_k\Gamma_k\alpha>>' would have
a superscript $\xi F_k\Gamma_k\alpha$ placed above the arrow.   These
arrows were originally developed for use in commutative diagrams but can
be used elsewhere also.  (See section
\ref{s:commdiag} for more information about the \opt{amscd} option.)

\section{\cs{overset}, \cs{underset} and
\cs{sideset}} \LaTeX{} provides \cs{stackrel} for
placing a superscript above a binary relation.  In \opt{amstex}
there are somewhat more general commands, \cs{overset} and
\cs{underset}, that can be used to place one symbol above or
below another symbol, whether it's a relation or something
else.  The input \verb'\overset{*}{X}' will place a
superscript-size $*$ above the  $X$; \cs{underset} performs
the parallel operation that you'd expect.

There's also a command called \cs{sideset}, for a rather special
purpose: putting symbols at the subscript and superscript
corners of a large operator symbol such as $\sum$ or $\prod$.
The prime example is the case when
you want to put a prime on a sum symbol.  If there are no
limits above or below the sum, you could just use \cs{nolimits}:
here's \verb+\sum\nolimits' E_n+ in display mode:
\begin{equation}\sum\nolimits' E_n.\end{equation}
But if you want not only the prime but also something below or
above the sum symbol, it's not so easy.  If you have
\begin{equation}\sum_{n<k,\;n\ \rm odd}nE_n\end{equation}
and you want to add a prime
on the sum symbol, use \cs{sideset} like this:
\begin{verbatim}
\sideset{}{'}\sum_{...}nE_n
\end{verbatim}
The extra pair of empty braces is explained by the fact that
\cs{sideset} has the capability of putting an extra symbol
or symbols at each corner of a large operator; to put an asterisk
at each corner of a product symbol, you would type
\begin{verbatim}
\sideset{_*^*}{_*^*}\prod
\end{verbatim}
(After you have installed \AmS-\LaTeX{}, you can
typeset and print the sample file \fn{testart.tex} to see
examples of the output.)

\section{The \cs{text} command}
The main use of the command \cs{text} is for words or phrases in a
display.  It is very similar to the \LaTeX{} command \cs{mbox} in its
effects, but has a couple of advantages.  If you want a word or phrase
of text in a subscript, you can type \verb'..._{\text{word or
phrase}}', which is slightly easier than the \cs{mbox} equivalent:
\verb'..._{\mbox{\scriptsize' \verb'word' \verb'or' \verb'phrase}}'.
The other advantage is the more descriptive name.

\section{Operator names}\label{s:opname}
Math functions such as $\log$, $\sin$, and $\lim$ are traditionally
typeset in roman type to help avoid confusion with single math
variables, set in math italic.  The more common ones have predefined
names, \cs{log}, \cs{sin}, \cs{lim}, and so forth, but new ones come up
all the time in mathematical papers, so \opt{amstex} provides a general
mechanism for producing such names: \cs{operatorname}\verb'{xxx}'
produces {\rm xxx} in the proper font and automatically adds proper
spacing on either side when necessary, so that you get $A\,{\rm xxx}\,B$
instead of $A{\rm xxx}B$.

Since \cs{operatorname} takes rather a lot of typing, you would usually
put some definitions in the preamble of the form
\begin{verbatim}
\newcommand{\xxx}{\operatorname{xxx}}
\newcommand{\yyy}{\operatorname{yyy}}
\end{verbatim}
for any operator names you're going to use frequently.

Some of the operator names, such as \cs{lim},
have actually been defined using
\cs{operator\-name\-with\-limits} rather than \cs{operatorname}, because in
displayed formulas if there is a subscript on \cs{lim} it is conventionally
placed underneath, like the limits on sums:
\begin{equation}
C_+f(x)=\lim_{t\to0}C(f)(x+it)
\end{equation}
You can use \cs{operatornamewithlimits} just like \cs{operatorname}; the
only difference is the placement of subscripts and superscripts.
A few special operator names with limits are defined for you in the
\opt{amstex} option:  \cs{varinjlim}, \cs{varprojlim}, \cs{varliminf},
and \cs{varlimsup}; there are some examples in the sample file
\fn{testart.tex}.

\section{\cs{mod} and its relatives} Commands \cs{mod},
\cs{bmod}, \cs{pmod}, \cs{pod} are provided to deal with the rather
special spacing conventions of ``mod'' notation.  \cs{bmod} and \cs{pmod}
are available in \LaTeX{}, but in the \opt{amstex} option the spacing of
\cs{pmod} will adjust to a smaller value if it's used in a
non-display-mode formula.  \cs{mod} and \cs{pod} are variants of
\cs{pmod} preferred by some authors; \cs{mod} omits the parentheses,
whereas \cs{pod} omits the ``mod'' and retains the parentheses.

\section{Fractions and related constructions}
\label{fracs}

In addition to \cs{frac} (which was already available in \LaTeX{}),
\opt{amstex} provides \cs{dfrac} and \cs{tfrac} as convenient
abbreviations for \verb'{\displaystyle\frac' \verb'...' \verb'}' and
\verb'{\textstyle\frac' \verb'...' \verb'}'. Furthermore, the
thickness of the fraction line can be varied, using
a new square-bracket option of the \cs{frac} command.\index{dimension@{\it
dimension}}
\cs{frac}\5\verb'['\<dimension>\verb']'\5\verb'{...}'\5\verb'{...}'
makes a fraction where the thickness of the horizontal rule is determined by
the given dimension.  The sample file \fn{testart.tex} shows an
example using a thickness of \verb'1.5pt'.

\cs{fracwithdelims}\5\<left delimiter>\5\<right
delimiter>\5\verb'['\<dimension>\verb']' is an extension of the same idea,
with delimiters on either side specified by the user.%
\setbox0=\hbox{\footnotesize
\verb'\left'\5\verb'(\frac'\5\verb'{...}'\5\verb'{...}'\5\verb'\right)'}%
\footnote{The perceptive reader may wonder why this command is necessary
when you can type things like \unhbox0.
The answer is that \cs{fracwithdelims} provides slightly better
spacing.}

For binomial expressions such as $\bigl({n\atop k}\bigr)$
\opt{amstex} has \cs{binom}, \cs{dbinom} and \cs{tbinom}.  \cs{binom} is
an abbreviation for \cs{fracwithdelims}\verb'()[0pt]'.

After you have installed \AmS-\LaTeX{}, you can
typeset and print the sample file \fn{testart.tex} to see
examples of \cs{frac} and \cs{binom}.

\section{Continued fractions}
The continued fraction
\begin{equation}
\def\cfrac#1#2{{\displaystyle\strut#1\over\displaystyle#2}%
 \kern-\nulldelimiterspace}
\cfrac{1}{\sqrt{2}+
 \cfrac{1}{\sqrt{2}+
  \cfrac{1}{\sqrt{2}+
   \cfrac{1}{\sqrt{2}+
    \cfrac{1}{\sqrt{2}+\cdots
}}}}}
\end{equation}
can be obtained by typing
{\samepage
\begin{verbatim}
\cfrac{1}{\sqrt{2}+
 \cfrac{1}{\sqrt{2}+
  \cfrac{1}{\sqrt{2}+
   \cfrac{1}{\sqrt{2}+
    \cfrac{1}{\sqrt{2}+\dotsb
}}}}}
\end{verbatim}
}% End of \samepage
Left or right placement of any of the numerators is accomplished by using
\cs{lcfrac} or \cs{rcfrac} instead of \cs{cfrac}.

\section{Smash options}\label{ss:smash}

The plain \TeX{} command \cs{smash} is used to typeset a subformula and
give it an effective height and depth of zero, which is sometimes
useful in adjusting the subformula's position with respect to adjacent
symbols.   In \opt{amstex} there are optional arguments \verb't' and
\verb'b' for \cs{smash}, because sometimes it is advantageous to be
able to ``smash'' only the top or only the bottom of something while
retaining the natural depth or height.  For example, to smash only the
part of a subformula that extends below the baseline, you would type
\verb'\smash[b]{'\5\<whatever>\5\verb'}'.

\section{New \protect\LaTeX{} environments}
\subsection{The ``cases'' environment}
``Cases'' constructions like the following are common in
mathematics:
\begin{equation} P_{r-j}=
  \left\{
  \begin{array}{ll}
    0                &\mbox{if $r-j$ is odd},\\
    r!\,(-1)^{(r-j)/2} &\mbox{if $r-j$ is even}.
  \end{array}
  \right.
\end{equation}
and in the \opt{amstex} option there is a \env{cases} environment:
\begin{verbatim}
\begin{equation} P_{r-j}=
  \begin{cases}
    0&  \text{if $r-j$ is odd},\\
    r!\,(-1)^{(r-j)/2}&  \text{if $r-j$ is even}.
  \end{cases}
\end{equation}
\end{verbatim}
Notice the use of \cs{text} and the embedded math.

\subsection{Matrix}\label{ss:matrix}

In the creation of the \opt{amstex} option, \AmS-\TeX{}'s \cs{matrix}
could have been discarded, since \LaTeX{}'s \env{array} environment has
the same function.  But we wanted to keep \AmS-\TeX{}'s \cs{pmatrix},
\cs{bmatrix}, \cs{vmatrix} and \cs{Vmatrix} commands, with delimiters
built in; for consistency, the basic \cs{matrix} has been retained
also.  It and the other matrix commands have been changed into \LaTeX{}
environments that work like \env{array}, except that they don't have
an argument specifying the format of the columns. Instead a default
format is provided: up to 10 centered columns. The maximum number of
columns is determined by the counter \verb'MaxMatrixCols', which you
can change if necessary using \LaTeX{}'s \cs{setcounter} or
\cs{addtocounter} commands.  I.e., suppose you have a big
matrix with 19 or 20 columns.  Then you'd do something like this:
\begin{verbatim}
\begin{equation}
\setcounter{MaxMatrixCols}{20}
A=\begin{pmatrix}
...&...&...&...&...&...&...&...&...&...&...&...&  ... \\
  ...  \\
  ...
\end{pmatrix}
\end{equation}
\end{verbatim}

To produce a small matrix suitable for use in text, use the
\env{smallmatrix} environment.
\begin{verbatim}
\begin{math}
  \bigl( \begin{smallmatrix}
      a&b\\ c&d
    \end{smallmatrix} \bigr)
\end{math}
\end{verbatim}

\cs{hdotsfor}\verb'{'\<number>\verb'}' produces a row of dots in a matrix
spanning the given number of columns.
\begin{verbatim}
\begin{matrix} a&b&c&d\\
e&\hdotsfor{3} \end{matrix}
\end{verbatim}
would give dots spanning the last three columns in the second row.
The spacing of the dots can be varied through use of a square-bracket
option, for example, \verb'\hdotsfor[1.5]{3}'.  The number in square brackets
will be used as a multiplier; the normal value is 1.

\subsection{The \env{Sb} and \env{Sp} environments}

The \env{Sb} and \env{Sp} environments can be used to typeset several
lines as a subscript or superscript:
for example
\begin{verbatim}
\begin{equation}
  \sum\begin{Sb}
        0\le i\le m\\ 0<j<n
      \end{Sb}
    P(i,j)
\end{equation}
\end{verbatim}
produces a two-line subscript underneath the sum:
\begin{equation}
  \sum_{0\le i\le m\atop 0<j<n}
    P(i,j)
\end{equation}
\env{Sb} and \env{Sp} can be used anywhere that an ordinary subscript or
superscript can be used.

\subsection{Commutative diagrams}
\label{s:commdiag}

The commutative diagram commands of \AmS-\TeX{} are not
included in the \opt{amstex} option, but are available as a separate
option, \opt{amscd}.  This conserves memory for users who don't
need commutative diagrams. The \env{picture} environment
can be used for complex commutative diagrams but for
simple diagrams without diagonal arrows the \opt{amscd}  commands are more
convenient.

The commutative diagram
\begin{equation}\label{e:cd}
\begin{array}{ccc}
S^{{\cal W}_\Lambda}\otimes T&
   \stackrel{j}{\longrightarrow}&  T\\
\Big\downarrow&
       &\Big\downarrow\vcenter{\rlap{$\scriptstyle{\rm End}\,P$}}\\
(S\otimes T)/I& =&  (Z\otimes T)/J
\end{array}
\end{equation}
can be produced in ordinary \LaTeX{} by
\begin{verbatim}
\begin{array}{ccc}
S^{{\cal W}_\Lambda}\otimes T&
    \stackrel{j}{\longrightarrow}&  T\\
\Big\downarrow&
        &\Big\downarrow\vcenter{%
          \rlap{$\scriptstyle{\rm End}\,P$}}\\
(S\otimes T)/I& =&  (Z\otimes T)/J
\end{array}
\end{verbatim}
When the \opt{amscd} option is used you would type instead
\begin{verbatim}
\begin{CD}
S^{{\cal W}_\Lambda}\otimes T   @>j>>   T\\
@VVV                                    @VV{\End P}V\\
(S\otimes T)/I                  @=      (Z\otimes T)/J
\end{CD}
\end{verbatim}
(with \cs{End} defined as \cs{operatorname}\verb'{End}'; see
\S\ref{s:opname}).
This would give longer horizontal arrows than in (\ref{e:cd})
and improved spacing between
elements of the diagram (see \fn{testart.tex}). In the \env{CD} environment
the commands \verb'@>>>', \verb'@<<<',
\verb'@VVV', and \verb'@AAA' give respectively right, left, down, and
up arrows.\index{"@VVV@{\tt{}\atsign VVV}}%
\index{"@AAA@{\tt{}\atsign AAA}}%
\index{"@<<<@{\tt{}\atsign<<<}}%
\index{"@>>>@{\tt{}\atsign>>>}}
For the horizontal arrows, material between the first and second
\verb'>' or \verb'<' symbols will be typeset as a superscript,
and material between the second and third will be typeset as
a subscript.  Similarly, material between the first and second
or second and third {\tt A}s or {\tt V}s of vertical arrows will be typeset as
left or right ``sidescripts''.

\section{Alignment structures for equations}
\label{s:displays}

In the \opt{amstex} option
several environments exist for creating multi-line displayed
equations.
They are similar in function to \LaTeX{}'s \env{equation} and
\env{eqnarray} environments.
These environments are:
\begin{verbatim}
 align      gather     alignat    xalignat   xxalignat
 multline   split
\end{verbatim}
Each environment, except for \env{split}, has both starred
and unstarred forms, where the unstarred forms have automatic
numbering, using \LaTeX{}'s \env{equation} counter.  You can suppress
the number on any particular line by putting \cs{notag} before the
\cs\bslash; you can also override it with a tag of your own using
\cs{tag}\verb'{'\<label>\verb'}', where \<label> means arbitrary text
such as \verb'$*$' or \verb'ii' used to ``number'' the equation.  There
is also a \cs{tag*} command that causes the tag to be typeset
absolutely literally, without putting parentheses around it.  \cs{tag}
and \cs{tag*} can also be used in the starred versions of all the
\opt{amstex} alignment structures.  See \fn{testart.tex}, Appendix~B,
for examples of the use of \cs{tag}.

\subsection{The \env{align} environment}

The \env{align} environment is used for two or more equations when
vertical alignment is desired (usually binary relations such as equal
signs are aligned).  The term ``equation'' is used rather loosely here
to mean any math formula that is intended by the author as a
self-contained subdivision of the larger display,  usually, but not
always, containing a binary relation.

\subsection{The \env{gather} environment} Like the \env{align}
environment, the \env{gather} environment is  used for two or more
equations, but when there is no alignment desired among them; each one
is centered separately between the left and right margins.

\subsection{The \env{alignat} environment}

The \env{align} environment takes up the whole width of a display. If
you want to have several ``align''-type structures side by side, there
is an \env{alignat} environment you can use.  There is one required
argument, to specify the number of ``align'' structures.%
%
\footnote{For an argument of $n$, the number of {\tt\char`\&}'s per line
is $2n-1$ (one ampersand for alignment within each align structure,
and ampersands to separate the align structures from one another).}
%
The \env{xalignat} and \env{xxalignat} environments are forms of the
\env{alignat} environment with expanded spacing between the component
align structures.  If we consider each align structure to be a
column, \env{xalignat} has equal spacing between columns and at the
margins; \env{xxalignat} has equal spacing between columns and zero
spacing at the margins.  See \fn{testart.tex} for examples.

\subsection{The \env{multline} environment} The \env{multline}
environment is a variation of the \env{equation} environment used for
equations that don't fit on a single line.  The first line of a
\env{multline} will be at the left margin and the last line at the right
margin, except for an indention on both sides whose amount is equal to
\cs{multlinegap}.  The value of \cs{multlinegap} can be changed using
\LaTeX{}'s \cs{setlength} and \cs{addtolength} commands.  If the
\env{multline} contains more than two lines, any lines other than the
first and last will be centered individually between the margins.

\subsection{The \env{split} environment} Like \env{multline}, the
\env{split} environment is for {\em single} equations that are too long
to fit on one line and hence must be split into multiple lines.  Unlike
\env{multline}, however, the \env{split} environment provides for
alignment among the split lines, using \verb'&' to mark alignment
points, as usual. In addition, unlike the other \opt{amstex} equation
structures, the \env{split} environment provides no numbering, because
it is intended to be used only inside some other displayed equation
structure, usually an \env{equation}, \env{align}, or \env{gather}
environment, which provides the numbering.

\subsection{Alignment environments that don't constitute an entire display}
In addition to the \env{split} environment, there are some other
equation alignment environments that do not constitute an entire
display.  They are self-contained units that
can be used inside of other formulas, or set side-by-side.  The
environment names are \env{aligned}, \env{gathered}, and
\env{alignedat}.  These environments take an optional argument
to specify their vertical positioning with respect to the material
on either side.  The default is {\tt[c]}.  A \env{gathered}
environment with the first line level with the material on either
side would be done like this.
\begin{verbatim}
\begin{gathered}[t]
...\\
...
\end{gathered}
\end{verbatim}


\subsection{Vertical spacing and page breaks  in the \opt{amstex}
equation structures} You can use the
\cs\bslash\verb'['\<dimension>\verb']' command to get extra vertical
space between lines in all the \opt{amstex} displayed equation
environments, as is usual in \LaTeX{}\@.  Unlike \env{eqnarray},  the
\opt{amstex} environments don't allow page breaks between lines, unless
\cs{display\-break} or \cs{allow\-display\-breaks} is used.  The
philosophy is that page breaks in such situations should receive
individual attention from the author.  \cs{display\-break} must go
before the \cs\bslash\ where it is supposed to take effect.  Like
\LaTeX{}'s \cs{page\-break}, \cs{displaybreak} takes an optional argument
between 0 and 4 denoting the desirability of the pagebreak.
\verb'\displaybreak[0]' means ``it is permissible to break here''
without encouraging a break; \verb'\displaybreak' with no optional
argument is the same as \verb'\displaybreak[4]' and forces a break.

There is also an optional argument for \cs{allow\-display\-breaks}.
\cs{allow\-display\-breaks} obeys the usual \LaTeX{} scoping rules; the
normal way of limiting its scope would be to put
\verb'{'\cs{allow\-display\-breaks} at the beginning and \verb'}' at the end
of the desired range.  Within the scope of an \cs{allow\-display\-breaks}
command, the \verb'\\*' command can be used to prohibit a pagebreak, as
usual.

\subsection{The \cs{intertext} command} The command \cs{intertext}
is used for a short interjection of one or two lines of text in the middle of a
display alignment.  Its salient feature is preservation of the
alignment, which would not be possible if you simply ended the display
and then started it up again afterwards. \cs{intertext} may only appear
right after a \verb'\\' or \verb'\\*' command.  An example of
its use follows.
\begin{verbatim}
\begin{align}
A_1&=N_0(\lambda;\Omega')-\phi(\lambda;\Omega'),\\
A_2&=\phi(\lambda;\Omega')-\phi(\lambda;\Omega),\\
\intertext{and}
A_3&=\cal N(\lambda;\omega).
\end{align}
\end{verbatim}
Here the word ``and'' would fall outside the display,
at the left margin.

\subsection{Equation numbering}

In \LaTeX{} if you wanted to have equations numbered within
sections---that is, have
equation numbers (1.1), (1.2), \dots, (2.1), (2.2),
\dots, in sections 1, 2, and so forth---you would probably redefine
\verb'\theequation':
\begin{verbatim}
\renewcommand{\theequation}{\thesection.\arabic{equation}}
\end{verbatim}
This works fine except that the equation counter won't be reset to
zero at the beginning of a new section or chapter, unless you do it
yourself using \verb'\setcounter'.  To make this a little more
convenient, the \opt{amstex} option provides a command
\cs{numberwithin}.  To have equation numbering tied to section
numbering, with automatic reset of the equation counter,
the command would be
\begin{verbatim}
\numberwithin{equation}{section}
\end{verbatim}
As the name implies, \cs{numberwithin} can be applied to other
counters besides the equation counter, but the results are not
guaranteed because of potential complications.  Normal \LaTeX{} methods
should be used where available, e.g., in \cs{newtheorem}.

To make cross-references to equations easier, an \cs{eqref}
command is provided.  This automatically supplies the parentheses
around the equation number, and adds an italic correction if necessary
(see Section~\ref{s:rom}).  To refer to an equation that
was labeled with the label {\tt e:baset}, the usage would be
\verb'\eqref{e:baset}'.

\subsection{Error messages}

One kind of error message in particular should be mentioned, since it
follows from a mistake that is easy to make.

\begin{verbatim}
Runaway argument?
 \left |\frac {\hat v(s)-\hat v(t)}{|\widetilde {D} \ETC.
! Paragraph ended before \multline* was complete.
<to be read again>
                   \par
l.17

? h
I suspect you've forgotten a `}', causing me to apply this
control sequence to too much text. How can we recover?
My plan is to forget the whole thing and hope for the best.

? e
\end{verbatim}

This usually means one of two things: Either you have an
equation alignment environment where the
end doesn't match the beginning---perhaps something
like
\begin{verbatim}
\begin{multline*}
...
\end{multline}
\end{verbatim}
(as in this case)---or else you have a missing \verb'}' or \verb'\right'
delimiter inside of the environment.  A \verb'}' is rather easy
to leave off accidentally when using certain commands,
such as \verb'\frac'.

\section{Miscellaneous}

In the \opt{amstex} option \verb'~'\index{~@{\tt\string~}}, \cs{/}, and
\cs{slash} will remove superfluous spaces on either side of them, as a
convenience to the user (in the case of \cs{/}, only a space on the
left will be removed).  For example, if you have typed
% \tt space was too wide, so used ~ instead:
\verb'p.'~\verb'63' and then realize you should add a \verb'~', you can
insert the \verb'~' without deleting the space.

In ordinary \LaTeX{} \cs{big}, \cs{bigg}, \cs{Big}, and \cs{Bigg}
delimiters don't adjust properly over the full range of \LaTeX{} font
sizes.  In the \opt{amstex} option they do.

\section{New documentstyle options available}
\label{s:options}

Several new documentstyle
options have been created.  About half of them
have to do with the positioning of ``limits'' or \cs{tag}s.
The abbreviation of the names reflects the MS-DOS limitation
of eight characters for file names,\footnote{Not including the file
extension.} which we need to allow for.%
\begin{center}
\begin{tabular}{lp{20pc}}
 \opt{nosumlim}&      No limits on sums\\
 \opt{intlim}&        Limits on integrals\\
 \opt{nonamelm}&      No limits on operatornames\\
 \opt{ctagsplt}&      Vertically centered tags
                        on the \env{split} environment\\
 \opt{righttag}&      Equation tags on the right
\end{tabular}
\end{center}

Some of the component parts of the \opt{amstex} option are also
available individually, that is, they can be used in the options list
of the \cs{documentstyle} command:
\begin{center}
\begin{tabular}{lp{20pc}}
\opt{amstext}&  defines \cs{text}\\
\opt{amsbsy}&   defines \cs{boldsymbol} and \cs{pmb}\\
\opt{amsfonts}& defines \cs{frak} and \cs{Bbb}
and  sets up the fonts {\sc msam} (extra math symbols A), {\sc msbm}
(extra math symbols B, and blackboard bold), {\sc eufm} (Euler Fraktur),
as well as extra sizes of {\sc cmmib} (bold math italic and bold
lowerCase Greek), and {\sc cmbsy} (bold math symbols and bold script),
for use in mathematics.  This requires the AMSFonts package, version 2.0
or later.
\end{tabular}
\end{center}

Another option, \opt{amssymb}, defines names for all
the symbols in the AMS math symbol fonts.   It can be used with or
without the \opt{amstex} option.  Like the \opt{amsfonts} option, it
requires version 2.0 or later of the AMSFonts package.

\subsection{Comments}
\label{s:comment}

A new \opt{verbatim} style option written by Rainer Sch\"opf (and
distributed along with the \AmS-\LaTeX{} package) provides a \env{comment}
environment;
anything you write between \verb'\begin{comment}' and
% Splitting up \end%{comment} here allows this part of the documentation to be
% commented out! mjd
\verb'\end{'\verb'comment}' is totally ignored by \LaTeX{}\@.  The
\verb'\end{'\verb'comment}' should be on a line by itself: any text
after \verb'\end{'\verb'comment}' on the same line would be ignored (and
you would receive a warning message that it was lost).  Inside the
\env{comment} environment \LaTeX{} is in a special state that is
ended by the first occurrence of the \verb'\end{'\verb'comment}'
command; you cannot have one \env{comment} environment nested inside
another.  The \opt{verbatim} option provides some
other nice features; see \fn{verbatim.doc} for further details.

\subsection{Syntax checking}

Another new style option is called \opt{syntonly}; if you include this
in your document options list, then you can put \cs{syntaxonly} in the
document preamble to run the file with syntax check only.  No output
will be produced, but any \LaTeX{} errors will be uncovered.  The
advantage of this is that \LaTeX{} will run significantly faster when
\cs{syntaxonly} is in effect.  How much faster depends on the particular
computer being used and other variables but 30\%--40\% is typical.

\section{Protecting fragile commands}

Many of the commands added by the \opt{amstex} option are fragile and
will need to be \cs{protect}ed in commands with ``moving
arguments''---\cs{section} and other sectioning commands, \cs{caption},
\cs{addcontentsline}, \cs{addtocontents}, \cs{markboth}, \cs{markright},
\verb'@'-expressions in an \env{array} or \env{tabular} environment, and
others (see the \LaTeX{} manual, Section~C.1.3).
\sloppypar

\section{Differences the \protect\LaTeX{} user should note}

In \opt{amstex} the \verb'@' character has a special use, in the extensible
arrows \verb'@>>>' and \verb'@<<<' and in the math microspacing commands
\verb'@,' and \verb'@!'.  In order to get an ordinary printed
\atsign{} character, type \verb'@@' instead of
\verb'@'.\index{"@@{{\tt\atsign} (at sign)}}

With the various alignment environments available in the \opt{amstex}
option, the \env{eqnarray} environment is no longer needed.  Furthermore,
since it does not prevent overlapping of the equation numbers with wide
formulas, as most of the \opt{amstex} alignments do, using the
\opt{amstex} alignments seems better.

\cs{nonumber} is interchangeable with \cs{notag}; the
latter seems slightly preferable, for consistency with the name of \cs{tag}.

In math \cs{bf}, \cs{rm}, and other text font commands
should not be used for single math variables; \cs{bold},
\cs{mathrm}, etc.\ should be used instead.
(See section \ref{s:mathfonts} for details.)

\message{Part IV}
\newpage
\part{The \sty{amsart} and \sty{amsbook} documentstyles}
\label{artbook}

\section{General remarks}

Two considerations controlled the development of the \sty{amsart} and
\sty{amsbook} documentstyles.   First of all, their intended use is for
articles and books submitted for publication to the American
Mathematical Society (in addition to giving the \LaTeX{} user some
additional output styles).  And second, because \sty{amsart} and
\sty{amsbook} not only load the \opt{amstex} option, but also add
several features not found in the standard \LaTeX{} styles, they don't
have much spare memory to work with (if used with a ``small''
implementation of \TeX{}).

Therefore some features of the standard
\LaTeX{} documentstyles that are needed rarely (or never) in
AMS publications have been omitted or minimized in an effort to conserve
memory. No special provisions have been made for setting up marginal
notes or two-column format, for example. And the 11pt and 12pt options
have been reduced to a minimal kernel: they do nothing except reset
the margins and a few font sizes.  More sophisticated adjustments that
are done in the standard \LaTeX{} \sty{article} and \sty{book} styles
are omitted.

The \opt{fleqn} option and the \opt{openbib} bibliography style are
not used in AMS publications, and therefore the necessary
work to make them available has not been done.

No provision is made for fonts in sizes larger than \cs{large}; the
\LaTeX{} commands  \cs{Large}, \cs{LARGE}, \cs{huge}, and \cs{Huge}
still function normally but the size they produce is the same as for
\cs{large}. The design of \sty{amsart} does not use internally
anything larger than \cs{normalsize}.

The \opt{amstex} option, which is part of the \sty{amsart} and
\sty{amsbook} styles, does the necessary setting up to allow the use of
fonts from the AMSFonts collection (version 2.0+), but it is  possible to use
\sty{amsart} and \sty{amsbook} without having the AMSFonts. There is one
restriction, however: the {\tt.tfm} files for the math symbol fonts of the
AMSfonts package (\fn{msam}, \fn{msbm}, \fn{cmmib}, \fn{cmbsy})
are required, even if your document doesn't actually
use any symbols from the fonts.

Detailed information/instructions about using the \sty{amsart} and
\sty{amsbook} documentstyles for electronic submissions to
AMS publications can be found in {\it Guidelines for preparing
electronic manuscripts---\AmS-\LaTeX{}\/} \cite{author-guidelines}.

\section{The \sty{amsart} documentstyle}

\subsection{Top matter}
We use the term ``top matter'' for the information found at the
beginning of an article, such as the title, author, addresses,
and abstract.
Compared to the standard \sty{article} documentstyle, the
\sty{amsart} documentstyle has a significantly expanded top
matter section.
\LaTeX{}'s \sty{article} style provides \cs{title}, \cs{author},
\cs{thanks}, \cs{date}, and an \env{abstract} environment.  The
complete list of top matter commands provided by the \sty{amsart}
style is:

\begin{center}
\begin{tabular}{@{\hspace{1em}}lll}
\cs{title}&         \cs{keywords}&      \cs{author}\\
\cs{subjclass}&     \cs{address}&       \cs{curraddr}\\
\cs{email}&         \cs{translator}&    \cs{dedicatory}\\
\cs{thanks}&        \cs{date}\\
\end{tabular}
\end{center}

All of these commands
should precede the \cs{maketitle}
command.  If the\linebreak[2] \env{abstract} environment is used,
it should follow immediately after \cs{maketitle}.
The address, current address, e-mail address,
and translator information print at the end of the document;
the key words, subject classification, and thanks information
print as footnotes at the bottom of the first page of the document.

An \cs{author} command should be used for each individual author,
when a paper has multiple authors.
Things like \cs{address}, \cs{curraddr}, \cs{email}, and
\cs{thanks} that pertain only to one author  should be placed after
the \cs{author} command that they go with (and before any other
\cs{author} commands).  The AMS custom is to list author names in
alphabetical order. (See {\bf Author names and addresses} in section
\ref{s:variations} for further details.)

In giving an e-mail address remember that
\verb'@'\index{"@@{{\tt\atsign} (at sign)}} characters
should be doubled in order for them to print properly.

For submissions to the
American Mathematical Society, please provide as a minimum the
following information: title, author, addresses, mathematics
subject classification numbers, translator (if applicable),
and acknowledgments of funding support (\cs{thanks}).

\subsection{Memory conservation measures}
To free up valuable memory, commands that are needed only at the
beginning of a document are undefined when they are no longer needed.
This includes the top matter commands \cs{title},
\cs{author}, etc., and the
\env{abstract} environment.

\subsection{Running heads}

Running heads on odd-numbered pages (right-hand pages) in the
\sty{amsart} style contain the text of the article title, and on
even-numbered pages they contain the author's name.  If the title is
too long to fit within the page width, a shorter version
for the running head text can be
specified with a square-bracket option of the \cs{title}
command:
\begin{verbatim}
\title[Short Version Here]{Long Version of the Title Here,\\
  Perhaps with Multiple Lines}
\end{verbatim}
The \cs{author} command has also been given the same
kind of square-bracket option.

\subsection{Non-English versions of automatically generated text}

If the base language of an article is some language other than
English, the user may wish to change some pieces of text that
are generated automatically.  To change ``Abstract'' to ``R\'esum\'e'',
use \cs{renewcommand} to redefine \cs{abstractname}:
\begin{verbatim}
\renewcommand{\abstractname}{R\'esum\'e}
\end{verbatim}
The user can change the following in the same way:%
\footnote{The names of the control sequences
were chosen to match the names used in \fn{babel.sty}.}
\begin{verbatim}
\abstractname   Abstract
\partname       Part
\indexname      Index
\figurename     Figure
\tablename      Table
\proofname      Proof
\refname        References
\appendixname   Appendix
\tocname        Contents
\end{verbatim}
This also allows the user to substitute,
e.g., ``Diagram'' instead of ``Figure'' for the labels of
figure environments.
In the \sty{amsbook} style, there are some additional
names available for changing:
\begin{verbatim}
\chaptername    Chapter
\listfigurename List of Figures
\listtablename  List of Tables
\bibname        Bibliography
\end{verbatim}
(The environment \env{thebibliography} uses \cs{bibname} in the
\sty{amsbook} style, and \cs{refname} in the
\sty{amsart} style.)

\subsection{Theorems, definitions, and similar structures}

\LaTeX{} provides \cs{newtheorem} to create theorem
environments.  The \sty{amsart} and \sty{amsbook} styles make
use of the \opt{theorem} documentstyle option to provide more
flexibility in the design of theorems, definitions, proofs,
remarks, and the like (for full details, see Frank
Mittelbach's article in {\it TUGboat\/}, vol.~10, no.~3,
November 1989, pp.~416--426).  Three levels of
theorem-type environments are provided through
three \cs{theoremstyle}s: {\tt plain}, {\tt definition},
and {\tt remark}.  The different styles receive different
typographical treatment that gives them visual emphasis
corresponding to their relative importance in the
author's exposition.

To create new theorem-type environments in the
different styles, use the\linebreak[1] \cs{newtheorem} command
in the normal way, but divide your \cs{newtheorem} commands
into groups and preface each group with the appropriate
\cs{theoremstyle}.
If no \cs{theoremstyle} command is given, the style used will
be \fn{plain}.  The\linebreak[1] \cs{theorembodyfont} and
\cs{theoremheaderfont} commands described in Mittelbach's article
are unnecessary in the AMS documentstyles.

Here is an example of a rather comprehensive \cs{newtheorem} section:
\begin{verbatim}
% theorem style plain --- default
\newtheorem{thm}{Theorem}[section]
\newtheorem{lem}{Lemma}[section]
\newtheorem{cor}{Corollary}[section]
\newtheorem{prop}{Proposition}[section]
\newtheorem{crit}{Criterion}[section]
\newtheorem{alg}{Algorithm}[section]

\theoremstyle{definition}

\newtheorem{defn}{Definition}[section]
\newtheorem{conj}{Conjecture}[section]
\newtheorem{exmp}{Example}[section]
\newtheorem{prob}{Problem}[section]

\theoremstyle{remark}

\newtheorem{rem}{Remark}        \renewcommand{\theremark}{}
\newtheorem{note}{Note}         \renewcommand{\thenote}{}
\newtheorem{claim}{Claim}
\newtheorem{summ}{Summary}      \renewcommand{\thesumm}{}
\newtheorem{case}{Case}
\newtheorem{ack}{Acknowledgment}  \renewcommand{\theack}{}
\end{verbatim}
If you would like an unnumbered environment, use \cs{renewcommand}
to undefine\linebreak[1] {\tt\bslash thexxxx} (where {\tt xxxx}
stands for the environment name), as shown in the
``remark'' section.  If you have a theorem with a special name,
such as ``Klein's Theorem,'' use a separate \cs{newtheorem}
command just for that theorem and make it unnumbered:
\begin{verbatim}
\newtheorem{kthm}{Klein's Theorem}
\renewcommand{\thekthm}{}
\end{verbatim}
This will give the normal format for a theorem in all respects
except the automatic numbering.

\subsection{Proofs}

A predefined \env{pf} environment and a starred form \env{pf*}
are provided for
proofs, and produce the heading ``Proof'' with appropriate
spacing and punctuation.
The proof environment is primarily intended for short proofs, less than
a page in length; longer proofs should probably be done as a separate
section or subsection in your document.

A ``Q.E.D.'' symbol,
$\vcenter{\hrule\hbox{\vrule height.6em\kern.6em\vrule}\hrule}$,
is automatically appended at the end of a proof.
To substitute
a different end-of-proof symbol, use \cs{renewcommand}
to redefine the command \cs{qedsymbol}.  For a long
proof that doesn't use the \env{pf} environment,
you can obtain the symbol and the usual amount of preceding space
by using \cs{qed}.

Placement of the \cs{qedsymbol} can be problematic if the
last part of a \env{pf} environment is a displayed equation
or list environment or something of that nature.  Reasonably
good results can generally be obtained by using \cs{qed} at the
appropriate spot and then undefining \cs{qed} before the end
of the proof.  (The effect will be automatically localized
to the current proof by normal \LaTeX{} scoping rules.)  For example:
\begin{verbatim}
\begin{pf}
...
\begin{equation}
G(t)=L\gamma!\,t^{-\gamma}+t^{-\delta}\eta(t) \qed
\end{equation}
\renewcommand{\qed}{}\end{pf}
\end{verbatim}

The starred form, \env{pf*}, of the proof environment takes an
argument in curly braces, which allows you to
substitute a different name for the standard ``Proof''.
If you want to substitute, say, ``Proof (sufficiency)'', then write
\begin{verbatim}
\begin{pf*}{Proof (sufficiency)}
\end{verbatim}
and close the proof with \verb'\end{pf*}' instead of
\verb'\end{pf}'.

\subsection{Miscellaneous notes}
\subsubsection{Variations from standard \protect\LaTeX{}}
\label{s:variations}

Variations from standard \LaTeX{} that are simple additions
(like \verb'\subjclass' for subject classification numbers)
will not be pointed out in this section.  However,
a couple of variations that involve contradictions of
statements in the \LaTeX{} manual need to be noted.

\paragraph{Starred forms of sectioning commands}
In the {\tt amsart} and {\tt amsbook} documentstyles starred forms of
the \cs{chapter} and \cs{section} commands produce a table of contents
entry.  This is a variation from standard \LaTeX{}
(see the \LaTeX{} manual, \S C.3.1), but more in keeping with
usual publishing practice.

\paragraph{Author names and addresses}
The standard \LaTeX{} format for specifying the names and addresses
 of a document's authors is this:
\begin{verbatim}
\author{First Author\\Address, Line 1\\Address, Line 2
        \and Second Author\\Address, Line 1\\Address, Line 2}
\end{verbatim}
In the \sty{amsart} and \sty{amsbook} documentstyles
there is a separate
\cs{address} command for addresses, and the author names and addresses are
specified individually like all the other elements of the
top matter:
\begin{verbatim}
\author{First Author}
\address{Address, Line 1\\Address, Line 2}
...
\author{Second Author}
\address{Address, Line 1\\Address, Line 2}
\end{verbatim}

Addresses (including current address and e-mail address) will be
associated with the nearest preceding \cs{author} command to determine
where they should be printed.

Author names and addresses typed in the standard \LaTeX{} format
will still print fine, without error messages, but the addresses
may not fall in the proper place (at the end of the document, in
the \sty{amsart} style).

\subsubsection{Numbers and punctuation in italic text}
\label{s:rom}

Mathematical typesetting poses a problem that rarely arises in
nonmathematical typesetting.  In mathematical formulas,
for consistency,  parentheses and other punctuation, as
well as numbers, are always set in an upright font rather than
varying with the surrounding text.   But then when a math formula with
non\-italic numbers and punctuation occurs in the middle of italicized
text---e.g., in a theorem---with italic numbers and punctuation
nearby, the visual discrepancy rises up to smite the reader
in the eye. Consider the following example.

\begin{prop}
(1)~If $\tau=(\tau_1,\dots,\tau_d)$ are
horocyclic coordinates on ${\bf T}(\Gamma)$, then the plumbing
coordinates defined in \S 7.3 ($t=(t_1,\dots,t_d)$)
are coordinates on the quotient space.

(2) We have ${\bf D}_0({\cal G})\cong{\bf T}(\Gamma)$:
hence ${\bf D}_0({\cal G})$ is a domain of holomorphy.
\end{prop}

Therefore it is conventional in mathematical publishing to use the
same upright style for numbers and punctuation in italic text as is
used in the mathematics: {\it {\rm(1)}~The quotient space \dots\
in \S{\rm7.3} {\rm(}$t=(t_1,\dots,t_d)${\rm)} \dots} and
{\it {\rm(2)}~We have ${\bf D}_0({\cal G})\cong{\bf T}(\Gamma)${\rm:}
hence \dots.}

At the present time, italic fonts with upright numbers and
punctuation aren't available.
The work of getting upright numbers and punctuation in italic text
therefore must be done some other way.  In order to
help the author, the \sty{amsart} and \sty{amsbook} documentstyles
take two steps.  First, they do as much as possible automatically,
behind the scenes.  For example, \cs{ref} usually produces a number
of some sort; the definition of \cs{ref} has been changed so
that the number will never print in italic.  Second, they provide
a control sequence \cs{rom}, with one argument, that can be used by the author
when necessary to make an individual punctuation mark or number
nonitalic.  For example:
\begin{verbatim}
... formal \rom{(} in the previous year, \rom{1989)} ...
\end{verbatim}
Italic corrections are inserted automatically by \cs{rom}.

The use of \cs{rom} is unnecessary for punctuation marks that are not
high enough to have a noticeable slant, such as commas and periods.

\section{The \sty{amsbook} documentstyle}

The \sty{amsbook} style has much in common with the \sty{amsart}
style; everything in the previous sections about \sty{amsart}
holds true for \sty{amsbook}, excepting some details
such as the placement of author addresses and other top matter
information.

\subsection{Front matter}
The ``top matter'' information for a book (more commonly called the front matter,
when discussing a book)
is usually specially made up on a title page, with the format
varying widely from book to book.  In the \sty{amsbook} design
\cs{maketitle} produces a simple title page with the title
and author; subject classification numbers, abstract, or
key words, if supplied, will print on the following page.
The style is rather plain because it's not intended for actual publication;
its purpose is to make it convenient for authors to provide
the necessary information to a publisher.

For submissions to the
American Mathematical Society, please provide as a minimum the
following information: title, author, addresses, mathematics
subject classification numbers, translator (if applicable),
and acknowledgments of funding support (\cs{thanks}).

\subsection{Running heads}

Right-hand running heads in the \sty{amsbook} style contain
the text of the current section heading; left-hand running
heads contain the current chapter title.  For special chapters
such as a preface or bibliography that don't have sections,
the right running head will be the same as the left.
Square-bracket options can be used, as normal, to change
the text used for running heads.

\section{Bibliography styles for use with \protect\BibTeX{}}

The \AmS-\LaTeX{} distribution includes two bibliography styles,
\fn{amsplain} and \fn{amsalpha},  \index{amsplain@{\tt amsplain}
bibliography  style}\index{amsalpha@{\tt amsalpha} bibliography style}
analogous to the standard \LaTeX{} \fn{plain} and \fn{alpha}
bibliography \index{plain@{\tt plain} bibliography
style}\index{alpha@{\tt alpha} bibliography style} styles.  In the AMS
styles an extra field ``language'' is provided, for giving the original
language of a reference, as an indication to the reader that the title,
author name, and so on are translated.

Also included is a file \fn{mrabbrev.bib} containing  standardized
abbreviations used by {\it Mathematical Reviews\/} for journal names in
the mathematical sciences and related fields. Because the full list is
too big to be handled by the current version of \BibTeX{}, individual
users should use it as a resource, extracting abbreviations for the
journals that they cite in their particular bibliography database and
adding them to their database.

\newpage
\part{Appendixes}
\appendix

\message{Appendix A}
\section{Installation instructions}
\label{a:install}

\subsection{Introduction}

\AmS-\LaTeX{} can be used with any standard implementation of \TeX. See
Subsection~\ref{as:capacity} for a listing of recommended \TeX{} capacities for
using this package. All the files for \AmS-\LaTeX{} can be obtained by
anonymous FTP from the AMS Internet archive, {\tt e-MATH.AMS.org} (IP number
130.44.1.100 (August 1991)). Those without Internet access may obtain the files
on diskette from the AMS (see section \ref{a:furtherinfo}), and they are also
included with many of the commercial versions of \TeX{}. The installation
instructions below describe a DOS installation of \AmS-\LaTeX, although the
procedures can be generalized to apply to non-DOS environments.
Section~\ref{Textures:install} contains supplementary installation instructions
for use with \Textures{} on the Macintosh.

If you do not have a recent version of the file \verb'latex.tex', then
you  may have difficulty using the \AmS-\LaTeX{} package.  Checking the
version number is not sufficient because there have been several
releases called Version 2.09.   There is,
however, a date that is printed on the terminal screen, after the
version number, whenever you run \LaTeX{}. Problems have been reported
with versions dated as late as August 1988. The latest version (as of
August 1991) is dated January 14, 1991. A date of \verb'<24 May 1989>'
or later will avoid known problems.

If you need a new copy of \fn{latex.tex}, the alternatives are: (1)~if
you have a commercial version of \TeX{}, contact the company to see
about getting a more recent copy of \fn{latex.tex}; (2)~retrieve it by
anonymous FTP from the Internet archive \fn{labrea.stanford.edu} (the
canonical source) or from the AMS archive \fn{e-MATH.AMS.org};
(3)~contact the \TeX{} Users Group and ask about alternative
sources---see Appendix~\ref{a:help}.

\subsection{Files included in this distribution}

The subdirectory \verb'doc' contains the following files:

\begin{center}
\tt
\begin{tabular}{l@{\hspace{3em}}l@{\hspace{3em}}l}
READ.ME&      amslatex.tex& testart.bbl\\
amsart.doc&   amslatex.toc& testart.tex\\
amsart10.doc& app.tex&      testbook.bbl\\
amsart11.doc& chap1.tex&    testbook.tex\\
amsart12.doc& chap2.tex&    theorem.doc\\
amsbk10.doc&  pref.tex&     thp.doc\\
amsbook.doc&  test.bib&     verbatim.doc\\  
amslatex.aux\\
\end{tabular}
\end{center}

\smallskip
\noindent The subdirectory \verb'fontsel' contains the following files:
{\samepage

\begin{center}
\tt
\begin{tabular}{l@{\hspace{3em}}l@{\hspace{3em}}l}
READ.ME&      fontsel.bug&     preload.med\\
array.sty&    fontsel.tex&     preload.min\\
basefont.tex& lfonts.new&      preload.ori\\
concrete.doc& margid.sty&      readme.mz\\
concrete.sty& newlfont.sty&    readme.mz3\\
euscript.sty& nomargid.sty&    syntonly.sty\\
fontdef.max&  oldlfont.sty&    tracefnt.sty\\
fontdef.ori\\
\end{tabular}
\end{center}
}% End of \samepage

\smallskip
\noindent The subdirectory \verb'inputs' contains the following files:

{\samepage
\begin{center}
\tt
\begin{tabular}{l@{\hspace{3em}}l@{\hspace{3em}}l}
READ.ME&      amscd.sty&    mrabbrev.bib\\
amsalpha.bst& amsfonts.sty& nonamelm.sty\\
amsart.sty&   amsplain.bst& nosumlim.sty\\
amsart10.sty& amssymb.sty&  preload.max\\
amsart11.sty& amstex.sty&   righttag.sty\\
amsart12.sty& amstext.sty&  theorem.sty\\
amsbk10.sty&  ctagsplt.sty& thp.sty\\
amsbook.sty&  fontdef.ams&  verbatim.sty\\
amsbsy.sty&   intlim.sty&   \\
\end{tabular}
\end{center}
}% End of \samepage

\smallskip
\noindent The subdirectory \verb'tfm' contains the following files:

{\samepage
\begin{center}
\tt
\begin{tabular}{l@{\hspace{2em}}l@{\hspace{2em}}l@{\hspace{2em}}l}
READ.ME&        cmex8.tfm&    dummy.tfm&      msbm10.tfm\\
cmmib5.tfm&     cmex9.tfm&    msam10.tfm&     msbm5.tfm\\
cmmib6.tfm&     cmbsy5.tfm&   msam5.tfm&      msbm6.tfm\\
cmmib7.tfm&     cmbsy6.tfm&   msam6.tfm&      msbm7.tfm\\
cmmib8.tfm&     cmbsy7.tfm&   msam7.tfm&      msbm8.tfm\\
cmmib9.tfm&     cmbsy8.tfm&   msam8.tfm&      msbm9.tfm\\
cmex7.tfm&      cmbsy9.tfm&   msam9.tfm \\
\end{tabular}
\end{center}
}% End of \samepage

\subsection{Copying files to the appropriate directories}
\label{inst-copy}

If you have an earlier version of \AmS-\LaTeX{} and wish to maintain
a backup copy, start by backing up the files from
that version (referring to the file list for that version).
Normal installation, as described below, will
overwrite the files listed above; however, two files
from version 1.0, named \fn{amsbk11.sty} and
\fn{amsbk12.sty}, must be deleted separately
because they are not present in the latest version.

On your system, create a directory named \fn{amslatex}, and three
subdirectories named \fn{doc}, \fn{fontsel}, and \fn{inputs}. Then, copy the
contents of the corresponding subdirectories  of the \AmS-\LaTeX{} distribution
into those subdirectories. If your \TeX\ inputs path is user-definable, you
should now append it to include the \fn{fontsel} and \fn{inputs}
subdirectories. If it is not user-definable (such as with {\it Textures\/}),
then you should copy the entire contents of these two directories into the
directory where \TeX\ looks for input files.

If you do not have AMSFonts (version 2.0 or later) installed, you will
also need at least the \fn{.tfm} files for the math symbol fonts in the
AMSFonts package (even if you do not intend to use any of the AMSFonts).
These \fn{.tfm} files are included with the \AmS-\LaTeX{} distribution
under a separate directory named \fn{tfm}. Copy these files into
the directory on your system where your implementation of \TeX{} looks
for \fn{.tfm} files.

Once the files have been copied to the appropriate places, the
main ones of interest will be:
\[\begin{tabular}{ll}
$\left.\begin{tabular}{l}
lfonts.new\\ fontdef.*\\ preload.*\\ basefont.tex\\ newlfont.sty
\end{tabular}\right\}$&
        Used in creating a new \LaTeX{} format file\\[6pt]
\ \ amstex.sty&
        \LaTeX{} option file\\[3pt]
\ \ amsart.sty&
        Main documentstyle for AMS article formatting\\[3pt]
\ \ amsbook.sty&
        Main documentstyle for AMS book formatting\\[3pt]
\ \ amslatex.tex&
        {\it\AmS-\LaTeX{} User's Guide}\\[6pt]
$\left.\begin{tabular}{l}
app.tex\\ chap*.tex\\ pref.tex\\ test*.*
\end{tabular}\right\}$&
Examples
\end{tabular}\]

Use of the \AmS-\LaTeX{} package is dependent on the font selection
scheme of Mittelbach and Sch\"opf described in {\it TUGboat} {\bf 11},
no.~2, June 1990, pp.~297--305. This means that you need either to
receive from someone else a new \LaTeX{} format file based on Mittelbach
and Sch\"opf's scheme (some of the distributors of \TeX{} may provide
this), or make one yourself (instructions follow), using {\sc initex}, a
version of \TeX{} with no preloaded format.

Note that the file for the complete {\it \AmS-\LaTeX{} User's Guide}
(\verb'amslatex.tex') can be typeset using your old \LaTeX{} format file, so,
if you haven't done so already, you can typeset it using \LaTeX{} in the usual
way and read it before proceeding further.

\subsection{Making a new format file}

In order to create a new \LaTeX{} format file that uses the
Mittelbach--Sch\"opf font selection
scheme, you must first rename the file \verb'lfonts.tex' that was distributed
with your original \LaTeX{} distribution, so that it won't interfere.
This file is in one of the directories
where your implementation of \TeX{} looks for input files;
e.g., for PC\TeX{} it would be  \verb'\pctex\texinput' or \verb'\pctex\latex'.
Rename the file to something
such as \verb'olfonts.tex' (``O'' for ``original'').  You may also want to
rename the file \verb'lplain.fmt', in the directory where your \TeX{} looks for
format files, to \verb'olplain.fmt',  if you want to continue to be able to use
the old version of \LaTeX{} as well as the new version.  Otherwise this
file will be overwritten during the installation process.

Then follow the directions for creating a new \LaTeX{} format file, using
{\sc initex}, in the documentation for your implementation of \TeX. The
directions will vary from one implementation of \TeX{} to another, but
basically what you want to do is run the program {\sc initex}
as if you were running \TeX{}, using \verb'lplain.tex' as the input file.
(Remark: For PC\TeX{} there is an ``i'' option for the {\tt tex} command rather
than a separate {\tt initex} command.)
(\verb'lplain.tex' is a standard \LaTeX{} file
that should already be installed on your system.  It is not included
in the \AmS-\LaTeX{}  distribution.)

After some initial processing, {\sc initex} will stop and ask for another
filename because it cannot find the file \verb'lfonts.tex'. (Remember, you
renamed it according to the first paragraph of this section.) When it stops,
type \fn{lfonts.new}
in response to the prompt (``Please type another input file name'') and
continue.  There are three other files that will be input in a similar
way. They are indicated in the following table.
\begin{eqnarray*}
\mbox{In place of:}&& \mbox{Substitute:}\\
\mbox{fontdef.tex}&& \left\{
\begin{tabular}{lp{16pc}}
% We use \rightskip here instead of \raggedright because
% \raggedright causes problems with \\ commands.
fontdef.ori& \rightskip0pt plus3em
             If you want to use only CM and \LaTeX{} fonts.\\
fontdef.ams& \rightskip0pt plus3em
             If you want to also use the basic math fonts from
             the AMSFonts collection (Euler Fraktur, bold math,
             extra math symbols MSAM and MSBM).\\
fontdef.max& \rightskip0pt plus3em
             If you want to also use other AMSFonts (Euler Roman or
             Script, cyrillic text fonts), or Concrete fonts.
\end{tabular}\right.\\ % end of second line of eqnarray
\mbox{preload.tex}&& \left\{
\begin{tabular}{lp{16pc}}
preload.min& \rightskip0pt plus3em
             To preload fewer fonts (uses less font memory,
             but may noticeably increase processing time).\\
preload.med& \rightskip0pt plus3em
             To preload a ``medium'' number of fonts.\\
preload.ori& \rightskip0pt plus3em
             To preload all basic \LaTeX{} fonts (leaves less
             memory for other fonts, but gives faster processing time).
\end{tabular}\right.\\ % end of third line of eqnarray
\mbox{xxxlfont.sty}&& \left\{
\begin{tabular}{lp{16pc}}
basefont.tex& \rightskip0pt plus3em If you intend to use AMSFonts.\\
newlfont.sty& \rightskip0pt plus3em If you don't intend to use
AMSFonts.
\end{tabular}\right.
\end{eqnarray*}
Note: When basefont.tex is used the following math symbols will be
undefined: \cs{mho},  \cs{Join},  \cs{Box},  \cs{Diamond},
\cs{leadsto},  \cs{lhd},  \cs{unlhd},  \cs{rhd}, and \cs{unrhd}.
Alternate versions of all these symbols exist in the AMSFonts math
symbol fonts (which presumably you have, if you used basefont.tex) and
they can be defined using \cs{newsymbol}.  See the {\it AMSFonts User's
Guide}.

After quite a bit of further processing you will wind up at a
\verb'*' prompt, whereupon you should type \cs{dump}
and return.  You will then have a file called \verb'lplain.fmt'.
Depending on your implementation of \TeX, this file may be automatically
placed in the directory
where \TeX{} looks for format files or it may be in the currently connected
directory. If the latter is true, you must copy it to the directory
where \TeX{} will look for format files
(see your \TeX's documentation for the name of that directory).

You might also want to copy the transcript file \verb'lplain.log' into the
directory with the format file, for future reference.

\subsection{Using the new format file}

The test files \verb'testart.tex' and \verb'testbook.tex',  in the
\verb'doc' subdirectory, are useful for testing your new
\verb'lplain.fmt' format file. These are examples using the
\verb'amsart' and \verb'amsbook' documentstyles.

To do this, follow the instructions for using a format file with your
implementation of \TeX. With PC\TeX{} and many other
implementations of \TeX, the command would be
\[\mbox{\tt tex \&lplain testart.tex\qquad ({\em or\/}
testbook.tex)}\]
although it may be different for your implementation.
Dumping the format file will take a long time on some slower machines.

\subsubsection{Math fonts}
\label{AMSmathfonts}

The \opt{amstex} option assumes the availability of the math symbol fonts MSAM
and MSBM from the AMSFonts package. If the \fn{.tfm} files for these fonts are
not available, there will be an error message at the first math formula in a
document, when \LaTeX{} attempts to load the MSAM and MSBM fonts. If you plan
to actually use the symbols in these fonts, or any of the other AMSFonts, you
will also need to obtain and install the \fn{.pk} files for AMSFonts which are
appropriate for your output device. However, if you do not need the extra
symbols from these fonts, you do not have to obtain the full AMSFonts package;
it is sufficient to install only the few \fn{.tfm} files that are referenced.
These \fn{.tfm} files are provided with the \AmS-\LaTeX{} distribution (see
Section~\ref{inst-copy}).

\subsection{Using the old format file}

Some of your old \LaTeX{} files may be incompatible with the new format
file. If you find this to be the case, you can ordinarily typeset such
a file by adding ``{\tt oldlfont}'' to the documentstyle options list.
Alternatively, of course, you could use the old \LaTeX{} format file
instead of the new one.  Assuming you saved the previous
\verb'lplain.fmt'  file under the name \verb'olplain.fmt', you would
use \verb'&olplain' instead of  \verb'&lplain' in the command line.

\subsection{Installation for use with \protect\Textures{}}
\label{Textures:install}

Users of \Textures{} on the Macintosh do not need to use {\sc initex} (in a
sense, {\sc initex} and ordinary \TeX{} are both contained in the \Textures{} 
program). Format files can be produced by running \Textures{} in the ordinary
way and inputting the desired macro files.

\subsubsection{Putting the Files in the Right Place}

The files in the \AmS-\LaTeX{} package are divided into the groups
`doc', `inputs', `fontsel' and `tfm'. Copy all the `inputs' and `fontsel'
files into the \fn{TeX Inputs} folder in your \fn{Textures} folder.
Copy the `doc' files into a separate folder named \fn{AMS-LaTeX}. The
final group consists of a single font metrics suitcase
called \fn{AMSFonts 2.1 Metrics}, which should be copied into your
\fn{TeX Fonts} folder.

\subsubsection{Creating a New Format}

Users of most implementations of \TeX{} create a \LaTeX{} format file by
starting with something called Vir\TeX{}, which has nothing preloaded---not
even the \fn{Plain} format. With Textures{} this is not normally possible,
because \fn{Plain} is always loaded when you start running \Textures. You may
create a \LaTeX{} format based on the \fn{Plain} format. However, it involves
some unnecessary overhead from \fn{Plain} definitions and preloaded fonts that
are not pertinent to \LaTeX{}, but  normally this extra overhead will not be a
problem. 

If you want to avoid that overhead, Blue Sky Research (the maker of
\Textures{}) distributes a special \Textures{} format called ``Vir\TeX{}'' that
can be used instead of the \fn{Plain} format in the installation procedure
described below. [There is one change in the installation procedure: you {\em
will} need to load the file \fn{hyphen.tex} instead of bypassing it as
instructed below.]  Vir\TeX{} is not currently included in the standard
\Textures{} distribution (because most users don't need it), but is available
to registered \Textures{} users  on request from Blue Sky Research at 534 SW
Third Avenue, Portland, OR 97204; (800) 622-8398 or (503) 222-9571.

Before making your new \LaTeX{} format file, there is one preliminary
step you may need to take. If there exists a file called
\fn{lfonts.tex} anywhere in your \fn{Textures} folder or subfolders,
rename it to something else, let's say \fn{Old lfonts.tex}. This is
to prevent \fn{lfonts.tex} from being automatically loaded before we
have a chance to substitute another file in its place.
If you don't happen to know whether you have such a copy of
\fn{lfonts.tex}, you can just run through the installation
procedure below; if you see \fn{lfonts.tex} being loaded
in the \fn{TeX log} window and no dialog box pops up to
say {\tt I can't read ``lfonts'' (not found)}, then
you do have an old \fn{lfonts.tex} file and you'll need to track
it down and rename it before trying again.

\begin{enumerate}
\item Open the \fn{Textures} folder if you haven't done so already.
\item Start up \Textures{}.
\item Open the file \fn{lplain.tex} on the diskette.
\item Go to the ``Typeset'' menu and make sure that the \fn{Plain}
format is marked as the current format. (I.e., it should have a
check mark next to it.)

\item Select {\tt Typeset} from the ``Typeset'' menu. Informational
messages will begin scrolling past in the \fn{TeX log} window
describing the progress of the format file creation. At various points
in the processing \Textures{} will inform you that it can't read various
files. Proceed as described below.

\item
If \Textures{} says it can't read ``hyphen'', click
the ``No'' button in the dialogue box, because \fn{hyphen.tex}
is already loaded inside the \fn{Plain} format. On the other
hand, \Textures{} might be able to find \fn{hyphen.tex} if you
already have a copy somewhere. If \Textures{} does read \fn{hyphen.tex},
you'll get an error message saying
\begin{verbatim}
! Too late for \patterns.
\end{verbatim}
Click on the ``Continue'' button or press the {\sc return} key
as instructed by the comment in \fn{hyphen.tex}; this error
is harmless because the \fn{Plain} format already contains
the hyphenation patterns.

\item Further processing will eventually lead to  a dialog box saying
{\tt I can't read ``lfonts'' (not found)}. Click the ``Yes'' button and
select \fn{lfonts.new} from the \fn{TeX inputs} folder. Other
substitutions will be made in the same manner; the complete substitution
list is  as follows.{\sloppypar}

\medskip
{\tt I can't read ...}
\begin{description}
\item[lfonts.tex:] Select \fn{lfonts.new}.
\item[fontdef.tex:] Select \fn{fontdef.ori}, \fn{fontdef.max}, or
       \fn{fontdef.ams}.
\item[preload.tex:] Select \fn{preload.ori}, \fn{preload.min}, or
       \fn{preload.med}.
\item[xxxlfont.sty:] Select \fn{oldlfont.sty}, \fn{newlfont.sty},
  or \fn{basefont.tex}.
\end{description}

(See section A.4 for an explanation of the choices in each case.)

\item At this point \Textures{} will stop and present a \verb"*" prompt
in the log window. Enter the command \cs{dump} (followed by a {\sc
return}) to cause everything loaded so far to be saved in a new format
file. You will be asked to give a name for the new format file; before
giving the name, proceed through the Macintosh folder hierarchy to reach
the \fn{TeX formats} folder, so that the format file will be saved into
that folder. Then enter \fn{AMS-LaTeX} as the name for the format file.

\end{enumerate}

That completes the installation process; the new format file can
now be used by adding it to the format list with the ``Add Format''
command, and then selecting it in the ``Typeset'' menu. To test it,
try typesetting the sample document, \fn{testart.tex}. However,
if you do not plan to install the AMSFonts package along with
\AmS-\LaTeX{}, see Section~\ref{AMSmathfonts} above.


\message{Appendix B}
\section{Files included in the \protect\AmS-\protect\LaTeX{} distribution}

The total number of files in the \AmS-\LaTeX{} package, including
documentation and option files, is more than sixty.
A majority of these files are for the
Mittelbach--Sch\"opf font selection scheme and other \LaTeX{}
option files written and maintained by Mittelbach and Sch\"opf.  They
are used by various parts of the \AmS-\LaTeX{} package but are not
inherently part of the \AmS-\LaTeX{} distribution; they are included at
the present time because they have not yet become widely available in
the United States.

\subsection{Files maintained by the American Mathematical Society}

\begin{filelist}
\fn{amslatex.tex}\newline\fn{amslatex.aux}\newline\fn{amslatex.toc}&

This {\it User's Guide}, describing the \AmS-\LaTeX{} package, and the auxiliary
files generated by \LaTeX for cross-references and for the table of contents.
\end{filelist}

\begin{filelist}
\fn{testart.tex}\newline \fn{test.bib}\newline \fn{testart.bbl}&

A sample file illustrating the use of commands from the \opt{amstex}
option, as well as the \opt{amsart} documentstyle.
\end{filelist}

\begin{filelist}
\fn{testbook.tex}\newline \fn{pref.tex}\newline \fn{chap1.tex}\newline
\fn{chap2.tex}\newline \fn{app.tex}\newline \fn{testbook.bbl}&

These files are sample files illustrating the use of the \opt{amsbook}
documentstyle.\sloppy
\end{filelist}

\begin{filelist}
\fn{amstex.sty}&

The \opt{amstex} documentstyle option. Defines
special \AmS-\TeX{} structures (like multiline display alignments) with
\LaTeX{} syntax.  It is a copy of \fn{amstex.tex}, version 2.1,
modified as necessary to make it usable from within \LaTeX{}.
\end{filelist}

\begin{filelist}
\fn{amstext.sty}\newline \fn{amsbsy.sty}\newline
\fn{amsfonts.sty}\newline \fn{amssymb.sty}&

These are extra option files that can be used apart from the
\opt{amstex} option.  All except \fn{amssymb.sty} are
input by \fn{amstex.sty}.
The file \fn{amsbsy.sty} defines the \cs{boldsymbol} command
and the \cs{pmb} command (``poor man's
bold'').  The file
\fn{amstext.sty} defines the \AmS-\TeX{} \cs{text} command.
The files \fn{amsfonts.sty} and \fn{amssymb.sty} are for
use with the AMSFonts package (version 2.0+).   \fn{amsfonts.sty}
defines commands, including \cs{newsymbol},
for using fonts in the AMSFonts collection,
and \fn{amssymb.sty} defines the names of
all the math symbols available in the AMSFonts
collection.\sloppy
\end{filelist}

\begin{filelist}
\fn{amscd.sty}& Commutative diagrams\\[3pt]&

The \opt{amscd}  option defines some commands for
convenient typesetting of commutative diagrams.  It can be
used as an add-on with the \opt{amstex} option, or
independently.
\end{filelist}

\begin{filelist}
\fn{intlim.sty}\newline \fn{nonamelm.sty}\newline
\fn{nosumlim.sty}\newline \fn{righttag.sty}\newline
\fn{ctagsplt.sty}&

Extra math style options that affect, for example, left or right
placement of equation numbers.  They are for use only with
the \opt{amstex} option. The \opt{intlim} option provides for
integral subscripts to be placed above and below rather than on the
side. The \opt{nosumlim} option provides for sum subscripts to be
placed on the side rather than above and below. The \opt{nonamelm}
option provides for ``operator name'' subscripts to be placed on the
side rather than above and below. The \opt{righttag} option puts
equation numbers on the right instead of on the left. The
\opt{ctagsplt} option gives equation numbers  vertically centered on
the height of a displayed equation that uses the \env{split}
environment.
\end{filelist}

\begin{filelist}
\fn{amsart.sty}\newline \fn{amsbook.sty}\newline
\fn{amsart.doc}\newline \fn{amsbook.doc}&

Primary documentstyles for submissions to the AMS, for articles and
books respectively, and technical documentation files.
Auxiliary files for 10-point, 11-point and
12-point options are also distributed: \fn{amsart10.sty},
\fn{amsart11.sty}, \fn{amsart12.sty},\newline \fn{amsbk10.sty},
\fn{amsbk10.doc},
\fn{amsart10.doc}.\sloppy\hbadness9999
\end{filelist}

\begin{filelist}
\fn{amsplain.bst}\newline \fn{amsalpha.bst}\newline \fn{mrabbrev.bib}&

Bibliography style files for use with \BibTeX{}, and a file
containing the {\it Mathematical Reviews\/} standard abbreviations
for the names of mathematical journals.
\end{filelist}

\begin{filelist}
\fn{*.tfm}&

Font metrics files for a subset of AMSFonts 2.1, included in the \AmS-\LaTeX{}
distribution as described in section A.3. 
\end{filelist}

\subsection{Files maintained by Mittelbach and Sch\"opf}

The official copies of the remaining files in this distribution are maintained
by Frank Mittelbach and Rainer Sch\"opf, who have given permission for the
American Mathematical Society to distribute them.

\begin{filelist}
\fn{theorem.sty}\newline
\fn{theorem.doc}\newline and related files&

Option for special treatment of theorems and similar structures,
written by Frank Mittelbach, and auxiliary files; used by \fn{amsart.sty}
and \fn{amsbook.sty}.
\end{filelist}

\begin{filelist}
\fn{verbatim.sty}\newline \fn{verbatim.doc}&
Option file implementing an improved \env{verbatim}
environment and a \env{comment} environment, written by
Rainer Sch\"opf.
\end{filelist}

\begin{filelist}
\fn{lfonts.new}\newline
\fn{preload.min}\newline
\fn{fontdef.max}\newline
\fn{newlfont.sty}\newline
and related files&

The files that implement the Mittelbach--Sch\"opf
font selection scheme.
\end{filelist}

\message{Appendix C}
\section{Differences between \protect\AmS-\protect\TeX{} (version 2.1)
and the {\tt amstex} option}
\label{a:diff}

This section describes the parts of \AmS-\TeX{} that were removed
during the creation of the \opt{amstex} option.  It will probably
be of interest primarily to users with \AmS-\TeX{} experience.

In general, \AmS-\TeX{} commands that were redundant with \LaTeX{}
commands were simply dropped.  Other commands were reimplemented
as documentstyle options or otherwise changed in form.

\subsection{Document structure commands}
These commands have all been
superseded by their \LaTeX{} equivalents (some of which
have the same name but function slightly differently):

\begin{center}
\begin{tabular}{|l|l|}
\AmS-\TeX{}&\LaTeX{}\\[3pt]
\hline
\rule{0pt}{10pt}%    To get more space below the \hline
\cs{document}&\verb'\begin{document}'\\
\cs{midspace}&\verb'\beginfigure[htp]...\endfigure'\\
\cs{footnote}&\cs{footnote}\\
\cs{cite}&\cs{cite}\\
\cs{pagewidth}&\cs{textwidth}\\
\cs{pagebreak}&\cs{pagebreak}\\
\cs{pageheight}&\cs{textheight}\\
\end{tabular}
\end{center}

For more information on document structure commands
refer to Part~\ref{artbook}, which
describes the \sty{amsart} and \sty{amsbook} documentstyles.

\subsection{Math font commands}

The names for \AmS-\TeX{} math font commands couldn't simply be carried over
to \LaTeX{} because there is a conflict with \cs{roman}, which is
preempted by \LaTeX{} for another use.  Therefore, in the \opt{amstex}
option, \AmS-\TeX{}'s \cs{roman} has been renamed \cs{mathrm}.  In
addition, the \cs{italic} and \cs{slanted} math font commands have been
dropped in \opt{amstex}, since their usefulness is debatable and memory
space for control sequence names is in short supply.  It appears that
\verb'\text'\5\verb'{\it...}' will serve everywhere that \cs{italic}
might be used, and the same goes for \cs{sl} and \cs{slanted}.

In \AmS-\TeX{} the text font commands \cs{bf}, \cs{rm}, \cs{sl}, etc.,
cause an error message if used in math mode, but in the \opt{amstex}
option this has been disabled.  This is intended to make it easier for
users who might want to add the \opt{amstex} option to a \LaTeX{} document
that has already been written or partially written.  However, using
these commands in math mode will have no effect on font changes.

In \AmS-\TeX{} 2.0+ there is a command \cs{boldkey} used to obtain bold
versions of math symbols such as \verb'=' and \verb'+' that are
present on keyboard keys.  In the \opt{amstex} option the use of the
new font selection scheme made it possible to generalize
\cs{boldsymbol} so that \cs{boldkey} is not needed.

\subsection{Matrices}
The \cs{format} option of \AmS-\TeX{}'s \cs{matrix} is not available for
\env{matrix}, \env{pmatrix} and related
environments; just use the \env{array} environment
instead if you need an unusual format for the columns.

\AmS-\TeX{}'s \cs{smallmatrix} command has also been reimplemented
as an environment:
\begin{verbatim}
\begin{math}
  \bigl( \begin{smallmatrix}
      a&b\\ c&d
    \end{smallmatrix} \bigr)
\end{math}
\end{verbatim}

\subsection{Displayed equation structures}

 In the \LaTeX{} \opt{amstex} option, commands for creating
multiple-line displays
have been converted to environments similar to \LaTeX{}'s
\env{eqnarray} and \env{equation}---they use
\cs{begin} and \cs{end}, and the \verb'$$' that would have been used
in \AmS-\TeX{} should not be used.  See Section~\ref{s:displays}
for more details.

\subsection{Math style commands}
As a matter of convenience, \AmS-\TeX{} provided
the abbreviations \cs{dsize}, \cs{tsize},
\cs{ssize}, and \cs{sssize} for
\cs{display\-style},
\cs{text\-style}, \cs{script\-style}, and \cs{script\-script\-style}.
In order to conserve control sequence names, these have
been dropped in \opt{amstex}, since they are merely synonyms.
If you need to use a math style command frequently because of the nature
of your material, you can add an abbreviation using
\cs{newcommand} in the preamble of your document,
and call it whatever you choose:
\begin{verbatim}
\newcommand{\sst}{\scriptstyle}
\end{verbatim}

\subsection{\cs{thickfrac}}

The \cs{thickfrac} and \cs{thickfracwithdelims} commands of
\AmS-\TeX{} have been replaced by square-bracket options on
the \cs{frac} and \cs{fracwithdelims} commands.  See
Section~\ref{fracs}.

\subsection{Commenting out a  large section of text}

The \cs{comment} command of \AmS-\TeX{} is replaced by the
\env{comment} environment of the \opt{verbatim} documentstyle
option.  See the description in Section~\ref{s:comment}.

\subsection{Page breaks inside a display}

In the \opt{amstex} option, \cs{displaybreak} should
{\em precede\/} the \cs\bslash\ where it is supposed to take effect.
In the original \AmS-\TeX{} it follows
immediately after the \cs\bslash.

\subsection{Special colons in math}
\LaTeX{} and \AmS-\TeX{} have different definitions for the \cs{:} command.
In \LaTeX{} it is a medium math space, whereas in \AmS-\TeX{} it is a colon
with spacing appropriate in certain notation for mappings:
$S\,{:}\;s\to s^t$ (\verb'$S\:s\to' \5\verb's^t$'). The \LaTeX{} version
is the one that has been retained, in order to avoid compatibility
problems.  \verb'\colon' is available as a substitute for the \AmS-\TeX{}
\verb'\:'.

\subsection{Paragraphed text within a displayed equation}

\AmS-\TeX{} has a \cs{foldedtext} command  for handling
a piece of text within a display that needs to be typeset
as a paragraph (perhaps to keep it from running over the right
margin).  In the \opt{amstex} option this was dropped because
it's redundant; \LaTeX{}'s \cs{parbox} command
can be used instead.

\subsection{Commutative diagrams}

In order to conserve memory, commutative diagram commands are a separate
option, \opt{amscd}, that must be loaded in the documentstyle options list
if it is desired.

The \cs{pretend}\dots\cs{haswidth} command is not available in the
\opt{amscd} option. Approximately the same results can be gotten by
inserting blank space using \cs{hspace} in the subscript or
superscript fields of the extensible arrow commands (\verb'@>>>' and
\verb'@<<<').\index{"@<<<@{\tt{}\atsign<<<}}\index{"@>>>@{\tt{}\atsign>>>}}

\subsection{Footnotes}

The \cs{footnote} command of the \sty{amsppt} documentstyle
is superseded by \LaTeX{}'s command of
the same name.  Instead of the \AmS-\TeX{} \cs{adjustfootnotemark}
command use \cs{addtocounter}\verb'{footnote}{'\<number>\verb'}'.
The literal footnote mark feature of \fn{amsppt}, where
double quotes can be used to specify a different kind of footnote mark,
is not available.

\subsection{Vertical spacing}

The \AmS-\TeX{} use of \cs{vspace} in alignment structures
\cs{align}, \cs{split}, etc. is superfluous in \LaTeX{}
because the same function is available
through the optional
argument of the \verb'\\' or \verb'\\*' commands.  Therefore
\AmS-\TeX{}'s version of \cs{vspace} has been dropped.

\subsection{Blank space for figures}

\AmS-\TeX{}'s \cs{midspace}, \cs{topspace}, \cs{caption}
and\index{figures}\index{floats}\index{floating environments}
\cs{captionwidth} are superfluous in the \opt{amstex} option and have
been dropped; use \LaTeX{}'s \env{figure} environment and \cs{caption}
command.

\subsection{\cs{hdotsfor}}

The original \AmS-\TeX{} syntax of \cs{hdotsfor} has been simplified
somewhat in the \opt{amstex} option; \AmS-\TeX{}'s \cs{innerhdotsfor} is
not needed.  The spacing between
dots is adjusted via a square-bracket
option rather than through a separate command \cs{spacehdotsfor} or
\cs{spaceinnerhdotsfor}.  See Subsection~\ref{ss:matrix}.

\subsection{\cs{topsmash} and \cs{botsmash}}
These have been changed in the \opt{amstex} option to square-bracket
options of the \cs{smash} command.  See
Subsection~\ref{ss:smash}.

\subsection{\cs{spreadlines} and other display options}

Some of the \AmS-\TeX{} options used inside displays,
such as \cs{spreadlines} and \cs{nopagebreak}, have been dropped.  For the
most part their effect can be obtained by other means available in
standard \LaTeX{}.

\subsection{The \cs{and} command}
There is a name conflict between \AmS-\TeX{}'s \cs{and} and
\LaTeX{}'s \cs{and}.  The function of \AmS-\TeX{}'s \cs{and}
can be obtained in the \opt{amstex} option by using
\cs{And}.

\subsection{Global options}
There are several \AmS-\TeX{} commands that change the global setting of
certain aspects of the document style.  For use with
\LaTeX{}, we've
done the natural thing, which is to make them into \LaTeX{}
documentstyle options (see Section~\ref{s:options}).  These commands are
\cs{Tags\-On\-Right}, \cs{Centered\-Tags\-On\-Splits}, \cs{Limits\-On\-Ints},
\cs{No\-Limits\-On\-Names}, and \cs{No\-Limits\-On\-Sums}.  The corresponding
opposites of these commands have been dropped because they
describe the default conditions in the \opt{amstex} option.
Because they seem to be of only marginal usefulness,
\cs{TagsAsMath} and \cs{TagsAsText} have been dropped completely.

\message{Appendix D}
\section{Memory statistics}
\label{a:memstats}

Combining all of \AmS-\TeX{} with all of \LaTeX{}, even after eliminating
redundancies, produces a large macro package that strains the current
limits of personal computer versions of \TeX{}\@.  After the
zealous application of efficiency measures, the current version of
\AmS-\LaTeX{} is probably more compact than anyone anticipated;
nevertheless, for some documents and some implementations of \TeX{}
it will still be too big to run.  Among other things, a
large number of bibliography items, cross-references, or personal
definitions will tend to cause an overrun in a particular area of
\TeX{}'s memory: the maximum limit on the number of control sequence
names.   Also, you are more likely to run out of main memory if your
document includes a large table or Pic\TeX{} diagram.
\begin{table}[htp]
\caption[]{Memory statistics}
\label{memtable}
\medskip
\footnotesize
\setlength{\tabcolsep}{.2\tabcolsep}
\noindent
\begin{tabular*}{\columnwidth}{|r|c*{6}{@{\extracolsep{\fill}}c}|}
\hline
&{\tt article}$^*$&
{\tt\begin{tabular}[b]{c}[M--S\\fonts]\\article\end{tabular}}&
{\tt\begin{tabular}[b]{c}[amstex]$^{\dag}$\\article\end{tabular}}&
{\tt\begin{tabular}[b]{c}[amstex,\\amscd,\\amssymb]\\article\end{tabular}}&
{\tt\begin{tabular}[b]{c}[amsfonts,\\amsbsy]\\article\end{tabular}}&
{\tt\begin{tabular}[b]{c}[amstex]$^{\dag}$\\amsart\end{tabular}}&
\begin{tabular}[b]{c}represen-\\tative\\maxima\end{tabular}\\
\hline
strings& 286& 386& 942& 1170& 500& 933& 7032\\ \hline
\begin{tabular}{r}string\\charac-\\ters$^{\ddag}$\end{tabular}&
        2421& 3183& 8444& 10505& 4218& 8419& 20798$^{\ddag}$\\ \hline
\begin{tabular}{r}main\\memory\end{tabular}&
        51376& 50126& 63506& 63880& 51059& 65445& 65501\\ \hline
\begin{tabular}{r}control\\sequences\end{tabular}&
        2234& 2410& 2938& 3160& 2498& 2946& 5000\\ \hline
\begin{tabular}{r}font\\infor-\\mation\end{tabular}&
        18941& 12721& 16842& 16842& 16842& 11462& 65504\\ \hline
\begin{tabular}{r}number\\of fonts\end{tabular}&
        72& 50& 70& 70& 70& 47& 220\\ \hline
\begin{tabular}{r}hyphen-\\ation\\excep-\\tions\end{tabular}&
        14& 14& 14& 14& 14& 14& 307\\ \hline
\begin{tabular}{r}input\\stack\end{tabular}&
        12& 16& 16& 16& 16& 19& 200\\ \hline
\begin{tabular}{r}nesting\\levels\end{tabular}&
        9& 9& 14& 14& 9& 14& 40\\ \hline
\begin{tabular}{r}param-\\eter\\stack\end{tabular}&
        25& 25& 25& 25& 25& 25& 60\\ \hline
\begin{tabular}{r}input\\buffer\end{tabular}&
        361& 361& 436& 436& 400& 436& 1500\\ \hline
\begin{tabular}{r}save\\stack\end{tabular}&
        233& 173& 192& 192& 187& 264& 2000\\
\hline
\end{tabular*}%
\par
\smallskip

\parbox{\columnwidth}{$^*$This column is the control,
using standard \LaTeX{} with the standard \sty{article} documentstyle,
without the Mittelbach--Sch\"opf font selection scheme.  The
adjacent column is the same, except for the font selection
scheme.

$^{\dag}$Recall that
the \opt{amstex} option includes the
\opt{amsfonts}, \opt{amsbsy}, and \opt{amstext} options, and
that the \sty{amsart} and \sty{amsbook}
documentstyles automatically include the \opt{amstex} option.

$^{\ddag}$For the number of string characters (also referred to
as string pool size), the values given here (as reported
at the end of a \LaTeX{} log file) are potentially
misleading.  They represent the
difference between the maximum value compiled into \TeX{}
and the total number of string characters used by {\em all}
current control sequences and other strings, including
primitive control sequences and the error messages that are
built into \TeX{}.  See Subsection~\ref{as:capacity} below.

}% end of parbox
\end{table}

For those who might be interested in the details, Table~\ref{memtable}
shows memory statistics
from \LaTeX{} runs using various combinations of option files
from the \AmS-\LaTeX{} distribution.  For comparison purposes,
the statistics in the first column are from
a sample run using the current standard \LaTeX{} without
the Mittelbach--Sch\"opf font scheme, with the
\sty{article} documentstyle, and the last column, ``representative
maxima'', shows the available memory in each category in the
implementation of \TeX{} used for testing (VAX/VMS Version 2.98a.0 (AMS)).
The test document used in each case
was a medium-size article with about 20 bibliography
entries, 50 author-defined commands, and 50 cross-reference
labels.

It can be seen that in all of the tests with the \opt{amstex} option
loaded the upper limit of 65500 words of main memory is nearly
exceeded.  And use of the \opt{amssymb} option in the fourth test
would cause control sequence memory to be exceeded for implementations
of \TeX{} that have a maximum of 3000 (currently true for
many implementations on small computer systems).

The font base used in all the tests, except for the control, was
\fn{fontdef.max}, \fn{preload.min}, and \fn{newlfont.sty}.  Obviously,
preloading more fonts by using a different preload file would
tend to increase font memory usage.  The reason for the comparatively
small number of fonts used by the \sty{amsart} documentstyle is that
no fonts larger than \cs{normalsize} are used in the title and
section headings.

Memory statistics for the \sty{amsbook} documentstyle are comparable
to the statistics for \sty{amsart}.  But bear in mind that books
will usually have larger bibliographies and more cross-references,
which means greater usage of control sequence memory and string
memory (``hash size'' and ``string pool'', respectively).

\subsection{Recommended values for the various \protect\TeX{} memory categories}
\label{as:capacity}

Table~\ref{t:cap} lists the recommended capacities in various categories
for successful use of the \AmS-\LaTeX{} major documentstyles or the
\opt{amstex} option.  Not all categories are listed; the ones that
appear are the ones where problems tend to occur in current
implementations of \TeX{}.

Note in particular that the compiled-in value for
string pool needs to be much larger than the values listed in
Table~\ref{memtable}. This is because the string pool capacity reported
by \TeX{} in response to a \verb'\tracingstats' command is not the
compiled-in value, but the result of subtracting from the compiled-in
value the number of characters in \TeX{}'s built-in error
messages, the names of primitive control sequences, and the
names of all additional control sequences defined in the format file (in
our case, the whole of \LaTeX{}), not to mention font names and file
names. Thus the reported value only measures the amount of string
capacity that remains to the user after the format file is loaded.  This
reported value should be at least 10000 for ordinary use, and 20000 if
memory-expensive options such as \opt{amssymb} or PIC\TeX{} are to be
made available to the user.  We recommend that the compiled value be
at least 60000.

The reported value for number of strings is reduced in the same way,
so that the recommended minimum value for compilation is 5000.

\begin{table}[h]
\caption[]{Recommended capacities}
\label{t:cap}
\medskip
\begin{center}
\begin{tabular}{|r|l|l|l|}
\hline
&\multicolumn{2}{c|}{Capacity}&\\
Category& Adequate& Generous& WEB variable\\
\hline
strings& 5000& 15000& \verb'max_strings'\\
string characters& 60000& 120000& \verb'pool_size'\\
macro string pool$^*$& 30000& 90000& \verb'string_vacancies'\\
main memory& 65000& 130000& \verb'main_mem'\\
control sequences& 4000& 7000& \verb'hash_size'\\
font information& 60000& 120000& \verb'font_mem_size'\\
number of fonts& 128& 256& \verb'font_max'\\
input buffer& 1000& 2000& \verb'buf_size'\\
save stack& 2000& 5000& \verb'save_size'\\
\hline
\end{tabular}

\smallskip

\parbox{.8\columnwidth}{\noindent\llap{$^*$}The number of string
characters left for macro packages and user commands, after all
primitives and built-in error messages have been loaded---i.e., the
total number of string characters available for a format file and
individual documents using that format file.}

\end{center}
\end{table}

\clearpage

\message{Appendix E}
\section{Getting help}\label{a:help}

Comments or questions on the \AmS-\LaTeX{} package should be sent to:
\medskip

\begin{raggedright}

\leftskip=4.25pc
American Mathematical Society\\
Technical Support\\
P. O. Box 6248\\
Providence, RI 02940\\[3pt]
Phone: 800-321-4AMS (321-4267) \quad or \quad 401-455-4080\\
Internet: {\tt tech-support@Math.AMS.org}\\
\end{raggedright}
\medskip

\noindent If you are reporting a problem you should include
the following information:
\begin{enumerate}
\item the source file---either in electronic form or printed---where
  the problem occurred, preferably with irrelevant
  material removed.
\item a \LaTeX{} transcript (log) file showing the error
 message (if applicable) and the version numbers of
 the documentstyle and option files being used.
\end{enumerate}

\subsection{Further information}\label{a:furtherinfo}
Information about obtaining AMSFonts or other \TeX{}-related
software from the AMS Internet archive e-MATH.ams.org
can be obtained by sending a request through electronic mail to:
\[\mbox{\tt e-math\atsign math.ams.org}\]

Information about obtaining \AmS-\LaTeX{} on diskette from the AMS is available
from:
\medskip

\begin{raggedright}

\leftskip=4.25pc
American Mathematical Society\\
Customer Services\\
P. O. Box 6248\\
Providence, RI 02940\\[3pt]
Phone: 800-321-4AMS (321-4267) \quad or \quad 401-455-4000\\
Internet: {\tt cust-serv@Math.AMS.org}\\
\end{raggedright}
\medskip

\goodbreak
The \TeX{} Users Group is a nonprofit organization that publishes
a journal ({\it TUGboat\/}), holds meetings, and offers other services to
members.

\medskip
\begin{raggedright}

\leftskip=4.25pc
\TeX{} Users Group\\
P. O. Box 869\\
Santa Barbara, CA 93102-0869\\[3pt]
Phone: (805) 963-1338\\
Internet: {\tt TUG@Math.AMS.org}\\
\end{raggedright}

\medskip

Membership in the \TeX{} User's Group is a good way to support
continued development of \TeX{}-related public-domain software.

\begin{thebibliography}{9}

\bibitem{amsfonts}{\it AMSFonts version 2.1---user's guide},
American Mathematical Society, Providence, R.I., 1991; distributed
with the AMSFonts package.

\bibitem{author-guidelines}{\it Guidelines for preparing
electronic manuscripts---\AmS-\LaTeX{}},
American Mathematical Society, Providence, R.I., 1990.

\bibitem{kn} Donald Knuth, {\it The \TeX book}, Addison-Wesley,
1984.

\bibitem{lm} Leslie Lamport, {\it\LaTeX{}: A document preparation
system}, Addison-Wesley, 1985.

\bibitem{msf} Frank Mittelbach and Rainer Sch\"opf,
{\it The new font family selection---user
interface to standard \LaTeX{}}, {\it TUGboat\/} {\bf11},
no.~2 (June 1990), pp.~297--305.

\bibitem{jt} Michael Spivak, {\it The joy of \TeX{}}, 2nd ed.,
American Mathematical Society, Providence, R.I., 1990.

\end{thebibliography}

%      Add the index to the table of contents.  This does not happen
%      automatically with the "article" documentstyle because there
%      \theindex uses the * form of the \section command.
\newpage
\addcontentsline{toc}{section}{Index}
\message{Index}

%%%%%%%%%%%INDEX
%% The following index is the output file from running the program MakeIndex
%% on the .idx file produced by running amslatex.tex through LaTeX. It is
%% included here for the benefit of users who do
%% not have use of the program MakeIndex.
%%%%%%%%%%%%
\begin{theindex}

  \item {\tt\bslash !}, 18
  \item {\ntt{}*.tfm}, 50
  \item {\ntt\bslash ,}, 18
  \item {\ntt{}.pk}, 46
  \item {\ntt{}.tfm}, 43, 45, 46
  \item {\ntt\bslash /}, 30
  \item {\ntt\bslash :}, 18, 53
  \item {\ntt\bslash ;}, 18
  \item {{\tt\atsign} (at sign)}, 32, 34
  \item {\tt{}\atsign!}, 18
  \item {\tt{}\atsign,}, 18
  \item {\tt{}\atsign<<<}, 26, 53
  \item {\tt{}\atsign>>>}, 26, 53
  \item {\tt{}\atsign AAA}, 26
  \item {\tt{}\atsign VVV}, 26
  \item {\ntt\bslash \bslash }, 27, 28, 53
  \item {\tt\string~}, 30

  \indexspace

  \item {\ntt{}abstract} environment, 34
  \item {\ntt\bslash abstractname}, 35
  \item {\ntt\bslash accentedsymbol}, 20
  \item {\ntt\bslash addcontentsline}, 32
  \item {\ntt\bslash address}, 34, 38
  \item {\ntt\bslash addtocontents}, 32
  \item {\ntt\bslash addtocounter}, 24, 54
  \item {\ntt\bslash addtolength}, 28
  \item {\ntt\bslash adjustfootnotemark}, 53
  \item {\ntt\bslash align}, 54
  \item {\ntt{}align} environment, 27, 28
  \item {\ntt{}alignat} environment, 27
  \item {\ntt{}aligned} environment, 28
  \item {\ntt{}alignedat} environment, 28
  \item {\ntt\bslash allowdisplaybreaks}, 28, 29
  \item {\ntt{}alpha}, 40
  \item {\tt alpha} bibliography style, 40
  \item {\ntt{}AMS-LaTeX}, 46, 48
  \item {\ntt{}amsalpha}, 40
  \item {\tt amsalpha} bibliography style, 40
  \item {\ntt{}amsalpha.bst}, 50
  \item {\ntt{}amsart} documentstyle, 3, 9, 33--36, 38, 39, 51, 55--57
  \item {\ntt{}amsart} option, 48
  \item {\ntt{}amsart.doc}, 50
  \item {\ntt{}amsart.sty}, 50
  \item {\ntt{}amsart10.doc}, 50
  \item {\ntt{}amsart10.sty}, 50
  \item {\ntt{}amsart11.sty}, 50
  \item {\ntt{}amsart12.sty}, 50
  \item {\ntt{}amsbk10.doc}, 50
  \item {\ntt{}amsbk10.sty}, 50
  \item {\ntt{}amsbk11.sty}, 43
  \item {\ntt{}amsbk12.sty}, 43
  \item {\ntt{}amsbook} documentstyle, 3, 9, 33, 35, 36, 38--40, 51, 
		56, 57
  \item {\ntt{}amsbook} option, 49
  \item {\ntt{}amsbook.doc}, 50
  \item {\ntt{}amsbook.sty}, 50
  \item {\ntt{}amsbsy} option, 31, 56
  \item {\ntt{}amsbsy.sty}, 49
  \item {\ntt{}amscd} option, 21, 25, 26, 49, 53
  \item {\ntt{}amscd.sty}, 49
  \item {\ntt{}amsfonts} option, 16, 31, 56
  \item {\ntt{}AMSFonts 2.1 Metrics}, 46
  \item {\ntt{}amsfonts.sty}, 49
  \item {\ntt{}amslatex}, 43
  \item {\ntt{}amslatex.aux}, 48
  \item {\ntt{}amslatex.tex}, 1, 48
  \item {\ntt{}amslatex.toc}, 48
  \item {\ntt{}amsplain}, 40
  \item {\tt amsplain} bibliography  style, 40
  \item {\ntt{}amsplain.bst}, 50
  \item {\ntt{}amsppt}, 54
  \item {\ntt{}amsppt} documentstyle, 3, 53
  \item {\ntt{}amssymb} option, 17, 31, 55, 57
  \item {\ntt{}amssymb.sty}, 49
  \item {\ntt{}amstex} option, 1--3, 12, 15--33, 45, 48--57
  \item {\ntt{}amstex.sty}, 49
  \item {\ntt{}amstex.tex}, 49
  \item {\ntt{}amstext} option, 31, 56
  \item {\ntt{}amstext.sty}, 49
  \item {\ntt\bslash And}, 54
  \item {\ntt\bslash and}, 54
  \item {\ntt{}app.tex}, 49
  \item {\ntt{}array} environment, 24, 32, 52
  \item {\ntt{}article} documentstyle, 3, 33, 34, 55, 56
  \item {\ntt\bslash author}, 34, 35, 38

  \indexspace

  \item {\ntt{}babel.sty}, 35
  \item {\ntt{}basefont.tex}, 48
  \item {\ntt\bslash Bbb}, 12, 13, 16, 31
  \item {\ntt\bslash begin}, 2, 52
  \item {\ntt\bslash bf}, 4, 5, 13, 32, 52
  \item {\ntt\bslash bfdefault}, 8
  \item {\ntt\bslash bibname}, 35
  \item {\ntt\bslash Big}, 30
  \item {\ntt\bslash big}, 30
  \item {\ntt\bslash Bigg}, 30
  \item {\ntt\bslash bigg}, 30
  \item {\ntt\bslash binom}, 23
  \item {\ntt\bslash bmatrix}, 24
  \item {\ntt\bslash bmod}, 22
  \item {\ntt\bslash bold}, 12, 13, 15, 32
  \item {\ntt\bslash boldkey}, 52
  \item {\ntt\bslash boldmath}, 15, 16
  \item {\ntt\bslash boldsymbol}, 13, 15, 16, 31, 49, 52
  \item {\ntt{}book} documentstyle, 3, 33
  \item {\ntt\bslash botsmash}, 54
  \item {\ntt\bslash Box}, 45
  \item {\ntt\bslash boxed}, 20

  \indexspace

  \item {\ntt\bslash cal}, 12, 13, 16
  \item {\ntt\bslash caption}, 32, 54
  \item {\ntt\bslash captionwidth}, 54
  \item {\ntt{}cases} environment, 24
  \item {\ntt{}CD} environment, 26
  \item {\ntt\bslash cdots}, 19
  \item {\ntt\bslash CenteredTagsOnSplits}, 55
  \item {\ntt\bslash cfrac}, 23
  \item {\ntt{}chap1.tex}, 49
  \item {\ntt{}chap2.tex}, 49
  \item {\ntt\bslash chapter}, 38
  \item {\ntt\bslash cite}, 51
  \item {\ntt{}cmbsy}, 33
  \item {\ntt{}cmmib}, 33
  \item {\ntt{}cmti7}, 11
  \item {\ntt\bslash comment}, 53
  \item {\ntt{}comment} environment, 31, 50, 53
  \item {\ntt{}ctagsplt} option, 31, 50
  \item {\ntt{}ctagsplt.sty}, 50
  \item {\ntt\bslash curraddr}, 34
  \item {\ntt\bslash cy}, 15

  \indexspace

  \item {\ntt\bslash date}, 34
  \item {\ntt\bslash dbinom}, 23
  \item {\ntt\bslash ddddot}, 20
  \item {\ntt\bslash dddot}, 20
  \item {\ntt\bslash ddot}, 20
  \item {\ntt\bslash dedicatory}, 34
  \item {\ntt\bslash dfrac}, 22
  \item {\ntt\bslash Diamond}, 45
  \item {\it dimension}, 1, 23
  \item {\ntt\bslash displaybreak}, 28, 53
  \item {\ntt\bslash displaystyle}, 52
  \item {\ntt{}doc}, 43
  \item {\ntt\bslash document}, 51
  \item {\ntt\bslash documentstyle}, 3, 31
  \item {\ntt\bslash dot}, 20
  \item {\ntt\bslash dots}, 18, 19
  \item {\ntt\bslash dotsb}, 19
  \item {\ntt\bslash dotsc}, 19
  \item {\ntt\bslash dotsi}, 19
  \item {\ntt\bslash dotsm}, 19
  \item {\ntt\bslash dsize}, 52
  \item {\ntt\bslash dump}, 45, 48

  \indexspace

  \item {\ntt{}e-MATH.AMS.org}, 41
  \item {\ntt\bslash em}, 13
  \item {\ntt\bslash email}, 34
  \item {\ntt\bslash End}, 26
  \item {\ntt\bslash end}, 2, 19, 52
  \item {\ntt\bslash endmatrix}, 2
  \item {\ntt\bslash endsomething}, 2
  \item {\ntt{}eqnarray} environment, 26, 28, 32, 52
  \item {\ntt\bslash eqref}, 30
  \item {\ntt{}equation} environment, 26--28, 52

  \indexspace

  \item {\ntt\bslash family}, 4, 7, 8, 13
  \item {\ntt\bslash fbox}, 20
  \item {\ntt{}figure} environment, 54
  \item figures, 54
  \item {\ntt{}fleqn} option, 33
  \item floating environments, 54
  \item floats, 54
  \item {\ntt\bslash foldedtext}, 53
  \item {\ntt{}fontdef.ams}, 9, 48
  \item {\ntt{}fontdef.max}, 8--10, 12, 48, 51, 55
  \item {\ntt{}fontdef.ori}, 9, 12, 48
  \item {\ntt{}fontsel}, 43
  \item {\ntt{}fontsel.tex}, 4
  \item {\ntt\bslash footnote}, 9, 53
  \item {\ntt\bslash format}, 52
  \item {\ntt\bslash frac}, 22, 23, 53
  \item {\ntt\bslash fracwithdelims}, 23, 53
  \item {\ntt\bslash frak}, 12, 13, 16, 31

  \indexspace

  \item {\ntt{}gather} environment, 27, 28
  \item {\ntt{}gathered} environment, 28

  \indexspace

  \item {\ntt\bslash haswidth}, 53
  \item {\ntt\bslash Hat}, 19
  \item {\ntt\bslash hat}, 19
  \item {\ntt\bslash hdotsfor}, 25, 54
  \item {\ntt\bslash hspace}, 53
  \item {\ntt\bslash Huge}, 6, 33
  \item {\ntt\bslash huge}, 33
  \item {\ntt{}hyphen.tex}, 47

  \indexspace

  \item {\ntt\bslash idotsint}, 18
  \item {\ntt\bslash iiiint}, 18
  \item {\ntt\bslash iiint}, 18
  \item {\ntt\bslash iint}, 18
  \item {\ntt\bslash innerhdotsfor}, 54
  \item {\ntt{}inputs}, 43
  \item {\ntt\bslash intertext}, 29
  \item {\ntt{}intlim} option, 31, 50
  \item {\ntt{}intlim.sty}, 50
  \item {\ntt\bslash it}, 4, 5, 13
  \item {\ntt\bslash italic}, 51, 52
  \item {\ntt\bslash itdefault}, 8

  \indexspace

  \item {\ntt\bslash Join}, 45

  \indexspace

  \item {\ntt\bslash keywords}, 34

  \indexspace

  \item {\ntt{}labrea.stanford.edu}, 41
  \item {\ntt\bslash LARGE}, 33
  \item {\ntt\bslash Large}, 33
  \item {\ntt\bslash large}, 33
  \item {\ntt{}latex.tex}, 41
  \item {\ntt\bslash lcfrac}, 23
  \item {\ntt\bslash ldots}, 19
  \item {\ntt\bslash leadsto}, 45
  \item {\ntt\bslash leftroot}, 20
  \item {\ntt{}lfonts.new}, 44, 47, 48, 51
  \item {\ntt{}lfonts.tex}, 47
  \item {\ntt\bslash lhd}, 45
  \item {\ntt\bslash lim}, 22
  \item {\ntt\bslash LimitsOnInts}, 55
  \item {\ntt\bslash loadmsam}, 16
  \item {\ntt\bslash loadmsbm}, 16
  \item {\ntt\bslash log}, 22
  \item {\ntt{}lplain.tex}, 47

  \indexspace

  \item {\ntt\bslash maketitle}, 34, 39
  \item {\ntt\bslash markboth}, 32
  \item {\ntt\bslash markright}, 32
  \item {\ntt\bslash mathrm}, 12, 13, 32, 51
  \item {\ntt\bslash matrix}, 2, 24, 52
  \item {\ntt{}matrix} environment, 52
  \item {\ntt\bslash mbox}, 21
  \item {\ntt\bslash mediumseries}, 5, 13
  \item {\ntt\bslash medspace}, 18
  \item {\ntt\bslash mho}, 45
  \item {\ntt\bslash midspace}, 51, 54
  \item {\ntt\bslash mit}, 12, 13
  \item {\ntt\bslash mod}, 22
  \item {\ntt{}mrabbrev.bib}, 40, 50
  \item {\ntt{}msam}, 33
  \item {\ntt{}msbm}, 33
  \item {\ntt{}multline} environment, 27, 28
  \item {\ntt\bslash multlinegap}, 28

  \indexspace

  \item {\ntt\bslash negmedspace}, 18
  \item {\ntt\bslash negthickspace}, 18
  \item {\ntt\bslash negthinspace}, 18
  \item {\ntt\bslash new\char 64\relax fontshape}, 11
  \item {\ntt\bslash newcommand}, 16, 20, 52
  \item {\ntt{}newlfont.sty}, 48, 51, 55
  \item {\ntt\bslash newmathalphabet}, 12
  \item {\ntt\bslash newsymbol}, 16, 17, 45, 49
  \item {\ntt\bslash newtheorem}, 29, 36, 37
  \item {\ntt\bslash nolimits}, 21
  \item {\ntt\bslash NoLimitsOnNames}, 55
  \item {\ntt\bslash NoLimitsOnSums}, 55
  \item {\ntt{}nonamelm} option, 31, 50
  \item {\ntt{}nonamelm.sty}, 50
  \item {\ntt\bslash nonumber}, 32
  \item {\ntt\bslash nopagebreak}, 54
  \item {\ntt\bslash normalshape}, 5, 9, 13
  \item {\ntt\bslash normalsize}, 33, 55
  \item {\ntt{}nosumlim} option, 31, 50
  \item {\ntt{}nosumlim.sty}, 50
  \item {\ntt\bslash notag}, 27, 32
  \item {\ntt\bslash numberwithin}, 29

  \indexspace

  \item {\ntt{}Old lfonts.tex}, 47
  \item {\ntt{}oldlfont} option, 8
  \item {\ntt{}oldlfont.sty}, 48
  \item {\ntt{}openbib} option, 33
  \item {\ntt\bslash operatorname}, 13, 22, 26
  \item {\ntt\bslash operatornamewithlimits}, 22
  \item {\ntt\bslash overleftarrow}, 18
  \item {\ntt\bslash overleftrightarrow}, 18
  \item {\ntt\bslash overrightarrow}, 18
  \item {\ntt\bslash overset}, 21

  \indexspace

  \item {\ntt\bslash pagebreak}, 28, 51
  \item {\ntt\bslash pageheight}, 51
  \item {\ntt\bslash pagewidth}, 51
  \item {\ntt\bslash parbox}, 53
  \item {\ntt{}pf} environment, 37
  \item {\ntt{}pf*} environment, 37
  \item {\ntt{}picture} environment, 26
  \item {\ntt{}Plain}, 46, 47
  \item {\ntt{}plain}, 36, 40
  \item {\tt plain} bibliography style, 40
  \item {\ntt\bslash pmatrix}, 24
  \item {\ntt{}pmatrix} environment, 52
  \item {\ntt\bslash pmb}, 13, 16, 31, 49
  \item {\ntt\bslash pmod}, 22
  \item {\ntt\bslash pod}, 22
  \item {\ntt{}pref.tex}, 49
  \item {\ntt{}preload.med}, 48
  \item {\ntt{}preload.min}, 48, 51, 55
  \item {\ntt{}preload.ori}, 48
  \item {\ntt\bslash pretend}, 53
  \item {\ntt\bslash protect}, 32

  \indexspace

  \item {\ntt\bslash qed}, 37
  \item {\ntt\bslash qedsymbol}, 37

  \indexspace

  \item {\ntt\bslash rcfrac}, 23
  \item {\ntt\bslash ref}, 39
  \item {\ntt\bslash refname}, 35
  \item {\ntt\bslash renewcommand}, 7, 35, 37
  \item {\ntt\bslash rhd}, 45
  \item {\ntt{}righttag} option, 31, 50
  \item {\ntt{}righttag.sty}, 50
  \item {\ntt\bslash rm}, 4, 5, 7, 8, 13, 16, 32, 52
  \item {\ntt\bslash rom}, 39
  \item {\ntt\bslash roman}, 51

  \indexspace

  \item {\ntt{}Sb} environment, 25
  \item {\ntt\bslash sc}, 5, 13
  \item {\ntt\bslash scdefault}, 8
  \item {\ntt\bslash scriptscriptstyle}, 52
  \item {\ntt\bslash scriptstyle}, 52
  \item {\ntt\bslash section}, 32, 38
  \item {\ntt\bslash selectfont}, 4, 5
  \item {\ntt\bslash series}, 4, 5, 13
  \item {\ntt\bslash setcounter}, 24
  \item {\ntt\bslash setlength}, 28
  \item {\ntt\bslash sf}, 7, 8, 13
  \item {\ntt\bslash shape}, 4, 13
  \item {\ntt\bslash sideset}, 21
  \item {\ntt\bslash sin}, 22
  \item {\ntt\bslash size}, 4, 6
  \item {\ntt\bslash sl}, 5, 13, 52
  \item {\ntt\bslash slanted}, 51, 52
  \item {\ntt\bslash slash}, 30
  \item {\ntt\bslash sldefault}, 8
  \item {\ntt\bslash smallmatrix}, 52
  \item {\ntt{}smallmatrix} environment, 25
  \item {\ntt\bslash smash}, 23, 24, 54
  \item {\ntt\bslash something}, 2
  \item {\ntt{}Sp} environment, 25
  \item {\ntt\bslash spacehdotsfor}, 54
  \item {\ntt\bslash spaceinnerhdotsfor}, 54
  \item {\ntt\bslash split}, 54
  \item {\ntt{}split} environment, 27, 28, 31, 50
  \item {\ntt\bslash spreadlines}, 54
  \item {\ntt\bslash ssize}, 52
  \item {\ntt\bslash sssize}, 52
  \item {\ntt\bslash stackrel}, 21
  \item {\ntt\bslash subjclass}, 34
  \item {\ntt\bslash subst\char 64\relax fontshape}, 12
  \item {\ntt\bslash syntaxonly}, 32
  \item {\ntt{}syntonly} option, 32

  \indexspace

  \item {\ntt{}tabular} environment, 32
  \item {\ntt\bslash tag}, 27, 31, 32
  \item {\ntt\bslash tag*}, 27
  \item {\ntt\bslash TagsAsMath}, 55
  \item {\ntt\bslash TagsAsText}, 55
  \item {\ntt\bslash TagsOnRight}, 55
  \item {\ntt\bslash tbinom}, 23
  \item {\ntt\bslash tenbf}, 9
  \item {\ntt{}test.bib}, 48
  \item {\ntt{}testart.bbl}, 48
  \item {\ntt{}testart.tex}, 1, 18--23, 26, 27, 48
  \item {\ntt{}testbook.bbl}, 49
  \item {\ntt{}testbook.tex}, 1, 49
  \item {\ntt{}TeX Fonts}, 46
  \item {\ntt{}TeX formats}, 48
  \item {\ntt{}TeX Inputs}, 46
  \item {\ntt{}TeX inputs}, 47
  \item {\ntt{}TeX log}, 47
  \item {\ntt\bslash text}, 2, 13, 21, 24, 31, 49
  \item {\ntt\bslash textheight}, 51
  \item {\ntt\bslash textstyle}, 52
  \item {\ntt{}Textures}, 46, 47
  \item {\ntt\bslash textwidth}, 51
  \item {\ntt{}tfm}, 43
  \item {\ntt\bslash tfrac}, 22
  \item {\ntt\bslash thanks}, 34, 39
  \item {\ntt{}thebibliography} environment, 35
  \item {\ntt{}theorem} option, 36
  \item {\ntt{}theorem.doc}, 50
  \item {\ntt{}theorem.sty}, 50
  \item {\ntt\bslash theorembodyfont}, 36
  \item {\ntt\bslash theoremheaderfont}, 36
  \item {\ntt\bslash theoremstyle}, 36
  \item {\ntt\bslash thickfrac}, 53
  \item {\ntt\bslash thickfracwithdelims}, 53
  \item {\ntt\bslash thickspace}, 18
  \item {\ntt\bslash thinspace}, 18
  \item {\ntt\bslash tiny}, 6
  \item {\ntt\bslash title}, 34, 35
  \item {\ntt\bslash topsmash}, 54
  \item {\ntt\bslash topspace}, 54
  \item {\ntt\bslash translator}, 34
  \item {\ntt\bslash tsize}, 52
  \item {\ntt\bslash tt}, 4, 7, 8, 13

  \indexspace

  \item {\ntt\bslash underleftarrow}, 18
  \item {\ntt\bslash underleftrightarrow}, 18
  \item {\ntt\bslash underrightarrow}, 18
  \item {\ntt\bslash underset}, 21
  \item {\ntt\bslash unlhd}, 45
  \item {\ntt\bslash unrhd}, 45
  \item {\ntt\bslash uproot}, 20

  \indexspace

  \item {\ntt\bslash varinjlim}, 22
  \item {\ntt\bslash varliminf}, 22
  \item {\ntt\bslash varlimsup}, 22
  \item {\ntt\bslash varprojlim}, 22
  \item {\ntt{}verbatim} environment, 50
  \item {\ntt{}verbatim} option, 31, 53
  \item {\ntt{}verbatim.doc}, 31, 50
  \item {\ntt{}verbatim.sty}, 50
  \item {\ntt\bslash Vmatrix}, 24
  \item {\ntt\bslash vmatrix}, 24
  \item {\ntt\bslash vspace}, 54

  \indexspace

  \item {\ntt\bslash widehat}, 20
  \item {\ntt\bslash widetilde}, 20

  \indexspace

  \item {\ntt{}xalignat} environment, 27
  \item {\ntt{}xxalignat} environment, 27

\end{theindex}
%%%%%%%%%%%INDEX ENDS HERE

\end{document}

%% \CharacterTable
%%  {Upper-case    \A\B\C\D\E\F\G\H\I\J\K\L\M\N\O\P\Q\R\S\T\U\V\W\X\Y\Z
%%   Lower-case    \a\b\c\d\e\f\g\h\i\j\k\l\m\n\o\p\q\r\s\t\u\v\w\x\y\z
%%   Digits        \0\1\2\3\4\5\6\7\8\9
%%   Exclamation   \!     Double quote  \"     Hash (number) \#
%%   Dollar        \$     Percent       \%     Ampersand     \&
%%   Acute accent  \'     Left paren    \(     Right paren   \)
%%   Asterisk      \*     Plus          \+     Comma         \,
%%   Minus         \-     Point         \.     Solidus       \/
%%   Colon         \:     Semicolon     \;     Less than     \<
%%   Equals        \=     Greater than  \>     Question mark \?
%%   Commercial at \@     Left bracket  \[     Backslash     \\
%%   Right bracket \]     Circumflex    \^     Underscore    \_
%%   Grave accent  \`     Left brace    \{     Vertical bar  \|
%%   Right brace   \}     Tilde         \~}
\endinput

