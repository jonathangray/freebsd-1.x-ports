%%@texfile{%
%% filename="app.tex",
%% version="1.1",
%% date="21-JUN-1991",
%% filetype="AMS-LaTeX: documentation",
%% copyright="Copyright (C) American Mathematical Society, all rights
%%   reserved.  Copying of this file is authorized only if either:
%%   (1) you make absolutely no changes to your copy, including name;
%%   OR (2) if you do make changes, you first rename it to some other
%%   name.",
%% author="American Mathematical Society",
%% address="American Mathematical Society,
%%   Technical Support Department,
%%   P. O. Box 6248,
%%   Providence, RI 02940,
%%   USA",
%% telephone="401-455-4080 or (in the USA) 800-321-4AMS",
%% email="Internet: Tech-Support@Math.AMS.org",
%% checksumtype="line count",
%% checksum="76",
%% codetable="ISO/ASCII",
%% keywords="latex, amslatex, ams-latex",
%% abstract="This file is part of the AMS-\LaTeX{} package, version 1.1.
%%   It is part of the monograph sample, testbook.tex (q.v.)."
%%}
%%% end of file header
%
\appendix
\chapter[Nonselfadjoint Equations]%
{On the Eigenvalues and Eigenfunctions\\
of Certain Classes of Nonselfadjoint Equations}

\section{Compact operators} In an appropriate Hilbert space, all 
the equations considered below can be reduced to the 
form
\begin{equation}
y=L(\lambda)y+f,\qquad L(\lambda)=K_0+\lambda K_1+\dots+\lambda^n
K_n,
\end{equation}
where $y$ and $f$ are elements of the Hilbert space, 
$\lambda$ is a complex parameter, and the $K_i$ are
compact operators.

A compact operator $R(\lambda)$ is the resolvent of 
$L(\lambda)$ if $(E+R)(E-L)=E$.  If the resolvent exists
for some $\lambda=\lambda_0$, it is a meromorphic function
of $\lambda$ on the whole plane.  We say that $y$ is
an eigenelement for the eigenvalue $\lambda=c$, and that
$y_1,\dots,y_k$ are elements associated with it (or
associated elements) if 
\begin{equation}
y=L(c)y,\quad y_k=L(c)y_k+\frac{1}{1!}\,\frac{\partial L(c)}{\partial c}
y_{k-1}+\dots+\frac{1}{k!}\,\frac{\partial^kL(c)}{\partial c^k}y.
\end{equation}
Note that if $y$ is an eigenelement and $y_1,\dots,y_k$
are elements associated with it, then $y(t)=e^{ct}(y_k
+y_{k-1}t/1!+\dots+yt^k/k!)$ is a solution of the equation
$y=K_0y+K_1\partial y/\partial t+\dots+K_n\partial^ny/
\partial t^n$.

%% \CharacterTable
%%  {Upper-case    \A\B\C\D\E\F\G\H\I\J\K\L\M\N\O\P\Q\R\S\T\U\V\W\X\Y\Z
%%   Lower-case    \a\b\c\d\e\f\g\h\i\j\k\l\m\n\o\p\q\r\s\t\u\v\w\x\y\z
%%   Digits        \0\1\2\3\4\5\6\7\8\9
%%   Exclamation   \!     Double quote  \"     Hash (number) \#
%%   Dollar        \$     Percent       \%     Ampersand     \&
%%   Acute accent  \'     Left paren    \(     Right paren   \)
%%   Asterisk      \*     Plus          \+     Comma         \,
%%   Minus         \-     Point         \.     Solidus       \/
%%   Colon         \:     Semicolon     \;     Less than     \<
%%   Equals        \=     Greater than  \>     Question mark \?
%%   Commercial at \@     Left bracket  \[     Backslash     \\
%%   Right bracket \]     Circumflex    \^     Underscore    \_
%%   Grave accent  \`     Left brace    \{     Vertical bar  \|
%%   Right brace   \}     Tilde         \~}
\endinput
