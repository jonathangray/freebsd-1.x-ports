%%% template.tex
%%% This is a template for making up an AMS-LaTeX file
%%% Version of December 16, 1992
%%%------------------------------------------------------------------
%%% The following documentstyle command chooses 12 point type (instead
%%% of the default 10 point), allows us to use the commutative
%%% diagram macros, and defines the standard names for all of the
%%% special symbols in the AMSfonts package:
\documentstyle[12pt,amscd,amssymb]{amsart}






%%% This part of the file (after the documentstyle command, but before
%%% the \begin{document}) is called the ``preamble''.  This is a good
%%% place to  put our macro definitions.

\newcommand{\tensor}{\otimes}
\newcommand{\homotopic}{\simeq}
\newcommand{\homeq}{\cong}
\newcommand{\iso}{\approx}
\newcommand{\ho}{\operatorname{Ho}}

% Homotopy direct limit:
\newcommand{\hodlim}{\underrightarrow
{\operatorname{\mathstrut holim}}}

% Homotopy inverse limit:
\newcommand{\hoilim}{\underleftarrow
{\operatorname{\mathstrut holim}}}





\newcommand{\C}{{\cal C}}
\newcommand{\M}{{\cal M}}
\newcommand{\W}{{\cal W}}



%%%-------------------------------------------------------------------
%%%-------------------------------------------------------------------
%%% The Theorem environments:
%%%
%%%
%%% The following commands set it up so that:
%%%
%%% All Theorems, Corollaries, Lemmas, Propositions, Definitions,
%%% Remarks, and Examples will be numbered in a single sequence, and
%%% the numbering will be within each section.
%%%
%%% Anything called `bigthm' in the TeXfile will be printed as
%%% Theorem, but will be numbered in a separate sequence, named
%%% Theorem A, Theorem B, Theorem C, etc.
%%%
%%%
%%% Notations and Terminologies will not be numbered.
%%%
%%% Theorems, Propositions, Lemmas, and Corollaries will have the most
%%% formal typesetting.
%%%
%%% Definitions will have the next level of formality.
%%%
%%% Remarks, Examples, Notations, and Terminologies will be the least
%%% formal.
%%%
%%% Theorem:
%%% \begin{thm}
%%%
%%% \end{thm}
%%%
%%% Theorem: (Numbered separately, as Theorem A, etc.)
%%% \begin{bigthm}
%%%
%%% \end{bigthm}
%%%
%%% Corollary:
%%% \begin{cor}
%%%
%%% \end{cor}
%%%
%%% Lemma:
%%% \begin{lem}
%%%
%%% \end{lem}
%%%
%%% Proposition:
%%% \begin{prop}
%%%
%%% \end{prop}
%%%
%%% Definition:
%%% \begin{defn}
%%%
%%% \end{defn}
%%%
%%% Remark:
%%% \begin{rem}
%%%
%%% \end{rem}
%%%
%%% Example:
%%% \begin{ex}
%%%
%%% \end{ex}
%%%
%%% Notation:
%%% \begin{notation}
%%%
%%% \end{notation}
%%%
%%% Terminology:
%%% \begin{terminology}
%%%
%%% \end{terminology}
%%%
%%%       Theorem environments

\theoremstyle{plain}   %% This is the default, anyway
\begingroup % Confine the \theorembodyfont command
\theorembodyfont{\sl}
\newtheorem{bigthm}{Theorem}   % Numbered separately, as A, B, etc.
\newtheorem{thm}{Theorem}[section]   % Numbered within each section
\newtheorem{cor}[thm]{Corollary}     % Numbered along with thm
\newtheorem{lem}[thm]{Lemma}         % Numbered along with thm
\newtheorem{prop}[thm]{Proposition}  % Numbered along with thm
\endgroup

%%% We need to do the following outside of any group,
%%% since it's not \global:
\renewcommand{\thebigthm}{\Alph{bigthm}}  % Number as "Theorem A."


\theoremstyle{definition}
\newtheorem{defn}[thm]{Definition}   % Numbered along with thm

\theoremstyle{remark}
\newtheorem{rem}[thm]{Remark}        % Numbered along with thm
\newtheorem{ex}[thm]{Example}        % Numbered along with thm
\newtheorem{notation}{Notation}
\renewcommand{\thenotation}{}  % to make the notation
                               % environment unnumbered
\newtheorem{terminology}{Terminology}
\renewcommand{\theterminology}{}  % to make the terminology
                                  % environment unnumbered
%%%-------------------------------------------------------------------
%%% The following causes equations to be numbered within sections:
\numberwithin{equation}{section}

%%%-------------------------------------------------------------------
%%%-------------------------------------------------------------------
%%%-------------------------------------------------------------------
%%%-------------------------------------------------------------------
%%%-------------------------------------------------------------------
%%%-------------------------------------------------------------------
%%%-------------------------------------------------------------------
\begin{document}

%%% In the title, use a double backslash "\\" to show a linebreak:
%%% Use one of the following two forms:
%%% \title{Text of the title}
%%% or
%%% \title[Short form for the running head]{Text of the title}
\title


\author{}

%%% In the address, show linebreaks with double backslashes:
\address{}

%%% Email address is optional.  If you include it, use a double at
%%% sign "@@" to produce a single at sign in the printed copy, e.g.,
%%% \email{nsteenrod@@math.princeton.edu}
\email{}

%%% To have the current date inserted, use \date{\today}:
\date{}


\maketitle

%%% To include a table of contents, uncomment the next line:
% \tableofcontents
%%%-------------------------------------------------------------------
%%%-------------------------------------------------------------------
%%% Start the body of the paper here!  E.G., maybe use:
%%% \section{Introduction}
%%% \label{sec:intro}

















%%%-------------------------------------------------------------------
%%%-------------------------------------------------------------------
%%% The number "10" that appears in the next command is a TOTALLY
%%% RANDOM NUMBER which is chosen so that if it was printed, it would
%%% be at least as wide as any number of an item in the bibliography:

\begin{thebibliography}{10}




%%% The format of bibliography items is as in the following examples:
%%%
%%% \bibitem{yellowmonster}
%%% A. K. Bousfield and D. M. Kan, {\em Homotopy Limits, Completions
%%% and Localizations,} Lecture Notes in Mathematics number 304,
%%% Springer-Verlag, New York, 1972.
%%%
%%% \bibitem{HA}
%%% D. G. Quillen, {\em Homotopical Algebra,} Lecture Notes in
%%% Mathematics number 43, Springer-Verlag, Berlin, 1967.








\end{thebibliography}
\end{document}
