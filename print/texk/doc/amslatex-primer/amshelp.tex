%%% amshelp.tex
%%% This is an AMS-LaTeX file

%%% This is an attempt to explain how to get up and running with
%%% AmS-LaTeX, if you've already got some familiarity with either
%%% TeX or AMS-TeX or LaTeX.

%%% Copyright (C) Philip Hirschhorn 1993

%%% Permission is granted to reproduce the document in any way providing
%%% that it is distributed for free, except for any reasonable charges for
%%% printing, distribution, staff time, etc.  Direct commercial
%%% exploitation is not permitted.  Extracts may be made from this
%%% document providing an acknowledgment of the original source is
%%% maintained.

%%% Philip Hirschhorn
%%% Department of Mathematics
%%% Wellesley College
%%% Wellesley, MA 02181
%%% phirschhorn@lucy.wellesley.edu
%%% psh@math.mit.edu

%%% This version is dated December 16, 1992
%%% (If this is changed, remember to change
%%% the \date command below.)

%---------------------------------------------------------------------
\documentstyle[12pt,amscd]{amsart}



\newcommand{\tensor}{\otimes}
\newcommand{\homotopic}{\simeq}
\newcommand{\homeq}{\cong}
\newcommand{\iso}{\approx}
\newcommand{\ho}{\operatorname{Ho}}







\newcommand{\C}{{\cal C}}
\newcommand{\M}{{\cal M}}
\newcommand{\W}{{\cal W}}


\newenvironment{technical}{\begingroup \small}{\endgroup}




%---------------------------------------------------------------------
%---------------------------------------------------------------------

%       Theorem environments

\theoremstyle{plain} %% This is the default, anyway
\begingroup          % Confine the \theorembodyfont command
\theorembodyfont{\sl}
\newtheorem{bigthm}{Theorem}  % Numbered separately, as A, B, etc.
\newtheorem{thm}{Theorem}[section]
\newtheorem{cor}[thm]{Corollary}
\newtheorem{lem}[thm]{Lemma}
\newtheorem{prop}[thm]{Proposition}
\endgroup

%%% We need to do the following outside of any group,
%%% since it's not \global:
\renewcommand{\thebigthm}{\Alph{bigthm}}  % Number as "Theorem A."


\theoremstyle{definition}
\newtheorem{defn}[thm]{Definition}

\theoremstyle{remark}
\newtheorem{rem}[thm]{Remark}
\newtheorem{ex}[thm]{Example}
\newtheorem{notation}{Notation}
\renewcommand{\thenotation}{}  % to make the notation
                               % environment unnumbered
\newtheorem{terminology}{Terminology}
\renewcommand{\theterminology}{}  % to make the terminology
                                  % environment unnumbered

\numberwithin{equation}{section}



%---------------------------------------------------------------------
%---------------------------------------------------------------------
%---------------------------------------------------------------------
%---------------------------------------------------------------------

\begin{document}


\title[Running \AmS-\LaTeX]
{Getting up and running\\
with \AmS-\LaTeX}

\author{Philip S. Hirschhorn}

\address{Department of Mathematics\\
         Wellesley College\\
         Wellesley, MA 02181}

\email{phirschhorn@@lucy.wellesley.edu}


\date{December 16, 1992}



\maketitle

\begin{abstract}
This is an attempt to tell you enough about \LaTeX{} and \AmS-\LaTeX{}
so that you can get started with it {\em without\/} having to read
the book.
\end{abstract}





\tableofcontents
%---------------------------------------------------------------------
%---------------------------------------------------------------------
\section{Introduction}

This is an attempt to get you up and running with \AmS-\LaTeX\ with
the least possible pain.  These instructions won't be a substitute
for the User's Guide, but they may get you started quickly enough so
that you'll only need to refer to the guide occasionally, which
should eliminate most of the pain.



\AmS-\LaTeX{} is really just a new documentstyle for \LaTeX,
but it will only work after you've installed the new font selection
scheme into \LaTeX.  (There are actually {\em two\/} new
documentstyles, \verb"amsart" and \verb"amsbook", plus an optional
argument \verb"amstex" that can be added to the standard \LaTeX{}
\verb"article" documentstyle, but all I'll discuss here is the
\verb"amsart" documentstyle.)  Thus, using \AmS-\LaTeX{} is really
using a variety of \LaTeX.  If you're new to \LaTeX{} and these last
few sentences made no sense to you at all, don't worry about it.  You
don't have to know what the old font selection scheme is in order to
use the new one.


I'll be assuming that you have at least some experience with either
plain \TeX, \AmS-\TeX{} or \LaTeX, and I'll try to tell you what
you need to know so that you can get started with \AmS-\LaTeX\ {\em
without\/} actually reading the \LaTeX{} User's Guide~\cite{latex},
or even taking much of a look at the \AmS-\LaTeX{} User's
Guide~\cite{amslatexusersguide}.


If you've never used {\em any\/} version of \TeX, then I recommend
Jon Warbrick's {\em Essential \LaTeX\/}~\cite{essential}. This is
intended for those with no knowledge of \TeX{} or \LaTeX, and
concisely gives a description of what a \LaTeX{} document looks like
and how you type text in a \LaTeX{} document.  You should then look
at Sections~3.3.1 and~3.3.2 of the \LaTeX{} User's Guide~\cite[pages
41--47]{latex} to learn how to type mathematics in a \LaTeX{}
document.

I've also given you a template file \verb"template.tex", which is an
attempt to give you enough to mostly fake your way through an
\AmS-\LaTeX{} file, {\em almost\/} without even reading these
instructions.  I've included the text of that file in these
instructions as Section~\ref{sec:template}, so you might want to take a
look at that now, and then just use the table of contents of these
instructions to get more information on whatever in that file
confuses you.

In case you haven't guessed, these instructions were printed using
\AmS-\LaTeX, so you can get some idea what it all looks like.



%---------------------------------------------------------------------
%---------------------------------------------------------------------
\section{Basic \LaTeX\ stuff}
\label{sec:basicstuff}

In this section, we'll describe the three commands that have to be
part of any \LaTeX\ document: \verb"\documentstyle",
\verb"\begin{document}", and \verb"\end{document}".  The complete
explanation of these can be found in the \LaTeX\ User's
Guide~\cite{latex} or in {\em Essential \LaTeX\/}~\cite{essential}.
We'll also explain how to begin a new section or subsection of the
paper, and how \LaTeX\ manages to get the cross-references right
(which is also the explanation of why you need to run a file through
\LaTeX\ {\em twice\/} to be sure that all the cross-references are
correct).


\subsection{The {\tt documentstyle} command}

Before you type anything that actually appears in the paper, you must
include a \verb"\documentstyle" command.  It's easiest to just put
the \verb"\documentstyle" command at the very beginning of the file,
possibly with a few lines of comments before it.

It's actually the choice of document style that determines whether
you're using \AmS-\LaTeX\ or just plain old \LaTeX.  There are two
document styles that are a part of \AmS-\LaTeX: \verb"amsart" and
\verb"amsbook".  There is also the \verb"amstex" option that can be
added to the standard \LaTeX\ \verb"article" document style.  I'll only
be discussing the \verb"amsart" document style here.  For the others,
see the \AmS-\LaTeX\ User's Guide~\cite{amslatexusersguide}.


The simplest version of the \verb"documentstyle" command is
\begin{center}
\verb"\documentstyle{amsart}"
\end{center}
This will give you the default type size, which is 10~point type.  If
you'd like to use 12~point type, then you should include the optional
argument \verb"[12pt]"; this makes the command
\begin{center}
\verb"\documentstyle[12pt]{amsart}"
\end{center}
There should be no spaces inside the square brackets that enclose the
optional argument.

There are two other optional arguments that might be of interest.
The first is for when you want to include the macros that make it
easier to draw commutative diagrams.  (These aren't included
automatically, since they take up a lot of memory, and not everyone
wants to use them.)  If you want 10~point type and
you want to use the commutative diagram macros, then the command
is
\begin{center}
\verb"\documentstyle[amscd]{amsart}"
\end{center}
If you want 12~point type and you want to use the commutative
diagram macros, then the command is
\begin{center}
\verb"\documentstyle[12pt,amscd]{amsart}"
\end{center}
The other optional argument is for use when you want to use some of
the special symbols contained in the \AmS{}Fonts package.  (These are
the fonts \verb"msam" and \verb"msbm".)  If you want the standard
names for these symbols to be defined for your use, then you need to
include the optional argument \verb"amssymb".  Thus, to use the
default 10~point type and have the special symbols defined, use the
command
\begin{center}
\verb"\documentstyle[amssymb]{amsart}"
\end{center}
If you want to use 12~point type, the commutative diagram macros, and
special symbols from the \AmS{}Fonts collection, then use the command
\begin{center}
\verb"\documentstyle[12pt,amscd,amssymb]{amsart}"
\end{center}
You can list the optional arguments \verb"12pt", \verb"amscd", and
\verb"amssymb" in
whatever order you like, but there should be {\em no spaces\/}
inside of the square brackets, and the optional arguments should be
separated by commas.



%---------------------------------------------------------------------

\subsection{{\tt begin\{document\}} and
{\tt end\{document\}}}

Everything that is to appear in the document must appear in between
the \verb"\begin{document}" and \verb"\end{document}" commands. There
are no optional arguments for these commands, so they always look the
same.  Anything following the \verb"\end{document}" command is
ignored.  You {\em are\/} allowed to have macro definitions (i.e.,
newcommands; see Section~\ref{sec:definitions}) before the
\verb"\begin{document}", and that's actually a good place for them, but
that's about all.

%---------------------------------------------------------------------
\subsection{Sections and subsections}
\label{sec:sections}

To begin a new section, you give the command
\begin{center}
\verb"\section{Section name}"
\end{center}
To begin the present section, I gave the command
\begin{center}
\verb"\section{Basic \LaTeX\ stuff}"
\end{center}
A section number is supplied automatically.  If you want to be able
to make reference to that section, then you need to {\em label\/} it.
Since I wanted to be able to demonstrate the cross-reference
commands, I actually began this section with the lines
%
\begin{verbatim}
\section{Basic \LaTeX\ stuff}
\label{sec:basicstuff}
\end{verbatim}
%
This allows me to say ``\verb"Section~\ref{sec:basicstuff}"'' and have
it printed as Section~\ref{sec:basicstuff}.

To begin a new subsection, you give the command
\begin{center}
\verb"\subsection{Subsection name}"
\end{center}
To begin the present subsection, I gave the command
\begin{center}
\verb"\subsection{Sections and subsections}"
\end{center}
A subsection number is supplied automatically.  If you want to be able
to make reference to that subsection, then you need to {\em label\/}
it.  This subsection was begun with the lines
%
\begin{verbatim}
\subsection{Sections and subsections}
\label{sec:sections}
\end{verbatim}
%
so if we say ``\verb"Section~\ref{sec:sections}",'' it is printed as
Section~\ref{sec:sections}.

Labels always take the number of the smallest enclosing structure.
Thus, a \verb"\label" command that's inside a section but {\em not\/}
inside a subsection or Theorem or anything else will take the value
of the section counter, while a \verb"\label" command that's inside
the statement of a Theorem will take the value of that Theorem
number.  For more information on this, see Section~\ref{sec:xreferences}.

\subsubsection{Yes, there are subsubsections too}

I began this subsubsection with the command
\begin{center}
\verb"\subsubsection{Yes, there are subsubsections too}"
\end{center}
I refuse to even experiment to see if there are subsubsubsections.



\subsubsection*{Sections without numbers}

I began this subsubsection with the command
\begin{verbatim}
\subsubsection*{Sections without numbers}
\end{verbatim}
and got a subsubsection that wasn't numbered.  If you give the
command
\begin{verbatim}
\section*{A Section Title}
\end{verbatim}
then you'll begin a new section, which will not have a number.





%---------------------------------------------------------------------
\subsection{Italics {\em for emphasis\/}}

If you want to use italics to emphasize a word or two, the \LaTeX\
convention is not to switch explicitly to  italics, but rather to use
the command \verb"\em" (which means {\em emphasize\/}).  This command
works just like a font change command, except that it switches you
{\em into\/} italics if the current font is roman, and switches you
{\em out of\/} italics if the current font is italics.

For example, if you type
\begin{center}
\verb"The whole is {\em more\/} than the sum of its parts."
\end{center}
you'll get
\begin{center}
The whole is {\em more\/} than the sum of its parts.
\end{center}
but if you type
\begin{verbatim}
\begin{thm}
The whole is\/ {\em more} than the sum of its parts.
\end{thm}
\end{verbatim}
you'll get
\begin{thm}
The whole is\/ {\em more} than the sum of its parts.
\end{thm}





%---------------------------------------------------------------------
\subsection{Once is never enough}

This is an explanation of how \LaTeX\ manages to fill in
cross-references to parts of the file it hasn't processed yet, and
what those \verb".aux" and \verb".toc" files are.


\subsubsection*{Cross-References}

Every time \LaTeX\ processes your file, it writes an {\em auxiliary\/}
file.  Since the file containing these instructions is called
\verb"amshelp.tex", the auxiliary file is called \verb"amshelp.aux".
The auxiliary file contains the definitions of all the keys
used for cross-references.  When \LaTeX\ begins to process your
file, it first looks for an \verb".aux" file, and reads it in if
it exists.  Of course, this is the \verb".aux" file that was produced
the {\em last \/} time that your file was processed, so the Theorem
numbers, Section numbers, etc.\ are all the ones from the last time
the file was processed.

The very first time that \LaTeX\ processes your file, there is no
\verb".aux" file, and so \LaTeX\ gives {\em lots\/} of warning
messages about undefined labels, or whatever.  Ignore all of this.
The {\em next\/} time that you run \LaTeX, there {\em will\/} be an
\verb".aux" file, and all the references will be filled in.  (Yes, it
is possible, at least in theory, for some page number to change
every time you run \LaTeX\ on your file, even without any
changes in the source file, but this isn't very likely.)

\subsubsection*{The Table of Contents}

If you give the command \verb"\tableofcontents", then \LaTeX\ will
try to write a table of contents, including the page numbers of the
sections.  Obviously, \LaTeX\ can't know those page numbers or
section titles yet, so as \LaTeX\ processes your file, it writes a
\verb".toc" file containing the information it needs.  (The
\verb".toc" file for these instructions is \verb"amshelp.toc".)
Once again, \LaTeX\ is always using the information from the {\em
last\/} time that it processed your file.

If you {\em do\/} include a table of contents in your document, and
if the table of contents takes up at least a page or so of space,
then you might have to run \LaTeX{} {\em three\/} times in order to
get all of the cross-references correct.  The reason for this is that
the first time you run \LaTeX{} there isn't any \verb".toc" file
listing the section titles, and so the table of contents has nothing
in it.  The second time you run \LaTeX{} you'll get a table of
contents that lists the page numbers for the sections from the last
time you ran \LaTeX, when the table of contents took up no space at
all.  Unfortunately, during this second run, the table of contents
will be created, and will take up enough space to change the page
numbers of the sections from what they were during the first run.
Only during the {\em third\/} run will the correct page numbers be
written into the table of contents.  Since this doesn't change the
amount of space that the table of contents occupies, this version
will be correct.

\subsubsection*{How do I know when everything is correct?}

Every time that you run \LaTeX{}, it checks to see that the
cross-reference numbers that it read from the \verb".aux"
file are correct.  If any of them are incorrect, it
prints a warning on the screen at the very end of the run advising
you that  labels may have changed, and that you should run \LaTeX{}
again to get the cross-references right.  Unfortunately, \LaTeX{}
doesn't seem to check that the table of contents entries are correct,
so if you change the name of a section in a way that doesn't make any
page references incorrect, you won't be warned to run \LaTeX{} again.





%---------------------------------------------------------------------
%---------------------------------------------------------------------
\section{Title, Author,  and the {\tt maketitle}
command}

This stuff should go right after the \verb"\begin{document}" command.
I'll give a quick sketch here, which is probably all you'll ever
need, but the full explanation is given in the \AmS-\LaTeX\ sample
paper \verb"testart.tex"~\cite[Page~3]{testart.tex}, with extra
explanation given in the comments of the source file
\verb"testart.tex" itself.  If you are already
familiar with \LaTeX, then you should be warned that this part is
slightly different from what you do when using the standard \LaTeX{}
\verb"article" documentstyle.

%---------------------------------------------------------------------

\subsection{The title}
You announce the title with the
command
\begin{center}
\verb"\title[Optional running title]{Actual title}"
\end{center}
These instructions used the title command
\begin{verbatim}
\title[Running \AmS-\LaTeX]
{Getting up and running\\
with \AmS-\LaTeX}
\end{verbatim}
Notice that you indicate line breaks in the title with a double
backslash.  If I had decided to let the {\em full\/} title be printed
in the head of the odd numbered pages, I would have used the command
\begin{verbatim}
\title{Getting up and running\\
with \AmS-\LaTeX}
\end{verbatim}


%---------------------------------------------------------------------
\subsection{The author, and the author's address}


The author is specified with an \verb"author" command:
\begin{center}
\verb"\author{Author's name}"
\end{center}
These directions used the command
\verb"\author{Philip S. Hirschhorn}".
The author's address is given in an address command, with double
backslashes to indicate line breaks.  These instructions used the
command
\begin{verbatim}
\address{Department of Mathematics\\
         Wellesley College\\
         Wellesley, MA 02181}
\end{verbatim}
You can also include an email address, but you have to enter a double
``at'' sign for each one in the address.  These instructions used
\begin{verbatim}
\email{phirschhorn@@lucy.wellesley.edu}
\end{verbatim}
To acknowledge support, use the command \verb"\thanks", e.g.,
%
\begin{verbatim}
\thanks{Supported in part by NSF grant 3.14159}
\end{verbatim}
%
This will be printed as a footnote on the first page.





\subsubsection*{Multiple authors}

If there are several authors, then each one should have a separate
\verb"\author" command, with each individual's address following that
individual's \verb"\author" command, in its own \verb"\address"
command.  If there {\em are\/} several authors, and their combined
names are too long for the running head on the even numbered pages,
you can give a running head as an optional argument in square
brackets.  (It's apparently a convention that the running head in a
multiple author paper should have only initials for the first and
middle names, but I don't think that I was invited to that
convention.)  The source file \verb"testart.tex" is an example
of a paper with three authors, two of whom are at the same
institution.  The comments in the source file carefully explain
what's going on.


%---------------------------------------------------------------------

\subsection{The date}

This is pretty straightforward:
\begin{center}
\verb"\date{Whatever date you please}"
\end{center}
To have the date of processing used, use the command
\verb"\date{\today}".

%---------------------------------------------------------------------
\subsection{{\tt maketitle}}


After you've given all of the commands mentioned in this section, you
can give the command \verb"\maketitle".  The exact arrangement is
determined by the document style.  In particular, the \verb"amsart"
document style puts the author's address at the {\em end\/} of the
paper.  If you {\em don't\/} give
the command \verb"\maketitle", a title won't be made.



%---------------------------------------------------------------------
%---------------------------------------------------------------------
\section{Theorems, Propositions, Lemmas, etc.}

The instructions in this section assume that you're using the
\verb"newtheorem" commands that I put in the file
\verb"template.tex".

\subsection{Stating theorems, propositions, etc.}

To state a theorem, you do the following:
\begin{verbatim}
\begin{thm}
The square of the hypotenuse of a right triangle is equal to the sum
of the squares of the two adjacent sides.
\end{thm}
\end{verbatim}
If you do that, you'll get the following:
\begin{thm}
\label{pythagthm}
The square of the hypotenuse of a right triangle is equal to the sum
of the squares of the two adjacent sides.
\end{thm}
If you thought that it was only a proposition, you'd use
\begin{verbatim}
\begin{prop}
The square of the hypotenuse of a right triangle is equal to the sum
of the squares of the two adjacent sides.
\end{prop}
\end{verbatim}
and you'd get
\begin{prop}
The square of the hypotenuse of a right triangle is equal to the sum
of the squares of the two adjacent sides.
\end{prop}

If you think it's a theorem again, but you'd like to make reference
to it in some other part of the paper, you have to choose a {\em
key\/} with which you'll refer to it, and then {\em label\/} the
theorem.  If you want to use the key {\em pythagthm}, then it would
look like the following:
\begin{verbatim}
\begin{thm}
\label{pythagthm}
The square of the hypotenuse of a right triangle is equal to the sum
of the squares of the two adjacent sides.
\end{thm}
\end{verbatim}
If you later give the command \verb"\ref{pythagthm}", then that
command will expand to the {\em number\/} that was assigned to that
theorem (in this case, \ref{pythagthm}).  For more explanation of
cross-references, see Section~\ref{sec:xreferences}.

If you'd like to state a theorem and give a {\em name\/} to it, then
you can add an optional argument to the \verb"\begin{thm}" command.
If you type
\begin{verbatim}
\begin{thm}[Pythagorus]
The square of the hypotenuse of a right triangle is equal to the sum
of the squares of the two adjacent sides.
\end{thm}
\end{verbatim}
you'll get
\begin{thm}[Pythagorus]
The square of the hypotenuse of a right triangle is equal to the sum
of the squares of the two adjacent sides.
\end{thm}


\subsubsection*{Summary of environments provided in the template}

Most of the following structures are numbered in the same sequence,
in the form SectionNumber.Number.  The exceptions are that the
\verb"bigthm" environment is numbered separately and uses letters
instead of numbers (i.e., {\bf Theorem~A}, {\bf Theorem~B}, etc.),
and the Notation and Terminology environments are unnumbered.

$$
\begin{tabular}{c@{\hspace{4em}}l@{\hspace{4em}}c}

\multicolumn{3}{c}{Theorem Environments}\\*[8pt]

\hspace{1em}Environment Name&  Printed Form&
                             Font shape used in body\\*[6pt]

{\tt bigthm}&        {\bf Theorem}&      Slanted\\
{\tt thm}&           {\bf Theorem}&      Slanted\\
{\tt cor}&           {\bf Corollary}&    Slanted\\
{\tt lem}&           {\bf Lemma}&        Slanted\\
{\tt prop}&          {\bf Proposition}&  Slanted\\
{\tt defn}&          {\bf Definition}&   Normal\\
{\tt rem}&           {\it Remark}&       Normal\\
{\tt ex}&            {\it Example}&      Normal\\
{\tt notation}&      {\it Notation}&     Normal\\
{\tt terminology}&   {\it Terminology}&  Normal\\*[3pt]
\end{tabular}
$$



For full details, see the beginning of the template file (reproduced
here in Section~\ref{sec:template}), after the comment ``The Theorem
Environments.''  To change the font used in the body of the
environment, see Section~\ref{sec:thmfonts}.






%---------------------------------------------------------------------
\subsection{Proofs}

To give a proof, you do the following:
\begin{verbatim}
\begin{pf}
As any fool can plainly see, it's true!
\end{pf}
\end{verbatim}
and you'll get the following:
\begin{pf}
As any fool can plainly see, it's true!
\end{pf}
If the theorem said that a condition was both necessary and sufficient
for something, and you want to prove each part separately,
you can do the following:
\begin{verbatim}
\begin{pf*}{Proof (sufficiency)}
Well, it's {\em obviously\/} sufficient!
\end{pf*}
\end{verbatim}
and you'll get
\begin{pf*}{Proof (sufficiency)}
Well, it's {\em obviously\/} sufficient!
\end{pf*}
that is, if you use the \verb"pf*" environment instead of the
\verb"pf" environment, you must include a second
argument to the \verb"\begin{pf*}" command
which will be printed in place of the word ``{\em Proof\/}''
that would have been supplied by \verb"\begin{pf}".

If the proof of Theorem~\ref{pythagthm} does not appear immediately
after its statement, you might use the following:
\begin{verbatim}
\begin{pf*}{Proof of Theorem~\ref{pythagthm}}
As any fool can plainly see, it's true!
\end{pf*}
\end{verbatim}
and you'd get
\begin{pf*}{Proof of Theorem~\ref{pythagthm}}
As any fool can plainly see, it's true!
\end{pf*}

%---------------------------------------------------------------------
\subsection{Changing the fonts used in theorems, etc.}
\label{sec:thmfonts}

This section is set in small type because it's technical,
and you probably don't need to worry about it.  We explain how
to change the font used in the body of theorem environments, in case
you don't like what we've set up.

\begin{technical}
Each `theorem' environment is defined by a \verb"\newtheorem"
command.  The fonts used in the header and the body of the
`theorem' are determined by the theoremstyle that is
current when the \verb"\newtheorem" command is given.

The \verb"amsart" documentstyle provides three theoremstyles:
\verb"plain", \verb"definition", and \verb"remark".  The \verb"plain"
theoremstyle chooses italics as the default font for the body of the
theorem.  In the template, we've used the \verb"\theorembodyfont"
command to change the font used in the body of the theorem to
slanted.

The template file includes the commands
%
\begin{verbatim}
\theoremstyle{plain}   %% This is the default, anyway
\begingroup % Confine the \theorembodyfont command
\theorembodyfont{\sl}
\newtheorem{bigthm}{Theorem}   % Numbered separately, as A, B, etc.
\newtheorem{thm}{Theorem}[section]   % Numbered within each section
\newtheorem{cor}[thm]{Corollary}     % Numbered along with thm
\newtheorem{lem}[thm]{Lemma}         % Numbered along with thm
\newtheorem{prop}[thm]{Proposition}  % Numbered along with thm
\endgroup

%%% We need to do the following outside of any group,
%%% since it's not \global:
\renewcommand{\thebigthm}{\Alph{bigthm}}  % Number as "Theorem A."
\end{verbatim}
%
This sets things up so that the font used for the body of theorems,
Propositions, etc.{} is slanted.  The \verb"\theorembodyfont" command
changes the font used, and the \verb"\begingroup" and
\verb"\endgroup" commands keep this from affecting any environments
that are defined later.

If you prefer italics in the body of theorems, then remove the
\verb"\theorembodyfont" command.  If you'd like to change the font
used in the body of `theorems' in either the \verb"definition" or
\verb"remark" theoremstyles (both of which use normal type), then
insert the appropriate \verb"\theorembodyfont" commands (enclosing the
\verb"\theorembodyfont" and \verb"\newtheorem" commands in a group,
to avoid unintentionally affecting the font of `theorems' that you
define later).

\end{technical}





%---------------------------------------------------------------------
%---------------------------------------------------------------------
\section{Cross-References}
\label{sec:xreferences}

This section explains how to make reference to numbered sections,
theorems, equations, and bibliography items, with the correct
reference numbers filled in automatically by \LaTeX.

%---------------------------------------------------------------------
\subsection{References to sections, theorems and equations}
\label{sec:thmrefs}

For each structure in the manuscript to which you'll be making
reference, you must assign a {\em key\/} that you'll use to refer to
that structure.  For sections, theorems and numbered equations, you assign
the key using the \verb"\label" command.  This command takes one
argument, which is the {\em key\/} you're assigning to the object.
The command \verb"\ref{key}" then produces the number that was
assigned to that structure.  If the structure is an equation, then
the command \verb"\eqref{key}" should be used instead of the command
\verb"\ref{key}".


Consider the following example.
\begin{thm}
\label{homotopy}
If the maps $f\colon X \to Y$ and $g\colon X \to Y$ are homotopic,
then $f_* = g_* \colon H_*X \to H_*Y$.
\end{thm}

We typed this theorem as follows.
\begin{verbatim}
\begin{thm}
\label{homotopy}
If the maps $f\colon X \to Y$ and $g\colon X \to Y$ are homotopic,
then $f_* = g_* \colon H_*X \to H_*Y$.
\end{thm}
\end{verbatim}
If we now type ``\verb"see Theorem~\ref{homotopy}",'' then it will be
printed as ``see Theorem~\ref{homotopy}.''





\subsubsection*{So, what exactly is the label labeling?}

We began this section by typing
\begin{verbatim}
\section{Cross-References}
\label{sec:xreferences}
\end{verbatim}
and we began this subsection by typing
\begin{verbatim}
\subsection{References to sections, theorems and equations}
\label{sec:thmrefs}
\end{verbatim}
The phrase ``\verb"See Section~\ref{sec:xreferences}"'' is
printed as ``See Section~\ref{sec:xreferences},'' while the phrase
``\verb"See Section~\ref{sec:thmrefs}"'' is printed as
``See Section~\ref{sec:thmrefs}.''


The command \verb"\label{key}" assigns to \verb"key" the value of the
{\em smallest enclosing structure}.  That's why the command
\verb"\ref{sec:xreferences}" is printed
as ~\ref{sec:xreferences}, while
\verb"\ref{sec:thmrefs}" is printed as~\ref{sec:thmrefs}:
the key \verb"sec:xreferences" was defined inside of
Section~\ref{sec:xreferences} but {\em outside\/} of
Section~\ref{sec:thmrefs}, while the key \verb"sec:thmrefs" was
defined {\em inside\/} of Section~\ref{sec:thmrefs}.


\subsubsection*{References to equations}

To make reference to a numbered equation, you assign the {\em key\/}
as before, but you replace \verb"\ref" with \verb"\eqref".  For
example, if you type
%
\begin{verbatim}
\begin{equation}
\label{additivity}
H_* \bigvee_{\alpha\in A} X_\alpha   \iso
          \bigoplus_{\alpha\in A}H_* X_\alpha
\end{equation}
\end{verbatim}
%
then you'll get
%
\begin{equation}
\label{additivity}
H_* \bigvee_{\alpha\in A} X_\alpha   \iso
          \bigoplus_{\alpha\in A}H_* X_\alpha
\end{equation}
%
If we now say
%
\begin{verbatim}
\begin{thm}
Equation~\eqref{additivity} is true for all sorts of functors $H$.
\end{thm}
\end{verbatim}
%
then we'll get
%
\begin{thm}
Equation~\eqref{additivity} is true for all sorts of functors $H$.
\end{thm}
Notice the parentheses around the equation number, and the fact that
even though the theorem is set in slanted type, the equation
number is set in an upright font.  This is the advantage of using
\verb"\eqref" instead of \verb"\ref"; the command \verb"\eqref"
arranges it so that the number and surrounding parentheses are in an
upright font no matter what the surrounding font, and supplies an
italic correction if it's needed.


%---------------------------------------------------------------------
\subsection{References to page numbers}

If you want to make reference to the {\em page\/} that contains a
label, rather than to the structure that is labeled, use the command
\verb"\pageref{key}".  For example, if you type
%
\begin{verbatim}
Look at page~\pageref{homotopy} to find Theorem~\ref{homotopy}.
\end{verbatim}
%
you'll get ``Look at page~\pageref{homotopy} to find
Theorem~\ref{homotopy}.''







%---------------------------------------------------------------------
\subsection{Bibliographic references}
\label{sec:bibreferences}


Bibliography entries receive a {\em key\/} as part of their basic
structure.  Each item in the bibliography is entered as
\begin{verbatim}
\bibitem{key} The actual bibliography item goes here.
\end{verbatim}
(For more detail on this, see Section~\ref{sec:bibliography}.)

You refer to bibliography items using the \verb"\cite" command.  For
example, the bibliography of these instructions contains the entry
%
\begin{verbatim}
\bibitem{HA}
D. G. Quillen, {\em Homotopical Algebra,} Lecture Notes in
Mathematics number 43, Springer-Verlag, Berlin, 1967.
\end{verbatim}
%
If we say ``\verb"This is the work of Quillen~\cite{HA}",'' then it
will be printed as ``This is the work of Quillen~\cite{HA}.''  Notice
that square brackets have been inserted around the bibliography item
number.

The \verb"\cite" command takes an optional second argument, which
allows you to annotate the reference.  If we say
``\verb"see~\cite[Chapter I]{HA}",'' then it will be printed as
``see~\cite[Chapter I]{HA}.''





%---------------------------------------------------------------------
%---------------------------------------------------------------------
\section{Mathematics}


\subsection{Mathematics in running text}

This is pretty much exactly as it is in plain \TeX, except that you
have an extra option (which you can ignore).  The simplest thing is
to just enclose between dollar signs (\verb"$") any material that should be
in math mode.  Thus, if you type
\begin{center}
\verb"This is true for $ i + j \le n$."
\end{center}
you'll get
\begin{center}
This is true for $ i + j \le n$.
\end{center}

The only novelty that \LaTeX\ introduces is that, instead of using
a dollar sign  to toggle math mode on and off, you can use
`\verb"\("' to {\em begin\/} math mode, and `\verb"\)"' to {\em end\/}
math mode.  Thus, the example above is equivalent to typing
\begin{center}
\verb"This is true for \( i + j \le n\)."
\end{center}
This provides a tiny bit more error checking, but can
otherwise be safely ignored.


%---------------------------------------------------------------------
\subsection{Displayed mathematics}

For simple displayed mathematics without an equation number, this is
very much like plain \TeX, again with extra choices that can be
ignored.  If you enclose material between double dollar
signs(\verb"$$"), it will be interpreted in math mode and displayed.
Thus, if you've previously given the command
\verb"\newcommand{\iso}{\approx}" (see
Section~\ref{sec:definitions}),
and you type
%
\begin{verbatim}
$$
\pi_1(X \vee Y) \iso \pi_1X * \pi_1Y
$$
\end{verbatim}
%
you'll get
%
$$
\pi_1(X \vee Y) \iso \pi_1X * \pi_1Y
$$
%
The new choices are that exactly the same thing will be obtained by
either
%
\begin{verbatim}
\[
\pi_1(X \vee Y) \iso \pi_1X * \pi_1Y
\]
\end{verbatim}
%
or by
%
\begin{verbatim}
\begin{displaymath}
\pi_1(X \vee Y) \iso \pi_1X * \pi_1Y
\end{displaymath}
\end{verbatim}
%
or by
%
\begin{verbatim}
\begin{equation*}
\pi_1(X \vee Y) \iso \pi_1X * \pi_1Y
\end{equation*}
\end{verbatim}


If you'd like the displayed formula to be {\em numbered}, then you
should use the \verb"equation" environment.  (\LaTeX{} calls all
formula numbers {\em equation numbers}, whether or not the
mathematics has anything to do with equations.)  If you type
%
\begin{verbatim}
\begin{equation}
\pi_1(X \vee Y) \iso \pi_1X * \pi_1Y
\end{equation}
\end{verbatim}
%
you'll get
%
\begin{equation}
\label{pi1eqn}
\pi_1(X \vee Y) \iso \pi_1X * \pi_1Y
\end{equation}
%
(Notice that the \verb"equation" environment produces an equation
number, while the \verb"equation*" environment doesn't.  This is a
standard \LaTeX ism: Adding an asterisk to the name of a numbered
\LaTeX{} environment often gives the unnumbered equivalent.)


If you'd like to be able to make reference to the equation number,
you need to {\em label\/} the equation, using a {\em key\/} that you
can use for referencing it:
%
\begin{verbatim}
\begin{equation}
\label{pi1eqn}
\pi_1(X \vee Y) \iso \pi_1X * \pi_1Y
\end{equation}
\end{verbatim}
%
If you later type ``\verb"see formula~\eqref{pi1eqn}"'' you'll get
``see formula~\eqref{pi1eqn}.''  (For more on cross-references to
formulas, see Section~\ref{sec:thmrefs}.)

\AmS-\LaTeX{} has several environments that make it easier to typeset
complicated multiline displays.  These are explained in the
\AmS-\LaTeX{} User's Guide~\cite[pages 26--30]{amslatexusersguide}
and illustrated in
\verb"testart.tex"~\cite[pages 30--42]{testart.tex}.



%---------------------------------------------------------------------
\subsection{Commutative diagrams}

\AmS-\LaTeX{} provides the \verb"CD" environment for drawing
commutative diagrams.  These only allow for {\em rectangular\/}
diagrams, but they're very convenient to use.  If you want diagonal
arrows, this section is of no use to you.

{\bf Important:} To use the \verb"CD" environment, your
\verb"documentstyle" command {\em must\/} include the \verb"amscd"
option.  For example, the template file uses the command
\begin{center}
\verb"\documentstyle[12pt,amscd]{amsart}"
\end{center}
which selects the \verb"amsart" documentstyle, with two options: the
first selects twelve point type as the default, and the second
includes the macros for the \verb"CD" environment.

\subsubsection*{A simple example}

To produce the diagram
$$
\begin{CD}
A    @>>>     B   @=     B\\
@AAA       @|        @VVV\\
X   @<<<   B   @>>>      Y
\end{CD}
$$
you type
\begin{verbatim}
$$
\begin{CD}
A    @>>>     B   @=     B\\
@AAA       @|        @VVV\\
X   @<<<   B   @>>>      Y
\end{CD}
$$
\end{verbatim}
This illustrates several things.  First of all, the \verb"CD"
environment must be inside of a displayed mathematics environment.
(Here we used the standard \verb"$$" toggle to get displayed
mathematics.  If we had used, e.g., \verb"\begin{equation}" and
\verb"\end{equation}", we would have had an equation number assigned
to the display.)  Right pointing arrows are obtained with
\verb"@>>>", left pointing arrows with \verb"@<<<", up
pointing arrows with \verb"@AAA", down pointing
arrows with \verb"@VVV", horizontal equals signs with \verb"@=", and
vertical equals signs with \verb"@|".  Each line except the last is
ended with a double backslash (\verb"\\").

\subsubsection*{Labeling the arrows}
The arrows can also be labeled.  For horizontal arrows, anything
between the first and second inequality sign goes above the arrow,
and anything between the second and third inequality sign goes below
it.  For downward arrows, anything between the first and second
\verb"V" goes to the left, and anything between the second and third
goes to the right (and similarly for upward arrows).  Thus, if you
type
\begin{verbatim}
\begin{equation*}
\begin{CD}
H_iX  @>f_*>>     H_iY   @<g_*<\iso<    E_fA\\
@V\phi VV        @V\psi VV            @AA\Omega A\\
\pi_iQ   @>\alpha\beta\gamma>>    \pi_i(R,S)
                   @<<\text{A long arrow}<   \prod_{k=1}^i H_kZ
\end{CD}
\end{equation*}
\end{verbatim}
you'll get
\begin{equation*}
\begin{CD}
H_iX  @>f_*>>     H_iY   @<g_*<\iso<    E_fA\\
@V\phi VV        @V\psi VV            @AA\Omega A\\
\pi_iQ   @>\alpha\beta\gamma>>    \pi_i(R,S)
                   @<<\text{A long arrow}<   \prod_{k=1}^i H_kZ
\end{CD}
\end{equation*}


\subsubsection*{Leaving out part of the rectangle}

If you want to end a line in the diagram early (omitting the right
end of that line), just type the double backslash.  If you want to
leave out the {\em beginning\/} of a line, you type ``\verb"@."''
(the ``at'' sign followed by a period) to denote an invisible arrow.
(It's the arrows that are the column markers in the \verb"CD"
environment.)  Thus, if you type
\begin{verbatim}
\begin{equation*}
\begin{CD}
X\\
@VfVV\\
Y  @=     Y\\
@.       @VVgV\\
  @.     Z
\end{CD}
\end{equation*}
\end{verbatim}
you'll get
\begin{equation*}
\begin{CD}
X\\
@VfVV\\
Y  @=     Y\\
@.       @VVgV\\
  @.     Z
\end{CD}
\end{equation*}





%---------------------------------------------------------------------
%---------------------------------------------------------------------
\section{Macro definitions, a.k.a.\ {\tt newcommand}}
\label{sec:definitions}


\LaTeX\ allows you to use the same \verb"\def" command that you use
in plain \TeX, but it's considered bad style.  Instead, \LaTeX\ has
the \verb"\newcommand" and \verb"\renewcommand" commands, which do a
little error checking for you.  In plain \TeX, you might use the
command
\begin{center}
\verb"\def\tensor{\otimes}"
\end{center}
but in \LaTeX, the preferred form is
\begin{center}
\verb"\newcommand{\tensor}{\otimes}"
\end{center}
The advantage of this is that \LaTeX\ will check to see if there
already is a command with the name \verb"\tensor", and give you an
error message if there is.  If you know that there is a previous
definition of \verb"\tensor" but you {\em want\/} to override it,
then you use the command
\begin{center}
\verb"\renewcommand{\tensor}{\otimes}"
\end{center}


If you want to use macros with replaceable parameters, the
\verb"newcommand" command allows this.  For the equivalent of the
plain \TeX\ command
\begin{center}
\verb"\def\pushout#1#2#3{#1\cup_{#2}#3}"
\end{center}
you use the \LaTeX\ command
\begin{center}
\verb"\newcommand{\pushout}[3]{#1\cup_{#2}#3}"
\end{center}
i.e., the command name is enclosed in braces, and the number of
parameters is enclosed in square brackets.



%---------------------------------------------------------------------
%---------------------------------------------------------------------
\section{Lists: {\tt itemize, enumerate, and description}}

There are three list making environments: {\tt itemize},
{\tt enumerate}, and {\tt description}.  The {\tt itemize}
environment just lists the items with a marker in front of each one.
If you type
%
\begin{verbatim}
\begin{itemize}
\item
This is the first item in the list, which runs on long enough to
spill over onto a second line.
\item
This is the second item in the list, which is a bit shorter.
\item
This is the last item.
\end{itemize}
\end{verbatim}
%
then you'll get
%
\begin{itemize}
\item
This is the first item in the list, which runs on long enough to
spill over onto a second line.
\item
This is the second item in the list, which is a bit shorter.
\item
This is the last item.
\end{itemize}



The {\tt enumerate} environment looks the same, except that the items
in the list are numbered.  If you type
\begin{verbatim}
\begin{enumerate}
\item
This is the first item in the list, which runs on long enough to
spill over onto a second line.
\item
This is the second item in the list, which is a bit shorter.
\item
This is the last item.
\end{enumerate}
\end{verbatim}
then you'll get
\begin{enumerate}
\item
This is the first item in the list, which runs on long enough to
spill over onto a second line.
\item
This is the second item in the list, which is a bit shorter.
\item
This is the last item.
\end{enumerate}

The {\tt description} environment requires an extra argument for each
\verb"\item" command, which will be printed at the beginning of the
item.  If you type
\begin{verbatim}
\begin{description}
\item[sedge]
A green plant, found in both wetlands and uplands.  Sedges are often
confused with grasses and rushes.
\item[grass]
A green plant, found in both wetlands and uplands.  Grasses are often
confused with sedges and rushes.
\item[rush]
A green plant, found in both wetlands and uplands.  Rushes are often
confused with sedges and grasses
\end{description}
\end{verbatim}
you'll get
\begin{description}
\item[sedge]
A green plant, found in both wetlands and uplands.  Sedges are often
confused with grasses and rushes.
\item[grass]
A green plant, found in both wetlands and uplands.  Grasses are often
confused with sedges and rushes.
\item[rush]
A green plant, found in both wetlands and uplands.  Rushes are often
confused with sedges and grasses
\end{description}


These environments can be inserted within each other, and the
\verb"enumerate" environment keeps track of what level it's at, and
numbers its items accordingly.  If you type
\begin{verbatim}
\begin{enumerate}
\item I went to the dry cleaners.
\item I went to the supermarket.  I bought
\begin{enumerate}
\item Bread.
\item Cheese.
\item Tabasco sauce.
\end{enumerate}
\item I went to the bank.
\end{enumerate}
\end{verbatim}
%
you'll get
%
\begin{enumerate}
\item I went to the dry cleaners.
\item I went to the supermarket.  I bought
\begin{enumerate}
\item Bread.
\item Cheese.
\item Tabasco sauce.
\end{enumerate}
\item I went to the bank.
\end{enumerate}








%---------------------------------------------------------------------
%---------------------------------------------------------------------
\section{The bibliography}
\label{sec:bibliography}

\subsection{{\tt begin\{thebibliography\}} and {\tt
end\{thebibliography\}}}


The bibliography is begun with the command
\begin{center}
\verb"\begin{thebibliography}{number}"
\end{center}
where {\em number\/} is a random number that, when printed, is as
wide as the widest number of any item in the bibliography.  (The only
use made of \verb"number" is that \LaTeX{} assumes that the numbers
that it will assign to the actual items in the bibliography will be
no wider (when printed) than \verb"number".) For example, if the
bibliography will contain between 10 and~19 items, you can use
\verb"\begin{thebibliography}{10}".

After listing each item in the bibliography, you end the bibliography
with the \verb"\end{thebibliography}" command.


\subsection{Bibliography items}

Each item is begun with a \verb"\bibitem" command.  The format is
\begin{center}
\verb"\bibitem{key for cross-references}Item entry"
\end{center}
For example, the bibliography in these instructions contains the
entry
%
\begin{verbatim}
\bibitem{yellowmonster}
A. K. Bousfield and D. M. Kan, {\em Homotopy Limits, Completions and
Localizations,} Lecture Notes in Mathematics number 304,
Springer-Verlag, New York, 1972.
\end{verbatim}


The above entry allows you to say
\begin{verbatim}
Homotopy inverse limits are discussed
in~\cite[Chapter 11]{yellowmonster}.
\end{verbatim}
and have it print as
``Homotopy inverse limits are discussed
in~\cite[Chapter 11]{yellowmonster}.''
For more on this, see Section~\ref{sec:bibreferences}.

















%---------------------------------------------------------------------
%---------------------------------------------------------------------
\section{The template file}
\label{sec:template}

The following is the text of the file \verb"template.tex".

\begin{verbatim}
%%% template.tex
%%% This is a template for making up an AMS-LaTeX file
%%% Version of December 16, 1992
%%%------------------------------------------------------------------
%%% The following documentstyle command chooses 12 point type (instead
%%% of the default 10 point), allows us to use the commutative
%%% diagram macros, and defines the standard names for all of the
%%% special symbols in the AMSfonts package:
\documentstyle[12pt,amscd,amssymb]{amsart}






%%% This part of the file (after the documentstyle command, but before
%%% the \begin{document}) is called the ``preamble''.  This is a good
%%% place to  put our macro definitions.

\newcommand{\tensor}{\otimes}
\newcommand{\homotopic}{\simeq}
\newcommand{\homeq}{\cong}
\newcommand{\iso}{\approx}
\newcommand{\ho}{\operatorname{Ho}}

% Homotopy direct limit:
\newcommand{\hodlim}{\underrightarrow
{\operatorname{\mathstrut holim}}}

% Homotopy inverse limit:
\newcommand{\hoilim}{\underleftarrow
{\operatorname{\mathstrut holim}}}





\newcommand{\C}{{\cal C}}
\newcommand{\M}{{\cal M}}
\newcommand{\W}{{\cal W}}



%%%-------------------------------------------------------------------
%%%-------------------------------------------------------------------
%%% The Theorem environments:
%%%
%%%
%%% The following commands set it up so that:
%%%
%%% All Theorems, Corollaries, Lemmas, Propositions, Definitions,
%%% Remarks, and Examples will be numbered in a single sequence, and
%%% the numbering will be within each section.
%%%
%%% Anything called `bigthm' in the TeXfile will be printed as
%%% Theorem, but will be numbered in a separate sequence, named
%%% Theorem A, Theorem B, Theorem C, etc.
%%%
%%%
%%% Notations and Terminologies will not be numbered.
%%%
%%% Theorems, Propositions, Lemmas, and Corollaries will have the most
%%% formal typesetting.
%%%
%%% Definitions will have the next level of formality.
%%%
%%% Remarks, Examples, Notations, and Terminologies will be the least
%%% formal.
%%%
%%% Theorem:
%%% \begin{thm}
%%%
%%% \end{thm}
%%%
%%% Theorem: (Numbered separately, as Theorem A, etc.)
%%% \begin{bigthm}
%%%
%%% \end{bigthm}
%%%
%%% Corollary:
%%% \begin{cor}
%%%
%%% \end{cor}
%%%
%%% Lemma:
%%% \begin{lem}
%%%
%%% \end{lem}
%%%
%%% Proposition:
%%% \begin{prop}
%%%
%%% \end{prop}
%%%
%%% Definition:
%%% \begin{defn}
%%%
%%% \end{defn}
%%%
%%% Remark:
%%% \begin{rem}
%%%
%%% \end{rem}
%%%
%%% Example:
%%% \begin{ex}
%%%
%%% \end{ex}
%%%
%%% Notation:
%%% \begin{notation}
%%%
%%% \end{notation}
%%%
%%% Terminology:
%%% \begin{terminology}
%%%
%%% \end{terminology}
%%%
%%%       Theorem environments

\theoremstyle{plain}   %% This is the default, anyway
\begingroup % Confine the \theorembodyfont command
\theorembodyfont{\sl}
\newtheorem{bigthm}{Theorem}   % Numbered separately, as A, B, etc.
\newtheorem{thm}{Theorem}[section]   % Numbered within each section
\newtheorem{cor}[thm]{Corollary}     % Numbered along with thm
\newtheorem{lem}[thm]{Lemma}         % Numbered along with thm
\newtheorem{prop}[thm]{Proposition}  % Numbered along with thm
\endgroup

%%% We need to do the following outside of any group,
%%% since it's not \global:
\renewcommand{\thebigthm}{\Alph{bigthm}}  % Number as "Theorem A."


\theoremstyle{definition}
\newtheorem{defn}[thm]{Definition}   % Numbered along with thm

\theoremstyle{remark}
\newtheorem{rem}[thm]{Remark}        % Numbered along with thm
\newtheorem{ex}[thm]{Example}        % Numbered along with thm
\newtheorem{notation}{Notation}
\renewcommand{\thenotation}{}  % to make the notation
                               % environment unnumbered
\newtheorem{terminology}{Terminology}
\renewcommand{\theterminology}{}  % to make the terminology
                                  % environment unnumbered
%%%-------------------------------------------------------------------
%%% The following causes equations to be numbered within sections:
\numberwithin{equation}{section}

%%%-------------------------------------------------------------------
%%%-------------------------------------------------------------------
%%%-------------------------------------------------------------------
%%%-------------------------------------------------------------------
%%%-------------------------------------------------------------------
%%%-------------------------------------------------------------------
%%%-------------------------------------------------------------------
\begin{document}

%%% In the title, use a double backslash "\\" to show a linebreak:
%%% Use one of the following two forms:
%%% \title{Text of the title}
%%% or
%%% \title[Short form for the running head]{Text of the title}
\title


\author{}

%%% In the address, show linebreaks with double backslashes:
\address{}

%%% Email address is optional.  If you include it, use a double at
%%% sign "@@" to produce a single at sign in the printed copy, e.g.,
%%% \email{nsteenrod@@math.princeton.edu}
\email{}

%%% To have the current date inserted, use \date{\today}:
\date{}


\maketitle

%%% To include a table of contents, uncomment the next line:
% \tableofcontents
%%%-------------------------------------------------------------------
%%%-------------------------------------------------------------------
%%% Start the body of the paper here!  E.G., maybe use:
%%% \section{Introduction}
%%% \label{sec:intro}

















%%%-------------------------------------------------------------------
%%%-------------------------------------------------------------------
%%% The number "10" that appears in the next command is a TOTALLY
%%% RANDOM NUMBER which is chosen so that if it was printed, it would
%%% be at least as wide as any number of an item in the bibliography:

\begin{thebibliography}{10}




%%% The format of bibliography items is as in the following examples:
%%%
%%% \bibitem{yellowmonster}
%%% A. K. Bousfield and D. M. Kan, {\em Homotopy Limits, Completions
%%% and Localizations,} Lecture Notes in Mathematics number 304,
%%% Springer-Verlag, New York, 1972.
%%%
%%% \bibitem{HA}
%%% D. G. Quillen, {\em Homotopical Algebra,} Lecture Notes in
%%% Mathematics number 43, Springer-Verlag, Berlin, 1967.








\end{thebibliography}
\end{document}
\end{verbatim}











%---------------------------------------------------------------------
%---------------------------------------------------------------------

\begin{thebibliography}{10}


\bibitem{amslatexusersguide}
{\em \AmS-\LaTeX\ Version 1.1 User's Guide}, American Mathematical
Society, August, 1991.

\bibitem{testart.tex}
{\em \AmS-\LaTeX{} Sample paper for the `AMSTEX' option and the
`AMSART' documentstyle}, filename: {\tt testart.tex}, Distributed
with the \AmS-\LaTeX\ User's Guide, American Mathematical Society,
1991.

\bibitem{latex}
Leslie Lamport,
{\em \LaTeX{} User's Guide and Reference Manual}, Addison-Wesley,
1986.

\bibitem{essential}
Jon Warbrick, {\em Essential \LaTeX}, 1989.  Available by ftp from
many \TeX{} archives.  To receive a copy by email, send an email
message to \verb"fileserv@shsu.edu" containing the single line
\verb"sendme essential", and a computer will respond.

\bibitem{yellowmonster}
A. K. Bousfield and D. M. Kan, {\em Homotopy Limits, Completions
and Localizations,} Lecture Notes in Mathematics number 304,
Springer-Verlag, New York, 1972.

\bibitem{HA}
D. G. Quillen, {\em Homotopical Algebra,} Lecture Notes in
Mathematics number 43, Springer-Verlag, Berlin, 1967.


\end{thebibliography}
\end{document}
