%		*****	  TUGBOT.TeX	*****			14 Sep 88
%
%	TUGboat is put together from individual files comprising individual
%	articles, by a "top-level" file which \input's the article files
%	in the appropriate order.  This is a model "top-level" file.
%
%	All articles available in general distribution are referenced in
%	this file.  However, most are commented out; they can be activated
%	by removing the % at the beginning of the line which \input's them.

\input tugbot.sty			% definitions to format TUGboat
% \input amstex
% \documentstyle{tugbot}
%	TUGboat does not require AMS-TeX, but may call for it as an alternate
%	to reading in tugbot.sty directly; the results should be equivalent.

\Lasertrue		% assume output will be on laser printer

%	Publication information
\vol 0, 0.		% volume, issue.
\issueseqno=0		% sequential issue number
\issdate Thermidor, 2001.

%	The TUGboat editor names article files thus:	tbNxxx
%	where tb indicates TUGboat; N = sequential issue number;
%	xxx is the specific file identifier.  Then
%		\input tb99editor  or  \Input editor
%	(\Input xxx  expands to \input tbNxxx as defined in TUGBOT.STY.)
%	Articles are arranged in sections; within sections, separate
%	articles by \secsep in preference to, e.g., \eject or \vskip<dimen>.
%
%	If a page must be ended explicitly, use of \newpage will ensure
%	adherence to 2-column format.
%
%	To reset page number:	\pageno=<integer>
%
%	TUGboat is divided into sections, each tagged by \sectitle ...<
%	This file contains (commented out) entries for all usual sections.

%	Here insert macros specific to this issue.

% \def \_{\hbox{\hskip .1pt
% 		\vrule height .2pt depth .2pt width .3333em \hskip .1pt}}
%		\_ used in Pascal data names

%	This replaces the ad hoc definition of \logo in TUGBOT.STY.
%\font\logo=manfnt % font used for the METAFONT logo
%	Note:  If the font is available, uncomment this line.

\def \PS{Post\-Script}
\def \TB{{\sl The \TeX book}}

%		end issue-specific macros


%	In production, write out page numbers for xrefs and contents.
%	Use two files to prevent loss of authoritative copy for final run.
%\Input pages			% page number file from prior run
%\xdef \ppoutfile{tb\number\issueseqno pp.tex}
%\openout1=\ppoutfile		% file to write out page numbers

%	In 2-column format, additional stretch (or ragged right) is needed
%	to avoid overfull boxes.
\StretchyTenPointSpacing
\StretchyNinePointSpacing
\StretchyEightPointSpacing

\twocol

\pageno=1001			% starting page number

%%%%%%%%%%%%%%%%%%%%%%%%%%%%%%%%%%%%%%%%%%%%%%%%%%%%%%%%%%%%%%%%%%%%%%%%

% \sectitle General Delivery<

%	The hyphenation list is too long for a normal page; change the
%	length of the page, before putting anything on it.  The page
%	length will be reset automatically at the end of the page.
%	A negative value will cause the page to be made longer.
\ShortenThisPage by -1\baselineskip.

\sectitle Software<

\Input hyf				% hyphenation exception log; annual,
%					% last issue of the year
\newpage


% \sectitle Fonts<		% included in data file

\Input cyr			% Barbara Beeton, cyrillic and math symbols
%				% 6#3, revised
% \secsep
% \Input apl			% Aarno Hohti/Okko Kanerva, APL font, 8#3
\newpage


% \sectitle Output Devices<


% \sectitle Site Reports<


% \sectitle Typesetting\\on Personal Computers<


% \sectitle Macros<

% \Input tree				% David Eppstein, Trees, 6#1


% \sectitle \LaTeX<


% \sectitle Problems<


% \sectitle Queries<


% \sectitle Letters<


% \sectitle News \& Announcements<


% \sectitle TUG Business<


% \sectitle Advertisements<
%\def \midrtitle{Advertisement}


% \onecol 
% \PageXref{TUGorderForm}
% \input TUGorder

\newpage
\end
