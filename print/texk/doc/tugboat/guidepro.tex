%% @texfile{
%%     filename="guidepro.tex",
%%     version="1.07",
%%     date="3-Feb-1994",
%%     filetype="Plain TeX instructions for TUG Proceedings",
%%     copyright="Copyright (C) TeX Users Group.
%%            Copying of this file is authorized only if either:
%%            (1) you make absolutely no changes to your copy, OR
%%            (2) if you do make changes, you first rename it to some
%%                other name.",
%%     author="TeX Users Group",
%%     address="TeX Users Group,
%%            P. O. Box 869,
%%            Santa Barbara, CA 93102,
%%            USA",
%%     telephone="805-963-1338",
%%     email="Internet: TUGboat@Math.AMS.org",
%%     codetable="ISO/ASCII",
%%     checksumtype="line count",
%%     checksum="1010",
%%     keywords="tex users group, tugboat, tug proceedings",
%%     abstract="This file is the source for the article
%%            Guidelines for Proceedings of the 1994 Annual
%%            Meeting of the TeX Users Group."
%%     }
%% *********************************************************
%%
%%  TeXing this file requires the following files:
%%      TUGPROC.STY (version 1.07+)
%%      TUGBOAT.STY (version 1.09+)
%%      TUGBOAT.CMN (version 1.08+) (loaded by TUGBOAT.STY)
%%
%%%%%%%%%%%%%%%%%%%%%%%%%%%%%%%%%%%%%%%%%%%%%%%%%%%%%%%%%%%%%%%%%%%%%%%%
%%%%%%%%%%%%%%%%%%%%%%%%%%%%%%%%%%%%%%%%%%%%%%%%%%%%%%%%%%%%%%%%%%%%%%%%
\overfullrule=0pt
\def\fileversion{v1.07}
\def\filedate{3 Feb 94}
% Change history at the bottom of the file.

\input tugproc.sty

\def\MtgYear{1994}
\def\TUBissue{15 (\MtgYear), No.\ 3}
\def\PrelimDeadline{April 4, 1994}%      -  prelim papers due
\def\PreprintDeadline{June 24, 1994}%    -  revised papers due
\def\MeetingDate{July 31--Aug 4, 1994}%  -  meeting
\def\CameraDeadline{August 26, 1994}%    -  final deadline
\def\editor{editor}
\def\Editor{Editor}

%  the following items are recent additions to various tugboat macro
%  files and are included here for the convenience of users who do
%  not have the latest versions

%  added to tugproc.sty, v1.09:
\def\pfoottext{{\smc Preprint}: \MtgYear\ \TUG\ Annual Meeting}

\def\rfoottext{\tenpoint\TUB, Volume \TUBissue\Dash
    Proceedings of the \MtgYear\ Annual Meeting}

% added to tugboat.cmn, v1.10:
\def\AllTeX{(\La)\TeX}
% end of recent additions

\LoadSansFonts  % Use if needed, cf. tugboat.sty & "METAFONT" (below)

\preprint % Comment this line out for final version following meeting

% \let\Now=\null % Uncovering this will remove time stamps on preprints

%************************************************************************

\title * Guidelines for {\sl Proceedings\/} of the \MtgYear{}
         Annual Meeting\\
         of the \TeX\ Users Group *

\shorttitle * {\sl Proceedings\/} Guidelines *  % for running head

\author * The Proceedings \Editor{}:\\\hbox{}\\
          Michel Goossens *
\address * CERN, CN Division\\
           CH 1211, Geneva-23, Switzerland\\
           Phone: 22-767-3363;  \ Fax:22-767-7155 *
\netaddress[\network{Internet}] tug94-papers@cern.ch
\endnetaddress

\shortauthor * The Proceedings \Editor{} *

\abstract
These Guidelines are for authors of papers being prepared for
presentation at the \MtgYear{} Annual Meeting of the \TUG\ and
subsequent publication in the {\sl Proceedings\/} issue of \TUB.  Their
purpose is to help assure consistency in the presentation of information
and in format.  This is important when discussing \TeX\ because of its
reputation for producing beautiful work.
\endabstract

\article

\head * Papers *

\subhead * Selection and preparation of papers *

Papers to be presented at the \TUG\ Annual Meeting are selected by the
Program Committee.  These Guidelines are concerned with preparation of
the papers for distribution as Preprints and for publication in the {\sl
Proceedings\/} of the meeting.  Papers (including the bibliography)
should be around six pages in length; in addition to the body of the
paper, examples of input and output may be given in an appendix, for a
maximum of 10~pages.

Papers are expected to maintain a technical, non-commercial orientation.
Vendors are given the opportunity to introduce their products through
exhibits, and are expected to use that forum for sales presentations and
product demonstrations.  However, technical presentations which deal
with issues of design, implementation and effective use are encouraged.

\figure[\bot]
{\parindent9pt%%%%MG (to center info in box)
\boxedlist{\hbox to 14pc{\hfil \HEADfont Deadlines\hfil}\par
Original versions due {\bf \PrelimDeadline}\par
Revised versions due {\bf \PreprintDeadline}\par
Meeting presentation {\bf \MeetingDate}\par
{\sl Proceedings\/} copy due {\bf \CameraDeadline}
}}%end special parindent%%%%MG
\endfigure

\subhead * Preliminary version of the paper *

Authors should submit {\it two\/} paper copies {\it and\/} the source
file(s) of their text to the Proceedings \Editor, at the address above, by
the announced date ({\bf \PrelimDeadline}). Electronic mail or ftp is
preferred for transmission of the source; {\SMC DOS} or Macintosh discs
can also be accepted, provided they are in {\SMC ASCII} or text-only
format.

Problems may occur with electronic transmission of files, some of which
are discussed in the section Electronic Submission.
If in doubt please contact the \editor{} for additional details.

The paper copies can be on regular paper rather than on
film or high-quality paper because the copy will be marked by the
\editor; photocopies are acceptable.
Art work (graphics insertions) should be originals.

As with papers submitted to regular issues of \TUB, each paper to appear
in the {\sl Proceedings\/} will be reviewed anonymously by a referee
competent in the particular subject area of the paper.  This referee will
examine the paper for accuracy, for clarity of presentation, and for
non-commercial orientation, and will make suggestions as necessary in
those areas.  The purpose of the review process is to achieve a
consistent high quality in all issues of \TUB.  The comments of the
referee will be conveyed to the author through the \editor, and the
referee will remain anonymous.

\baselineskip.98\baselineskip


\subhead * Preprints *

Once the papers are refereed and edited, the marked-up paper copy,
revised and annotated source file, any suggestions from the referee, and
a clean proof copy will be returned to the author.  Recommendations from
the referee {\it must\/} be addressed at this time.  Authors will then
prepare their revised versions, and send the new source file (via e-mail
if possible).  The deadline for receipt of the electronic
copy for the preprints by the \editor{} is {\bf \PreprintDeadline}.

The preprints will be made available at the Meeting. The author will be
given another opportunity to refine the paper before publication in the
{\sl Proceedings\/} issue of \TUB.

Should there be any questions regarding the review, or should an author
respond other than by incorporating suggested changes, the referee may
be asked to review the revised article and approve it or make additional
comments as appropriate.  All comments are to be resolved to the
satisfaction of referee and \editor{} before the article is published.

Papers that do not conform to these Guidelines will encounter delays in
production; authors are therefore urged to prepare their manuscripts
with particular attention to the sections headed Deadlines, Macro
Packages, Format, and Contents.  The Appendix should be consulted for
additional {\sl Proceedings\/} conventions.

\subhead * Final version *

Following the Meeting, authors may elect to add new or updated
information to their articles, resulting from their presentations and
subsequent discussion with participants at the Meeting.
Revised source files, and any necessary hard-copy material
that has changed since the first version, should be
returned directly to the \editor{} prior to the deadline, {\bf
\CameraDeadline}.  To avoid damage, place hard-copy material between
cardboard and mark the package ``Do Not Bend!''

\subhead * Technical Note *

Some papers are at the cutting edge of \TeX\ and technology.  Should
typesetting of an article be particularly difficult, the author may be
required to provide camera copy for the preprints.  In addition, the
article may be moved to a subsequent issue of \TUB, in order not to
delay publication of the {\sl Proceedings}.  The \editor{} will discuss
any difficulties with the author.

\head * Timetable *

Publication of the {\sl Proceedings\/} of the Annual Meeting requires
adherence to a rigid timetable.  There are several important deadlines
(see box, first page) which authors have an obligation to meet.

Authors should contact the \editor, at least 24 hours in advance, if
there is any difficulty in meeting any one of the deadlines, so that
alternatives may be explored that will not delay production.

\head * Copyright *

The {\sl Proceedings\/} will be copyright by the \TUG, with the actual
copyright statement in the spirit of the GNU General Public License.
That is, permission ``will be granted to make and distribute verbatim
copies \dots\ provided the copyright notice and \dots\ permission
notice are preserved on all copies.''

Authors are encouraged to agree to this form of copyright; a copyright
clearance form will be sent to each author, to be filled out and
returned to the \editor.  However, any author wishing to retain full
personal rights to an article may do so by including a copyright notice
in the article file; the \editor{} will confirm such a notice with the
author.

\head * Electronic submission *

Some extra care is required when your file is to be transferred to or
from the \editor{} in electronic form.  Even though you may send it
on a disc or transfer it via ftp, the \editor{} will be communicating
with referees via electronic mail, so the file should be prepared
in the most robust manner possible in order to avoid problems and
resulting delays.

Problems arising in e-mail are often the result of communications
mismatches between different types of computer systems joined together in
the networks.  Among the causes are {\SMC ASCII}-to-{\SMC EBCDIC}
conversions and restrictions on line length in some mail systems.
Tabs should not be used at all; sometimes they do not survive
transmission, and require deciphering at the receiving end.

Lines that are too wide can also cause problems.  Some systems along
the net truncate lines which have more than 80 characters, and others
simply break such lines (the latter therefore can insert spaces in the
middle of words).  It is safest to set auto-fill or word-wrap at
something fewer than 80 characters (65 to 70 usually works well) and to
be careful to reformat after making insertions.  Finally, avoid placing
a period at the beginning of any line, as in some systems this indicates
the end of a file, and everything following will be lost.

If you have access to a network connection that permits ftp, you
should request the \editor{} to make arrangements to use it.  This
technique preserves the integrity of the files and is especially
important for bitmapped graphics, which cannot be safely e-mailed.

\head * Macro Packages *

These macro packages for the {\sl Proceedings\/} of the TUG Annual
Meeting are distributed with these Guidelines ({\tt guidepro.tex}):
\verbatim
tugboat.sty   \ use with plain.tex
tugproc.sty   / use with plain.tex
tugboat.cmn   - use with BOTH
                plain and latex
ltugboat.sty  \ use with latex.tex
ltugproc.sty  / use with latex.tex
\endverbatim
The three {\tt *tugboat} files are described in Whitney and Beeton
(1989).  The remaining two ({\tt *tugproc}) files contain supplementary
\TeX\ and \LaTeX\ macros for the {\sl Proceedings\/} issue of \TUB.

If there are any macros that you define yourself for use in your paper,
but whose definitions you do not print in the article (including its
appendix), their definitions should be incorporated near the top of your
source file.

If some of your macros are both listed and used, they should be put into
a separate file which is |\input| at the top of your text file, {\it as
well as\/} read in when it is to be listed; cf.\ |\verbfile| in Whitney
and Beeton (1989, page 380).  At this writing, no \LaTeX{} equivalent has
yet been installed; please notify the \editor{} as soon as possible if you
need this facility.

These Guidelines were produced by processing the file {\tt guidepro.tex}
with {\tt tugproc.sty}, {\tt tugboat.sty}, and {\tt tugboat.cmn}.

Any questions regarding any of these files should be directed to the
\editor.

\head * Format and Structure *

While these Guidelines were prepared with the plain \TeX{} macros,
they describe both the plain \TeX{} and \LaTeX{} commands to be used
to prepare articles for the {\sl Proceedings}.  (The relevant files
are identified in the section Macro Packages, and the source file for
these Guidelines can be used as a model for a plain-based article.)
The {\sl Proceedings\/} macros modify and augment the basic \TUB\/
macros.  The present document describes only the {\sl Proceedings\/}
macros; authors should also consult the article on the basic \TUB{}
macros in Whitney and Beeton (1989).  In addition, sample uses of
typical plain and \LaTeX{} commands are shown in the Appendix.

\subhead * Title and author(s) *

The text used for the \hbox{|\title|} and |\author| macros will appear
both in the top matter on the first page of the article, and in the
running heads on subsequent pages.  Try to keep titles short, yet
informative; full names are preferred to initials.  If your title
requires, respectively, slant font, italics, or the \MF\ logo, use
|\sl|, |\it|, or |logo10 scaled 1440| (the latter if you have it; if
not, use something similar at that size or rewrite the title); the title
and the running heads should then appear as shown in this document.  If
your title is too long to fit in the running head, use
|\shorttitle #1\endshorttitle|
in a plain \TeX{} article or
|\shortTitle{...}|
in a \LaTeX{} article to provide an abbreviated version.  If there is
more than one author, use
|\shortauthor #1\endshortauthor|
or
|\shortAuthor{...}|
for a plain or \LaTeX{} article, respectively,
to provide an appropriate list of them (if necessary, last names only
or, e.g., |Smith, {\it et al.}|).

\subhead * Author(s) addresses *

These should contain the author's complete mailing address, with
(optional) telephone number and, whenever possible, appropriate
electronic mail addresses (including identification of the network).
The macros involved are |\address| and |\netaddress|.

\subhead * Abstract *

The abstract should be a quick overview of the article, between 60 and
100 words in length.  It should comprise a summary of the main points of
the paper, with only brief references to the literature, if any.  The
abstracts will be used for translations and summaries to appear
elsewhere.  The abstract is the last piece of the top matter or title
block; the text of the article begins immediately after it.

Some authors find that a simple way to construct the abstract is to
select key sentences from the article, once it has been written.  It is
essential that the abstract be self-contained:\ include no bibliographic
citations and no footnotes.  (You could mention \TB\ by name, but don't
refer to it using cryptic codes like Knuth (1984), etc.  Leave that kind
of thing for the body of your article.)

\subhead * Body of Article *

Articles should be 6--10 pages in length (10pt font on 12pt baseline)\Dash
approximately the equivalent of a 25-minute presentation.  Creation
of headers and footers is, on the whole, automatic.  At the top of your
file include the items listed at the beginning of this one, including
the |\preprint| switch, which substitutes one form of footer for
another.  The \editor{} will comment out the |\preprint| line when
producing the final camera copy.
The actual page numbers for the published {\sl Proceedings\/}
will be pasted on when the files are being prepared by
TUG for the printer.  Authors should try to keep formatting fairly
straightforward.  Cross-referencing, if used, should be to sections and
not to page numbers.

\subhead * Bibliography *

The {\sl Proceedings\/} serve not only as a record of the Meeting, but
also as a reference document for members unable to attend and for others
who are not members of the \TUG.  Hence authors must not assume their
readers are familiar with even the more commonly cited documents.  All
sources referred to in the body of the article must be properly
identified in the bibliography.  The following paragraphs describe the
content and order of appearance of the various elements, and the format
used to typeset the bibliographic entries.  See previous TUG {\sl
Proceedings\/} for further examples; uniformity and consistency of style
are the main objectives, to the extent possible.

\subsubhead * Contents of entries: *

Authors' names should be listed in alphabetical order (last name first),
followed by either the full name or initials; in the case of multiple
authors, only the first author's name should be in reverse order.

\TUB\/ has a policy of strict adherence to the pattern, established in
\TB, of setting names of books and periodicals in the {\sl slant font\/}
(i.e., |{\sl slant font\/}|).  The names of a number of books and
periodicals which you may have occasion to cite have macros in the file
|tugboat.cmn|,\footnote{$^1$}{{\bf Note:} This file, which contains
macros common to both \TeX{} and \LaTeX{} usage, was previously called
{\tt tugboat.com}; however, in order to avoid confusion with {\SMC DOS}
naming conventions, as well as to better indicate the {\it common\/}
nature of its content, the name has been changed to {\tt tugboat.cmn}.}
including |\TUB| for \TUB, |\TB| for \TB, etc.; other
macros in the file do useful things, such as refining the kerning for
\AW, etc.  Authors are urged to print out the file {\tt tugboat.cmn}
and use it for reference.

For articles published in journals, enclose the title of the article in
quotation marks, and use the slant font for the name of the journal.
The volume and issue numbers should be included, as well as the page
range of the article, followed by the year.

For articles published in books, put the article title in quotation
marks, followed by the inclusive page numbers, using an en-dash.  The
title of the book should appear in the slant font.  In the case of
edited collections, the editor's name should be included.  Place of
publication is followed by the publisher.  The date is the final element
in the entry.

For books, each entry should include the full name(s) of the author(s),
complete title of the book, place of publication and publisher, and date
of publication.

For all titles, use capital letters on the main words.  Multiple entries
with the same author(s) should be listed in chronological order.

In general, style ``A'', as described in {\sl The Chicago Manual of
Style\/} (pages 438ff), is to be followed; our style diverges with
respect to the position of page ranges for articles within books
(between the article title and that of the book), and the position of
the date for journal articles (end of the entry, as for books).

\subsubhead * Format of entries: *

|tugproc.sty| contains the macro |\entry|, which sets the individual
bibliographic entries.  Bibliographies prepared with {\tt ltugproc.sty}
should use |\bibentry|.
Below are some sample entries (see also {\sl The Chicago Manual of
Style\/} for examples of various types of document references):

\medskip

\entry{Chen, Peehong, Michael A.~Harrison, Jeffrey W.~McCarrell, John
   Coker, and Steve Procter.  ``An Improved User Environment for \TeX.''
   Pages~32--44 in {\sl \TeX\ for Scientific Documentation}, Jacques
   D\'esarm\'enien, ed.  (Lecture Notes in Computer Science 236).
   Heidelberg: Springer, 1986.}

\entry{Knuth, Donald E.  \TB.  Reading, Mass.: \AW, 1984.}

\entry{Parks, Berkeley.  ``\TeX\ Tips for Getting Started.'' {\sl \TeX
   niques\/} {\bf 7}, pages 129--138, 1988.}

\entry{Wujastyk, Dominik.  ``The Many Faces of \TeX.  A Survey of
   Digital \MF\ Fonts.'' \TUB\/ {\bf 9} (2), pages 131--151, 1988.}

\medskip

When referring to publications in the body of the article, use the
following form for consistency throughout the volume: ``$\ldots$\
according to Knuth (page 420) $\ldots$'', ``$\ldots$\ according to
Knuth (1984, page 420) $\ldots$'', or ``$\ldots$\ according to Knuth
(\TB, page 420) $\ldots$'', where the choice depends on how much is
required for positive identification.  But please do not say ``$\ldots$\
according to (Knuth, page 420)\ $\ldots$''.\footnote{$^2$}{Example of a
footnote.}

\subhead * Appendix *

Authors who wish to provide samples of input or output that will not
comfortably fit in the two-column format of their articles are
encouraged to put them in an appendix.  An |\appendix| macro is supplied
(in |tugproc.sty|) for the heading; it automatically starts on a new
page and prints the material in a one-column full-width format.  No
special provision has been made yet for appendices in |ltugproc.sty|;
if you require this facility, please notify the \editor{} as soon as
possible.  Recall that the page limit {\it includes\/} any appendix
material.

\head * Contents *

In the previous section the basic elements of an article were outlined,
from the top matter (title, authors, abstract) to the final items
(bibliography and/or appendix).  This section continues the discussion,
with attention to questions about specific elements in your paper.

\subhead * Headings *

{\sl Proceedings\/} style allows for up to three levels of headings.
For those using {\tt tugproc.sty}, the syntax |\head * ... *|,
|\subhead * ... *|, and |\subsubhead * ... *| is provided.
\LaTeX\ provides |\section{...}|, |\subsection|\hskip0pt\relax|{...}|,
and |\subsubsection{...}|.  The same headings are used in regular issues of
\TUB, except that the first level differs from the \TUB\/ main head, by
using 12 point extended bold instead of the usual 10 point that appears
in ordinary issues.  (These Guidelines illustrate all three levels of
headings.)

There is no requirement that any of these headings be used; they are
available if you feel a need for them.  The only constraint is that when
more than one level is used, the hierarchy first-, second-, third-level
should be followed.  Articles which use other forms of headings (e.g.,
with different fonts or positions) will be made consistent with this
style.  The only centered headings used for the {\sl Proceedings\/} are
for the Abstract, titles or captions for tables and figures, and for the
Appendix, if there is one.

\list[\lettered]
\item Level-one headings should be in mixed case.  The main
   words in the heading should begin with capitals, but {\it not\/} words
   such as |of|, |on|, |by|, |from|, |the|.  Do not use all capitals!

\item For level-two headings, only the first word should
   begin with a capital. {\tt tugproc.sty} users should omit
   punctuation at the end of this heading; a period will be supplied
   by the macro. Users of {\tt ltugproc.sty} must insert the period.

\item For level-three headings only the first word
   should begin with a capital.  Include punctuation at the end of this
   heading, unless the heading itself contains the first word or so of the
   new paragraph.  No punctuation will be supplied by the macro.
   Paragraphs with this heading are indented.
\endlist

\subhead * Footnotes *

Footnotes are for brief comments, not bibliographic information (see the
subsection Bibliography, above).  \TUB\/ style calls for footnotes
referring to the text to be set at 10 point; in plain-based articles
they must be numbered by hand
(|\footnote{$^|$n$|$}|\penalty-2\hskip0pt\relax|{...}|).
Footnotes in tables
and figures should appear immediately below the table or figure, above
the caption.  The footnote flags in tables and figures should be marked
with raised lowercase italic letters; asterisks, daggers, etc., should
be avoided.  Trademark acknowledgements should {\it not\/} be included in
footnotes.  (More on this below.) Other acknowledgements may be made in
the first sentences of the paper, as a footnote, or, if lengthy, kept
for a section at the end of the paper (just before the Bibliography)
using a heading at the appropriate level, e.g., in plain style:
\verbatim
\head * Acknowledgements *
\endverbatim
or the |\subhead| macro if you find you have no need in your article for
a higher level.

\subhead * Verbatim \TeX\ code *

When source code is to be represented in the article, verbatim
techniques should be used.  There are differences between plain \TeX\
and \LaTeX\ verbatim techniques, which will be explained separately.
Remember that displayed verbatim text will reflect the line breaks of
the input file, and will be set in the |\tt| font.  Vertical space above
and below the displayed verbatim text will be inserted automatically.

If you desire particular visual effects in your verbatim text (e.g.,
column effects or indented macro continuation lines in macro
definitions), use spaces and {\it not\/} tabs, which may hamper
electronic transmission of your file (see section below) and will, in
any case, not yield the desired results.

Extensive listings of files should be assigned to the Appendix, where
they will be set one column per page.  Use headings such as
``Listing~1'', etc., for identification.  Please consider using smaller
type, as discussed below, which might save space and paper.  Passages in
verbatim displays break automatically between columns and pages.

\subsubhead * Plain \TeX\ ({\tt tugproc.sty}): *

|\verbatim| and |\endverbatim| are provided for displayed verbatim text.
There is an automatic indent from the left margin.  Initial spaces in
the first line are ignored unless the null-switch is used:\
|\verbatim[]|.

If the text following displayed verbatim text is the continuation of the
paragraph above the display, do not leave a blank line.  A blank line
will cause the automatic paragraph indent to be activated.

For short verbatim items in text (``in-line verbatim''), simply enclose
the item between a pair of vertical bars, for example:\
\verbatim[\inline]|verbatim|\endverbatim.

You might consider using the |\smallcode| switch to reduce a verbatim
example to 9pt, in order to make a long line fit the narrow column,
and/or to save space overall.\endgraf
||
\verbatim[\smallcode]
This is nine point code.
\endverbatim
||
which will produce type of the following size
\verbatim[\smallcode]
This is nine point code.
\endverbatim
instead of this size:
\verbatim
This is ten point code.
\endverbatim

If a significant portion of your displayed verbatim text requires this
smaller size, it is better to use |\everyverbatim| for consistency:\endgraf
% bug in parsing of verbatim options applies baselineskip of \smallcode
% to preceding text; avoid problem by ending the paragraph
\verbatim[\smallcode]
\everyverbatim{\displaystyle{\smallcode}}
\endverbatim
This is placed at the top of your text file (or wherever you want it to
take effect), which avoids having to switch the size for each individual
display.

\subsubhead * \LaTeX\ ({\tt ltugproc.sty}): *

Displayed verbatim text uses the usual verbatim macro environment:
|\begin{verbatim}| and |\end{verbatim}|.

If the displayed verbatim text is to be followed by a continuation of the
paragraph, don't leave a blank line.  A blank line will activate the
usual paragraph indent.

For in-line verbatim text, use the |\verb+...+| construction (almost any
character can be used as matching delimiters; see the \LaTeX\ manual,
pp.\ 65--66, 168).

Where verbatim material is either too long for the line or there is a
need to conserve space, enclose the verbatim environment in a |\small|
group, as shown below:

\verbatim
{\small
\begin{verbatim}
Your text
On these lines
\end{verbatim}
}
\endverbatim

\subhead * Figures and tables *

It is left to authors to decide where to place figures and tables; these
may be inserted in the text, or gathered together in an appendix.
Captions should be centered; table captions should appear above, and
figure captions below their respective elements:
$$
   \hbox{Figure~1: Sample of new font |cmxxfr|}
$$

\subhead * Spelling conventions *

Spelling consistency covers both common words and accepted spellings of
commercial products.  Authors are advised to look at various \TeX\
publications, including \TB\ and issues of \TUB, for general guidance.
See also the Appendix for some pertinent spelling conventions.  Many
items of this kind may be invoked easily using macros in the file
|tugboat.cmn|, which is included in the macro package.  The main
question here is consistency.  Either British or American spelling is
acceptable; pick one and stick with it.

When two words describe a third (``left-justified text''), there is
frequently a hyphen between the first two.  {\sl The Chicago Manual of
Style\/} (1982) has a fairly extensive set of guidelines on dealing with
such noun phrases (pages~176--181), which authors may find useful.
The third edition of {\sl Words into Type\/} (1974) also has a good
section (pages~223--239).

Do not use |---| for em-dashes.  Use {\tt\char'134{Dash}\char'040}
instead.  This inserts thin spaces before and after the dash, and
provides proper control at line breaks.

By the way, in \TUB, commas, etc., do not go inside quotes unless they
are part of what is being quoted.

\subhead * Trademark acknowledgements *

Since the {\sl Proceedings\/} are part of the regular \TUB\/ series,
there is no need to acknowledge trademarks\Dash these are covered in a
general statement at the front of the issue.  Authors are therefore
asked {\it not\/} to include such information either in the text or in
footnotes.  If a product name is newly trademarked or may not be known
to the \editor, this fact may be conveyed in a comment or a covering
message to the \editor.

\subhead * Font use *

Font use should be consistent, and restrained.  In other words, as
with headings and footnotes, your use of fonts should be limited to
what is essential to your exposition.  The following conventions are
suggested:

\list[\lettered]
\item {\it Italics}, rather than {\bf boldface}, should be used for
   emphasis; the {\sl slant\/} font {\it could\/} also be used, but is
better reserved for names of books and periodicals.  Never use
underlining.  Use italic corrections (|\/|) where appropriate.
Italics should not be used for titles of articles; see section on
Bibliography.

\item The |typewriter| or |teletype| font (|\tt|) should be used for
   macro names or anything else to be keyed in, and will often include
the backslash character (e.g., |\entry|); font names should also be in
this font (e.g., |cmr12|).  These examples are produced by using the
appropriate verbatim techniques (see the subsection Verbatim \TeX\
code, above). The |\tt| font is automatically used for verbatim text.
See the Appendix to these Guidelines for a list of \TeX-related words
which are customarily set in this font.

\item The {\smc smallcaps} (|\smc|) font may be used if there are
   terms, especially product names, in uppercase (more or less), such as
   {\smc Unix} or {\smc PostScript}.  This is, however, a choice left to
   the author.  For these particular examples key |{\smc smallcaps}|,
   |{\smc Unix}|, |{\smc PostScript}|, respectively.
   Note: \LaTeX's |\sc| is the same as |\smc|.

   A slightly larger variation of small caps, obtained with
   |\SMC| and the term keyed in all caps, as {\SMC ASCII} or
   {\SMC WYSIWYG} (keyed as |{\SMC ASCII}| or |{\SMC WYSIWYG}|,
   respectively), is particularly suitable for acronyms.  Again, the
   choice is left to the author.

\item There are certain font-related conventions for trademarks and
   other citations.  See the Appendix for one list. The file
   {\tt tugboat.cmn}, required for both plain \TeX\ and \LaTeX\ macro
   packages, provides definitions for a number of frequently used terms
   and logos.  If you do not have the \MF\ font (|logo*|), for example,
   use sans serif in an appropriate size; users of the plain \TeX\
   macros can type |{\niness METAFONT}|, to get {\niness
   METAFONT}.\footnote{$^3$}{Sans serif will not
   be available unless the macro |\LoadSansFonts| has been invoked.
   Do not use |\LoadSansFonts| unless needed, as it will occupy memory
   unnecessarily. For an example of its use, see the top of this file
   and {\tt tugboat.sty}.} \LaTeX\ users should use |\sf| if they need it.
   Macro definitions for non-standard fonts should
   be put at the top of the file, in the preamble.
\endlist

\head * Special effects *

Special effects include traditional art work (photographs, diagrams,
etc.), which must be pasted into space left for that purpose.  Special
effects may also include special font work, or \PS\ material which can
be provided in electronic form.  All three types require special
handling.

\subhead * Art work *

If physical art work is required,
originals should be sent with the preliminary version.
Inserts (including photographs) should be clear\-ly identified on the
back (and for photographs, especially, very carefully, at the edge),
e.g., ``Fig.~1''; their location in the text should be shown
explicitly.

\subhead * Special fonts *

If your paper requires fonts which are not generally available, please
indicate this clearly to the \editor.  You will most likely be asked to
supply the \MF\ source(s).  Additional time and effort may be required
to process your paper; final output for the {\sl Proceedings\/} will be
prepared on a phototypesetter and unexpected effects have appeared in
the past when fonts have not previously been tested at typesetter
resolutions.  This testing must commence soon after receipt of the
preliminary version, in order to resolve any potential problems.

\subhead * PostScript and other graphics inser\-tions *

\PS\ and other graphics have been included successfully in \TUB\/ issues,
including the {\sl Proceedings}.  However, the author must inform the
\editor{} about such special elements.  As with special fonts, testing
must be done as early as possible, in order to avoid unexpected problems
at later stages of production.

With respect to \PS\ material, encapsulated \PS\ is the most acceptable
form of \PS\ file, the one that gives the fewest problems in processing.
Color figures should be converted to black and white.  Ask for the {\tt
epsf.sty} file from the \editor.

If using a non-\PS\ graphics program to generate non-text material,
clearly identify the program (correct file names, version number,
equipment and working environments).  Make a special note to the \editor{}
that these additional items are required.
You may be asked to submit them along with the electronic
file of your preliminary version.

If problems encountered cannot be solved in a timely manner, the author
may be asked to provide clean camera copy (300dpi minimum).  This may be
unavoidable for the preprints; an effort will be made to solve any such
problems before final publication in the {\sl Proceedings}.

Final production of the {\sl Proceedings\/} issue of \TUB\ will be done
on a high-resolution phototypesetter or \PS{} imagesetter, the latter
using Radical Eye Software's |dvips|.  Any |\special| commands must be
compatible with this output device driver.

\head * Updating your Article\\ after Presentation *

Following presentation of your paper at the Annual Meeting, you may wish
to add information, or report on responses to it.  Rather than
incorporate these changes in the body of the article, you may find it
simpler just to add a new section, ``Update'', for such additions.  Use
the appropriate heading level.

\head * Bibliography *

\entry{{\sl Chicago Manual of Style}, 13th ed. Chicago: University
   of Chicago Press, 1982.}

\entry{{\sl Words into Type}, 3rd ed., based on studies by
   Marjorie E. Skillin, Robert M. Gay, {\it et al.}, Englewood Cliffs,
   N.J.: Prentice Hall, 1974.}

\entry{Whitney, Ron, and Barbara Beeton, ``\TUB\/ Authors' Guide'',
   \TUB\/ {\bf 10}(4), pages 378--385, 1989. Also available
   electronically as the file {\tt tubguide.tex},
   via ftp from the usual archives, or by request to the \editor.}

\medskip

The following article should be required reading for every person
interested in computer composition.  It demonstrates how
much one may achieve with a minimum of fancy features (fonts, format,
etc.), each of which presents risks for impeding comprehension.

\medskip

\entry{Southall, Richard.  ``First Principles of Typographic Design for
   Document Production'' \TUB\/ {\bf 5} (2), pages 79--90, 1984; corrigenda,
   {\it ibid}., {\bf 6} (1), page 6, 1985.}

\appendix Appendix \endappendix

\message{Begin appendix}

\head * Spelling Conventions *

\halign{\qquad\qquad#\hss\kern4em&#\hss\quad\cr
braces, curly braces (not ``brackets'') & proof copy               \cr
database                                & proofreaders             \cr
formatting, formatted                   & re-key                   \cr
left justified {\it vs.}\ left-justified text
        & right justified {\it vs.}\ margins                     \cr
minicomputer                            & uppercase, lowercase     \cr
multilevel                              & word processing          \cr
on-line                                 & workstation              \cr
PCs, Macs, 1980s (no apostrophe)        &                          \cr
}

\medskip

\head * Typographic Representation *
In addition to the following, consult previous {\sl Proceedings\/} and
other \TeX-related documents. It is also strongly recommended that
authors print out a copy of the file {\tt tugboat.cmn}, which contains
many already defined macros for logos and commonly used names.
(Some of these are listed below.)

\medskip

\halign{\qquad\qquad#\hss\kern5em&#\hss\quad\cr
ASCII\quad(or {\SMC ASCII}\ \ |{\SMC ASCII}|)
        &|pk|\quad(\verbatim[\inline]|pk|\endverbatim)         \cr
AT\&T
        & |plain.tex|\quad(\verbatim[\inline]|plain.tex|\endverbatim)\cr
Bitnet, Netnorth
          & |pxl|\quad(\verbatim[\inline]|pxl|\endverbatim)      \cr
|dvi|\quad(\verbatim[\inline]|dvi|\endverbatim)
        & {\it Textures\/} (italics; do not use |\TeX|)        \cr
EARN
        & |TANGLE|\quad(\verbatim[\inline]|TANGLE|\endverbatim)\cr
Emacs
        & UNIX\quad(or {\smc Unix}\ \ |{\smc Unix}|)           \cr
|gf|\quad(\verbatim[\inline]|gf|\endverbatim)
        & uucp                                                 \cr
HP LaserJet
        & {\it vi\/}                                           \cr
imPRESS
        & |VIRTEX|\quad(\verbatim[\inline]|VIRTEX|\endverbatim)\cr
|INITEX|\quad(\verbatim[\inline]|INITEX|\endverbatim)
        & WordPerfect                                          \cr
|log|\quad(\verbatim[\inline]|log|\endverbatim)
                                                                \cr
}

\medskip
\noindent
The following terms are already defined in {\tt tugboat.cmn}.   Please
avoid redefining them, as the definitions provided are the official
definitions used in all TUG publications.

\medskip

\halign{\qquad\qquad#\hss\kern10em&#\hss\quad\cr
\AW\quad(|\AW|)
        & \SGML\quad (|\SGML|)                                 \cr
\AmSTeX\quad(|\AmSTeX|)
        & \SliTeX\quad(|\SliTeX|)                              \cr
\BibTeX\quad(|\BibTeX|)
        & \TeX\quad (|\TeX|)                                   \cr
\La\quad (|\La|)
        & \TTN\quad (|\TTN|)                                   \cr
\LAMSTeX\quad (|\LAMSTeX|)
        & \TUG\quad(|\TUG|)                                    \cr
\LaTeX\quad (|\LaTeX|)
        & \TB\quad(|\TB|)                                      \cr
\AllTeX\quad(|\AllTeX|)
        & \TeXhax\quad(|\TeXhax|)                              \cr
{\MF}\quad (|\MF|)
        & \TeXXeT\quad(|\TeXXeT|)                              \cr
{\PiC}\quad (|\PiC|)
        & \TUB\quad(|\TUB|)                                    \cr
{\PiCTeX}\quad (|\PiCTeX|)
        & |WEB|\quad(|\WEB|)                                   \cr
\PS\quad(|\PS|)
        \cr
}

\newpage
\head * Sample Commands *

\everyverbatim{\enablemetacode}

\halign{\tabskip=0pt#\hfil\tabskip=1em & #\hfil \cr
\qquad\TeX{}                    & \qquad \LaTeX{} \cr
\noalign{\medskip}
|\input tugproc.sty|            & |\documentstyle{ltugproc}| \cr
|\title * <title> *|            & |\title{<title>}| \cr
|\shorttitle * <abbrev title> *| & |\shorttitle{<short title>}| \cr
|\author * <name> *|            & |\author{<name(s)>}|\cr
|\address * <address with linebreaks> *|
                & |\address{<address with linebreaks>}| \cr
|\netaddress[\network{<Netid>}] *<id>@<node> *|
                &|\netaddress[\network{<Netid>}]{<id>@<node>}|\cr
|\abstract|                     & |\begin{abstract}| \cr
|\endabstract|                  & |\end{abstract}| \cr
|\article|                      & |\maketitle| \cr
|\head * <Heading Text> *|      & |\section{<Heading Text>}| \cr
|\subhead * <Subhead Text> *|   & |\subsection{<Subhead Text>.}| \cr
\quad(Note: OMIT punctuation)   & \quad(Note: insert period) \cr
|\subsubhead * <Subsubhead text>. *|
                & |\subsubsection{<Subsubhead text>.}| \cr
\multispan2{\quad(Note: insert period in subsubhead; if part of first
  sentence, no punctuation is necessary)\hfil} \cr
|\verbatim|                     & |\begin{verbatim}| \cr
|\endverbatim|                  & |\end{verbatim}| \cr
\vrt|<whatever>|\vrt{}          & |\verb+<whatever>+| \cr
|\figure[\top]| or |\figure[\bot]|      & |\begin{figure}| \cr
\quad|<figure content>|         & \quad|<figure content>| \cr
|\caption{<The caption>}|       & |\caption{<The caption>}| \cr
|\endfigure|                    & |\end{figure}| \cr
\quad(Note: for tables use |\halign|)   & |\begin{table}| \cr
\quad|<no command>|             & |\end{table}| \cr
|\head * Bibliography *|        & |\section{Bibliography}| \cr
|\entry |                       & |\bibentry| \cr
}

\head * Checklist *

Careful adherence to the established procedures will help avoid errors
and delays in processing.  Before mailing your paper to the \editor,
make sure you have done the following:

\medskip
\noindent{\bf Original version}
\list[\tag{\bull}]
\item Be sure your paper conforms to the style you have chosen (plain
 \TeX\ or \LaTeX).
\item Check files for adherence to format conventions.
\item Proofread your file for spelling errors.
\item Are your graphics encapsulated PostScript?
\item Are your graphics black and white images?
\item Arrange for transfer of all necessary files and additional
 material to the \editor{} (e-mail, ftp, or disc, plus paper copies).
\item Send originals of camera-ready graphics or other hard-copy material,
      packing it carefully to avoid damage during shipment.
\endlist
\medskip
\noindent{\bf Preprint  version}
\list[\tag{\bull}]
\item As above for original version.
\item Updates have been made to the file returned to you by the
 \editor, and this is the file you send back to the editor.  Do
 {\it not\/} send a new version of any other file.
\item Have you incorporated referee's recommendations?  If not, then
 provide a separate explanation of suggestions not followed.
\endlist
\medskip
\noindent{\bf Proceedings version}
\list[\tag{\bull}]
\item As above for original and preprint versions.
\item Return signed copyright transfer form to \editor.
\endlist

\endarticle

%%%%%%%%%%%%%%%%%%%%%%%%%%%%%%%%%%%%%%%%%%%%%%%%%%%%%%%%%%%%%%%%%%%%%%%%
%
%      *** Change History ***

 3 Feb 94: comments and minor adjustments (bb)
 1 Feb 94: modifications for 1994 (MG)
 8 Mar 92: final cleanup after e-mail discussion (bb)
           add standard headers, prepare for archive installation
 3 Mar 92: input changes from UK discussion (bb)
           parameterize \MtgYear
           problem remaining with verbatim file to be input to LaTeX;
             looking for code to install in ltugproc.sty; hold for 1993
20 FEB 92: input changes from UK discussion (on train to Birmingham.
           We only got to page 5 ...) (Ch.)
12 FEB 92: discuss changes with Barbara Beeton (at Chris Rowley's
           in London) (Ch.)
12 MAR 91: input final changes to 1991 TUG Proceedings Guidelines (Ch.)
 9 MAR 91: input further changes, following discussions with RW (Ch.)
26 FEB 91: input changes/updates for 1991 Guidelines (Ch.)
           Note: these changes are being input into a file
                 which Barb Beeton has already worked on.
26 JAN 90: finished changes to both the Guidelines and the macros.
25 JAN 90: worked in updating Guidelines (Ch.)
22 NOV 89: continued updating file (Ch.)
21 NOV 89: began updating Guidelines, based on 1989 experiences (Ch.)
